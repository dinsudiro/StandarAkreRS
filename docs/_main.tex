% Options for packages loaded elsewhere
\PassOptionsToPackage{unicode}{hyperref}
\PassOptionsToPackage{hyphens}{url}
%
\documentclass[
]{book}
\usepackage{amsmath,amssymb}
\usepackage{lmodern}
\usepackage{ifxetex,ifluatex}
\ifnum 0\ifxetex 1\fi\ifluatex 1\fi=0 % if pdftex
  \usepackage[T1]{fontenc}
  \usepackage[utf8]{inputenc}
  \usepackage{textcomp} % provide euro and other symbols
\else % if luatex or xetex
  \usepackage{unicode-math}
  \defaultfontfeatures{Scale=MatchLowercase}
  \defaultfontfeatures[\rmfamily]{Ligatures=TeX,Scale=1}
\fi
% Use upquote if available, for straight quotes in verbatim environments
\IfFileExists{upquote.sty}{\usepackage{upquote}}{}
\IfFileExists{microtype.sty}{% use microtype if available
  \usepackage[]{microtype}
  \UseMicrotypeSet[protrusion]{basicmath} % disable protrusion for tt fonts
}{}
\makeatletter
\@ifundefined{KOMAClassName}{% if non-KOMA class
  \IfFileExists{parskip.sty}{%
    \usepackage{parskip}
  }{% else
    \setlength{\parindent}{0pt}
    \setlength{\parskip}{6pt plus 2pt minus 1pt}}
}{% if KOMA class
  \KOMAoptions{parskip=half}}
\makeatother
\usepackage{xcolor}
\IfFileExists{xurl.sty}{\usepackage{xurl}}{} % add URL line breaks if available
\IfFileExists{bookmark.sty}{\usepackage{bookmark}}{\usepackage{hyperref}}
\hypersetup{
  pdftitle={Standar Akreditasi Rumah Sakit},
  pdfauthor={Achmaddudin Sudiro},
  hidelinks,
  pdfcreator={LaTeX via pandoc}}
\urlstyle{same} % disable monospaced font for URLs
\usepackage{longtable,booktabs,array}
\usepackage{calc} % for calculating minipage widths
% Correct order of tables after \paragraph or \subparagraph
\usepackage{etoolbox}
\makeatletter
\patchcmd\longtable{\par}{\if@noskipsec\mbox{}\fi\par}{}{}
\makeatother
% Allow footnotes in longtable head/foot
\IfFileExists{footnotehyper.sty}{\usepackage{footnotehyper}}{\usepackage{footnote}}
\makesavenoteenv{longtable}
\usepackage{graphicx}
\makeatletter
\def\maxwidth{\ifdim\Gin@nat@width>\linewidth\linewidth\else\Gin@nat@width\fi}
\def\maxheight{\ifdim\Gin@nat@height>\textheight\textheight\else\Gin@nat@height\fi}
\makeatother
% Scale images if necessary, so that they will not overflow the page
% margins by default, and it is still possible to overwrite the defaults
% using explicit options in \includegraphics[width, height, ...]{}
\setkeys{Gin}{width=\maxwidth,height=\maxheight,keepaspectratio}
% Set default figure placement to htbp
\makeatletter
\def\fps@figure{htbp}
\makeatother
\setlength{\emergencystretch}{3em} % prevent overfull lines
\providecommand{\tightlist}{%
  \setlength{\itemsep}{0pt}\setlength{\parskip}{0pt}}
\setcounter{secnumdepth}{5}
\usepackage{booktabs}
\ifluatex
  \usepackage{selnolig}  % disable illegal ligatures
\fi
\usepackage[]{natbib}
\bibliographystyle{plainnat}

\title{Standar Akreditasi Rumah Sakit}
\author{Achmaddudin Sudiro}
\date{2022-05-11}

\begin{document}
\maketitle

{
\setcounter{tocdepth}{1}
\tableofcontents
}
\hypertarget{perkenalan}{%
\chapter*{Perkenalan}\label{perkenalan}}
\addcontentsline{toc}{chapter}{Perkenalan}

Hi, Salam kenal dengan saya Achmaddudin Sudiro (Din Sudiro), anda bisa memanggil saya dengan sebutan apapun yang anda suka.

Dalam rangka memudahkan penyelenggara layanan kesehatan, lembaga akreditasi, akademisi, maupun masyarakat umum untuk mengakses informasi standar akreditasi rumah sakit yang tertuang dalam Keputusan Menteri Kesehatan Republik Indonesia (Kepmenkes) No.~1128 tahun 2022 tentang Standar Akreditasi Rumah Sakit, saya berinisiatif untuk mengonversi dokumen Kepmenkes tersebut kedalam format buku elektronik ini dibuat menggunakan program r-studio (bookdown).

Ada beberapa fitur yang dapat anda manfaatkan untuk menunjang kenyamanan anda dalam menelusuri informasi terkait akreditasi kepmenkes dalam format buku elektronik ini, seperti fitur pencarian, penyesuaian ukuran font, juga fitur unduh (format pdf).

Saya mengharapkan anda memberikan masukan juga pandangan anda yang berharga untuk perbaikan buku elektronik ini, maupun untuk mendiskusikan hal lain. Untuk dapat menghubungi saya, Anda dapat mengirimkan pesan melalui \textbf{\href{mailto:achmaddudinsudiro2@gmail.com}{E-mail}} , \textbf{\href{https://mobile.twitter.com/din_sudiro}{Twitter}}, atau melalui \textbf{\href{https://www.linkedin.com/in/din-sudiro/}{LinkedIn}}.

Semoga buku elektronik standar akreditasi rumah sakit ini dapat dimanfaatkan dengan baik untuk membantu proses penyelenggaraan akreditasi di rumah sakit.
Terima Kasih.

Salam hangat,

Rumpin, 11 Mei 2022

\textbf{Din Sudiro}

\hypertarget{bab-1-pendahuluan}{%
\chapter*{Bab 1 Pendahuluan}\label{bab-1-pendahuluan}}
\addcontentsline{toc}{chapter}{Bab 1 Pendahuluan}

\hypertarget{a.-latar-belakang}{%
\section*{A. Latar Belakang}\label{a.-latar-belakang}}
\addcontentsline{toc}{section}{A. Latar Belakang}

Rumah Sakit adalah institusi pelayanan kesehatan yang menyelenggarakan pelayanan kesehatan perorangan secara paripurna yang menyediakan pelayanan rawat inap, rawat jalan, dan gawat darurat. Dalam memberikan pelayanan, rumah sakit harus memperhatikan mutu dan keselamatan pasien. Pelayanan kesehatan yang bermutu adalah pelayanan yang memiliki karakter aman, tepat waktu, efisien, efektif, berorientasi pada pasien, adil dan terintegrasi. Pemenuhan mutu pelayanan di rumah sakit dilakukan dengan dua cara yaitu peningkatan mutu secara internal dan peningkatan mutu secara eksternal.

Peningkatan Mutu Internal ( \emph{Internal Continous Quality Improvement} ) yaitu rumah sakit melakukan upaya peningkatan mutu secara berkala antara lain penetapan, pengukuran, pelaporan dan evaluasi indikator mutu serta pelaporan insiden keselamatan pasien. Peningkatan mutu secara internal ini menjadi hal terpenting bagi rumah sakit untuk menjamin mutu pelayanan. Peningkatan Mutu Eksternal ( \emph{External Continous Quality Improvement} ) merupakan bagian dari upaya peningkatan mutu pelayanan di rumah sakit secara keseluruhan. Beberapa kegiatan yang termasuk peningkatan mutu eksternal adalah perizinan, sertifikasi, dan akreditasi. Rumah sakit melakukan peningkatan mutu internal dan eksternal secara berkesinambungan ( \emph{continuous quality improvement} ).

Akreditasi adalah pengakuan terhadap mutu pelayanan rumah sakit setelah dilakukan penilaian bahwa rumah sakit telah memenuhi standar akreditasi yang disetujui oleh Pemerintah. Pada bulan Desember 2021 Kementerian Kesehatan mencatat 3.120 rumah sakit telah teregistrasi. Sebanyak 2.482 atau 78,8\% rumah sakit telah terakreditasi dan 638 rumah sakit atau 21,2\% belum terakreditasi.

Upaya percepatan akreditasi rumah sakit mengalami beberapa kendala antara lain adanya isu atau keluhan terkait lembaga penilai akreditasi yang juga melakukan workshop atau bimbingan, penilaian akreditasi dianggap mahal, masih kurangnya peran pemerintah daerah dan pemilik rumah sakit dalam pemenuhan syarat akreditasi, akuntabilitas lembaga, dan lain-lain.

Pemerintah mengharapkan pada tahun 2024 seluruh rumah sakit di Indonesia telah terakreditasi sesuai dengan target RPJMN tahun 2020 - 2024. Dalam upaya meningkatkan cakupan akreditasi rumah sakit, Pemerintah mendorong terbentuknya lembaga-lembaga independen penyelenggara akreditasi serta transformasi sistem akreditasi rumah sakit. Sejalan dengan terbentuknya lembaga-lembaga independen penyelenggara akreditasi maka perlu ditetapkan standar akreditasi rumah sakit yang akan dipergunakan oleh seluruh lembaga independen penyelenggara akreditasi rumah sakit dalam melaksanakan penilaian akreditasi.

Proses penyusunan standar akreditasi rumah sakit diawali dengan pembentukan tim yang melakukan sandingan dan benchmarking standar akreditasi dengan menggunakan referensi Standar Nasional Akreditasi Rumah Sakit Edisi 1.1 dari Komisi Akreditasi Rumah Sakit, \emph{Joint Commission International Standards for Hospital} edisi 7, regulasi perumahsakitan serta panduan prinsip-prinsip standar akreditasi edisi 5 yang dikeluarkan oleh \emph{The International Society for Quality in Health Care} (ISQua). Selanjutnya dilakukan pembahasan dengan melibatkan perwakilan dari lembaga independen penyelenggara akreditasi rumah sakit, organisasi profesi, asosiasi perumahsakitan, rumah sakit dan akademisi. Selanjutnya hasil diskusi tersebut dibahas lebih lanjut oleh panelis penyusunan standar akreditasi rumah sakit dengan mendapat masukan secara tertulis dari lembaga independen penyelenggara akreditasi rumah sakit. Penyusunan standar akreditasi rumah sakit mempertimbangkan penyederhanaan standar akreditasi agar lebih mudah dipahami dan dapat dilaksanakan oleh rumah sakit.

\hypertarget{b.-tujuan}{%
\section*{B. Tujuan}\label{b.-tujuan}}
\addcontentsline{toc}{section}{B. Tujuan}

\begin{enumerate}
\def\labelenumi{\arabic{enumi}.}
\tightlist
\item
  Untuk meningkatkan mutu dan keselamatan pasien di rumah sakit.
\item
  Menjadi acuan bagi lembaga independen penyelenggara akreditasi rumah sakit dan rumah sakit dalam penyelenggaraan akreditasi rumah sakit.
\item
  Menjadi acuan bagi Kementerian Kesehatan, dinas kesehatan daerah provinsi, dan dinas kesehatan daerah kabupaten/kota dalam pembinaan dan evaluasi mutu dan keselamatan pasien di rumah sakit.
\end{enumerate}

\hypertarget{c.-ruang-lingkup}{%
\section*{C. Ruang Lingkup}\label{c.-ruang-lingkup}}
\addcontentsline{toc}{section}{C. Ruang Lingkup}

\begin{enumerate}
\def\labelenumi{\arabic{enumi}.}
\tightlist
\item
  Penyelenggaraan akreditasi rumah sakit yaitu persiapan, pelaksanaan penilaian akreditasi, dan pasca akreditasi.
\item
  Standar akreditasi rumah sakit meliputi gambaran umum, maksud dan tujuan, serta elemen penilaian pada setiap kelompok standar akreditasi rumah sakit.
\end{enumerate}

\hypertarget{d.-kelompok-standar-akreditasi-rumah-sakit}{%
\section*{D. Kelompok Standar Akreditasi Rumah Sakit}\label{d.-kelompok-standar-akreditasi-rumah-sakit}}
\addcontentsline{toc}{section}{D. Kelompok Standar Akreditasi Rumah Sakit}

Standar Akreditasi Rumah Sakit dikelompokkan menurut fungsi- fungsi penting yang umum dalam organisasi perumahsakitan. Standar dikelompokkan menurut fungsi yang terkait dengan penyediaan pelayanan bagi pasien ( \emph{good clinical governance} ) dan upaya menciptakan organisasi rumah sakit yang aman, efektif, dan dikelola dengan baik ( \emph{good corporate governance} ). Standar Akreditasi Rumah Sakit dikelompokkan sebagai berikut:

\begin{enumerate}
\def\labelenumi{\arabic{enumi}.}
\tightlist
\item
  Kelompok Manajemen Rumah Sakit terdiri atas: Tata Kelola Rumah Sakit (TKRS), Kualifikasi dan Pendidikan Staf (KPS), Manajemen Fasilitas dan Keselamatan (MFK), Peningkatan Mutu dan Keselamatan Pasien (PMKP), Manajemen Rekam Medik dan Informasi Kesehatan (MRMIK), Pencegahan dan Pengendalian Infeksi (PPI), dan Pendidikan dalam Pelayanan Kesehatan (PPK).
\item
  Kelompok Pelayanan Berfokus pada Pasien terdiri atas: Akses dan Kontinuitas Pelayanan (AKP), Hak Pasien dan Keluarga (HPK), Pengkajian Pasien (PP), Pelayanan dan Asuhan Pasien (PAP), Pelayanan Anestesi dan Bedah (PAB), Pelayanan Kefarmasian dan Penggunaan Obat (PKPO), dan Komunikasi dan Edukasi (KE).
\item
  Kelompok Sasaran Keselamatan Pasien (SKP).
\item
  Kelompok Program Nasional (PROGNAS).
\end{enumerate}

\hypertarget{bab-2-penyelenggara-akreditasi-rumah-sakit}{%
\chapter*{Bab 2 Penyelenggara Akreditasi Rumah Sakit}\label{bab-2-penyelenggara-akreditasi-rumah-sakit}}
\addcontentsline{toc}{chapter}{Bab 2 Penyelenggara Akreditasi Rumah Sakit}

\hypertarget{a.-persiapan-akreditasi}{%
\section*{A. Persiapan Akreditasi}\label{a.-persiapan-akreditasi}}
\addcontentsline{toc}{section}{A. Persiapan Akreditasi}

Persiapan dilakukan sepenuhnya oleh rumah sakit secara mandiri atau dengan pembinaan dari Kementerian Kesehatan, dinas kesehatan daerah provinsi, dan dinas kesehatan daerah kabupaten/kota maupun lembaga lain yang kompeten. Kegiatan persiapan akreditasi antara lain pemenuhan syarat untuk dapat diakreditasi dengan pemenuhan kelengkapan dokumen pelayanan dan perizinan, peningkatan kompetensi staf melalui pelatihan, dan kesiapan fasilitas pelayanan sesuai dengan peraturan perundang-undangan.

Rumah Sakit dapat melakukan penilaian mandiri secara periodik tentang pemenuhan standar akreditasi rumah sakit sehingga tergambar kemampuan rumah sakit dalam memenuhi standar akreditasi yang ditetapkan. Setelah dinilai mampu oleh pimpinan rumah sakit, maka rumah sakit dapat mengajukan permohonan survei kepada lembaga independen penyelenggara akreditasi yang dipilih oleh rumah sakit. Pemilihan lembaga dilaksanakan secara sukarela oleh rumah sakit dan tidak atas paksaan pihak manapun.

Rumah sakit yang mengajukan permohonan survei akreditasi paling sedikit harus memenuhi persyaratan sebagai berikut:

\begin{enumerate}
\def\labelenumi{\arabic{enumi}.}
\tightlist
\item
  Rumah sakit memiliki perizinan berusaha yang masih berlaku dan teregistrasi di Kementerian Kesehatan;
\item
  Kepala atau direktur rumah sakit harus seorang tenaga medis yang mempunyai kemampuan dan keahlian di bidang perumahsakitan;
\item
  Rumah sakit memiliki Izin Pengelolaan Limbah Cair (IPLC) yang masih berlaku;
\item
  Rumah sakit memiliki kerja sama dengan pihak ketiga yang mempunyai izin sebagai pengolah dan/atau sebagai transporter limbah B3 yang masih berlaku atau izin alat pengolah limbah B3;
\item
  Seluruh tenaga medis di rumah sakit yang menyelenggarakan pelayanan kesehatan (pemberi asuhan) memiliki Surat Tanda Registrasi (STR) dan Surat Izin Praktik (SIP) yang masih berlaku atau surat tugas sesuai dengan ketentuan peraturan perundang- undangan;
\item
  Rumah sakit bersedia melaksanakan kewajiban dalam meningkatkan mutu dan keselamatan pasien; dan
\item
  Pemenuhan Sarana Prasarana dan Alat Kesehatan (SPA) minimal 60\% berdasarkan ASPAK dan telah tervalidasi 100\% oleh Kementerian Kesehatan atau dinas kesehatan daerah setempat sesuai dengan kewenangannya.
\end{enumerate}

\hypertarget{b.-pelaksanaan-penilaian-akreditasi}{%
\section*{B. Pelaksanaan Penilaian Akreditasi}\label{b.-pelaksanaan-penilaian-akreditasi}}
\addcontentsline{toc}{section}{B. Pelaksanaan Penilaian Akreditasi}

Lembaga independen penyelenggara akreditasi melaksanakan penilaian persyaratan rumah sakit yang mengajukan permohonan kemudian Lembaga menetapkan waktu pelaksanaan akreditasi setelah persyaratan dipenuhi rumah sakit. Penilaian akreditasi dilakukan dengan metode daring dan/atau luring sesuai tahapan pelaksanaan akreditasi. Adapun tahapan pelaksanaan penilaian akreditasi adalah sebagai berikut:

\hypertarget{persiapan-dan-penjelasan-survei}{%
\subsection*{1. Persiapan dan penjelasan survei}\label{persiapan-dan-penjelasan-survei}}
\addcontentsline{toc}{subsection}{1. Persiapan dan penjelasan survei}

Pada tahap ini lembaga penyelenggara akreditasi menyampaikan seluruh rangkaian kegiatan akreditasi dimulai dari persiapan survei, pelaksanaan survei dan setelah survei. Penjelasan dapat dilakukan dengan metode daring menggunakan media informasi yang tersedia dan dapat diakses oleh rumah sakit.

\hypertarget{penyampaian-dan-pemeriksaan-dokumen}{%
\subsection*{2. Penyampaian dan pemeriksaan dokumen}\label{penyampaian-dan-pemeriksaan-dokumen}}
\addcontentsline{toc}{subsection}{2. Penyampaian dan pemeriksaan dokumen}

Rumah Sakit menyampaikan dokumen kepada lembaga independen penyelenggara akreditasi melalui sistem informasi yang telah disediakan oleh lembaga independen penyelenggara akreditasi yang bersangkutan. Jenis dokumen yang akan disampaikan oleh rumah sakit mengikuti permintaan dari surveior lembaga independen penyelenggara akreditasi yang disesuaikan dengan standar akreditasi. Lembaga independen penyelenggara akreditasi melakukan evaluasi dan analisis dokumen dan melakukan klarifikasi kepada rumah sakit terhadap dokumen-dokumen tersebut. Kegiatan ini dilakukan secara daring menggunakan sistem informasi yang dapat diakses oleh rumah sakit.

\hypertarget{telusur-dan-kunjungan-lapangan}{%
\subsection*{3. Telusur dan kunjungan lapangan}\label{telusur-dan-kunjungan-lapangan}}
\addcontentsline{toc}{subsection}{3. Telusur dan kunjungan lapangan}

Telusur dan kunjungan lapangan dilakukan oleh lembaga independen penyelenggara akreditasi setelah melakukan klarifikasi dokumen yang disampaikan oleh rumah sakit. Telusur dan kunjungan lapangan bertujuan untuk memastikan kondisi lapangan sesuai dengan dokumen yang disampaikan, serta untuk mendapatkan hal-hal yang masih perlu pembuktian lapangan oleh surveior. Pada saat telusur, surveior akan melakukan observasi, wawancara staf, pasien, keluarga, dan pengunjung serta simulasi.

Lembaga independen penyelenggara akreditasi menentukan jadwal pelaksanaan telusur dan kunjungan lapangan. Jumlah hari dan jumlah surveior yang melaksanakan telusur dan kunjungan lapangan sesuai dengan ketentuan sebagai berikut:

\begin{longtable}[]{@{}lccc@{}}
\toprule
Klasifikasi RS & Kelas RS & Jumlah Hari & Jumlah Surveior \\
\midrule
\endhead
RS Umum & A & 3 & 4 \\
& B & 2 & 3 \\
& C & 2 & 2 \\
& D & 2 & 2 \\
RS Khusus & A & 2 & 3 \\
& B & 2 & 2 \\
& C & 2 & 2 \\
\bottomrule
\end{longtable}

\hypertarget{penilaian}{%
\subsection*{4. Penilaian}\label{penilaian}}
\addcontentsline{toc}{subsection}{4. Penilaian}

Lembaga independen penyelenggara akreditasi menetapkan tata cara dan tahapan penilaian akreditasi dengan berpedoman pada standar akreditasi yang dipergunakan saat survei akreditasi. Tahapan penilaian ditentukan lembaga independen penyelenggara akreditasi dengan menerapkan prinsip-prinsip keadilan, profesionalisme, dan menghindari terjadinya konflik kepentingan. Lembaga independen penyelenggara akreditasi membuat instrumen, daftar tilik dan alat bantu untuk surveior dalam melakukan penilaian agar hasil yang diperoleh objektif dan dapat dipertanggungjawabkan.

Penentuan skor dari elemen penilaian dilakukan dengan memperhatikan kelengkapan dokumen, hasil telusur, kunjungan lapangan, simulasi kepada petugas, wawancara, dan klarifikasi yang ada di standar akreditasi mengikuti ketentuan sebagai berikut:

\begin{longtable}[]{@{}
  >{\raggedright\arraybackslash}p{(\columnwidth - 10\tabcolsep) * \real{0.00}}
  >{\raggedright\arraybackslash}p{(\columnwidth - 10\tabcolsep) * \real{0.23}}
  >{\raggedright\arraybackslash}p{(\columnwidth - 10\tabcolsep) * \real{0.25}}
  >{\raggedright\arraybackslash}p{(\columnwidth - 10\tabcolsep) * \real{0.34}}
  >{\raggedright\arraybackslash}p{(\columnwidth - 10\tabcolsep) * \real{0.16}}
  >{\raggedright\arraybackslash}p{(\columnwidth - 10\tabcolsep) * \real{0.02}}@{}}
\toprule
No & Kriteria & Skor 10 (TL) & Skor 5 (TS) & Skor 0 (TT) & TDD \\
\midrule
\endhead
1 & Pemenuhan elemen penilaian & ≥80\% & 20\% s.d \textless80\% & \textless20\% & Tidak dapat diterapkan \\
2 & Bukti kepatuhan & Bukti kepatuhan ditemukan secara konsisten pada semua bagian/unit di mana persyaratan- persyaratan tersebut berlaku. Catatan: Hasil pengamatan tidak dapat dianggap sebagai temuan apabila hanya terjadi pada 1 (satu) pengamatan (observasi). & Bukti kepatuhan ditemukan tidak konsisten/ hanya pada sebagian unit di mana persyaratan-persyaratan tersebut berlaku (misalnya ditemukan kepatuhan di IRI, namun tidak di IRJ, patuh pada ruang operasi namun tidak patuh di unit rawat sehari (day surgery), patuh pada area-area yang menggunakan sedasi namun tidak patuh di klinik gigi) & Bukti kepatuhan tidak ditemukan pada semua bagian/unit di mana & \\
3 & Hasil wawancara dari pemenuhan persyaratan yang ada di EP & Hasil wawancara menjelaskan sesuai standar dan dibuktikan dengan dokumen dan pengamatan & Hasil wawancara menjelaskan sebagian sesuai standar dan dibuktikan dengan dokumen dan pengamatan & Hasil wawancara tidak sesuai standar dan dibuktikan dengan dokumen dan pengamatan & \\
4 & Regulasi sesuai dengan yang dijelaskan di maksud dan tujuan pada standar & Regulasi yang meliputi Kebijakan dan SPO lengkap sesuai dengan maksud dan tujuan pada standar & Regulasi yang meliputi Kebijakan dan SPO sesuai dengan maksud dan tujuan pada standar hanya sebagian/tidak lengkap & Regulasi yang meliputi Kebijakan dan SPO sesuai dengan maksud dan tujuan pada standar tidak ada & \\
5 & Dokumen rapat/pertemu an: seperti undangan, materi rapat, absensi/daftar hadir, notulen rapat & Kelengkapan bukti dokumen rapat 80\% s.d 100\% (cross check dengan wawancara) & Kelengkapan bukti dokumen rapat 50\% s.d \textless80\% & Kelengkapan bukti dokumen rapat \textless50\% & \\
6 & Dokumen pelatihan seperti kerangka acuan (TOR) pelatihan yang dilampiri jadwal acara, undangan, materi/bahan pelatihan, absensi/daftar hadir, laporan pelatihan & Kelengkapan bukti dokumen pelatihan 80\% s.d 100\% & Kelengkapan bukti dokumen pelatihan 50\% s.d \textless80\% & Kelengkapan bukti dokumen pelatihan \textless50\% & \\
7 & Dokumen orientasi staf seperti kerangka acuan (TOR) orientasi yang dilampiri jadwal acara, undangan, absensi/daftar hadir, laporan, penilaian hasil orientasi dari kepala SDM (orientasi umum) atau kepala unit (orientasi khusus) & Kelengkapan bukti dokumen orientasi 80\% s.d 100\% & Kelengkapan bukti dokumen orientasi 50\% s.d \textless80\% & Kelengkapan bukti dokumen orientasi \textless50\% & \\
8 & Hasil observasi pelaksanaan kegiatan/ pelayanan sesuai regulasi dan standar & Pelaksanaan kegiatan/ pelayanan sesuai regulasi dan standar 80\% s.d 100\%. Contoh: 9 dari 10 kegiatan/ pelayanan yang diobservasi sudah memenuhi EP & Pelaksanaan kegiatan/ pelayanan sesuai regulasi dan standar 50\% s.d \textless80\% Contoh: 5 dari 10 kegiatan/pelayanan yang diobservasi sudah memenuhi EP & Pelaksanaan kegiatan/ pelayanan sesuai regulasi dan standar \textless50\%. Contoh: hanya 4 dari 10 kegiatan/ pelayanan yang diobservasi memenuhi EP & \\
9 & Hasil simulasi staf sesuai regulasi/ standar & Staf dapat memperagakan/ mensimulasikan sesuai regulasi/ standar: 80\% s.d 100\%. Contoh: 9 dari 10 staf yang diminta simulasi sudah memenuhi regulasi/standar & Staf dapat memperagakan/ mensimulasikan sesuai regulasi/ standar 50\% s.d \textless80\%. Contoh: 5 dari 10 staf yang diminta simulasi sudah memenuhi regulasi/standar & Staf dapat memperagakan/ mensimulasikan sesuai regulasi/ standar \textless50\%. Contoh: hanya 4 dari 10 staf yang diminta simulasi sudah memenuhi regulasi/standar & \\
10 & Kelengkapan rekam medik (Telaah rekam medik tertutup), pada survei awal 4 bulan sebelum survei, pada survei ulang 12 bulan sebelum survei & Rekam medik lengkap 80\% s.d 100\% saat di lakukan telaah. Contoh hasil telaah: 9 dari 10 rekam medik yang lengkap & Rekam medik lengkap 50\% s.d \textless80\% saat di lakukan telaah. Contoh hasil telaah: 5 dari 10 rekam medik yang lengkap & Rekam medik lengkap kurang dari 50\% saat di lakukan telaah. Contoh hasil telaah: hanya 4 dari 10 rekam medik yang lengkap & \\
\bottomrule
\end{longtable}

Keterangan:

TL : Terpenuhi Lengkap

TS : Terpenuhi Sebagian

TT : Tidak Terpenuhi

TDD : Tidak Dapat Diterapkan

\hypertarget{penutupan}{%
\subsection*{5. Penutupan}\label{penutupan}}
\addcontentsline{toc}{subsection}{5. Penutupan}

Setelah dilakukan telusur dan kunjungan lapangan termasuk klarifikasi kepada rumah sakit, maka surveior dapat menyampaikan hal-hal penting yang berkaitan dengan pelaksanaan akreditasi kepada rumah sakit secara langsung/luring. Tujuan tahapan ini adalah untuk memberi gambaran kepada rumah sakit bagaimana proses akreditasi yang telah dilaksanakan dan hal-hal yang perlu mendapat perbaikan untuk meningkatkan mutu pelayanan.

\hypertarget{c.-pasca-akreditasi}{%
\section*{C. Pasca Akreditasi}\label{c.-pasca-akreditasi}}
\addcontentsline{toc}{section}{C. Pasca Akreditasi}

\hypertarget{hasil-akreditasi-dan-akreditasi-ulang}{%
\subsection*{1. Hasil Akreditasi dan Akreditasi Ulang}\label{hasil-akreditasi-dan-akreditasi-ulang}}
\addcontentsline{toc}{subsection}{1. Hasil Akreditasi dan Akreditasi Ulang}

Lembaga independen penyelenggara akreditasi menyampaikan hasil akreditasi kepada Kementerian Kesehatan melalui Direktur Jenderal Pelayanan Kesehatan paling lambat 5 (lima) hari kerja setelah dilakukan survei. Hasil akreditasi berdasarkan pemenuhan standar akreditasi dalam Keputusan Menteri ini, dilaksanakan dengan mengikuti ketentuan sebagai berikut:

\begin{longtable}[]{@{}
  >{\raggedright\arraybackslash}p{(\columnwidth - 2\tabcolsep) * \real{0.10}}
  >{\raggedright\arraybackslash}p{(\columnwidth - 2\tabcolsep) * \real{0.90}}@{}}
\toprule
Hasil Akreditasi & Kriteria \\
\midrule
\endhead
Paripurna & Seluruh Bab mendapat nilai minimal 80\% \\
Utama & 12 -- 15 Bab mendapatkan nilai 80\% dan Bab SKP mendapat nilai minimal 80\%. Untuk rumah sakit selain rumah sakit pendidikan/wahana pendidikan maka kelulusan adalah 12 -- 14 bab dan bab SKP minimal 80 \% \\
Madya & 8 sampai 11 Bab mendapat nilai minimal 80\% dan Bab SKP mendapat nilai minimal 70\% \\
Tidak Terakreditasi & a. Kurang dari 8 Bab yang mendapat nilai minimal 80\%; dan/atau b. Bab SKP mendapat nilai kurang dari 70\% \\
\bottomrule
\end{longtable}

Rumah sakit diberikan kesempatan mengulang pada standar yang pemenuhannya kurang dari 80\%. Akreditasi ulang dapat dilakukan paling cepat 3 (tiga) bulan dan paling lambat 6 (enam) bulan sejak survei terakhir dilaksanakan.

\hypertarget{penyampaian-sertifikat-akreditasi}{%
\subsection*{2. Penyampaian Sertifikat Akreditasi}\label{penyampaian-sertifikat-akreditasi}}
\addcontentsline{toc}{subsection}{2. Penyampaian Sertifikat Akreditasi}

Penyampaian sertifikat akreditasi rumah sakit dilakukan paling lambat 14 (empat belas) hari setelah survei akreditasi dilakukan. Sertifikat akreditasi mencantumkan masa berlaku akreditasi sesuai dengan ketentuan peraturan perundang-undangan.

\hypertarget{penyampaian-rekomendasi}{%
\subsection*{3. Penyampaian Rekomendasi}\label{penyampaian-rekomendasi}}
\addcontentsline{toc}{subsection}{3. Penyampaian Rekomendasi}

Rekomendasi hasil penilaian akreditasi disampaikan oleh lembaga independen penyelenggara akreditasi kepada rumah sakit berisikan hal-hal yang harus ditindaklanjuti atau diperbaiki oleh rumah sakit. Penyampaian rekomendasi dilakukan bersamaan dengan penyerahan sertifikat akreditasi.

\hypertarget{penyampaian-rencana-perbaikan}{%
\subsection*{4. Penyampaian Rencana perbaikan}\label{penyampaian-rencana-perbaikan}}
\addcontentsline{toc}{subsection}{4. Penyampaian Rencana perbaikan}

Rumah sakit membuat Perencanaan Perbaikan Strategi (PPS) berdasarkan rekomendasi yang disampaikan oleh lembaga penyelenggara akreditasi. Penyampaian rencana perbaikan dilakukan dalam waktu 45 (empat puluh lima) hari sejak menerima rekomendasi dari lembaga penyelenggara akreditasi. Strategi rencana perbaikan disampaikan kepada lembaga yang melakukan akreditasi, dinas kesehatan setempat untuk rumah sakit kelas B, kelas C dan Kelas D, dan untuk rumah sakit kelas A disampaikan ke Kementerian Kesehatan.

\hypertarget{penyampaian-laporan-akreditasi}{%
\subsection*{5. Penyampaian Laporan Akreditasi}\label{penyampaian-laporan-akreditasi}}
\addcontentsline{toc}{subsection}{5. Penyampaian Laporan Akreditasi}

Lembaga menyampaikan pelaporan kegiatan akreditasi kepada Kementerian Kesehatan melalui sistem informasi akreditasi rumah sakit. Laporan berisi rekomendasi perbaikan yang harus dilakukan oleh rumah sakit, dan tingkat akreditasi yang dicapai oleh rumah sakit. Laporan kegiatan akreditasi dalam sistem informasi tersebut dapat diakses oleh pemerintah daerah provinsi dan dinas kesehatan daerah kabupaten/kota.

\hypertarget{umpan-balik-pelaksanaan-survei-akreditasi-oleh-rumah-sakit}{%
\subsection*{6. Umpan Balik Pelaksanaan Survei Akreditasi Oleh Rumah Sakit}\label{umpan-balik-pelaksanaan-survei-akreditasi-oleh-rumah-sakit}}
\addcontentsline{toc}{subsection}{6. Umpan Balik Pelaksanaan Survei Akreditasi Oleh Rumah Sakit}

Untuk menjamin akuntabilitas dan kualitas pelaksanaan survei, maka setiap survei harus diikuti dengan permintaan umpan balik kepada rumah sakit terkait penyelenggaraan survei akreditasi dan kinerja dan perilaku surveior. Umpan balik disampaikan kepada lembaga independen penyelenggara akreditasi di rumah sakit tersebut dan Kementerian Kesehatan melalui sistem informasi akreditasi rumah sakit. Umpan balik digunakan sebagai dasar untuk upaya peningkatan kualitas penyelenggaraan survei akreditasi. Kementerian Kesehatan dapat memanfaatkan informasi dari umpan balik tersebut untuk melakukan pembinaan dan pengawasan penyelenggaraan survei akreditasi kepada lembaga independen penyelenggara akreditasi rumah sakit.

\hypertarget{bab-3-standar-akreditasi-rumah-sakit}{%
\chapter*{Bab 3 Standar Akreditasi Rumah Sakit}\label{bab-3-standar-akreditasi-rumah-sakit}}
\addcontentsline{toc}{chapter}{Bab 3 Standar Akreditasi Rumah Sakit}

\hypertarget{a.-kelompok-manajemen-rumah-sakit}{%
\chapter*{A. Kelompok Manajemen Rumah Sakit}\label{a.-kelompok-manajemen-rumah-sakit}}
\addcontentsline{toc}{chapter}{A. Kelompok Manajemen Rumah Sakit}

\hypertarget{tata-kelola-rumah-sakit-tkrs}{%
\section*{1. Tata Kelola Rumah Sakit (TKRS)}\label{tata-kelola-rumah-sakit-tkrs}}
\addcontentsline{toc}{section}{1. Tata Kelola Rumah Sakit (TKRS)}

\textbf{Gambaran Umum}

Rumah sakit adalah institusi pelayanan kesehatan yang menyelenggarakan pelayanan kesehatan perorangan secara paripurna yang menyediakan pelayanan medis bagi rawat inap, rawat jalan, gawat darurat serta pelayanan penunjang seperti laboratorium, radiologi serta layanan lainnya. Untuk dapat memberikan pelayanan prima kepada pasien, rumah sakit dituntut memiliki kepemimpinan yang efektif. Kepemimpinan efektif ini ditentukan oleh sinergi yang positif antara Pemilik Rumah Sakit/Representasi Pemilik/Dewan Pengawas, Direktur Rumah Sakit, para pimpinan di rumah sakit, dan kepala unit kerja unit pelayanan.

Direktur rumah sakit secara kolaboratif mengoperasionalkan rumah sakit bersama dengan para pimpinan, kepala unit kerja, dan unit pelayanan untuk mencapai visi misi yang ditetapkan serta memiliki tanggung jawab dalam pengelolaan pengelolaan peningkatan mutu dan keselamatan pasien, pengelolaan kontrak, serta pengelolaan sumber daya. Operasional rumah sakit berhubungan dengan seluruh pemangku kepentingan yang ada mulai dari pemilik, jajaran direksi, pengelolaan secara keseluruhan sampai dengan unit fungsional yang ada. Setiap pemangku kepentingan memiliki tugas dan tanggung jawab sesuai ketentuan peraturan dan perundangan yang berlaku.

Fokus pada Bab TKRS mencakup:

\begin{enumerate}
\def\labelenumi{\alph{enumi}.}
\tightlist
\item
  Representasi Pemilik/Dewan Pengawas
\item
  Akuntabilitas Direktur Utama/Direktur/Kepala Rumah Sakit
\item
  Akuntabilitas Pimpinan Rumah Sakit
\item
  Kepemimpinan Rumah Sakit Untuk Mutu dan Keselamatan Pasien
\item
  Kepemimpinan Rumah Sakit Terkait Kontrak
\item
  Kepemimpinan Rumah Sakit Terkait Keputusan Mengenai Sumber Daya
\item
  Pengorganisasian dan Akuntabilitas Komite Medik, Komite Keperawatan, dan Komite Tenaga Kesehatan Lain
\item
  Akuntabilitas Kepala unit klinis/non klinis
\item
  Etika Rumah Sakit
\item
  Kepemimpinan Untuk Budaya Keselamatan di Rumah Sakit
\item
  Manajemen risiko
\item
  Program Penelitian Bersubjek Manusia di Rumah Sakit
\end{enumerate}

Catatan: Semua standar Tata Kelola rumah sakit mengatur peran dan tanggung jawab Pemilik atau Representasi Pemilik, Direktur, Pimpinan rumah sakit dan Kepala Instalasi/Kepala Unit. Hierarki kepemimpinan dalam Standar ini terdiri dari:

\begin{enumerate}
\def\labelenumi{\alph{enumi}.}
\tightlist
\item
  Pemilik/Representasi Pemilik: satu atau sekelompok orang sebagai Pemilik atau sebagai Representasi Pemilik, misalnya Dewan Pengawas.
\item
  Direktur/Direktur Utama/Kepala rumah sakit: satu orang yang dipilih oleh Pemilik untuk bertanggung jawab mengelola rumah sakit
\item
  Para Wakil direktur (Pimpinan rumah sakit): beberapa orang yang dipilih untuk membantu Direktur Apabila rumah sakit tidak mempunyai Wakil direktur, maka kepala bidang/manajer dapat dianggap sebagai pimpinan rumah sakit.
\item
  Kepala Unit klinis/Unit non klinis: beberapa orang yang dipilih untuk memberikan pelayanan termasuk Kepala IGD, Kepala Radiologi, Kepala Laboratorium, Kepala Keuangan, dan lainnya.
\end{enumerate}

Rumah sakit yang menerapkan tata kelola yang baik memberikan kualitas pelayanan yang baik yang secara kasat mata, terlihat dari penampilan keramahan staf dan penerapan budaya 5 R (rapi, resik, rawat, rajin, ringkas) secara konsisten pada seluruh bagian rumah sakit, serta pelayanan yang mengutamakan mutu dan keselamatan pasien.

\hypertarget{a.-representasi-pemilikdewan-pengawas}{%
\subsection*{a. Representasi Pemilik/Dewan Pengawas}\label{a.-representasi-pemilikdewan-pengawas}}
\addcontentsline{toc}{subsection}{a. Representasi Pemilik/Dewan Pengawas}

Struktur organisasi serta wewenang pemilik/representasi pemilik dijelaskan di dalam aturan internal rumah sakit (Hospital by Laws) yang ditetapkan oleh pemilik rumah sakit.

\hypertarget{standar-tkrs-1}{%
\paragraph*{1. Standar TKRS 1}\label{standar-tkrs-1}}
\addcontentsline{toc}{paragraph}{1. Standar TKRS 1}

Struktur organisasi serta wewenang pemilik/representasi pemilik dijelaskan di dalam aturan internal rumah sakit (Hospital by Laws) yang ditetapkan oleh pemilik rumah sakit.

\hypertarget{maksud-dan-tujuan-tkrs-1}{%
\paragraph*{2. Maksud dan Tujuan TKRS 1}\label{maksud-dan-tujuan-tkrs-1}}
\addcontentsline{toc}{paragraph}{2. Maksud dan Tujuan TKRS 1}

Pemilik dan representasi pemilik memiliki tugas pokok dan fungsi secara khusus dalam pengolaan rumah sakit. Regulasi yang mengatur hal tersebut dapat berbentuk peraturan internal rumah sakit atau Hospital by Laws atau dokumen lainnya yang serupa. Struktur organisasi pemilik termasuk representasi pemilik terpisah dengan struktur organisasi rumah sakit sesuai dengan bentuk badan hukum pemilik dan peraturan perundang-undangan. Pemilik rumah sakit tidak diperbolehkan menjadi Direktur/Direktur Utama/Kepala Rumah Sakit, tetapi posisinya berada di atas representasi pemilik.

Pemilik rumah sakit mengembangkan sebuah proses untuk melakukan komunikasi dan kerja sama dengan Direktur/Direktur Utama/Kepala Rumah Sakit dalam rangka mencapai misi dan perencanaan rumah sakit. Representasi pemilik, sesuai dengan bentuk badan hukum kepemilikan rumah sakit memiliki wewenang dan tanggung jawab untuk memberi persetujuan, dan pengawasan agar rumah sakit mempunyai kepemimpinan yang jelas, dijalankan secara efisien, dan memberikan pelayanan kesehatan yang bermutu dan aman.

Berdasarkan hal tersebut maka pemilik/representasi pemilik perlu menetapkan Hospital by Laws/peraturan internal rumah sakit yang mengatur:

\begin{enumerate}
\def\labelenumi{\alph{enumi}.}
\tightlist
\item
  Pengorganisasian pemilik atau representasi pemilik sesuai dengan bentuk badan hukum kepemilikan rumah sakit serta peraturan perundang-undangan yang berlaku.
\item
  Peran, tugas dan kewenangan pemilik atau representasi pemilik
\item
  Peran, tugas dan kewenangan Direktur rumah sakit
\item
  Pengorganisasian tenaga medis
\item
  Peran, tugas dan kewenangan tenaga medis.
\end{enumerate}

Tanggung jawab representasi pemilik harus dilakukan agar rumah sakit mempunyai kepemimpinan yang jelas, dapat beroperasi secara efisien, dan menyediakan pelayanan kesehatan bermutu tinggi. Tanggung jawabnya mencakup namun tidak terbatas pada:

\begin{enumerate}
\def\labelenumi{\alph{enumi}.}
\tightlist
\item
  Menyetujui dan mengkaji visi misi rumah sakit secara periodik dan memastikan bahwa masyarakat mengetahui misi rumah sakit.
\item
  Menyetujui berbagai strategi dan rencana operasional rumah sakit yang diperlukan untuk berjalannya rumah sakit sehari-hari.
\item
  Menyetujui partisipasi rumah sakit dalam pendidikan profesional kesehatan dan dalam penelitian serta mengawasi mutu dari program-program tersebut.
\item
  Menyetujui dan menyediakan modal serta dana operasional dan sumber daya lain yang diperlukan untuk menjalankan rumah sakit dan memenuhi misi serta rencana strategis rumah sakit.
\item
  Melakukan evaluasi tahunan kinerja Direksi dengan menggunakan proses dan kriteria yang telah ditetapkan.
\item
  Mendukung peningkatan mutu dan keselamatan pasien dengan menyetujui program peningkatan mutu dan keselamatan pasien.
\item
  Melakukan pengkajian laporan hasil pelaksanaan program Peningkatan Mutu dan Keselamatan Pasien (PMKP) setiap 3 (tiga) bulan sekali serta memberikan umpan balik perbaikan yang harus dilaksanakan dan hasilnya di evaluasi kembali pada pertemuan berikutnya secara tertulis.
\item
  Melakukan pengkajian laporan Manajemen Risiko setiap 6 (enam) bulan sekali dan memberikan umpan balik perbaikan yang harus dilaksanakan dan hasilnya di evaluasi kembali pada pertemuan berikutnya secara tertulis.
\end{enumerate}

Khusus mengenai struktur organisasi rumah sakit, hal ini sangat bergantung pada kebutuhan dalam pelayanan dan ketentuan peraturan perundangan yang ada.

\hypertarget{elemen-penilaian-tkrs-1}{%
\paragraph*{3. Elemen Penilaian TKRS 1}\label{elemen-penilaian-tkrs-1}}
\addcontentsline{toc}{paragraph}{3. Elemen Penilaian TKRS 1}

\begin{enumerate}
\def\labelenumi{\alph{enumi}.}
\tightlist
\item
  Representasi pemilik/Dewan Pengawas dipilih dan ditetapkan oleh Pemilik.
\item
  Tanggung jawab dan wewenang representasi pemilik meliputi poin a) sampai dengan h) yang tertera di dalam maksud dan tujuan serta dijelaskan di dalam peraturan internal rumah sakit.
\item
  Representasi pemilik/Dewan Pengawas di evaluasi oleh pemilik setiap tahun dan hasil evaluasinya didokumentasikan.
\item
  Representasi pemilik/Dewan Pengawas menetapkan visi misi rumah sakit yang diarahkan oleh pemilik.
\end{enumerate}

\hypertarget{b.-akuntabilitas-direktur-utamadirekturkepala-rumah-sakit}{%
\subsection*{b. Akuntabilitas Direktur Utama/Direktur/Kepala Rumah Sakit}\label{b.-akuntabilitas-direktur-utamadirekturkepala-rumah-sakit}}
\addcontentsline{toc}{subsection}{b. Akuntabilitas Direktur Utama/Direktur/Kepala Rumah Sakit}

\hypertarget{standar-tkrs-2}{%
\paragraph*{1. Standar TKRS 2}\label{standar-tkrs-2}}
\addcontentsline{toc}{paragraph}{1. Standar TKRS 2}

Direktur rumah sakit bertanggung jawab untuk menjalankan rumah sakit dan mematuhi peraturan dan perundang- undangan.

\hypertarget{maksud-dan-tujuan-tkrs-2}{%
\paragraph*{2. Maksud dan Tujuan TKRS 2}\label{maksud-dan-tujuan-tkrs-2}}
\addcontentsline{toc}{paragraph}{2. Maksud dan Tujuan TKRS 2}

Pimpinan tertinggi organisasi Rumah Sakit adalah kepala atau Direktur Rumah Sakit dengan nama jabatan kepala, direktur utama atau direktur, dalam standar akreditasi ini disebut Direktur Rumah Sakit. Dalam menjalankan operasional Rumah Sakit, direktur dapat dibantu oleh wakil direktur atau direktur (bila pimpinan tertinggi disebut direktur utama) sesuai kebutuhan, kelompok ini disebut direksi.

Persyaratan untuk direktur Rumah Sakit sesuai dengan peraturan perundangan adalah tenaga medis yang mempunyai kemampuan dan keahlian di bidang perumahsakitan.

Pendidikan dan pengalaman Direktur tersebut telah memenuhi persyaratan untuk melaksanakan tugas yang termuat dalam uraian tugas serta peraturan dan perundangan.

Tanggung jawab Direktur dalam menjalankan rumah sakit termasuk namun tidak terbatas pada:

\begin{enumerate}
\def\labelenumi{\alph{enumi}.}
\tightlist
\item
  Mematuhi perundang-undangan yang berlaku.
\item
  Menjalankan visi dan misi rumah sakit yang telah ditetapkan.
\item
  Menetapkan kebijakan rumah sakit.
\item
  Memberikan tanggapan terhadap setiap laporan pemeriksaan yang dilakukan oleh regulator.
\item
  Mengelola dan mengendalikan sumber daya manusia, keuangan dan sumber daya lainnya.
\item
  Merekomendasikan sejumlah kebijakan, rencana strategis, dan anggaran kepada Representatif pemilik/Dewan Pengawas untuk mendapatkan persetujuan.
\item
  Menetapkan prioritas perbaikan tingkat rumah sakit yaitu perbaikan yang akan berdampak luas/menyeluruh di rumah sakit yang akan dilakukan pengukuran sebagai indikator mutu prioritas rumah sakit.
\item
  Melaporkan hasil pelaksanaan program mutu dan keselamatan pasien meliputi pengukuran data dan laporan semua insiden keselamatan pasien secara berkala setiap 3 (tiga) bulan kepada Representasi pemilik/Dewan Pengawas.
\item
  Melaporkan hasil pelaksanaan program manajemen risiko kepada Representasi pemilik/Dewan Pengawas setiap 6 (enam) bulan.
\end{enumerate}

\hypertarget{elemen-penilaian-tkrs-2}{%
\paragraph*{3. Elemen Penilaian TKRS 2}\label{elemen-penilaian-tkrs-2}}
\addcontentsline{toc}{paragraph}{3. Elemen Penilaian TKRS 2}

\begin{enumerate}
\def\labelenumi{\alph{enumi}.}
\tightlist
\item
  Telah menetapkan regulasi tentang kualifikasi Direktur, uraian tugas, tanggung jawab dan wewenang sesuai dengan persyaratan dan peraturan perundang- undangan yang berlaku.
\item
  Direktur menjalankan operasional rumah sakit sesuai tanggung jawabnya yang meliputi namun tidak terbatas pada poin a) sampai dengan i) dalam maksud dan tujuan yang dituangkan dalam uraian tugasnya.
\item
  Memiliki bukti tertulis tanggung jawab Direktur telah dilaksanakan dan dievaluasi oleh pemilik/representasi pemilik setiap tahun dan hasil evaluasinya didokumentasikan.
\end{enumerate}

\hypertarget{c.-akuntabilitas-pimpinan-rumah-sakit}{%
\subsection*{c.~Akuntabilitas Pimpinan Rumah Sakit}\label{c.-akuntabilitas-pimpinan-rumah-sakit}}
\addcontentsline{toc}{subsection}{c.~Akuntabilitas Pimpinan Rumah Sakit}

\hypertarget{standar-tkrs-3}{%
\paragraph*{1. Standar TKRS 3}\label{standar-tkrs-3}}
\addcontentsline{toc}{paragraph}{1. Standar TKRS 3}

Pimpinan rumah sakit menyusun misi, rencana kerja dan kebijakan untuk memenuhi misi rumah sakit serta merencanakan dan menentukan jenis pelayanan klinis untuk memenuhi kebutuhan pasien yang dilayani rumah sakit.

\hypertarget{maksud-dan-tujuan-tkrs-3}{%
\paragraph*{2. Maksud dan Tujuan TKRS 3}\label{maksud-dan-tujuan-tkrs-3}}
\addcontentsline{toc}{paragraph}{2. Maksud dan Tujuan TKRS 3}

Direktur melibatkan wakil direktur rumah sakit dan kepala- kepala unit dalam proses menyusun misi dan nilai yang dianut rumah sakit. Apabila rumah sakit tidak mempunyai wakil direktur, maka kepala bidang/manajer dapat dianggap sebagai pimpinan rumah sakit.

Berdasarkan misi tersebut, pimpinan bekerja sama untuk menyusun rencana kerja dan kebijakan yang dibutuhkan. Apabila misi dan rencana kerja dan kebijakan telah ditetapkan oleh pemilik atau Dewan Pengawas, maka pimpinan bekerja sama untuk melaksanakan misi dan kebijakan yang telah dibuat.

Jenis pelayanan yang akan diberikan harus konsisten dengan misi rumah sakit. Pimpinan rumah sakit menentukan pimpinan setiap unit klinis dan unit layanan penting lainnya.

Pimpinan rumah sakit bersama dengan para pimpinan tersebut:

\begin{enumerate}
\def\labelenumi{\alph{enumi}.}
\tightlist
\item
  Merencanakan cakupan dan intensitas pelayanan yang akan disediakan oleh rumah sakit, baik secara langsung maupun tidak langsung.
\item
  Meminta masukan dan partisipasi masyarakat, rumah sakit jejaring, fasilitas pelayanan kesehatan dan pihak- pihak lain untuk memenuhi kebutuhan kesehatan masyarakat. Bentuk pelayanan ini akan dimasukkan dalam penyusunan rencana strategis rumah sakit dan perspektif pasien yang akan dilayani rumah sakit.
\item
  Menentukan komunitas dan populasi pasien, mengidentifikasi pelayanan yang dibutuhkan oleh komunitas, dan merencanakan komunikasi berkelanjutan dengan kelompok pemangku kepentingan utama dalam komunitas. Komunikasi dapat secara langsung ditujukan kepada individu,
\item
  melalui media massa, melalui lembaga dalam komunitas atau pihak ketiga.
\end{enumerate}

Jenis informasi yang disampaikan meliputi:

\begin{enumerate}
\def\labelenumi{\alph{enumi}.}
\tightlist
\item
  informasi tentang layanan, jam kegiatan kerja, dan proses untuk mendapatkan pelayanan; dan
\item
  informasi tentang mutu layanan, yang disediakan kepada masyarakat dan sumber rujukan.
\end{enumerate}

\hypertarget{elemen-penilaian-tkrs-3}{%
\paragraph*{3. Elemen Penilaian TKRS 3}\label{elemen-penilaian-tkrs-3}}
\addcontentsline{toc}{paragraph}{3. Elemen Penilaian TKRS 3}

\begin{enumerate}
\def\labelenumi{\alph{enumi}.}
\tightlist
\item
  Direktur menunjuk pimpinan rumah sakit dan kepala unit sesuai kualifikasi dalam persyaratan jabatan yang telah ditetapkan beserta uraian tugasnya.
\item
  Pimpinan rumah sakit bertanggung jawab untuk melaksanakan misi yang telah ditetapkan dan memastikan kebijakan serta prosedur dilaksanakan.
\item
  Pimpinan rumah sakit bersama dengan pimpinan unit merencanakan dan menentukan jenis pelayanan klinis untuk memenuhi kebutuhan pasien yang dilayani rumah sakit.
\item
  Rumah sakit memberikan informasi tentang pelayanan yang disediakan kepada tokoh masyarakat, para pemangku kepentingan, fasilitas pelayanan kesehatan di sekitar rumah sakit, dan terdapat proses untuk menerima masukan bagi peningkatan pelayanannya.
\end{enumerate}

\hypertarget{standar-tkrs-3.1}{%
\paragraph*{1. Standar TKRS 3.1}\label{standar-tkrs-3.1}}
\addcontentsline{toc}{paragraph}{1. Standar TKRS 3.1}

Pimpinan rumah sakit memastikan komunikasi yang efektif telah dilaksanakan secara menyeluruh di rumah sakit.

\hypertarget{maksud-dan-tujuan-tkrs-3.1}{%
\paragraph*{2. Maksud dan Tujuan TKRS 3.1}\label{maksud-dan-tujuan-tkrs-3.1}}
\addcontentsline{toc}{paragraph}{2. Maksud dan Tujuan TKRS 3.1}

Komunikasi yang efektif baik antara para profesional pemberi asuhan (PPA); antara unit dengan unit baik pelayanan maupun penunjang, antara PPA dengan kelompok nonprofesional; antara PPA dengan manajemen, antara PPA dengan pasien dan keluarga; serta antara PPA dengan organisasi di luar rumah sakit merupakan tanggung jawab pimpinan rumah sakit. Pimpinan rumah sakit tidak hanya mengatur parameter komunikasi yang efektif, tetapi juga memberikan teladan dalam melakukan komunikasi efektif tentang misi, rencana strategi dan informasi terkait lainnya. Para pimpinan memperhatikan keakuratan dan ketepatan waktu dalam pemberian informasi dan pelaksanaan komunikasi dalam lingkungan rumah sakit.

Untuk mengoordinasikan dan mengintegrasikan pelayanan kepada pasien, pimpinan menetapkan Tim/Unit yang menerapkan mekanisme pemberian informasi dan komunikasi misalnya melalui pembentukan Tim/Unit PKRS. Metode komunikasi antar layanan dan staf dapat berupa buletin, poster, \emph{story board}, dan lain-lainnya.

\hypertarget{elemen-penilaian-tkrs-3.1}{%
\paragraph*{3. Elemen Penilaian TKRS 3.1}\label{elemen-penilaian-tkrs-3.1}}
\addcontentsline{toc}{paragraph}{3. Elemen Penilaian TKRS 3.1}

\begin{enumerate}
\def\labelenumi{\alph{enumi}.}
\tightlist
\item
  Pimpinan rumah sakit memastikan bahwa terdapat proses untuk menyampaikan informasi dalam lingkungan rumah sakit secara akurat dan tepat waktu.
\item
  Pimpinan rumah sakit memastikan bahwa komunikasi yang efektif antara unit klinis dan nonklinis, antara PPA dengan manajemen, antar PPA dengan pasien dan keluarga serta antar staf telah dilaksanakan.
\item
  Pimpinan rumah sakit telah mengkomunikasikan visi, misi, tujuan, rencana strategis dan kebijakan, rumah sakit kepada semua staf.
\end{enumerate}

\hypertarget{d.-kepemimpinan-rumah-sakit-untuk-mutu-dan-keselamatan-pasien}{%
\subsection*{d.~Kepemimpinan Rumah Sakit Untuk Mutu dan Keselamatan Pasien}\label{d.-kepemimpinan-rumah-sakit-untuk-mutu-dan-keselamatan-pasien}}
\addcontentsline{toc}{subsection}{d.~Kepemimpinan Rumah Sakit Untuk Mutu dan Keselamatan Pasien}

\hypertarget{standar-tkrs-4}{%
\paragraph*{1. Standar TKRS 4}\label{standar-tkrs-4}}
\addcontentsline{toc}{paragraph}{1. Standar TKRS 4}

Pimpinan rumah sakit merencanakan, mengembangkan, dan menerapkan program peningkatan mutu dan keselamatan pasien.

\hypertarget{maksud-dan-tujuan-tkrs-4}{%
\paragraph*{2. Maksud dan Tujuan TKRS 4}\label{maksud-dan-tujuan-tkrs-4}}
\addcontentsline{toc}{paragraph}{2. Maksud dan Tujuan TKRS 4}

Peran para pimpinan rumah sakit termasuk dalam mengembangkan program mutu dan keselamatan pasien sangat penting. Diharapkan pelaksanaan program mutu dan keselamatan dapat membangun budaya mutu di rumah sakit.
Pimpinan rumah sakit memilih mekanisme pengukuran data untuk meningkatkan mutu dan keselamatan pasien. Di samping itu, pimpinan rumah sakit juga memberikan arahan dan dukungan terhadap pelaksanaan program, misalnya menyediakan sumber daya yang cukup agar Komite/Tim Penyelenggara Mutu dapat bekerja secara efektif.

Pimpinan rumah sakit juga menerapkan mekanisme dan proses untuk memantau dan melakukan koordinasi secara menyeluruh terhadap penerapan program di rumah sakit. Koordinasi ini dapat tercapai melalui pemantauan dari Komite/Tim Penyelenggara Mutu, atau struktur lainnya. Koordinasi menggunakan pendekatan sistem untuk pemantauan mutu dan aktivitas perbaikan sehingga mengurangi duplikasi; misalnya terdapat dua unit yang secara independen mengukur suatu proses atau luaran yang sama.

Komunikasi dan pemberian informasi tentang hasil program peningkatan mutu dan keselamatan pasien secara berkala setiap triwulan kepada Direktur dan staf merupakan hal yang penting. Informasi yang diberikan mencakup hasil pengukuran data, proyek perbaikan mutu yang baru akan dilaksanakan atau proyek perbaikan mutu yang sudah diselesaikan, hasil pencapaian Sasaran Keselamatan Pasien,penelitian terkini dan program kaji banding.

Saluran komunikasi ditetapkan oleh pimpinan rumah sakit menggunakan jalur yang efektif serta mudah dipahami, meliputi:

\begin{enumerate}
\def\labelenumi{\alph{enumi}.}
\tightlist
\item
  Informasi hasil pengukuran data kepada Direktur, misalnya Dashboard.
\item
  Informasi hasil pengukuran data kepada staf misalnya buletin, papan cerita ( \emph{story board} ), pertemuan staf, dan proses lainnya.
\end{enumerate}

\hypertarget{elemen-penilaian-tkrs-4}{%
\paragraph*{3. Elemen Penilaian TKRS 4}\label{elemen-penilaian-tkrs-4}}
\addcontentsline{toc}{paragraph}{3. Elemen Penilaian TKRS 4}

\begin{enumerate}
\def\labelenumi{\alph{enumi}.}
\tightlist
\item
  Direktur dan Pimpinan rumah sakit berpartisipasi dalam merencanakan mengembangkan dan menerapkan program peningkatan mutu dan keselamatan pasien di lingkungan rumah sakit.
\item
  Pimpinan rumah sakit memilih dan menetapkan proses pengukuran, pengkajian data, rencana perbaikan dan mempertahankan peningkatan mutu dan keselamatan pasien di lingkungan rumah sakit
\item
  Pimpinan rumah sakit memastikan terlaksananya program PMKP termasuk memberikan dukungan teknologi dan sumber daya yang adekuat serta menyediakan pendidikan staf tentang peningkatan mutu dan keselamatan pasien di rumah sakit agar dapat berjalan secara efektif.
\item
  Pimpinan rumah sakit menetapkan mekanisme pemantauan dan koordinasi program peningkatan mutu dan keselamatan pasien.
\end{enumerate}

\hypertarget{standar-tkrs-5}{%
\paragraph*{4. Standar TKRS 5}\label{standar-tkrs-5}}
\addcontentsline{toc}{paragraph}{4. Standar TKRS 5}

Direktur dan Pimpinan rumah sakit berpartisipasi dalam menetapkan prioritas perbaikan di tingkat rumah sakit yang merupakan proses yang berdampak luas/menyeluruh di rumah sakit termasuk di dalamnya kegiatan keselamatan pasien serta analisis dampak dari perbaikan yang telah dilakukan.

\hypertarget{maksud-dan-tujuan-tkrs-5}{%
\paragraph*{5. Maksud dan Tujuan TKRS 5}\label{maksud-dan-tujuan-tkrs-5}}
\addcontentsline{toc}{paragraph}{5. Maksud dan Tujuan TKRS 5}

Tanggung jawab direktur dan pimpinan rumah sakit adalah menetapkan Prioritas perbaikan di tingkat rumah sakit yaitu perbaikan yang akan berdampak luas/menyeluruh dan dapat dilakukan di berbagai unit klinis maupun non klinis. Prioritas perbaikan tersebut harus dilakukan pengukuran dalam bentuk indikator mutu prioritas rumah sakit (IMP-RS).

Pengukuran prioritas perbaikan tingkat rumah sakit mencakup:

\begin{enumerate}
\def\labelenumi{\alph{enumi}.}
\tightlist
\item
  Sasaran keselamatan pasien meliputi enam Sasaran Keselamatan Pasien (SKP)
\item
  Pelayanan klinis prioritas untuk dilakukan perbaikan misalnya pada pelayanannya berisiko tinggi dan terdapat masalah dalam pelayanan tersebut, seperti pada pelayanan hemodialisa serta pelayanan kemoterapi. Pemilihan pelayanan klinis prioritas dapat menggunakan kriteria pemilihan prioritas pengukuran dan perbaikan.
\item
  Tujuan strategis rumah sakit misalnya rumah sakit ingin menjadi rumah sakit rujukan untuk pasien kanker. Maka prioritas perbaikannya dapat dalam bentuk \emph{Key Performance indicator} (KPI) dapat berupa peningkatkan efisiensi, mengurangi angka readmisi, mengurangi masalah alur pasien di IGD atau memantau mutu layanan yang diberikan oleh pihak lain yang dikontrak.
\item
  Perbaikan sistem adalah perbaikan yang jika dilakukan akan berdampak luas/menyeluruh di rumah sakit yang dapat diterapkan di beberapa unit misalnya sistem pengelolaan obat, komunikasi serah terima dan lain-lainnya.
\item
  Manajemen risiko untuk melakukan perbaikan secara proaktif terhadap proses berisiko tinggi misalnya yang telah dilakukan analisis FMEA atau dapat diambil dari profil risiko
\item
  Penelitian klinis dan program pendidikan kesehatan (apabila ada).
\end{enumerate}

Untuk memilih prioritas pengukuran dan perbaikan menggunakan kriteria prioritas mencakup:

\begin{enumerate}
\def\labelenumi{\alph{enumi}.}
\tightlist
\item
  Masalah yang paling banyak di rumah sakit.
\item
  Jumlah yang banyak ( \emph{High volume} ).
\item
  Proses berisiko tinggi ( \emph{High process} ).
\item
  Ketidakpuasan pasien dan staf.
\item
  Kemudahan dalam pengukuran.
\item
  Ketentuan Pemerintah / Persyaratan Eksternal.
\item
  Sesuai dengan tujuan strategis rumah sakit.
\item
  Memberikan pengalaman pasien lebih baik ( \emph{patient experience} ).
\end{enumerate}

Direktur dan Pimpinan rumah sakit berpartisipasi dalam penentuan pengukuran perbaikan. Penentuan prioritas terukur dapat menggunakan skoring prioritas.
Direktur dan pimpinan rumah sakit akan menilai dampak perbaikan dapat berupa:

\begin{enumerate}
\def\labelenumi{\alph{enumi}.}
\tightlist
\item
  Dampak primer adalah hasil capaian setelah dilakukan perbaikan misalnya target kepuasan pasien tercapai 90\%, atau hasil kepatuhan terhadap proses yang ditetapkan misalnya, kepatuhan pelaporan hasil kritis \textless{} 30 menit tercapai 100\%.
\item
  Dampak sekunder adalah dampak terhadap efisiensi setelah dilakukan perbaikan misalnya efisiensi pada proses klinis yang kompleks, perubahan alur pelayanan yang kompleks, penghematan biaya pengurangan sumber daya, perubahan ruangan yang dibutuhkan yang digunakan dalam proses pelayanan tersebut.
\end{enumerate}

Penilaian dampak perbaikan akan memberikan pemahaman tentang biaya yang dikeluarkan untuk investasi mutu, sumber daya manusia, keuangan, dan keuntungan lain dari investasi tersebut. Direktur dan pimpinan rumah sakit akan menetapkan cara/tools sederhana untuk membandingkan sumber daya yang digunakan pada proses yang lama dibandingkan proses yang baru dengan membandingkan dampak perbaikan pada hasil keluaran pasien dan atau biaya yang menyebabkan efisiensi. Hal ini akan menjadi pertimbangan dalam penentuan prioritas perbaikan pada periode berikutnya, baik di tingkat rumah sakit maupun di tingkat unit klinis/non klinis. Apabila semua informasi ini digabungkan secara menyeluruh, maka direktur dan pimpinan rumah sakit dapat lebih memahami bagaimana mengalokasikan sumber daya mutu dan keselamatan pasien yang tersedia.

\hypertarget{elemen-penilaian-tkrs-5}{%
\paragraph*{6. Elemen Penilaian TKRS 5}\label{elemen-penilaian-tkrs-5}}
\addcontentsline{toc}{paragraph}{6. Elemen Penilaian TKRS 5}

\begin{enumerate}
\def\labelenumi{\alph{enumi}.}
\tightlist
\item
  Direktur dan pimpinan rumah sakit menggunakan data yang tersedia (data based) dalam menetapkan indikator prioritas rumah sakit yang perbaikannya akan berdampak luas/menyeluruh meliputi poin a) -- f) dalam maksud dan tujuan.
\item
  Dalam memilih prioritas perbaikan di tingkat rumah sakit maka Direktur dan pimpinan mengggunakan kriteria prioritas meliputi poin a) -- h) dalam maksud dan tujuan.
\item
  Direktur dan pimpinan rumah sakit mengkaji dampak perbaikan primer dan dampak perbaikan sekunder pada indikator prioritas rumah sakit yang ditetapkan di tingkat rumah sakit maupun tingkat unit.
\end{enumerate}

\hypertarget{e.-kepemimpinan-rumah-sakit-terkait-kontrak}{%
\subsection*{e. Kepemimpinan Rumah Sakit Terkait Kontrak}\label{e.-kepemimpinan-rumah-sakit-terkait-kontrak}}
\addcontentsline{toc}{subsection}{e. Kepemimpinan Rumah Sakit Terkait Kontrak}

\hypertarget{standar-tkrs-6}{%
\paragraph*{1. Standar TKRS 6}\label{standar-tkrs-6}}
\addcontentsline{toc}{paragraph}{1. Standar TKRS 6}

Pimpinan Rumah Sakit bertanggung jawab untuk mengkaji, memilih, dan memantau kontrak klinis dan nonklinis serta melakukan evaluasi termasuk inspeksi kepatuhan layanan sesuai kontrak yang disepakati.

\hypertarget{maksud-dan-tujuan-tkrs-6}{%
\paragraph*{2. Maksud dan Tujuan TKRS 6}\label{maksud-dan-tujuan-tkrs-6}}
\addcontentsline{toc}{paragraph}{2. Maksud dan Tujuan TKRS 6}

Rumah sakit dapat memilih pelayanan yang akan diberikan kepada pasien apakah akan memberikan pelayanan secara langsung atau tidak langsung misalnya rujukan, konsultasi atau perjanjian kontrak lainnya. Pimpinan rumah sakit menetapkan jenis dan ruang lingkup layanan yang akan di kontrakkan baik kontrak klinis maupun kontrak manajemen. Jenis dan ruang lingkup layanan tersebut kemudian dituangkan dalam kontrak/perjanjian untuk memastikan bahwa pelayanan yang diberikan memenuhi kebutuhan pasien.

Kontrak pelayanan klinis disebut kontrak klinis adalah perjanjian pelayanan klinis yang diberikan oleh pihak ketiga kepada pasien misalnya layanan laboratorium, layanan radiologi dan pencitraan diagnostik dan lain-lainnya. Kontrak pelayanan manajemen disebut kontrak manajemen adalah perjanjian yang menunjang kegiatan rumah sakit dalam memberikan pelayanan kepada pasien misalnya: layanan kebersihan, kemanan, rumah tangga/tata graha/housekeeping, makanan, linen, dan lain-lainnya.

Kontrak klinis bisa juga berhubungan dengan staf profesional kesehatan. misalnya, kontrak perawat untuk pelayanan intensif, dokter tamu/dokter paruh waktu, dan lain-lainnya. Dalam kontrak tersebut harus menyebutkan bahwa staf profesional tersebut telah memenuhi persyaratan yang ditetapkan Rumah Sakit. Manajemen rumah sakit menetapkan kriteria dan isi kontrak agar kerjasama dapat berjalan dengan baik dan rumah sakit memperoleh manfaat dan pelayanan yang bermutu.

Pimpinan unit berpartisipasi dalam mengkaji dan memilih semua kontrak klinis dan nonklinis serta bertanggung jawab untuk memantau kontrak tersebut.
Kontrak dan perjanjian- perjanjian merupakan bagian dalam program mutu dan keselamatan pasien. Untuk memastikan mutu dan keselamatan pasien, perlu dilakukan evaluasi untuk semua layanan yang diberikan baik secara langsung oleh rumah sakit
maupun melalui kontrak.

Karena itu, rumah sakit perlu meminta informasi mutu (misalnya quality control), menganalisis, kemudian mengambil tindakan terhadap informasi mutu yang diberikan pihak yang di kontrak. Isi kontrak dengan pihak yang dikontrak harus mencantumkan apa yang diharapkan untuk menjamin mutu dan keselamatan pasien, data apa yang harus diserahkan kepada rumah sakit, frekuensi penyerahan data, serta formatnya. Pimpinan unit layanan menerima laporan mutu dari pihak yang dikontrak tersebut, untuk kemudian ditindaklanjuti dan memastikan bahwa laporan- laporan tersebut diintegrasikan ke dalam proses penilaian mutu rumah sakit.

\hypertarget{elemen-penilaian-tkrs-6}{%
\paragraph*{3. Elemen Penilaian TKRS 6}\label{elemen-penilaian-tkrs-6}}
\addcontentsline{toc}{paragraph}{3. Elemen Penilaian TKRS 6}

\begin{enumerate}
\def\labelenumi{\alph{enumi}.}
\tightlist
\item
  Pimpinan rumah sakit bertanggung jawab terhadap kontrak untuk memenuhi kebutuhan pasien dan manajemen termasuk ruang lingkup pelayanan tersebut yang dicantumkan dalam persetujuan kontrak.
\item
  Tenaga kesehatan yang dikontrak perlu dilakukan kredensial sesuai ketentuan di rumah sakit.
\item
  Pimpinan rumah sakit menginspeksi kepatuhan layanan kontrak sesuai kebutuhan
\item
  Apabila kontrak dinegosiasikan ulang atau dihentikan, rumah sakit tetap mempertahankan kelanjutan dari pelayanan pasien
\item
  Semua kontrak menetapkan data mutu yang harus dilaporkan kepada rumah sakit, disertai frekuensi dan mekanisme pelaporan, serta bagaimana rumah sakit akan merespons jika persyaratan atau ekspektasi mutu tidak terpenuhi.
\item
  Pimpinan klinis dan non klinis yang terkait layanan yang dikontrak melakukan analisis dan memantau informasi mutu yang dilaporkan pihak yang dikontrak yang merupakan bagian dalam program penigkatan mutu dan keselamatan pasien rumah sakit.
\end{enumerate}

\hypertarget{f.-kepemimpinan-rumah-sakit-terkait-keputusan-mengenai-sumber-daya}{%
\subsection*{f.~Kepemimpinan Rumah Sakit Terkait Keputusan Mengenai Sumber Daya}\label{f.-kepemimpinan-rumah-sakit-terkait-keputusan-mengenai-sumber-daya}}
\addcontentsline{toc}{subsection}{f.~Kepemimpinan Rumah Sakit Terkait Keputusan Mengenai Sumber Daya}

\hypertarget{standar-tkrs-7}{%
\paragraph*{1. Standar TKRS 7}\label{standar-tkrs-7}}
\addcontentsline{toc}{paragraph}{1. Standar TKRS 7}

Pimpinan rumah sakit membuat keputusan tentang pengadaan dan pembelian. Penggunaan sumber daya manusia dan sumber daya lainnya harus berdasarkan pertimbangan mutu dan dampaknya pada keselamatan.

\hypertarget{maksud-dan-tujuan-tkrs-7}{%
\paragraph*{2. Maksud dan Tujuan TKRS 7}\label{maksud-dan-tujuan-tkrs-7}}
\addcontentsline{toc}{paragraph}{2. Maksud dan Tujuan TKRS 7}

Pimpinan rumah sakit akan mengutamakan mutu dan keselamatan pasien daripada biaya pada saat akan mengambil keputusan terkait pembelian dan keputusan terhadap sumber daya lainnya seperti pengurangan atau pemindahan staf keperawatan. Misalnya pada saat diputuskan untuk membeli pompa infus baru, maka informasi tingkat kegagalan dan insiden keselamatan pasien terkait alat yang akan dibeli, preferensi dari staf, catatan terkait adanya masalah dengan alarm dari pompa infus, pemeliharaan alat, pelatihan yang diperlukan dan hal lain terkait mutu dan keselamatan pasien di pakai sebagai dasar untuk membuat keputusan pembelian.

Pimpinan rumah sakit mengembangkan proses untuk mengumpulkan data dan informasi untuk pembelian ataupun keputusan mengenai sumber daya untuk memastikan bahwa keputusannya sudah berdasarkan pertimbangan mutu dan keselamatan.
Data terkait keputusan mengenai sumber daya adalah memahami kebutuhan dan rekomendasi peralatan medis,perbekalan dan obat-obatan yang dibutuhkan untuk pelayanan. Rekomendasi dapat diperoleh dari pemerintah, organisasi profesional nasional dan internasional serta sumber berwenang lainnya.

Investasi untuk teknologi informasi kesehatan (TIK) merupakan sumber daya yang penting bagi rumah sakit. TIK meliputi berbagai teknologi yang mencakup metode pendokumentasian dan penyebaran informasi pasien, seperti rekam medis elektronik. Selain itu, TIK juga meliputi metode untuk menyimpan dan menganalisis data, mengomunikasikan informasi antarpraktisi kesehatan agar dapat mengoordinasikan pelayanan lebih baik, serta untuk menerima informasi yang dapat membantu menegakkan diagnosis dan memberikan pelayanan yang aman bagi pasien. Implementasi sumber daya TIK membutuhkan arahan, dukungan, dan pengawasan dari pimpinan rumah sakit. Ketika keputusan mengenai pengadaan sumber daya dibuat oleh pihak ketiga misalnya Kementerian Kesehatan, maka pimpinan rumah sakit menginformasikan kepada Kementerian Kesehatan pengalaman dan preferensi sumber daya tersebut sebagai dasar untuk membuat keputusan.

\hypertarget{elemen-penilaian-tkrs-7}{%
\paragraph*{3. Elemen Penilaian TKRS 7}\label{elemen-penilaian-tkrs-7}}
\addcontentsline{toc}{paragraph}{3. Elemen Penilaian TKRS 7}

\begin{enumerate}
\def\labelenumi{\alph{enumi}.}
\tightlist
\item
  Pimpinan rumah sakit menggunakan data dan informasi mutu serta dampak terhadap keselamatan untuk membuat keputusan pembelian dan penggunaan peralatan baru.
\item
  Pimpinan rumah sakit menggunakan data dan informasi mutu serta dampak terhadap keselamatan dalam pemilihan, penambahan, pengurangan dan melakukan rotasi staf.
\item
  Pimpinan rumah sakit menggunakan rekomendasi dari organisasi profesional dan sumber berwenang lainnya dalam mengambil keputusan mengenai pengadaan sumber daya.
\item
  Pimpinan rumah sakit memberikan arahan, dukungan, dan pengawasan terhadap penggunaan sumber daya Teknologi informasi Kesehatan (TIK)
\item
  Pimpinan rumah sakit memberikan arahan, dukungan, dan pengawasan terhadap pelaksanaan program penanggulangan kedaruratan dan bencana.
\item
  Pimpinan rumah sakit memantau hasil keputusannya dan menggunakan data tersebut untuk mengevaluasi dan memperbaiki mutu keputusan pembelian dan pengalokasian sumber daya.
\end{enumerate}

\hypertarget{standar-tkrs-7.1}{%
\paragraph*{4. Standar TKRS 7.1}\label{standar-tkrs-7.1}}
\addcontentsline{toc}{paragraph}{4. Standar TKRS 7.1}

Pimpinan rumah sakit mencari dan menggunakan data serta informasi tentang keamanan dalam rantai perbekalan untuk melindungi pasien dan staf terhadap produk yang tidak stabil, terkontaminasi, rusak, dan palsu.

\hypertarget{maksud-dan-tujuan-tkrs-7.2}{%
\paragraph*{5. Maksud dan Tujuan TKRS 7.2}\label{maksud-dan-tujuan-tkrs-7.2}}
\addcontentsline{toc}{paragraph}{5. Maksud dan Tujuan TKRS 7.2}

Pengelolaan rantai perbekalan merupakan hal penting untuk memastikan keamanan dan mutu perbekalan rumah sakit. Rantai perbekalan meliputi serangkaian proses dimulai dari produsen hingga pengantaran perbekalan ke rumah sakit. Jenis dan jumlah perbekalan yang digunakan rumah sakit sangat bervariasi, oleh karena itu rumah sakit harus mengelola begitu banyak rantai perbekalan. Karena staf dan sumber daya yang terbatas, tidak semua rantai perbekalan dapat dilacak dan dievaluasi di saat yang sama. Oleh karena itu, rumah sakit harus menentukan obat- obatan, perbekalan medis, serta peralatan medis yang paling berisiko tidak stabil, mengalami kontaminasi, rusak, atau ditukar dengan produk palsu atau imitasi.

Untuk perbekalan-perbekalan yang berisiko tersebut, rumah sakit menentukan langkah-langkah untuk mengelola rantai perbekalannya. Meskipun informasi ini mungkin tidak lengkap dan sulit untuk dirangkaikan menjadi satu, minimal rumah sakit harus memutuskan di manakah terdapat risiko yang paling tinggi, misalnya dengan membuat bagan alur/flow chart untuk memetakan setiap langkah, atau titik dalam rantai perbekalan dengan mencantumkan produsen, fasilitas gudang, vendor, distributor, dan lain-lainnya.

Rumah sakit dapat menunjukkan titik mana di dalam bagan alur tersebut yang memiliki risiko paling signifikan. misalnya, rumah sakit menentukan obat insulin sebagai obat yang paling berisiko di rumah sakit, kemudian membuat bagan alur yang menunjukkan setiap langkah dalam rantai perbekalan obat insulin. Rumah sakit kemudian menentukan titik-titik mana yang berisiko, seperti titik produsen, vendor, gudang, dan pengiriman, serta dapat menentukan elemen-elemen penting lainnya yang harus dipertimbangkan seperti kepatuhan produsen terhadap regulasi, pengendalian dan pemantauan suhu di gudang, serta pembatasan jarak tempuh antar satu titik ke titik yang lain dalam rantai perbekalan.

Pada saat meninjau risiko potensial dalam suatu rantai perbekalan, rumah sakit mengetahui bahwa ternyata vendor baru saja menandatangani kontrak dengan perusahaan pengiriman logistik yang layanannya kurang memuaskan, termasuk pengiriman yang terlambat dan pencatatan pemantauan suhu yang tidak konsisten selama pengiriman. Setelah mengkaji situasi ini, rumah sakit dapat menggolongkan hal ini sebagai risiko yang signifikan dalam rantai perbekalan. Pimpinan rumah sakit harus mengambil keputusan untuk membuat perubahan terhadap rantai perbekalan dan menentukan prioritas pengambilan keputusan terkait pembelian berdasarkan informasi titik risiko dalam rantai perbekalan tersebut.

Pengelolaan rantai perbekalan bukan hanya mengenai evaluasi prospektif terhadap perbekalan yang berisiko tinggi, proses ini juga meliputi pelacakan retrospektif terhadap perbekalan yang ada setelah perbekalan tersebut diantarkan ke rumah sakit. Rumah sakit harus memiliki proses untuk mengidentifikasi obat-obatan, perbekalan medis, serta peralatan medis yang tidak stabil, terkontaminasi, rusak atau palsu dan melacak kembali perbekalan-perbekalan tersebut untuk menentukan sumber atau penyebab masalah yang ada, jika memungkinkan. Rumah sakit harus memberitahu produsen dan/atau distributor apabila ditemukan perbekalan yang tidak stabil, terkontaminasi, rusak atau palsu dalam pelacakan retrospektif.

Ketika perbekalan rumah sakit dibeli, disimpan dan didistribusikan oleh pemerintah, rumah sakit dapat berpartisipasi untuk mendeteksi dan melaporkan jika menemukan perbekalan yang diduga tidak stabil, terkontaminasi, rusak, atau palsu serta melakukan tindakan untuk mencegah kemungkinan bahaya bagi pasien. Meskipun rumah sakit pemerintah mungkin tidak tahu integritas dari setiap pemasok dalam rantai perbekalan, rumah sakit perlu ikut memantau perbekalan yang dibeli dan dikelola oleh pemerintah ataupun nonpemerintah.

\hypertarget{elemen-penilaian-tkrs-7.3}{%
\paragraph*{7. Elemen Penilaian TKRS 7.3}\label{elemen-penilaian-tkrs-7.3}}
\addcontentsline{toc}{paragraph}{7. Elemen Penilaian TKRS 7.3}

\begin{enumerate}
\def\labelenumi{\alph{enumi}.}
\tightlist
\item
  Pimpinan rumah sakit menentukan obat-obatan, perbekalan medis, serta peralatan medis yang paling berisiko dan membuat bagan alur rantai perbekalannya.
\item
  Pimpinan rumah sakit menentukan titik paling berisiko dalam bagan alur rantai perbekalan dan membuat keputusan berdasarkan risiko dalam rantai perbekalan tersebut.
\item
  Rumah sakit memiliki proses untuk melakukan pelacakan retrospektif terhadap perbekalan yang diduga tidak stabil, terkontaminasi, rusak, atau palsu.
\item
  Rumah sakit memberitahu produsen dan/atau distributor bila menemukan perbekalan yang tidak stabil, terkontaminasi, rusak, atau palsu.
\end{enumerate}

\hypertarget{g.-pengorganisasian-dan-akuntabilitas-komite-medik-komite-keperawatan-dan-komite-tenaga-kesehatan-lain}{%
\subsection*{g. Pengorganisasian dan Akuntabilitas Komite Medik, Komite Keperawatan, dan Komite Tenaga Kesehatan Lain}\label{g.-pengorganisasian-dan-akuntabilitas-komite-medik-komite-keperawatan-dan-komite-tenaga-kesehatan-lain}}
\addcontentsline{toc}{subsection}{g. Pengorganisasian dan Akuntabilitas Komite Medik, Komite Keperawatan, dan Komite Tenaga Kesehatan Lain}

\hypertarget{standar-tkrs-8}{%
\paragraph*{1. Standar TKRS 8}\label{standar-tkrs-8}}
\addcontentsline{toc}{paragraph}{1. Standar TKRS 8}

Komite Medik, Komite Keperawatan dan Komite Tenaga Kesehatan Lain menerapkan pengorganisasisannya sesuai peraturan perundang-undangan untuk mendukung tanggung jawab serta wewenang mereka.

\hypertarget{maksud-dan-tujuan-tkrs-8}{%
\paragraph*{2. Maksud dan Tujuan TKRS 8}\label{maksud-dan-tujuan-tkrs-8}}
\addcontentsline{toc}{paragraph}{2. Maksud dan Tujuan TKRS 8}

Struktur organisasi Komite Medik, Komite Keperawatan, dan Komite Tenaga Kesehatan Lain ditetapkan oleh Direktur sesuai peraturan perundang-undangan yang berlaku. Dalam menjalankan fungsinya, Komite Medik, Komite Keperawatan dan Komite Tenaga Kesehatan Lain mempunyai tanggung jawab kepada pasien dan kepada rumah sakit yaitu:

\begin{enumerate}
\def\labelenumi{\alph{enumi}.}
\tightlist
\item
  Mendukung komunikasi yang efektif antar tenaga profesional;
\item
  Menyusun kebijakan, pedoman, prosedur serta protokol, tata hubungan kerja, alur klinis, dan dokumen lain yang mengatur layanan klinis;
\item
  Menyusun kode etik profesi; dan
\item
  Memantau mutu pelayanan pasien lainnya.
\end{enumerate}

\hypertarget{elemen-penilaian-tkrs-8}{%
\paragraph*{3. Elemen Penilaian TKRS 8}\label{elemen-penilaian-tkrs-8}}
\addcontentsline{toc}{paragraph}{3. Elemen Penilaian TKRS 8}

\begin{enumerate}
\def\labelenumi{\alph{enumi}.}
\tightlist
\item
  Terdapat struktur organisasi Komite Medik, Komite Keperawatan, dan Komite Tenaga Kesehatan Lain yang ditetapkan Direktur sesuai peraturan perundang- undangan yang berlaku.
\item
  Komite Medik, Komite Keperawatan dan Komite Tenaga Kesehatan Lain melaksanakan tanggung jawabnya mencakup (a-d) dalam maksud dan tujuan.
\item
  Untuk melaksanakan tanggung jawabnya Komite Medik, Komite Keperawatan, dan Komite Tenaga Kesehatan Lain menyusun Program kerja setiap tahun dan ditetapkan oleh Direktur.
\end{enumerate}

\hypertarget{h.-akuntabilitas-kepala-unit-klinisnon-klinis}{%
\subsection*{h. Akuntabilitas Kepala Unit Klinis/Non Klinis}\label{h.-akuntabilitas-kepala-unit-klinisnon-klinis}}
\addcontentsline{toc}{subsection}{h. Akuntabilitas Kepala Unit Klinis/Non Klinis}

\hypertarget{standar-tkrs-9}{%
\paragraph*{1. Standar TKRS 9}\label{standar-tkrs-9}}
\addcontentsline{toc}{paragraph}{1. Standar TKRS 9}

Unit layanan di rumah sakit dipimpin oleh kepala unit yang ditetapkan oleh Direktur sesuai dengan kompetensinya untuk mengarahkan kegiatan di unitnya.

\hypertarget{maksud-dan-tujuan-tkrs-9}{%
\paragraph*{2. Maksud dan Tujuan TKRS 9}\label{maksud-dan-tujuan-tkrs-9}}
\addcontentsline{toc}{paragraph}{2. Maksud dan Tujuan TKRS 9}

Kinerja yang baik di unit layanan membutuhkan kepemimpinan yang kompeten dalam melaksanakan tanggung jawabnya yang dituangkan dalam urain tugas.

Setiap kepala unit merencanakan dan melaporkan kebutuhan staf dan sumber daya misalnya ruangan, peralatan dan sumber daya lainnya kepada pimpinan rumah sakit untuk memenuhi pelayanan sesuai kebutuhan pasien. Meskipun para kepala unit layanan telah membuat rencana kebutuhan sumber daya manusia dan sumber daya lainnya, namun terkadang terdapat perubahan prioritas di dalam rumah sakit yang mengakibatkan tidak terpenuhinya sumber daya yang dibutuhkan. Oleh karena itu, kepala unit harus memiliki proses untuk merespon kekurangan sumber daya agar memastikan pemberian pelayanan yang aman dan efektif bagi semua pasien.

Kepala unit layanan menyusun kriteria berdasarkan pendidikan, keahlian, pengetahuan, dan pengalaman yang diperlukan professional pemberi asuhan (PPA) dalam memberikan pelayanan di unit layanan tersebut. Kepala unit layanan juga bekerja sama dengan Unit SDM dan unit lainnya dalam melakukan proses seleksi staf.
Kepala unit layanan memastikan bahwa semua staf dalam unitnya memahami tanggung jawabnya dan mengadakan kegiatan orientasi dan pelatihan bagi staf baru.

Kegiatan orientasi mencakup misi rumah sakit, lingkup pelayanan yang diberikan, serta kebijakan dan prosedur yang terkait pelayanan yang diberikan di unit tersebut, misalnya semua staf telah memahami prosedur pencegahan dan pengendalian infeksi rumah sakit dan di unit layanan tersebut. Bila terdapat revisi kebijakan atau prosedur baru, staf akan diberikan pelatihan ulang.

Para kepala unit kerja menyusun program kerja di masing- masing unit setiap tahun, menggunakan format yang seragam yang telah ditetapkan rumah sakit. Kepala unit kerja melakukan koordinasi dan integrasi dalam unitnya dan antar unit layanan untuk mencegah duplikasi pelayanan, misalnya koordinasi dan integrasi antara pelayanan medik dan pelayanan keperawatan.

\hypertarget{elemen-penilaian-tkrs-9}{%
\paragraph*{3. Elemen Penilaian TKRS 9}\label{elemen-penilaian-tkrs-9}}
\addcontentsline{toc}{paragraph}{3. Elemen Penilaian TKRS 9}

\begin{enumerate}
\def\labelenumi{\alph{enumi}.}
\tightlist
\item
  Kepala unit kerja diangkat sesuai kualifikasi dalam persyaratan jabatan yang ditetapkan.
\item
  Kepala unit kerja menyusun pedoman pengorganisasian, pedoman pelayanan dan prosedur sesuai proses bisnis di unit kerja.
\item
  Kepala unit kerja menyusun program kerja yang termasuk di dalamnya kegiatan peningkatan mutu dan keselamatan pasien serta manajemen risiko setiap tahun.
\item
  Kepala unit kerja mengusulkan kebutuhan sumber daya mencakup ruangan, peralatan medis, teknologi informasi dan sumber daya lain yang diperlukan unit layanan serta terdapat mekanisme untuk menanggapi kondisi jika terjadi kekurangan tenaga.
\item
  Kepala unit kerja telah melakukan koordinasi dan integrasi baik dalam unitnya maupun antar unit layanan.
\end{enumerate}

\hypertarget{standar-tkrs-10}{%
\paragraph*{4. Standar TKRS 10}\label{standar-tkrs-10}}
\addcontentsline{toc}{paragraph}{4. Standar TKRS 10}

Kepala unit layanan berpartisipasi dalam meningkatkan mutu dan keselamatan pasien dengan melakukan pengukuran indikator mutu rumah sakit yang dapat diterapkan di unitnya dan memantau serta memperbaiki pelayanan pasien di unit layanannya.

\hypertarget{maksud-dan-tujuan-tkrs-10}{%
\paragraph*{5. Maksud dan Tujuan TKRS 10}\label{maksud-dan-tujuan-tkrs-10}}
\addcontentsline{toc}{paragraph}{5. Maksud dan Tujuan TKRS 10}

Kepala unit layanan melibatkan semua stafnya dalam kegiatan pengukuran indikator prioritas rumah sakit yang perbaikan akan berdampak luas/menyeluruh di rumah sakit baik kegiatan klinis maupun non klinis yang khusus untuk unit layanan tersebut. Misalnya indikator prioritas rumah sakit adalah komunikasi saat serah terima yang perbaikannya akan berdampak luas/menyeluruh di semua unit klinis maupun non klinis. Hal yang sama juga dapat dilakukan pada unit non klinis untuk memperbaiki komunikasi serah terima dengan menerapkan proyek otomatisasi untuk memonitor tingkat keakurasian saat pembayaran pasien.

Kepala unit klinis memilih indikator mutu yang akan dilakukan pengukuran sesuai dengan pelayanan di unitnya mencakup hal-hal sebagai berikut:

\begin{enumerate}
\def\labelenumi{\alph{enumi}.}
\tightlist
\item
  Pengukuran indikator nasional mutu (INM).
\item
  Pengukuran indikator mutu prioritas rumah sakit (IMP-RS) yang berdampak luas dan menyeluruh di rumah sakit.
\item
  Pengukuran indikator mutu prioritas unit (IMP-unit) untuk mengurangi variasi, meningkatkan keselamatan pada prosedur/tindakan berisiko tinggi dan meningkatkan kepuasan pasien serta efisiensi sumber daya.
\end{enumerate}

Pemilihan pengukuran berdasarkan pelayanan dan bisnis proses yang membutuhkan perbaikan di setiap unit layanan. Setiap pengukuran harus ditetapkan target yang diukur dan dianalisis capaian dan dapat dipertahankan dalam waktu 1 (satu) tahun. Jika target telah tercapai dan dapat dipertahankan untuk dalam waktu 1 (satu) tahun maka dapat diganti dengan indikator yang baru.

Kepala unit layanan klinis dan non klinis bertanggung jawab memberikan penilaian kinerja staf yang bekerja di unitnya. Karena itu penilaian kinerja staf harus mencakup kepatuhan terhadap prioritas perbaikan mutu di unit yaitu indikator mutu prioritas unit (IMP-unit) sebagai upaya perbaikan di setiap unit untuk meningkatkan mutu dan keselamatan pasien tingkat unit.

\hypertarget{elemen-penilaian-tkrs-10}{%
\paragraph*{6. Elemen Penilaian TKRS 10}\label{elemen-penilaian-tkrs-10}}
\addcontentsline{toc}{paragraph}{6. Elemen Penilaian TKRS 10}

\begin{enumerate}
\def\labelenumi{\alph{enumi}.}
\tightlist
\item
  Kepala unit klinis/non klinis melakukan pengukuran INM yang sesuai dengan pelayanan yang diberikan oleh unitnya
\item
  Kepala unit klinis/non klinis melakukan pengukuran IMP-RS yang sesuai dengan pelayanan yang diberikan oleh unitnya, termasuk semua layanan kontrak yang menjadi tanggung jawabnya.
\item
  Kepala unit klinis/non klinis menerapkan pengukuran IMP-Unit untuk mengurangi variasi dan memperbaiki proses dalam unitnya,
\item
  Kepala unit klinis/non klinis memilih prioritas perbaikan yang baru bila perbaikan sebelumnya sudah dapat dipertahankan dalam waktu 1 (satu) tahun.
\end{enumerate}

\hypertarget{standar-tkrs-11}{%
\paragraph*{7. Standar TKRS 11}\label{standar-tkrs-11}}
\addcontentsline{toc}{paragraph}{7. Standar TKRS 11}

Kepala unit klinis mengevaluasi kinerja para dokter, perawat dan tenaga kesehatan profesional lainnya menggunakan indikator mutu yang diukur di unitnya.

\hypertarget{maksud-dan-tujuan-tkrs-11}{%
\paragraph*{8. Maksud dan Tujuan TKRS 11}\label{maksud-dan-tujuan-tkrs-11}}
\addcontentsline{toc}{paragraph}{8. Maksud dan Tujuan TKRS 11}

Kepala unit klinis bertanggung jawab untuk memastikan bahwa mutu pelayanan yang diberikan oleh stafnya dilakukan secara konsisten dengan melakukan evaluasi kinerja terhadap stafnya. Kepala unit klinis juga terlibat dalam memberikan rekomendasi tentang penunjukan, delineasi kewenangan, evaluasi praktik profesional berkelanjutan ( \emph{On going Professional Practice Evaluation} ), serta penugasan kembali dokter/perawat/tenaga kesehatan lain yang bertugas dalam unitnya.

\hypertarget{elemen-penilaian-tkrs-11}{%
\paragraph*{9. Elemen Penilaian TKRS 11}\label{elemen-penilaian-tkrs-11}}
\addcontentsline{toc}{paragraph}{9. Elemen Penilaian TKRS 11}

\begin{enumerate}
\def\labelenumi{\alph{enumi}.}
\tightlist
\item
  Penilaian praktik profesional berkelanjutan ( \emph{On going Professional Practice Evaluation} ) para dokter dalam memberikan pelayanan untuk meningkatkan mutu dan keselamatan pasien menggunakan indikator mutu yang diukur di unit tersebut.
\item
  Penilaian kinerja para perawat dalam memberikan pelayanan untuk meningkatkan mutu dan keselamatan pasien menggunakan indikator mutu yang diukur di unit tersebut.
\item
  Penilaian kinerja tenaga kesehatan lainnya memberikan pelayanan untuk meningkatkan mutu dan keselamatan pasien menggunakan indikator mutu yang diukur di unit tersebut.
\end{enumerate}

\hypertarget{i.-etika-rumah-sakit}{%
\subsection*{i. Etika Rumah Sakit}\label{i.-etika-rumah-sakit}}
\addcontentsline{toc}{subsection}{i. Etika Rumah Sakit}

\hypertarget{standar-tkrs-12}{%
\paragraph*{1. Standar TKRS 12}\label{standar-tkrs-12}}
\addcontentsline{toc}{paragraph}{1. Standar TKRS 12}

Pimpinan rumah sakit menetapkan kerangka kerja pengelolaan etik rumah sakit untuk menangani masalah etik rumah sakit meliputi finansial, pemasaran, penerimaan pasien, transfer pasien, pemulangan pasien dan yang lainnya termasuk konflik etik antar profesi serta konflik kepentingan staf yang mungkin bertentangan dengan hak dan kepentingan pasien.

\hypertarget{maksud-dan-tujuan-tkrs-12}{%
\paragraph*{2. Maksud dan Tujuan TKRS 12}\label{maksud-dan-tujuan-tkrs-12}}
\addcontentsline{toc}{paragraph}{2. Maksud dan Tujuan TKRS 12}

Rumah sakit menghadapi banyak tantangan untuk memberikan pelayanan yang aman dan bermutu. Dengan kemajuan dalam teknologi medis, pengaturan finansial, dan harapan yang terus meningkat, dilema etik dan kontroversi telah menjadi suatu hal yang lazim terjadi.

Pimpinan rumah sakit bertanggung jawab secara profesional dan hukum untuk menciptakan dan mendukung lingkungan dan budaya etik dan dan memastikan bahwa pelayanan pasien diberikan dengan mengindahkan norma bisnis, keuangan, etika dan hukum, serta melindungi pasien dan hak-hak pasien serta harus menunjukkan teladan perilaku etik bagi stafnya.

Untuk melaksanakan hal tersebut Direktur menetapkan Komite Etik rumah sakit untuk menangani masalah dan dilema etik dalam dalam pelayanan klinis (misalnya perselisihan antar profesional dan perselisihan antara pasien dan dokter mengenai keputusan dalam pelayanan pasien) dan kegiatan bisnis rumah sakit (misalnya kelebihan input pada pembayaran tagihan pasien yang harus dikembalikan oleh rumah sakit).

Dalam melaksanakan tugasnya Komite Etik:

\begin{enumerate}
\def\labelenumi{\alph{enumi}.}
\tightlist
\item
  Menyusun kode etik rumah sakit yang mengacu pada kode etik rumah sakit Indonesia (KODERSI)
\item
  Menyusun kerangka kerja pengelolaan etik rumah sakit mencakup tapi tidak terbatas pada:
\end{enumerate}

\begin{enumerate}
\def\labelenumi{\arabic{enumi}.}
\tightlist
\item
  Menjelaskan pelayanan yang diberikan pada pasien secara jujur;
\item
  Melindungi kerahasiaan informasi pasien;
\item
  Mengurangi kesenjangan dalam akses untuk mendapatkan pelayanan kesehatan dan dampak klinis.
\item
  Menetapkan kebijakan tentang pendaftaran pasien, transfer, dan pemulangan pasien;
\item
  Mendukung transparansi dalam melaporkan pengukuran hasil kinerja klinis dan kinerja non klinis
\item
  Keterbukaan kepemilikan agar tidak terjadi konflik kepentingan misalnya hubungan kepemilikan antara dokter yang memberikan instruksi pemeriksaan penunjang dengan fasilitas laboratorium atau fasilitas radiologi di luar rumah sakit yang akan melakukan pemeriksaan.
\item
  Menetapkan mekanisme bahwa praktisi kesehatan dan staf lainnya dapat melaporkan kesalahan klinis (clinical error) atau mengajukan kekhawatiran etik tanpa takut dihukum, termasuk melaporkan perilaku staf yang merugikan terkait masalah klinis ataupun operasional;
\item
  Mendukung keterbukaan dalam sistem pelaporan mengenai masalah/isu etik tanpa takut diberikan sanksi;
\item
  Memberikan solusi yang efektif dan tepat waktu untuk masalah etik yang terjadi;
\item
  Memastikan praktik nondiskriminasi dalam pelayanan pasien dengan mengingat norma hukum dan budaya negara; dan
\item
  Tagihan biaya pelayanan harus akurat dan dipastikan bahwa insentif dan pengelolaan pembayaran tidak menghambat pelayanan pasien.
\item
  Pengelolaan kasus etik pada konflik etik antar profesi di rumah sakit, serta penetapan Code of Conduct bagi staf sebagai pedoman perilaku sesuai dengan standar etik di rumah sakit.
\end{enumerate}

Komite Etik mempertimbangkan norma-norma nasional dan internasional terkait dengan hak asasi manusia dan etika profesional dalam menyusun etika dan dokumen pedoman lainnya.
Pimpinan rumah sakit mendukung pelaksanaan kerangka kerja pengeloaan etik rumah sakit seperti pelatihan untuk praktisi kesehatan dan staf lainnya.

\hypertarget{elemen-penilaian-tkrs-12}{%
\paragraph*{3. Elemen Penilaian TKRS 12}\label{elemen-penilaian-tkrs-12}}
\addcontentsline{toc}{paragraph}{3. Elemen Penilaian TKRS 12}

\begin{enumerate}
\def\labelenumi{\alph{enumi}.}
\tightlist
\item
  Direktur rumah sakit menetapkan Komite Etik rumah sakit.
\item
  Komite Etik telah menyusun kode etik rumah sakit yang mengacu pada Kode Etik Rumah Sakit Indonesia (KODERSI) dan ditetapkan Direktur.
\item
  Komite Etik telah menyusun kerangka kerja pelaporan dan pengelolaan etik rumah sakit serta pedoman pengelolaan kode etik rumah sakit meliputi poin (1) sampai dengan (12) dalam maksud dan tujuan sesuai dengan visi, misi, dan nilai-nilai yang dianut rumah sakit.
\item
  Rumah sakit menyediakan sumber daya serta pelatihan kerangka pengelolaan etik rumah sakit bagi praktisi kesehatan dan staf lainnya dan memberikan solusi yang efektif dan tepat waktu untuk masalah etik.
\end{enumerate}

\hypertarget{j.-kepemimpinan-untuk-budaya-keselamatan-di-rumah-sakit}{%
\subsection*{j. Kepemimpinan Untuk Budaya Keselamatan di Rumah Sakit}\label{j.-kepemimpinan-untuk-budaya-keselamatan-di-rumah-sakit}}
\addcontentsline{toc}{subsection}{j. Kepemimpinan Untuk Budaya Keselamatan di Rumah Sakit}

\hypertarget{standar-tkrs-13}{%
\paragraph*{1. Standar TKRS 13}\label{standar-tkrs-13}}
\addcontentsline{toc}{paragraph}{1. Standar TKRS 13}

Pimpinan rumah sakit menerapkan, memantau dan mengambil tindakan serta mendukung Budaya Keselamatan di seluruh area rumah sakit.

\hypertarget{maksud-dan-tujuan-tkrs-13}{%
\paragraph*{2. Maksud dan Tujuan TKRS 13}\label{maksud-dan-tujuan-tkrs-13}}
\addcontentsline{toc}{paragraph}{2. Maksud dan Tujuan TKRS 13}

Budaya keselamatan di rumah sakit merupakan suatu lingkungan kolaboratif di mana para dokter saling menghargai satu sama lain, para pimpinan mendorong kerja sama tim yang efektif dan menciptakan rasa aman secara psikologis serta anggota tim dapat belajar dari insiden keselamatan pasien, para pemberi layanan menyadari bahwa ada keterbatasan manusia yang bekerja dalam suatu sistem yang kompleks dan terdapat suatu proses pembelajaran serta upaya untuk mendorong perbaikan.

Budaya keselamatan juga merupakan hasil dari nilai-nilai, sikap, persepsi, kompetensi, dan pola perilaku individu maupun kelompok yang menentukan komitmen terhadap, serta kemampuan mengelola pelayanan kesehatan maupun keselamatan.

Keselamatan dan mutu berkembang dalam suatu lingkungan yang membutuhkan kerja sama dan rasa hormat satu sama lain, tanpa memandang jabatannya. Pimpinan rumah sakit menunjukkan komitmennya mendorong terciptanya budaya keselamatan tidak mengintimidasi dan atau mempengaruhi staf dalam memberikan pelayanan kepada pasien. Direktur menetapkan Program Budaya Keselamatan di rumah sakit yang mencakup:

\begin{enumerate}
\def\labelenumi{\alph{enumi}.}
\tightlist
\item
  Perilaku memberikan pelayanan yang aman secara konsisten untuk mencegah terjadinya kesalahan pada pelayanan berisiko tinggi.
\item
  Perilaku di mana para individu dapat melaporkan kesalahan dan insiden tanpa takut dikenakan sanksi atau teguran dan diperlakuan secara adil (just culture)
\item
  Kerja sama tim dan koordinasi untuk menyelesaikan masalah keselamatan pasien.
\item
  Komitmen pimpinan rumah sakit dalam mendukung staf seperti waktu kerja para staf, pendidikan, metode yang aman untuk melaporkan masalah dan hal lainnya untuk menyelesaikan masalah keselamatan.
\item
  Identifikasi dan mengenali masalah akibat perilaku yang tidak diinginkan (perilaku sembrono).
\item
  Evaluasi budaya secara berkala dengan metode seperti kelompok fokus diskusi (FGD), wawancara dengan staf, dan analisis data.
\item
  Mendorong kerja sama dan membangun sistem, dalam mengembangkan budaya perilaku yang aman.
\item
  Menanggapi perilaku yang tidak diinginkan pada semua staf pada semua jenjang di rumah sakit, termasuk manajemen, staf administrasi, staf klinis dan nonklinis, dokter praktisi mandiri, representasi pemilik dan anggota Dewan pengawas.
\end{enumerate}

Perilaku yang tidak mendukung budaya keselamatan di antaranya adalah: perilaku yang tidak layak seperti kata- kata atau bahasa tubuh yang merendahkan atau menyinggung perasaan sesama staf, misalnya mengumpat dan memaki, perilaku yang mengganggu, bentuk tindakan verbal atau nonverbal yang membahayakan atau mengintimidasi staf lain, perilaku yang melecehkan ( \emph{harassment} ) terkait dengan ras, agama, dan suku termasuk gender serta pelecehan seksual.
Seluruh pemangku kepentingan di rumah sakit bertanggungjawab mewujudkan budaya keselamatan dengan berbagai cara.

Saat ini di rumah sakit masih terdapat budaya menyalahkan orang lain ketika terjadi suatu kesalahan ( \emph{blaming culture} ), yang akhirnya menghambat budaya keselamatan sehingga pimpinan rumah sakit harus menerapkan perlakuan yang adil ( \emph{just culture} ) ketika terjadi kesalahan, dimana ada saatnya staf tidak disalahkan ketika terjadi kesalahan, misalnya pada kondisi:

\begin{enumerate}
\def\labelenumi{\alph{enumi}.}
\tightlist
\item
  Komunikasi yang kurang baik antara pasien dan staf.
\item
  Perlu pengambilan keputusan secara cepat.
\item
  Kekurangan staf dalam pelayanan pasien.
\end{enumerate}

Di sisi lain terdapat kesalahan yang dapat diminta pertanggungjawabannya ketika staf dengan sengaja melakukan perilaku yang tidak diinginkan (perilaku sembrono) misalnya:

\begin{enumerate}
\def\labelenumi{\alph{enumi}.}
\tightlist
\item
  Tidak mau melakukan kebersihan tangan.
\item
  Tidak mau melakukan time-out (jeda) sebelum operasi.
\item
  Tidak mau memberi tanda pada lokasi pembedahan.
\end{enumerate}

Rumah sakit harus meminta pertanggungjawaban perilaku yang tidak diinginkan (perilaku sembrono) dan tidak mentoleransinya. Pertanggungjawaban dibedakan atas:

\begin{enumerate}
\def\labelenumi{\alph{enumi}.}
\tightlist
\item
  Kesalahan manusia (human error) adalah tindakan yang tidak disengaja yaitu melakukan kegiatan tidak sesuai dengan apa yang seharusnya dilakukan.
\item
  Perilaku berisiko (risk behaviour) adalah perilaku yang dapat meningkatkan risiko (misalnya, mengambil langkah pada suatu proses layanan tanpa berkonsultasi dengan atasan atau tim kerja lainnya yang dapat menimbulkan risiko).
\item
  Perilaku sembrono (reckless behavior) adalah perilaku yang secara sengaja mengabaikan risiko yang substansial dan tidak dapat dibenarkan.
\end{enumerate}

\hypertarget{elemen-penilaian-tkrs-13}{%
\paragraph*{3. Elemen Penilaian TKRS 13}\label{elemen-penilaian-tkrs-13}}
\addcontentsline{toc}{paragraph}{3. Elemen Penilaian TKRS 13}

\begin{enumerate}
\def\labelenumi{\alph{enumi}.}
\tightlist
\item
  Pimpinan rumah sakit menetapkan Program Budaya Keselamatan yang mencakup poin a) sampai dengan h) dalam maksud dan tujuan serta mendukung penerapannya secara akuntabel dan transparan.
\item
  Pimpinan rumah sakit menyelenggarakan pendidikan dan menyediakan informasi (kepustakaan dan laporan) terkait budaya keselamatan bagi semua staf yang bekerja di rumah sakit.
\item
  Pimpinan rumah sakit menyediakan sumber daya untuk mendukung dan mendorong budaya keselamatan di rumah sakit.
\item
  Pimpinan rumah sakit mengembangkan sistem yang rahasia, sederhana dan mudah diakses bagi staf untuk mengidentifikasi dan melaporkan perilaku yang tidak diinginkan dan menindaklanjutinya.
\item
  Pimpinan rumah sakit melakukan pengukuran untuk mengevaluasi dan memantau budaya keselamatan di rumah sakit serta hasil yang diperoleh dipergunakan untuk perbaikan penerapannya di rumah sakit.
\item
  Pimpinan rumah sakit menerapkan budaya adil (just culture) terhadap staf yang terkait laporan budaya keselamatan tersebut.
\end{enumerate}

\hypertarget{k.-manajemen-risiko}{%
\subsection*{k. Manajemen Risiko}\label{k.-manajemen-risiko}}
\addcontentsline{toc}{subsection}{k. Manajemen Risiko}

\hypertarget{standar-tkrs-14}{%
\paragraph*{1. Standar TKRS 14}\label{standar-tkrs-14}}
\addcontentsline{toc}{paragraph}{1. Standar TKRS 14}

Program manajemen risiko yang terintegrasi digunakan untuk mencegah terjadinya cedera dan kerugian di rumah sakit.

\hypertarget{maksud-dan-tujuan-tkrs-14}{%
\paragraph*{2. Maksud dan Tujuan TKRS 14}\label{maksud-dan-tujuan-tkrs-14}}
\addcontentsline{toc}{paragraph}{2. Maksud dan Tujuan TKRS 14}

Manajemen risiko adalah proses yang proaktif dan berkesinambungan meliputi identifikasi, analisis, evaluasi, pengendalian, informasi komunikasi, pemantauan, dan pelaporan risiko, termasuk berbagai strategi yang dijalankan untuk mengelola risiko dan potensinya. Tujuan penerapan manajemen risiko untuk mencegah terjadinya cedera dan kerugian di rumah sakit. Rumah sakit perlu menerapkan manajemen risiko dan rencana penanganan risiko untuk memitigasi dan mengurangi risiko bahaya yang ada atau mungkin terjadi.

Beberapa kategori risiko yang harus diidentifikasi meliputi namun tidak terbatas pada risiko:

\begin{enumerate}
\def\labelenumi{\alph{enumi}.}
\item
  Operasional adalah risiko yang terjadi saat rumah sakit memberikan pelayanan kepada pasien baik klinis maupun non klinis.
  Risiko klinis yaitu risiko operasional yang terkait dengan pelayanan kepada pasien (keselamatan pasien) meliputi risiko yang berhubungan dengan perawatan klinis dan pelayanan penunjang seperti kesalahan diagnostik, bedah atau pengobatan.
  Risiko non klinis yang juga termasuk risiko operasional adalah risiko PPI (terkait pengendalian dan pencegahan infeksi misalnya sterilisasi, laundry, gizi, kamar jenazah dan lain-lainnya), risiko MFK (terkait dengan fasilitas dan lingkungan, seperti kondisi bangunan yang membahayakan, risiko yang terkait dengan ketersediaan sumber air dan listrik, dan lain lain. Unit klinis maupun non klinis dapat memiliki risiko yang lain sesuai dengan proses bisnis/kegiatan yang dilakukan di unitnya. Misalnya unit humas dapat mengidentifikasi risiko reputasi dan risiko keuangan;
\item
  Risiko keuangan; risiko kepatuhan (terhadap hukum dan peraturan yang berlaku);
\item
  Risiko reputasi (citra rumah sakit yang dirasakan oleh masyarakat);
\item
  Risiko strategis (terkait dengan rencana strategis termasuk tujuan strategis rumah sakit); dan
\item
  Risiko kepatuhan terhadap hukum dan regulasi.
\end{enumerate}

Proses manajemen risiko yang diterapkan di rumah sakit meliputi:

\begin{enumerate}
\def\labelenumi{\alph{enumi}.}
\tightlist
\item
  Komunikasi dan konsultasi.
\item
  Menetapkan konteks.
\item
  Identifikasi risiko sesuai kategori risiko pada poin a) - e)
\item
  Analisis risiko.
\item
  Evaluasi risiko.
\item
  Penanganan risiko.
\item
  Pemantauan risiko.
\end{enumerate}

Program manajemen risiko rumah sakit harus disusun setiap tahun berdasarkan daftar risiko yang diprioritaskan dalam profil risiko meliputi:

\begin{enumerate}
\def\labelenumi{\alph{enumi}.}
\tightlist
\item
  Proses manajemen risiko (poin a)-g)).
\item
  Integrasi manajemen risiko di rumah sakit.
\item
  Pelaporan kegiatan program manajemen risiko.
\item
  Pengelolaan klaim tuntunan yang dapat menyebabkan tuntutan.
\end{enumerate}

\hypertarget{elemen-penilaian-tkrs-14}{%
\paragraph*{3. Elemen Penilaian TKRS 14}\label{elemen-penilaian-tkrs-14}}
\addcontentsline{toc}{paragraph}{3. Elemen Penilaian TKRS 14}

\begin{enumerate}
\def\labelenumi{\alph{enumi}.}
\tightlist
\item
  Direktur dan pimpinan rumah sakit berpartisipasi dan menetapkan program manajemen risiko tingkat rumah sakit meliputi poin a) sampai dengan d) dalam maksud dan tujuan.
\item
  Direktur memantau penyusunan daftar risiko yang diprioritaskan menjadi profil risiko di tingkat rumah sakit.
\end{enumerate}

\hypertarget{l.-program-penelitian-bersubjek-manusia-di-rumah-sakit}{%
\subsection*{l. Program Penelitian Bersubjek Manusia Di Rumah Sakit}\label{l.-program-penelitian-bersubjek-manusia-di-rumah-sakit}}
\addcontentsline{toc}{subsection}{l. Program Penelitian Bersubjek Manusia Di Rumah Sakit}

\hypertarget{standar-tkrs-15}{%
\paragraph*{1. Standar TKRS 15}\label{standar-tkrs-15}}
\addcontentsline{toc}{paragraph}{1. Standar TKRS 15}

Pimpinan rumah sakit bertanggung jawab terhadap mutu dan keamanan dalam program penelitian bersubjek manusia.

\hypertarget{maksud-dan-tujuan-tkrs-15}{%
\paragraph*{2. Maksud dan Tujuan TKRS 15}\label{maksud-dan-tujuan-tkrs-15}}
\addcontentsline{toc}{paragraph}{2. Maksud dan Tujuan TKRS 15}

Penelitian bersubjek manusia merupakan sebuah proses yang kompleks dan signifikan bagi rumah sakit. Direktur menetapkan penanggung jawab penelitian di rumah sakit untuk melakukan pemantauan proses penelitian di rumah sakit (mis. Komite penelitian). Pimpinan rumah sakit harus memiliki komitmen yang diperlukan untuk menjalankan penelitian dan pada saat yang bersamaan melindungi pasien yang telah setuju untuk mengikuti proses pengobatan dan atau diagnostik dalam penelitian.

Komitmen pemimpin unit terhadap penelitian dengan subjek manusia tidak terpisah dari komitmen mereka terhadap perawatan pasien-komitmen terintegrasi di semua tingkatan. Dengan demikian, pertimbangan etis, komunikasi yang baik, pemimpin unit dan layanan yang bertanggung jawab, kepatuhan terhadap peraturan, dan sumber daya keuangan dan non-keuangan merupakan komponen dari komitmen ini. Pimpinan rumah sakit mengakui kewajiban untuk melindungi pasien terlepas dari sponsor penelitian.

\hypertarget{elemen-penilaian-tkrs-15}{%
\paragraph*{3. Elemen Penilaian TKRS 15}\label{elemen-penilaian-tkrs-15}}
\addcontentsline{toc}{paragraph}{3. Elemen Penilaian TKRS 15}

\begin{enumerate}
\def\labelenumi{\alph{enumi}.}
\tightlist
\item
  Pimpinan rumah sakit menetapkan penanggung jawab program penelitian di dalam rumah sakit yang memastikan semua proses telah sesuai dengan kode etik penelitian dan persyaratan lainnya sesuai peraturan perundang-undangan.
\item
  Terdapat proses untuk menyelesaian konflik kepentingan (finansial dan non finansial) yang terjadi akibat penelitian di rumah sakit.
\item
  Pimpinan rumah sakit telah mengidentifikasi fasilitas dan sumber daya yang diperlukan untuk melakukan penelitian, termasuk di dalam nya kompetensi sumber daya yang akan berpartisipasi di dalam penelitian sebagai pimpinan dan anggota tim peneliti.
\item
  Terdapat proses yang memastikan bahwa seluruh pasien yang ikut di dalam penelitian telah melalui proses persetujuan tertulis (informed consent) untuk melakukan penelitian, tanpa adanya paksaan untuk mengikuti penelitian dan telah mendapatkan informasi mengenai lamanya penelitian, prosedur yang harus dilalui, siapa yang dapat dikontak selama penelitian berlangsung, manfaat, potensial risiko serta alternatif pengobatan lainnya.
\item
  Apabila penelitian dilakukan oleh pihak ketiga (kontrak), maka pimpinan rumah sakit memastikan bahwa pihak ketiga tersebut bertanggung jawab dalam pemantauan dan evaluasi dari mutu, keamanan dan etika dalam penelitian.
\item
  Penanggung jawab penelitian melakukan kajian dan evaluasi terhadap seluruh penelitian yang dilakukan di rumah sakit setidaknya 1 (satu) tahun sekali.
\item
  Seluruh kegiatan penelitian merupakan bagian dari program mutu rumah sakit dan dilakukan pemantauan serta evaluasinya secara berkala sesuai ketetapan rumah sakit.
\end{enumerate}

\hypertarget{kualifikasi-dan-pendidikan-staf-kps}{%
\section*{2. Kualifikasi dan Pendidikan Staf (KPS)}\label{kualifikasi-dan-pendidikan-staf-kps}}
\addcontentsline{toc}{section}{2. Kualifikasi dan Pendidikan Staf (KPS)}

\textbf{Gambaran Umum}

Rumah sakit membutuhkan staf yang memiliki keterampilan dan kualifikasi untuk mencapai misinya dan memenuhi kebutuhan pasien. Para pimpinan rumah sakit mengidentifikasi jumlah dan jenis staf yang dibutuhkan berdasarkan rekomendasi dari unit.

Perekrutan, evaluasi, dan pengangkatan staf dilakukan melalui proses yang efisien, dan seragam. Di samping itu perlu dilakukan kredensial kepada tenaga medis, tenaga perawat, dan tenaga kesehatan lainnya, karena mereka secara langsung terlibat dalam proses pelayanan klinis.

Orientasi terhadap rumah sakit, dan orientasi terhadap tugas pekerjaan staf merupakan suatu proses yang penting. Rumah sakit menyelenggarakan program kesehatan dan keselamatan staf untuk memastikan kondisi kerja yang aman, kesehatan fisik dan mental, produktivitas, kepuasan kerja.

Program ini bersifat dinamis, proaktif, dan mencakup hal-hal yang mempengaruhi kesehatan dan kesejahteraan staf seperti pemeriksaan kesehatan kerja saat rekrutmen, pengendalian pajanan kerja yang berbahaya, vaksinasi/imunisasi, cara penanganan pasien yang aman, staf dan kondisi-kondisi umum terkait kerja.
Fokus pada standar ini adalah:

\begin{enumerate}
\def\labelenumi{\alph{enumi}.}
\tightlist
\item
  Perencanaan dan pengelolaan staf;
\item
  Pendidikan dan pelatihan;
\item
  Kesehatan dan keselamatan kerja staf;
\item
  Tenaga medis;
\item
  Tenaga keperawatan; dan
\item
  Tenaga kesehatan lain.
\end{enumerate}

\hypertarget{a.-perencanaan-dan-pengelolaan-staf}{%
\subsection*{a. Perencanaan dan Pengelolaan Staf}\label{a.-perencanaan-dan-pengelolaan-staf}}
\addcontentsline{toc}{subsection}{a. Perencanaan dan Pengelolaan Staf}

\hypertarget{standar-kps-1}{%
\paragraph*{1. Standar KPS 1}\label{standar-kps-1}}
\addcontentsline{toc}{paragraph}{1. Standar KPS 1}

Kepala unit merencanakan dan menetapkan persyaratan pendidikan, keterampilan, pengetahuan, dan persyaratan lainnya bagi semua staf di unitnya sesuai kebutuhan pasien.

\hypertarget{maksud-dan-tujuan-kps-1}{%
\paragraph*{2. Maksud dan Tujuan KPS 1}\label{maksud-dan-tujuan-kps-1}}
\addcontentsline{toc}{paragraph}{2. Maksud dan Tujuan KPS 1}

Kepala unit menetapkan persyaratan pendidikan, kompetensi dan pengalaman setiap staf di unitnya untuk memberikan asuhan kepada pasien. Kepala unit mempertimbangkan faktor berikut ini untuk menghitung kebutuhan staf:

\begin{enumerate}
\def\labelenumi{\alph{enumi}.}
\tightlist
\item
  Misi rumah sakit.
\item
  Populasi pasien yang dilayani dan kompleksitas serta kebutuhan pasien.
\item
  Layanan diagnostik dan klinis yang disediakan rumah sakit.
\item
  Jumlah pasien rawat inap dan rawat jalan.
\item
  Peralatan medis yang digunakan untuk pelayanan pasien.
\end{enumerate}

Rumah sakit mematuhi peraturan dan perundang- undangan tentang syarat pendidikan, keterampilan atau persyaratan lainnya yang dibutuhkan staf.

Perencanaan kebutuhan staf disusun secara kolaboratif oleh kepala unit dengan mengidentifikasi jumlah, jenis, dan kualifikasi staf yang dibutuhkan. Perencanaan tersebut ditinjau secara berkelanjutan dan diperbarui sesuai kebutuhan.
Proses perencanaan menggunakan metode-metode yang diakui sesuai peraturan perundang-undangan. Perencanaan kebutuhan mempertimbangkan hal-hal dibawah ini:

\begin{enumerate}
\def\labelenumi{\alph{enumi}.}
\tightlist
\item
  Terjadi peningkatan jumlah pasien atau kekurangan stad di satu unit sehingga dibutuhkan rotasi staf dari satu unit ke unit lain.
\item
  Pertimbangan permintaan staf untuk rotasi tugas berdasarkan nilai-nilai budaya atau agama dan kepercayaan.
\item
  kepatuhan terhadap peraturan dan perundang- undangan.
  Perencanaan staf, dipantau secara berkala dan diperbarui sesuai kebutuhan.
\end{enumerate}

\hypertarget{elemen-penilaian-kps-1}{%
\paragraph*{3. Elemen Penilaian KPS 1}\label{elemen-penilaian-kps-1}}
\addcontentsline{toc}{paragraph}{3. Elemen Penilaian KPS 1}

\begin{enumerate}
\def\labelenumi{\alph{enumi}.}
\tightlist
\item
  Direktur telah menetapkan regulasi terkait Kualifikasi Pendidikan dan staf meliputi poin a - f pada gambaran umum.
\item
  Kepala unit telah merencanakan dan menetapkan persyaratan pendidikan, kompetensi dan pengalaman staf di unitnya sesuai peraturan dan perundang- undangan.
\item
  Kebutuhan staf telah direncanakan sesuai poin a)-e) dalam maksud dan tujuan.
\item
  Perencanaan staf meliputi penghitungan jumlah, jenis, dan kualifikasi staf menggunakan metode yang diakui sesuai peraturan perundang-undangan.
\item
  Perencanaan staf termasuk membahas penugasan dan rotasi/alih fungsi staf.
\item
  Efektivitas perencanaan staf dipantau secara berkelanjutan dan diperbarui sesuai kebutuhan.
\end{enumerate}

\hypertarget{standar-kps-2}{%
\paragraph*{4. Standar KPS 2}\label{standar-kps-2}}
\addcontentsline{toc}{paragraph}{4. Standar KPS 2}

Tanggung jawab tiap staf dituangkan dalam uraian tugas

\hypertarget{maksud-dan-tujuan-kps-2}{%
\paragraph*{5. Maksud dan Tujuan KPS 2}\label{maksud-dan-tujuan-kps-2}}
\addcontentsline{toc}{paragraph}{5. Maksud dan Tujuan KPS 2}

Setiap staf yang bekerja di rumah sakit harus mempunyai uraian tugas. Pelaksanaan tugas, orientasi, dan evaluasi kinerja staf didasarkan pada uraian tugasnya.
Uraian tugas juga dibutuhkan untuk tenaga kesehatan jika:

\begin{enumerate}
\def\labelenumi{\alph{enumi}.}
\tightlist
\item
  Tenaga kesehatan ditugaskan di bidang manajerial, misalnya kepala bidang, kepala unit.
\item
  Tenaga kesehatan melakukan dua tugas yaitu di bidang manajerial dan di bidang klinis, misalnya dokter spesialis bedah melakukan tugas manajerialnya sebagai kepala kamar operasi maka harus mempunyai uraian tugas sedangkan tugas klinisnya sebagai dokter spesialis bedah harus mempunyai Surat Penugasan Klinis (SPK) dan Rincian Kewenangan Klinis (RKK).
\item
  Tenaga kesehatan yang sedang mengikuti pendidikan dan bekerja dibawah supervisi, maka program pendidikan menentukan batasan kewenangan apa yang boleh dan apa yang tidak boleh dikerjakan sesuai dengan tingkat pendidikannya.
\item
  Tenaga kesehatan yang diizinkan untuk memberikan pelayanan sementara dirumah sakit; misalnya, perawat paruhwaktu yang membantu dokter di poliklinik.
\end{enumerate}

Uraian tugas untuk standar ini berlaku bagi semua staf baik staf purnawaktu, staf paruhwaktu, tenaga sukarela, atau sementara yang membutuhkan

\hypertarget{elemen-penilaian-kps-2}{%
\paragraph*{6. Elemen Penilaian KPS 2}\label{elemen-penilaian-kps-2}}
\addcontentsline{toc}{paragraph}{6. Elemen Penilaian KPS 2}

\begin{enumerate}
\def\labelenumi{\alph{enumi}.}
\tightlist
\item
  Setiap staf telah memiliki uraian tugas sesuai dengan tugas yang diberikan.
\item
  Tenaga kesehatan yang diidentifikasi dalam a) hingga d) dalam maksud dan tujuan, memiliki uraian tugas yang sesuai dengan tugas dan tanggung jawabnya.
\end{enumerate}

\hypertarget{standar-kps-3}{%
\paragraph*{7. Standar KPS 3}\label{standar-kps-3}}
\addcontentsline{toc}{paragraph}{7. Standar KPS 3}

Rumah sakit menyusun dan menerapkan proses rekrutmen, evaluasi, dan pengangkatan staf serta prosedur- prosedur terkait lainnya.

\hypertarget{maksud-dan-tujuan-kps-3}{%
\paragraph*{8. Maksud dan Tujuan KPS 3}\label{maksud-dan-tujuan-kps-3}}
\addcontentsline{toc}{paragraph}{8. Maksud dan Tujuan KPS 3}

Rumah sakit menetapkan proses yang terpusat, efisien dan terkoordinasi, agar terlaksana proses yang seragam mencakup:

\begin{enumerate}
\def\labelenumi{\alph{enumi}.}
\tightlist
\item
  Rekrutmen staf sesuai kebutuhan rumah sakit.
\item
  Evaluasi kompetensi kandidat calon staf.
\item
  Pengangkatan staf baru.
\end{enumerate}

Kepala unit berpartisipasi merekomendasikan jumlah dan kualifikasi staf serta jabatan nonklinis yang dibutuhkan untuk memberikan pelayanan pada pasien, pendidikan, penelitian ataupun tanggung jawab lainnya.

\hypertarget{elemen-penilaian-kps-3}{%
\paragraph*{9. Elemen Penilaian KPS 3}\label{elemen-penilaian-kps-3}}
\addcontentsline{toc}{paragraph}{9. Elemen Penilaian KPS 3}

\begin{enumerate}
\def\labelenumi{\alph{enumi}.}
\tightlist
\item
  Rumah sakit telah menetapkan regulasi terkait proses rekrutmen, evaluasi kompetensi kandidat calon staf dan mekanisme pengangkatan staf di rumah sakit.
\item
  Rumah sakit telah menerapkan proses meliputi poin a) -- c) di maksud dan tujuan secara seragam.
\end{enumerate}

\hypertarget{standar-kps-4}{%
\paragraph*{10. Standar KPS 4}\label{standar-kps-4}}
\addcontentsline{toc}{paragraph}{10. Standar KPS 4}

Rumah sakit menetapkan proses untuk memastikan bahwa kompetensi Profesional Pemberi Asuhan (PPA) sesuai dengan persyaratan jabatan atau tanggung jawabnya untuk memenuhi kebutuhan rumah sakit

\hypertarget{maksud-dan-tujuan-kps-4}{%
\paragraph*{11. Maksud dan Tujuan KPS 4}\label{maksud-dan-tujuan-kps-4}}
\addcontentsline{toc}{paragraph}{11. Maksud dan Tujuan KPS 4}

\begin{enumerate}
\def\labelenumi{\alph{enumi}.}
\item
  Staf yang direkrut rumah sakit melalui proses untuk menyesuaikan dengan persyaratan jabatan/posisi staf. Untuk para PPA, proses tersebut meliputi penilaian kompetensi awal untuk memastikan apakah PPA dapat melakukan tanggung jawab sesuai uraian tugasnya. Penilaian dilakukan sebelum atau saat mulai bertugas. Rumah sakit dapat menetapkan kontrak kerja sebagai masa percobaan untuk mengawasi dan mengevaluasi PPA tersebut. Ada proses untuk memastikan bahwa PPA yang memberikan pelayanan berisiko tinggi atau perawatan bagi pasien berisiko tinggi dievaluasi pada saat mereka mulai memberikan perawatan, sebelum masa percobaan atau orientasi selesai.
\item
  Penilaian kompetensi awal dilakukan oleh unit di mana PPA tersebut ditugaskan
  Penilaian kompetensi yang diinginkan juga mencakup penilaian kemampuan PPA untuk mengoperasikan alat medis, alarm klinis, dan mengawasi pengelolaan obat- obatan yang sesuai dengan area tempat ia akan bekerja (misalnya, PPA yang bekerja di unit perawatan intensif harus dapat mengoperasikan ventilator pompa infus, dan lain-lainnya, dan sedangkan PPA yang bekerja di unit obstetri harus dapat menggunakan alat pemantauan janin).
\end{enumerate}

c.Rumah sakit menetapkan proses evaluasi kemampuan PPA dan frekuensi evaluasi secara berkesinambungan.

Penilaian yang berkesinambungan dapat digunakan untuk menentukan rencana pelatihan sesuai kebutuhan, kemampuan staf untuk memikul tanggung jawab baru atau untuk melakukan perubahan tanggung jawab dari PPA tersebut. Sekurang-kurangnya terdapat satu penilaian terkait uraian tugas tiap PPA yang didokumentasikan setiap tahunnya.

\hypertarget{elemen-penilaian-kps-4}{%
\paragraph*{12. Elemen Penilaian KPS 4}\label{elemen-penilaian-kps-4}}
\addcontentsline{toc}{paragraph}{12. Elemen Penilaian KPS 4}

\begin{enumerate}
\def\labelenumi{\alph{enumi}.}
\tightlist
\item
  Rumah sakit telah menetapkan dan menerapkan proses untuk menyesuaikan kompetensi PPA dengan kebutuhan pasien.
\item
  Para PPA baru dievaluasi pada saat mulai bekerja oleh kepala unit di mana PPA tersebut ditugaskan
\item
  Terdapat setidaknya satu atau lebih evaluasi yang didokumentasikan untuk tiap PPA sesuai uraian tugas setiap tahunnya atau sesuai ketentuan rumah sakit.
\end{enumerate}

\hypertarget{standar-kps-5}{%
\paragraph*{13. Standar KPS 5}\label{standar-kps-5}}
\addcontentsline{toc}{paragraph}{13. Standar KPS 5}

Rumah sakit menetapkan proses untuk memastikan bahwa kompetensi staf nonklinis sesuai dengan persyaratan jabatan/posisinya untuk memenuhi kebutuhan rumah sakit.

\hypertarget{maksud-dan-tujuan-kps-5}{%
\paragraph*{14. Maksud dan Tujuan KPS 5}\label{maksud-dan-tujuan-kps-5}}
\addcontentsline{toc}{paragraph}{14. Maksud dan Tujuan KPS 5}

Rumah sakit mengidentifikasi dan mencari staf yang memenuhi persyaratan jabatan/posisi nonklinis. Staf nonklinis diberikan orientasiuntuk memastikan bahwa staf tersebut melakukan tanggung jawabnya sesuai uraian tugasnya. Rumah Sakit melakukan pengawasan dan evaluasi secara berkala untuk memastikan kompetensi secara terus menerus pada jabatan/posisi tersebut.

\hypertarget{elemen-penilaian-kps-5}{%
\paragraph*{15. Elemen Penilaian KPS 5}\label{elemen-penilaian-kps-5}}
\addcontentsline{toc}{paragraph}{15. Elemen Penilaian KPS 5}

\begin{enumerate}
\def\labelenumi{\alph{enumi}.}
\tightlist
\item
  Rumah sakit telah menetapkan dan menerapkan proses untuk menyesuaikan kompetensi staf non klinis dengan persyaratan jabatan/posisi.
\item
  Staf non klinis yang baru dinilai kinerjanya pada saat akan memulai pekerjaannya oleh kepala unit di mana staf tersebut ditugaskan.
\item
  Terdapat setidaknya satu atau lebih evaluasi yang didokumentasikan untuk tiap staf non klinis sesuai uraian tugas setiap tahunnya atau sesuai ketentuan rumah sakit.
\end{enumerate}

\hypertarget{standar-kps-6}{%
\paragraph*{16. Standar KPS 6}\label{standar-kps-6}}
\addcontentsline{toc}{paragraph}{16. Standar KPS 6}

Terdapat informasi kepegawaian yang terdokumentasi dalam file kepegawaian setiap staf.

\hypertarget{maksud-dan-tujuan-kps-6}{%
\paragraph*{17. Maksud dan Tujuan KPS 6}\label{maksud-dan-tujuan-kps-6}}
\addcontentsline{toc}{paragraph}{17. Maksud dan Tujuan KPS 6}

File kepegawaian yang terkini berisikan dokumentasi setiap staf rumah sakit yang mengandung informasi sensitif yang harus dijaga kerahasiaannya. File kepegawaian memuat:

\begin{enumerate}
\def\labelenumi{\alph{enumi}.}
\tightlist
\item
  Pendidikan, kualifikasi, keterampilan, dan kompetensi staf.
\item
  Bukti orientasi.
\item
  Uraian tugas staf.
\item
  Riwayat pekerjaan staf.
\item
  Penilaian kinerja staf.
\item
  Salinan sertifikat pelatihan di dalam maupun di luar rumah sakit yang telah diikuti.
\item
  Informasi kesehatan yang dipersyaratkan, seperti vaksinasi/imunisasi, hasil medical check up.
\end{enumerate}

File kepegawaian tersebut distandardisasi dan terus diperbarui sesuai dengan kebijakan rumah sakit.

\hypertarget{elemen-penilaian-kps-6}{%
\paragraph*{18. Elemen Penilaian KPS 6}\label{elemen-penilaian-kps-6}}
\addcontentsline{toc}{paragraph}{18. Elemen Penilaian KPS 6}

\begin{enumerate}
\def\labelenumi{\alph{enumi}.}
\tightlist
\item
  File kepegawaian staf distandardisasi dan dipelihara serta dijaga kerahasiaannya sesuai dengan kebijakan rumah sakit.
\item
  File kepegawaian mencakup poin a) ? g) sesuai maksud dan tujuan.
\end{enumerate}

\hypertarget{standar-kps-7}{%
\paragraph*{1. Standar KPS 7}\label{standar-kps-7}}
\addcontentsline{toc}{paragraph}{1. Standar KPS 7}

Semua staf diberikan orientasi mengenai rumah sakit dan unit tempat mereka ditugaskan dan tanggung jawab pekerjaannya pada saat pengangkatan staf.

\hypertarget{maksud-dan-tujuan-kps-7}{%
\paragraph*{2. Maksud dan Tujuan KPS 7}\label{maksud-dan-tujuan-kps-7}}
\addcontentsline{toc}{paragraph}{2. Maksud dan Tujuan KPS 7}

Keputusan pengangkatan staf melalui sejumlah tahapan. Pemahaman terhadap rumah sakit secara keseluruhan dan tanggung jawab klinis maupun nonklinis berperan dalam tercapainya misi rumah sakit. Hal ini dapat dicapai melalui orientasi kepada staf.
Orientasi umum meliputi informasi tentang rumah sakit, program mutu dan keselamatan pasien, serta program pencegahan dan pengendalian infeksi.

Orientasi khusus meliputi tugas dan tanggung jawab dalam melakukan pekerjaannya. Hasil orientasi ini dicatat dalam file kepegawaian. Staf paruh waktu, sukarelawan, dan mahasiswa atau trainee juga diberikan orientasi umum dan orientasi khusus

\hypertarget{elemen-penilaian-kps-7}{%
\paragraph*{3. Elemen Penilaian KPS 7}\label{elemen-penilaian-kps-7}}
\addcontentsline{toc}{paragraph}{3. Elemen Penilaian KPS 7}

\begin{enumerate}
\def\labelenumi{\alph{enumi}.}
\tightlist
\item
  Rumah sakit telah menetapkan regulasi tentang orientasi bagi staf baru di rumah sakit.
\item
  Tenaga kesehatan baru telah diberikan orientasi umum dan orientasi khusus sesuai.
\item
  Staf nonklinis baru telah diberikan orientasi umum dan orientasi khusus.
\item
  Tenaga kontrak, paruh waktu, mahasiswa atau trainee dan sukarelawan telah diberikan orientasi umum dan orientasi khusus (jika ada).
\end{enumerate}

\hypertarget{b.-pendidikan-dan-pelatihan}{%
\subsection*{b. Pendidikan dan Pelatihan}\label{b.-pendidikan-dan-pelatihan}}
\addcontentsline{toc}{subsection}{b. Pendidikan dan Pelatihan}

\hypertarget{standar-kps-8}{%
\paragraph*{1. Standar KPS 8}\label{standar-kps-8}}
\addcontentsline{toc}{paragraph}{1. Standar KPS 8}

Tiap staf diberikan pendidikan dan pelatihan yang berkelanjutan untuk mendukung atau meningkatkan keterampilan dan pengetahuannya.

\hypertarget{maksud-dan-tujuan-kps-8}{%
\paragraph*{2. Maksud dan Tujuan KPS 8}\label{maksud-dan-tujuan-kps-8}}
\addcontentsline{toc}{paragraph}{2. Maksud dan Tujuan KPS 8}

Rumah sakit mengumpulkan data dari berbagai sumber dalam penyusunan Program pendidikan dan pelatihan untuk memenuhi kebutuhan pasien dan/atau memenuhi persyaratan pendidikan berkelanjutan. Sumber informasi untuk menentukan kebutuhan pendidikan staf mencakup:

\begin{enumerate}
\def\labelenumi{\alph{enumi}.}
\tightlist
\item
  Hasil kegiatan pengukuran data mutu dan keselamatan pasien.
\item
  Hasil analisislaporan insiden keselamatan pasien.
\item
  Hasil survei budaya keselamatan pasien.
\item
  Hasil pemantauan program manajemen fasilitas dan keselamatan.
\item
  Pengenalan teknologi termasuk penambahan peralatan medis baru, keterampilan dan pengetahuan baru yang diperoleh dari penilaian kinerja.
\item
  Prosedur klinis baru.
\item
  Rencana untuk menyediakan layanan baru di masa yang akan datang.
\item
  Kebutuhan dan usulan dari setiap unit.
\end{enumerate}

Rumah sakit memiliki suatu proses untuk mengumpulkan dan mengintegrasikan data dari berbagai sumber untuk merencanakan program pendidikan dan pelatihan staf. Selain itu, rumah sakit menentukan staf mana yang diharuskan untuk mendapatkan pendidikan berkelanjutan untuk menjaga kemampuan mereka dan bagaimana pendidikan staf tersebut akan dipantau dan didokumentasikan.

Pimpinan rumah sakit meningkatkan dan mempertahankan kinerja staf dengan mendukung program pendidikan dan pelatihan termasuk menyediakan sarana prasarana termasuk peralatan, ruangan, tenaga pengajar, dan waktu. Program pendidikan dan pelatihan dibuat setiap tahun untuk memenuhi kebutuhan pasien dan/atau memenuhi persyaratan pendidikan berkelanjutan misalnya tenaga medis diberikan pelatihan PPI, perkembangan praktik medis, atau peralatan medis baru. Hasil pendidikan dan pelatihan staf didokumentasikan dalam file kepegawaian.

Ketersediaan teknologi dan informasi ilmiah yang aktual tersedia untuk mendukung pendidikan dan pelatihan disediakan di satu atau beberapa lokasi yang yang tersebar di rumah sakit. Pelatihan diatur sedemikian rupa agar tidak mengganggu pelayanan pasien.

\hypertarget{elemen-penilaian-kps-8}{%
\paragraph*{3. Elemen Penilaian KPS 8}\label{elemen-penilaian-kps-8}}
\addcontentsline{toc}{paragraph}{3. Elemen Penilaian KPS 8}

\begin{enumerate}
\def\labelenumi{\alph{enumi}.}
\tightlist
\item
  Rumah sakit telah mengidentifikasi kebutuhan pendidikan staf berdasarkan sumber berbagai informasi, mencakup a) ? h) dalam maksud dan tujuan.
\item
  Program pendidikan dan pelatihan telah disusun berdasarkan hasil identifikasi sumber informasi pada EP 1.
\item
  Pendidikan dan pelatihan berkelanjutan diberikan kepada staf rumah sakit baik internal maupun eksternal.
\item
  Rumah sakit telah menyediakan waktu, anggaran, sarana dan prasarana yang memadai bagi semua staf untuk mendapat kesempatan mengikuti pendidikan dan pelatihan yang dibutuhkan.
\end{enumerate}

\hypertarget{standar-kps-8.1}{%
\paragraph*{4. Standar KPS 8.1}\label{standar-kps-8.1}}
\addcontentsline{toc}{paragraph}{4. Standar KPS 8.1}

Staf yang memberikan asuhan pasien dan staf yang ditentukan rumah sakit dilatih dan dapat mendemonstrasikan teknik resusitasi jantung paru dengan benar.

\hypertarget{maksud-dan-tujuan-kps-8.1}{%
\paragraph*{5. Maksud dan Tujuan KPS 8.1}\label{maksud-dan-tujuan-kps-8.1}}
\addcontentsline{toc}{paragraph}{5. Maksud dan Tujuan KPS 8.1}

Semua staf yang merawat pasien, termasuk dokter dan staf lain yang ditentukan rumah sakit telah diberikan pelatihan teknik resusitasi dasar. Rumah sakit menentukan pelatihan Bantuan Hidup Dasar (BHD) atau bantuan hidup tingkat lanjut untuk setiap staf, sesuai dengan tugas dan perannya di rumah sakit. Misalnya rumah sakit menentukan semua staf yang merawat pasien di unit gawat darurat, di unit perawatan intensif, semua staf yang akan melaksanakan dan memantau prosedur sedasi prosedural serta tim kode biru (code blue) harus mendapatkan pelatihan sampai bantuan hidup tingkat lanjut. Rumah sakit juga menentukan bahwa staf lain yang tidak merawat pasien, seperti pekarya atau staf registrasi, harus mendapatkan pelatihan bantuan hidup dasar.

Tingkat pelatihan bagi staf tersebut harus diulang berdasarkan persyaratan dan/atau jangka waktu yang diidentifikasi oleh program pelatihan yang diakui, atau setiap 2 (dua) tahun jika tidak menggunakan program pelatihan yang diakui. Terdapat bukti yang menunjukkan bahwa tiap anggota staf yang menghadiri pelatihan benar- benar memenuhi tingkat kompetensi yang diinginkan.

\hypertarget{elemen-penilaian-kps-8.1}{%
\paragraph*{6. Elemen Penilaian KPS 8.1}\label{elemen-penilaian-kps-8.1}}
\addcontentsline{toc}{paragraph}{6. Elemen Penilaian KPS 8.1}

\begin{enumerate}
\def\labelenumi{\alph{enumi}.}
\tightlist
\item
  Rumah sakit telah menetapkan pelatihan teknik resusitasi jantung paru tingkat dasar (BHD) pada seluruh staf dan bantuan hidup tingkat lanjut bagi staf yang ditentukan oleh rumah sakit.
\item
  Terdapat bukti yang menunjukkan bahwa staf yang mengikuti pelatihan BHD atau bantuan hidup tingkat lanjut telah lulus pelatihan tersebut.
\item
  Tingkat pelatihan yang ditentukan untuk tiap staf harus diulang berdasarkan persyaratan dan/atau jangka waktu yang ditetapkan oleh program pelatihan yang diakui, atau setiap 2 (dua) tahun jika tidak menggunakan program pelatihan yang diakui.
\end{enumerate}

\hypertarget{c.-kesehatan-dan-keselamatan-kerja-staf}{%
\subsection*{c.~Kesehatan dan Keselamatan Kerja Staf}\label{c.-kesehatan-dan-keselamatan-kerja-staf}}
\addcontentsline{toc}{subsection}{c.~Kesehatan dan Keselamatan Kerja Staf}

\hypertarget{standar-kps-9}{%
\paragraph*{1. Standar KPS 9}\label{standar-kps-9}}
\addcontentsline{toc}{paragraph}{1. Standar KPS 9}

Rumah sakit menyelenggarakan pelayanan kesehatan dan keselamatan staf.

\hypertarget{maksud-dan-tujuan-kps-9}{%
\paragraph*{2. Maksud dan Tujuan KPS 9}\label{maksud-dan-tujuan-kps-9}}
\addcontentsline{toc}{paragraph}{2. Maksud dan Tujuan KPS 9}

Staf rumah sakit mempunyai risiko terpapar infeksi karena pekerjaannya yang berhubungan baik secara langsung dan maupun tidak langsung dengan pasien. Pelayanan kesehatan dan keselamatan staf merupakan hal penting untuk menjaga kesehatan fisik, kesehatan mental, kepuasan, produktivitas, dan keselamatan staf dalam bekerja. Karena hubungan staf dengan pasien dan kontak dengan bahan infeksius maka banyak petugas kesehatan berisiko terpapar penularan infeksi. Identifikasi sumber infeksi berdasar atas epidemiologi sangat penting untuk menemukan staf yang berisiko terpapar infeksi. Pelaksanaan program pencegahan serta skrining seperti imunisasi, vaksinasi, dan profilaksis dapat menurunkan insiden infeksi penyakit menular secara signifikan.

Staf rumah sakit juga dapat mengalami kekerasan di tempat kerja. Anggapan bahwa kekerasan tidak terjadi di rumah sakit tidak sepenuhnya benar mengingat jumlah tindak kekerasan di rumah sakit semakin meningkat. Untuk itu rumah sakit diminta menyusun program pencegahan kekerasan.

Cara rumah sakit melakukan orientasi dan pelatihan staf, penyediaan lingkungan kerja yang aman, pemeliharaan peralatan dan teknologi medis, pencegahan atau pengendalian infeksi terkait perawatan kesehatan (Health care-Associated Infections), serta beberapa faktor lainnya menentukan kesehatan dan kesejahteraan staf.
Dalam pelaksanaan program kesehatan dan keselamatan staf rumah sakit, maka staf harus memahami:

\begin{enumerate}
\def\labelenumi{\alph{enumi}.}
\tightlist
\item
  Cara pelaporan dan mendapatkan pengobatan, menerima konseling, dan menangani cedera yang mungkin terjadi akibat tertusuk jarum suntik, terpapar penyakit menular, atau mendapat kekerasan di tempat kerja;
\item
  Identifikasi risiko dan kondisi berbahaya di rumah sakit; dan
\item
  Masalah kesehatan dan keselamatan lainnya.
\end{enumerate}

Program kesehatan dan keselamatan staf rumah sakit tersebut mencakup hal-hal sebagai berikut:

\begin{enumerate}
\def\labelenumi{\alph{enumi}.}
\tightlist
\item
  Skrining kesehatan awal
\item
  Tindakan-tindakan untuk mengendalikan pajanan kerja yang berbahaya, seperti pajanan terhadap obat- obatan beracun dan c.~tingkat kebisingan yang berbahaya
\item
  Pendidikan, pelatihan, dan intervensi terkait cara pemberian asuhan pasien yang aman
\item
  Pendidikan, pelatihan, dan intervensi terkait pengelolaan kekerasan di tempat kerja
\item
  Pendidikan, pelatihan, dan intervensi terhadap staf yang berpotensi melakukan kejadian tidak diharapkan (KTD) atau f.~kejadian sentinel
\item
  Tata laksana kondisi terkait pekerjaan yang umum dijumpai seperti cedera punggung atau cedera lain yang lebih darurat.
\item
  Vaksinasi/Imunisasi pencegahan, dan pemeriksaan kesehatan berkala.
\item
  Pengelolaan kesehatan mental staf, seperti pada saat kondisi kedaruratan penyakit infeksi/pandemi.
\end{enumerate}

Penyusunan program mempertimbangkan masukan dari staf serta penggunaan sumber daya klinis yang ada di rumah sakit dan di masyarakat.

\hypertarget{elemen-penilaian-kps-9}{%
\paragraph*{3. Elemen Penilaian KPS 9}\label{elemen-penilaian-kps-9}}
\addcontentsline{toc}{paragraph}{3. Elemen Penilaian KPS 9}

\begin{enumerate}
\def\labelenumi{\alph{enumi}.}
\tightlist
\item
  Rumah sakit telah menetapkan program kesehatan dan keselamatan staf.
\item
  Program kesehatan dan keselamatan staf mencakup setidaknya a) hingga h) yang tercantum dalam maksud dan tujuan.
\item
  Rumah sakit mengidentifikasi penularan penyakit infeksi atau paparan yang dapat terjadi pada staf serta melakukan upaya pencegahan dengan vaksinasi.
\item
  Berdasar atas epidemologi penyakit infeksi maka rumah sakit mengidentifikasi risiko staf terpapar atau tertular serta melaksanakan pemeriksaan kesehatan dan vaksinasi.
\item
  Rumah sakit telah melaksanakan evaluasi, konseling, dan tata laksana lebih lanjut untuk staf yang terpapar penyakit infeksi serta dikoordinasikan dengan program pencegahan dan pengendalian infeksi.
\item
  Rumah sakit telah mengidentifikasi area yang berpotensi untuk terjadi tindakan kekerasan di tempat kerja (workplace violence) dan menerapkan upaya untuk mengurangi risiko tersebut.
\item
  Rumah sakit telah melaksanakan evaluasi, konseling, dan tata laksana lebih lanjut untuk staf yang mengalami cedera akibat tindakan kekerasan di tempat kerja.
\end{enumerate}

\hypertarget{d.-tenaga-medis}{%
\subsection*{d.~Tenaga medis}\label{d.-tenaga-medis}}
\addcontentsline{toc}{subsection}{d.~Tenaga medis}

\hypertarget{standar-kps-10}{%
\paragraph*{1. Standar KPS 10}\label{standar-kps-10}}
\addcontentsline{toc}{paragraph}{1. Standar KPS 10}

Rumah sakit menyelenggarakan proses kredensial yang seragam dan transparan bagi tenaga medis yang diberi izin memberikan asuhan kepada pasien secara mandiri.

\hypertarget{standar-kps-10.1}{%
\paragraph*{2. Standar KPS 10.1}\label{standar-kps-10.1}}
\addcontentsline{toc}{paragraph}{2. Standar KPS 10.1}

Rumah sakit melaksanakan verifikasi terkini terhadap pendidikan, registrasi/izin, pengalaman, dan lainnya dalam proses kredensialing tenaga medis.

\hypertarget{maksud-dan-tujuan-kps-10-sampai-kps-10.1}{%
\paragraph*{3. Maksud dan Tujuan KPS 10 sampai KPS 10.1}\label{maksud-dan-tujuan-kps-10-sampai-kps-10.1}}
\addcontentsline{toc}{paragraph}{3. Maksud dan Tujuan KPS 10 sampai KPS 10.1}

Penjelasan mengenai istilah dan ekspektasi yang ditemukan dalam standar-standar ini adalah sebagai berikut:

\begin{enumerate}
\def\labelenumi{\alph{enumi}.}
\item
  Kredensial adalah proses evaluasi (memeriksa dokumen dari pelamar), wawancara, dan ketentuan lain sesuai dengan kebutuhan rumah sakit yang dilakukan rumah sakit terhadap seorang tenaga medis untuk menentukan apakah yang bersangkutan layak diberi penugasan klinis dan kewenangan klinis untuk menjalankan asuhan/tindakan medis tertentu di lingkungan rumah sakit tersebut untuk periode tertentu. Dokumen kredensial adalah dokumen yang dikeluarkan oleh badan resmi untuk menunjukkan bukti telah dipenuhinya persyaratan seperti ijazah dari fakultas kedokteran, surat tanda registrasi, izin praktik, fellowship, atau bukti pendidikan dan pelatihan yang telah mendapat pengakuan dari organisasi profesi kedokteran. Dokumen dokumen ini harus diverifikasi ke sumber utama yang mengeluarkan dokumen tersebut atau website verifikasi ijazah Kementerian Pendidikan, Kebudayaan, Riset dan Teknologi.
  Dokumen kredensial dapat juga diperoleh dari rumah sakit, perorangan, badan hukum yang terkait dengan riwayat profesional, atau riwayat kompetensi dari pelamar seperti surat rekomendasi, semua riwayat pekerjaan sebagai tenaga medis di tempat kerja yang lalu,catatan asuhan klinis yang lalu, riwayat kesehatan, dan foto. Dokumen ini akan diminta rumah sakit sebagai bagian dari proses kredensial dan ijazah serta STR harus diverifikasi ke sumber utamanya. Syarat untuk verifikasi kredensial disesuaikan dengan posisi pelamar. Sebagai contoh, pelamar untuk kedudukan kepala departemen/unit layanan di rumah sakit dapat diminta verifikasi terkait jabatan dan pengalaman administrasi di masa lalu. Juga untuk posisi tenaga medis di rumah sakit dapat diminta verifikasi riwayat pengalaman kerja beberapa tahun yang lalu.
\item
  Tenaga medis adalah semua dokter dan dokter gigi yang memberikan layanan promotif, preventif, kuratif, rehabilitatif, bedah, atau layanan medis/gigi lain kepada pasien, atau yang memberikan layanan interpretatif terkait pasien seperti patologi, radiologi, laboratorium, serta memiliki surat tanda registrasi (STR) dan surat izin praktik (SIP).
\item
  Verifikasi adalah proses untuk memeriksa validitas dan kelengkapan kredensial dari sumber yang mengeluarkan kredensial. Proses dapat dilakukan ke fakultas/rumah sakit/perhimpunan di dalam maupun di luar negeri melalui email/surat konvensional/pertanyaan on line/atau melalui telepon. Jika verifikasi dilakukan melalui email maka alamat email harus sesuai dengan alamat email yang ada pada website resmi universitas/rumah sakit/perhimpunan profesi tersebut dan bila melalui surat konvensional harus dengan pos tercatat. Jika verifikasi dilakukan pada website verifikasi ijazah Kementerian Pendidikan, Kebudayaan, Riset dan Teknologi maka akan ada bukti ijazah tersebut terverifikasi.
\item
  Rekredensial adalah proses kredensial ulang setiap 3 (tiga) tahun. Dokumen kredensial dan rekredensial meliputi:
\end{enumerate}

\begin{enumerate}
\def\labelenumi{\arabic{enumi}.}
\tightlist
\item
  STR, SIP yang masih berlaku;
\item
  File pelanggaran etik atau disiplin termasuk infomasi dari sumber luar seperti dari MKEK dan MKDKI;
\item
  Rekomendasi mampu secara fisik maupun mental memberikan asuhan kepada pasien tanpa supervisi dari profesi dokter yang ditentukan;
\item
  Bila tenaga medis mengalami gangguan kesehatan, kecacatan tertentu, atau proses penuaan yang menghambat pelaksanaan kerja maka kepada yang bersangkutan dilakukan penugasan klinis ulang;
\item
  Jika seorang anggota tenaga medis mengajukan kewenangan baru terkait pelatihan spesialisasi canggih atau subspesialisasi maka dokumen kredensial harus segera diverifikasi dari sumber yang mengeluarkan sertifikat tersebut. Keanggotaan tenaga medis mungkin tidak dapat diberikan jika rumah sakit tidak mempunyai teknologi, peralatan medis khusus untuk mendukung kewenangan klinis tertentu. Sebagai contoh, seorang nefrolog melamar untuk memberikan layanan dialisis di rumah sakit bila rumah sakit tidak memiliki pelayanan ini maka kewenangan klinis untuk melakukan haemodialisis tidak dapat diberikan.
\end{enumerate}

Pengecualian untuk KPS 10.1 EP 1, hanya untuk survei awal. Pada saat survei akreditasi awal rumah sakit diwajibkan telah menyelesaikan verifikasi untuk tenaga medis baru yang bergabung dalam 12 (dua belas) bulan menjelang survei awal. Selama 12 (dua belas) bulan setelah survei awal, rumah sakit diwajibkan untuk menyelesaikan verifikasi sumber primer untuk seluruh anggota tenaga medis lainnya. Proses ini dicapai dalam kurun waktu 12 (dua belas) bulan setelah survei sesuai dengan rencana yang memprioritaskan verifikasi kredensial bagi tenaga medis aktif yang memberikan pelayanan berisiko tinggi.
Catatan: Pengecualian ini hanya untuk verifikasi kredensial saja. Semua kredensial anggota tenaga medis harus dikumpulkan dan ditinjau, dan kewenangan mereka diberikan.

\begin{enumerate}
\def\labelenumi{\alph{enumi}.}
\setcounter{enumi}{4}
\tightlist
\item
  Pengangkatan/penugasan merupakan proses peninjauan kredensial awal pelamar untuk memutuskan apakah orang tersebut memenuhi syarat untuk memberikan pelayanan yang dibutuhkan pasien rumah sakit dan dapat didukung rumah sakit dengan staf yang kompeten dan dengan kemampuan teknis rumah sakit. Untuk pelamar pertama, informasi yang ditinjau kebanyakan berasal dari sumber luar. Individu atau mekanisme yang berperan pada peninjauan, kriteria yang digunakan untuk membuat keputusan, dan bagaimana keputusan didokumentasikan diidentifikasi dalam kebijakan rumah sakit. Kebijakan rumah sakit mengidentifikasi proses pengangkatan praktisi kesehatan mandiri untuk keperluan gawat darurat atau untuk sementara waktu. Pengangkatan dan identifikasi kewenangan untuk praktisi kesehatan tersebut tidak dibuat sampai setidaknya verifikasi izin telah dilakukan.
\item
  Pengangkatan/penugasan kembali merupakan proses peninjauan dokumen anggota tenaga medis untuk verifikasi:
\end{enumerate}

\begin{enumerate}
\def\labelenumi{\arabic{enumi}.}
\tightlist
\item
  Perpanjangan izin;
\item
  Bahwa anggota tenaga medis tidak dikenai sanksi disipliner oleh badan perizinan dan sertifikasi;
\item
  Bahwa berkas berisi dokumentasi yang cukup untuk pencarian kewenangan atau tugas baru/perluasan di rumah sakit; dan
\item
  Anggota tenaga medis mampu secara fisik dan mental untuk memberikan perawatan dan tata laksana terhadap pasien tanpa supervisi.
\end{enumerate}

Informasi untuk peninjauan ini berasal dari sumber internal maupun eksternal. Jika suatu departemen/unit layanan klinis (misalnya, pelayanan subspesialis) tidak memiliki kepala/pimpinan, rumah sakit mempunyai kebijakan untuk mengidentifikasi siapa yang melakukan peninjauan untuk para tenaga profesional di departemen/unit layanan tersebut. Berkas kredensial anggota tenaga medis harus merupakan sumber informasi yang dinamis dan ditinjau secara konstan. Sebagai contoh, ketika anggota tenaga medis mendapatkan sertifikat pencapaian yang berhubungan dengan peningkatan gelar atau pelatihan khusus lanjutan, kredensial yang baru harus segera diverifikasi dari sumber yang mengeluarkan. Demikian pula jika ada badan luar yang melakukan investigasi tentang kejadian sentinel yang berkaitan dengan anggota tenaga medis dan mengeluarkan sanksi, informasi ini harus segera digunakan untuk mengevaluasi ulang kewenangan klinis dari anggota tenaga medis tersebut. Untuk memastikan bahwa berkas tenaga medis lengkap dan terkini, berkas ditinjau sedikitnya setiap 3 (tiga) tahun, dan terdapat catatan pada berkas tentang tindakan yang telah dilakukan atau tidak diperlukannya tindak lanjut sehingga pengangkatan tenaga medis dilanjutkan.

Keanggotaan tenaga medis dapat tidak diberikan jika rumah sakit tidak memiliki peralatan medis khusus atau staf untuk mendukung praktik profesi tersebut. Sebagai contoh, ahli nefrologi yang ingin melakukan pelayanan dialisis di rumah sakit, dapat tidak diberikan kewenangan (privilege) bila rumah sakit tidak menyelenggarakan pelayanan dialisis.
Akhirnya, jika izin/registrasi pelamar telah diverifikasi dengan sumber yang mengeluarkan, tetapi dokumen lain seperti edukasi dan pelatihan belum diverifikasi, staf tersebut dapat diangkat menjadi anggota tenaga medis dan kewenangan klinis dapat diberikan untuk orang tersebut untuk kurun waktu yang tidak melebihi 90 (sembilan puluh) hari. Pada kondisi di atas, orang-orang tersebut tidak boleh melakukan praktik secara mandiri dan memerlukan supervisi hingga seluruh kredensial telah diverifikasi. Supervisi secara jelas didefinisikan dalam kebijakan rumah sakit, dan berlangsung tidak lebih dari 90 (sembilan puluh) hari.

\hypertarget{elemen-penilaian-kps-10}{%
\paragraph*{4. Elemen Penilaian KPS 10}\label{elemen-penilaian-kps-10}}
\addcontentsline{toc}{paragraph}{4. Elemen Penilaian KPS 10}

\begin{enumerate}
\def\labelenumi{\alph{enumi}.}
\tightlist
\item
  Rumah sakit telah menetapkan peraturan internal tenaga medis (medical staf bylaws) yang mengatur proses penerimaan, kredensial, penilaian kinerja, dan rekredensial tenaga medis
\item
  Rumah sakit telah melaksanakan proses kredensial dan pemberian kewenangan klinis untuk pelayanan diagnostik, konsultasi, dan tata laksana yang diberikan oleh dokter praktik mandiri di rumah sakit secara seragam
\item
  Rumah sakit telah melaksanakan proses kredensial dan pemberian kewenangan klinis kepada dokter praktik mandiri dari luar rumah sakit seperti konsultasi kedokteran jarak jauh (telemedicine), radiologi jarak jauh (teleradiology), dan interpretasi untuk pemeriksaan diagnostik lain: elektrokardiogram (EKG), elektroensefalogram (EEG), elektromiogram (EMG), serta pemeriksaan lain yang serupa.
\item
  Setiap tenaga medis yang memberikan pelayanan di rumah sakit wajib menandatangani perjanjian sesuai dengan regulasi rumah sakit.
\item
  Rumah sakit telah melaksanakan verifikasi ke Lembaga/Badan/Instansi pendidikan atau organisasi profesional yang diakui yang mengeluarkan izin/sertifikat, dan kredensial lain dalam proses kredensial sesuai dengan peraturan perundang- undangan atau yang
\item
  Ada bukti dilaksanakan kredensial tambahan ke sumber yang mengeluarkan apabila tenaga medis yang meminta kewenangan klinis tambahan yang canggih atau subspesialisasi.
\end{enumerate}

\hypertarget{elemen-penilaian-kps-10.1}{%
\paragraph*{5. Elemen Penilaian KPS 10.1}\label{elemen-penilaian-kps-10.1}}
\addcontentsline{toc}{paragraph}{5. Elemen Penilaian KPS 10.1}

\begin{enumerate}
\def\labelenumi{\alph{enumi}.}
\tightlist
\item
  Pengangkatan tenaga medis dibuat berdasar atas kebijakan rumah sakit dan konsisten dengan populasi pasien rumah sakit, misi, dan pelayanan yang diberikan untuk memenuhi kebutuhan pasien.
\item
  Pengangkatan tidak dilakukan sampai setidaknya izin/surat tanda registrasi sudah diverifikasi dari sumber utama yang mengeluarkan surat tersebut dan tenaga medis dapat memberikan pelayanan kepada pasien di bawah supervisi sampai semua kredensial yang disyaratkan undang-undang dan peraturan sudah diverifikasi dari sumbernya.
\item
  Untuk tenaga medis yang belum mendapatkan kewenangan mandiri, dilakukan supervisi dengan mengatur frekuensi supervisi dan supervisor yang ditunjuk serta didokumentasikan di file kredensial staf tersebut.
\end{enumerate}

\hypertarget{standar-kps-11}{%
\paragraph*{6. Standar KPS 11}\label{standar-kps-11}}
\addcontentsline{toc}{paragraph}{6. Standar KPS 11}

Rumah sakit menetapkan proses yang seragam, objektif, dan berdasar bukti (evidence based) untuk memberikan wewenang kepada tenaga medis untuk memberikan layanan klinis kepada pasien sesuai dengan kualifikasinya

\hypertarget{maksud-dan-tujuan-kps-11}{%
\paragraph*{7. Maksud dan Tujuan KPS 11}\label{maksud-dan-tujuan-kps-11}}
\addcontentsline{toc}{paragraph}{7. Maksud dan Tujuan KPS 11}

Pemberian kewenangan (privileging) adalah penentuan kompetensi klinis terkini tenaga medis dan pengambilan keputusan tentang pelayanan klinis yang diizinkan kepada tenaga medis. Pemberian kewenangan (privileging) ini merupakan penentuan paling penting yang harus dibuat rumah sakit untuk melindungi keselamatan pasien dan meningkatkan mutu pelayanan klinis.
Pertimbangan untuk pemberian kewenangan klinis pada pengangkatan awal termasuk hal-hal berikut:

\begin{enumerate}
\def\labelenumi{\alph{enumi}.}
\tightlist
\item
  Keputusan tentang kewenangan klinis yang akan diberikan kepada seorang tenaga medis didasarkan terutama atas informasi dan dokumentasi yang diterima dari sumber luar rumah sakit. Sumber luar ini dapat berasal dari program pendidikan spesialis, surat rekomendasi dari penempatan sebagai tenaga medis yang lalu, atau dari organisasi profesi, kolega dekat, dan setiap data informasi yang mungkin diberikan kepada rumah sakit. Secara umum, sumber informasi ini terpisah dari yang diberikan oleh institusi pendidikan seperti program dokter spesialis, tidak diverifikasi dari sumber kecuali ditentukan lain oleh kebijakan rumah sakit, paling sedikit area kompetensi sudah dapat dianggap benar.
  Evaluasi praktik profesional berkelanjutan (ongoing professional practice evaluation/OPPE) untuk anggota tenaga medis memberikan informasi penting untuk proses pemeliharaan keanggotaan tenaga medis dan terhadap proses pemberian kewenangan klinis.
\item
  Program pendidikan spesialis menentukan dan membuat daftar secara umum tentang kompetensinya di area diagnosis dan tindakan profesi dan Konsil kedokteran Indonesia (KKI) mengeluarkan standar kompetensi atau kewenangan klinis. Perhimpunan profesi lain membuat daftar secara detail jenis/tindak medis yang dapat dipakai sebagai acuan dalam proses pemberian kewenangan klinis;
  c Di dalam setiap area spesialisasi proses untuk merinci kewenangan ini seragam;
\item
  Seorang dokter dengan spesialisasi yang sama dimungkinkan memiliki kewenangan klinis berbeda yang disebabkan oleh perbedaan pendidikan dan pelatihan tambahan, pengalaman, atau hasil kinerja yang bersangkutan selama bekerja, serta kemampuan motoriknya;
\item
  Keputusan kewenangan klinis dirinci dan akan direkomendasikan kepada pimpinan rumah sakit di area spesialisasi terkait dengan mempertimbangkan proses lain, diantaranya:
\end{enumerate}

\begin{enumerate}
\def\labelenumi{\arabic{enumi}.}
\tightlist
\item
  Pemilihan proses apa yang akan dimonitor menggunakan data oleh pimpinan unit pelayanan klinis;
\item
  Penggunaan data tersebut dalam OPPE dari tenaga medis tersebut di unit pelayanan klinis; dan
\item
  Penggunaan data yang dimonitor tersebut untuk proses penugasan ulang dan pembaharuan kewenangan klinis.
\end{enumerate}

\begin{enumerate}
\def\labelenumi{\alph{enumi}.}
\setcounter{enumi}{5}
\item
  Penilaian kinerja tenaga medis berkelanjutan setiap tahun yang dikeluarkan oleh rumah sakit yang berisi jumlah pasien per penyakit/tindakan yang ditangani per tahun, rerata lama dirawat, serta angka kematiannya. Angka Infeksi Luka Operasi (ILO) dan kepatuhan terhadap Panduan Praktik Klinis (PPK) meliputi penggunaan obat, penunjang diagnostik, darah, produk darah, dan lainnya;
\item
  Hasil evaluasi praktik professional berkelanjutan (OPPE) dan terfokus (FPPE);
\item
  Hasil pendidikan dan pelatihan tambahan dari pusat pendidikan, kolegium, perhimpunan profesi, dan rumah sakit yang kompeten mengeluarkan sertifikat;
\item
  Untuk kewenangan tambahan pada pelayanan risiko tinggi maka rumah sakit menentukan area pelayanan risiko tinggi seperti prosedur cathlab, penggantian sendi lutut dan panggul, pemberian obat kemoterapi, obat radioaktif, obat anestesi, dan lainnya. Prosedur dengan risiko tinggi tersebut maka tenaga medis dapat diberikan kewenangan klinis secara khusus. Prosedur risiko tinggi, obat-obat, atau layanan yang lain ditentukan di kelompok spesialisasi dan dirinci kewenangannya secara jelas. Beberapa prosedur mungkin digolongkan berisiko tinggi disebabkan oleh peralatan yang digunakan seperti dalam kasus penggunaan robot atau penggunaan tindakan dari jarak jauh melalui komputer. Juga pemasangan implan yang memerlukan kaliberasi, presisi, dan monitor jelas membutuhkan kewenangan klinis secara spesifik.
\item
  Kewenangan klinis tidak dapat diberikan jika rumah sakit tidak mempunyai peralatan medis khusus atau staf khusus untuk mendukung pelaksanaan kewenangan klinis. Sebagai contoh, seorang nefrolog kompeten melakukan dialisis atau kardiolog kompeten memasang sten tidak dapat diberi kewenangan klinis jika rumah sakit tidak memiliki peralatannya.
  Catatan: jika anggota tenaga medis juga mempunyai tanggung jawab administrasi seperti ketua kelompok tenaga medis (KSM), administrator rumah sakit, atau posisi lain maka tanggung jawab peran ini diuraikan di uraian tugas atau job description. Rumah sakit menetapkan sumber utama untuk memverifikasi peran administrasi ini.
  Proses pemberian rincian kewenangan klinis:
\item
  Terstandar, objektif, berdasar atas bukti (evidence based).
\item
  Terdokumentasi di kebijakan rumah sakit.
\item
  Aktif dan berkelanjutan mengikuti perubahan kredensial tenaga medis.
\item
  Diikuti semua anggota tenaga medis.
\item
  Dapat dibuktikan bahwa prosedur yang digunakan efektif.
\end{enumerate}

Surat penugasan klinis (SPK) dan rincian kewenangan klinis (RKK) tersedia dalam bentuk salinan cetak, salinan elektronik, atau cara lainnya sesuai lokasi/tempat tenaga medis memberikan pelayanan (misalnya, di kamar operasi, unit gawat darurat). Tenaga medis juga diberikan salinan kewenangan klinisnya.
Pembaruan informasi dikomunikasikan jika kewenangan klinis anggota tenaga medis berubah.

\hypertarget{elemen-penilaian-kps-11}{%
\paragraph*{8. Elemen Penilaian KPS 11}\label{elemen-penilaian-kps-11}}
\addcontentsline{toc}{paragraph}{8. Elemen Penilaian KPS 11}

\begin{enumerate}
\def\labelenumi{\alph{enumi}.}
\tightlist
\item
  Direktur menetapkan kewenangan klinis setelah mendapat rekomendasi dari Komite Medik termasuk kewenangan tambahan dengan mempertimbangan poin a) - j) dalam maksud dan tujuan.
\item
  Ada bukti pemberian kewenangan klinis berdasar atas rekomendasi kewenangan klinis dari Komite Medik.
\item
  Ada bukti pelaksanaan pemberian kewenangan tambahan setelah melakukan verifikasi dari sumber utama yang mengeluarkan ijazah/sertifikat.
\item
  Surat penugasan klinis dan rincian kewenangan klinis anggota tenaga medis dalam bentuk cetak atau elektronik (softcopy) atau media lain tersedia di semua unit pelayanan.
\item
  Setiap tenaga medis hanya memberikan pelayanan klinis sesuai kewenangan klinis yang diberikan kepadanya.
\end{enumerate}

\hypertarget{standar-kps-12}{%
\paragraph*{9. Standar KPS 12}\label{standar-kps-12}}
\addcontentsline{toc}{paragraph}{9. Standar KPS 12}

Rumah sakit menerapkan evaluasi praktik profesional berkelanjutan (OPPE) tenaga medis secara seragam untuk menilai mutu dan keselamatan serta pelayanan pasien yang diberikan oleh setiap tenaga medis.

\hypertarget{maksud-dan-tujuan-kps-12}{%
\paragraph*{10. Maksud dan Tujuan KPS 12}\label{maksud-dan-tujuan-kps-12}}
\addcontentsline{toc}{paragraph}{10. Maksud dan Tujuan KPS 12}

Penjelasan istilah dan ekspektasi yang terdapat dalam standar ini adalah sebagai berikut:

\begin{enumerate}
\def\labelenumi{\alph{enumi}.}
\tightlist
\item
  Evaluasi praktik profesional berkelanjutan (OPPE) adalah proses pengumpulan data dan informasi secara berkesinambungan untuk menilai kompetensi klinis dan perilaku profesional tenaga medis.
  Informasi tersebut akan dipertimbangkan dalam pengambilan keputusan untuk mempertahankan, merevisi, atau mencabut kewenangan klinis sebelum berakhirnya siklus 3 (tiga) tahun untuk pembaruan kewenangan klinis.
  Pimpinan medik, kepala unit, Subkomite Mutu Profesi Komite Medik dan Ketua Kelompok Staf Medik (KSM) bertanggung jawab mengintegrasikan data dan informasi tenaga medis dan pengambilan kesimpulan dalam memberikan penilaian.
  Jika terjadi kejadian insiden keselamatan pasien atau pelanggaran perilaku etik maka dilakukan tindakan terhadap tenaga medis tersebut secara adil (just culture) berdasarkan hasil analisis terkait kejadian tersebut.
  Tindakan jangka pendek dapat dalam bentuk konseling, menempatkan kewenangan tertentu di bawah supervisi, pembatasan kewenangan, atau tindakan lain untuk membatasi risiko terhadap pasien, dan untuk meningkatkan mutu serta keselamatan pasien. Tindakan jangka panjang dalam bentuk membuat rekomendasi terkait kelanjutan keanggotaan tenaga medis dan kewenangan klinis.
  Proses ini dilakukan sedikitnya setiap 3 (tiga) tahun. Monitor dan evaluasi berkelanjutan tenaga medis menghasilkan informasi kritikal dan penting terhadap proses mempertahankan tenaga medis dan proses pemberian kewenangan klinis. Walaupun dibutuhkan 3 (tiga) tahun untuk memperpanjang keanggotaan tenaga medis dan kewenangan kliniknya, prosesnya dimaksudkan berlangsung sebagai proses berkelanjutan dan dinamis.
  Masalah mutu dan insiden keselamatan pasien dapat terjadi jika kinerja klinis tenaga medis tidak dikomunikasikan dan dilakukan tindak lanjut.
\end{enumerate}

Proses monitor penilaian OPPE tenaga medis untuk:

\begin{enumerate}
\def\labelenumi{\arabic{enumi}.}
\tightlist
\item
  Meningkatkan praktik individual terkait mutu dan asuhan pasien yang aman;
\item
  Digunakan sebagai dasar mengurangi variasi di dalam kelompok tenaga medis (KSM) dengan cara membandingkan antara kolega, penyusunan panduan praktik klinis (PPK), dan clinical pathway; dan
\item
  Digunakan sebagai dasar memperbaiki kinerja kelompok tenaga medis/unit dengan cara membandingkan acuan praktik di luar rumah sakit, publikasi riset, dan indikator kinerja klinis nasional bila tersedia.
\end{enumerate}

Penilaian OPPE tenaga medis memuat 3 (tiga) area umum yaitu:

\begin{enumerate}
\def\labelenumi{\arabic{enumi}.}
\tightlist
\item
  Perilaku;
\item
  Pengembangan professional; dan
\item
  Kinerja klinis.
\end{enumerate}

\begin{enumerate}
\def\labelenumi{\alph{enumi}.}
\setcounter{enumi}{1}
\tightlist
\item
  Perilaku tenaga medis adalah sebagai model atau mentor dalam menumbuhkan budaya keselamatan (safety culture) di rumah sakit. Budaya keselamatan ditandai dengan partisipasi penuh semua staf untuk melaporkan bila ada insiden keselamatan pasien tanpa ada rasa takut untuk melaporkan dan disalahkan (no blame culture).
\end{enumerate}

Budaya keselamatan juga sangat menghormati satu sama lain, antar kelompok profesional, dan tidak terjadi sikap saling mengganggu. Umpan balik staf dalam dapat membentuk sikap dan perilaku yang diharapkan dapat mendukung staf medik menjadi model untuk menumbuhkan budaya aman. Evaluasi perilaku memuat:

\begin{enumerate}
\def\labelenumi{\arabic{enumi}.}
\tightlist
\item
  Evaluasi apakah seorang tenaga medis mengerti dan mendukung kode etik dan disiplin profesi dan rumah sakit serta dilakukan identifikasi perilaku yang dapat atau tidak dapat diterima maupun perilaku yang mengganggu;
\item
  Tidak ada laporan dari anggota tenaga medis tentang perilaku yang dianggap tidak dapat diterima atau mengganggu; dan
\item
  Mengumpulkan, analisis, serta menggunakan data dan informasi berasal dari survei staf serta survei lainnya tentang budaya aman di rumah sakit.
\end{enumerate}

Proses pemantauan OPPE harus dapat mengenali hasil pencapaian, pengembangan potensial terkait kewenangan klinis dari anggota tenaga medis, dan layanan yang diberikan. Evaluasi perilaku dilaksanakan secara kolaboratif antara Subkomite Etik dan Disiplin, manajer SDM, manajer pelayanan, dan kepala unit kerja.

Pengembangan profesional anggota tenaga medis berkembang dengan menerapkan teknologi baru dan pengetahuan klinis baru. Setiap anggota tenaga medis dari segala tingkatan akan merefleksikan perkembangan dan perbaikan pelayanan kesehatan dan praktik profesional sebagai berikut:

\begin{enumerate}
\def\labelenumi{\arabic{enumi}.}
\tightlist
\item
  Asuhan pasien, penyediaan asuhan penuh kasih, tepat dan efektif dalam promosi kesehatan, pencegahan penyakit, pengobatan penyakit, dan asuhan di akhir hidup. Alat ukurnya adalah layanan preventif dan laporan dari pasien serta keluarga
\item
  Pengetahuan medik/klinik termasuk pengetahuan biomedik, klinis, epidemiologi, ilmu pengetahuan sosial budaya, dan pendidikan kepada pasien. Alat ukurnya adalah penerapan panduan praktik klinis (clinical practice guidelines) termasuk revisi pedoman hasil pertemuan profesional dan publikasi.
\item
  Praktik belajar berdasar bukti (practice-bases learning) dan pengembangan, penggunaan bukti ilmiah dan metode pemeriksaan, evaluasi, serta perbaikan asuhan pasien berkelanjutan berdasar atas evaluasi dan belajar terus menerus (contoh alat ukur survei klinis, memperoleh kewenangan berdasar atas studi dan keterampilan klinis baru, dan partisipasi penuh pada pertemuan ilmiah).
\item
  Kepandaian berkomunikasi antarpersonal termasuk menjaga dan meningkatkan pertukaran informasi dengan pasien, keluarga pasien, dan anggota tim layanan kesehatan yang lain (contoh, partisipasi aktif di ronde ilmiah, konsultasi tim, dan kepemimpinan tim).
\item
  Profesionalisme, janji mengembangkan profesionalitas terus menerus, praktik etik, pengertian terhadap perbedaan, serta perilaku bertanggung jawab terhadap pasien, profesi, dan masyarakat (contoh, alat ukur: pendapat pimpinan di tenaga medis terkait isu klinis dan isu profesi, aktif membantu diskusi panel tentang etik, ketepatan waktu pelayanan di rawat jalan maupun rawat inap, dan partisipasi di masyarakat).
\item
  Praktik berbasis sistem, serta sadar dan tanggap terhadap jangkauan sistem pelayanan kesehatan yang lebih luas (contoh alat ukur: pemahaman terhadap regulasi rumah sakit yang terkait dengan tugasnya seperti sistem asuransi medis, asuransi kesehatan (JKN), sistem kendali mutu, dan biaya. Peduli pada masalah resistensi antimikrob).
\item
  Mengelola sumber daya, memahami pentingnya sumber daya dan berpartisipasi melaksanakan asuhan yang efisien, serta menghindari penyalahgunaan pemeriksaan untuk diagnostik dan terapi yang tidak ada manfaatnya bagi pasien serta meningkatkan biaya pelayanan kesehatan (contoh alat ukur: berpartisipasi dalam kendali mutu dan biaya, kepedulian terhadap biaya yang ditanggung pasien, serta berpatisipasi dalam proses seleksi pengadaan)
\item
  Sebagai bagian dari proses penilaian, proses pemantauan dan evaluasi berkelanjutan, serta harus mengetahui kinerja anggota tenaga medis yang relevan dengan potensi pengembangan kemampuan profesional tenaga medis.
  Proses pemantauan OPPE tenaga medis harus dapat menjadi bagian dari proses peninjauan kinerja tenaga medis terkait dengan upaya mendukung budaya keselamatan.
\end{enumerate}

Penilaian atas informasi bersifat umum berlaku bagi semua anggota tenaga medis dan juga tentang informasi spesifik terkait kewenangan anggota tenaga medis dalam memberikan pelayanannya. Rumah sakit mengumpulkan berbagai data untuk keperluan manajemen, misalnya membuat laporan ke pimpinan rumah sakit tentang alokasi sumber daya atau sistem pembiayaan rumah sakit. Agar bermanfaat bagi evaluasi berkelanjutan seorang tenaga medis maka sumber data rumah sakit:

\begin{enumerate}
\def\labelenumi{\arabic{enumi}.}
\tightlist
\item
  Harus dikumpulkan sedemikian rupa agar teridentifikasi tenaga medis yang berperan. Harus terkait dengan praktik klinis seorang anggota tenaga medis; dan
\item
  Dapat menjadi rujukan (kaji banding) di dalam KSM/Unit layanan atau di luarnya untuk mengetahui pola individu tenaga medis.
\end{enumerate}

Sumber data potensial seperti itu misalnya adalah lama hari rawat (length of stay), frekuensi (jumlah pasien yang ditangani), angka kematian, pemeriksaan diagnostik, pemakaian darah, pemakaian obat-obat tertentu, angka ILO, dan lain sebagainya.

Pemantauan dan evaluasi anggota tenaga medis berdasar atas berbagai sumber data termasuk data cetak, data elektronik, observasi dan, interaksi teman sejawat. Simpulan proses monitor dan evaluasi anggota tenaga medis:

\begin{enumerate}
\def\labelenumi{\arabic{enumi}.}
\tightlist
\item
  Jenis anggota tenaga medis, jenis KSM, jenis unit layanan terstandar;
\item
  Data pemantauan dan informasi dipergunakan untuk perbandingan internal, mengurangi variasi perilaku, serta pengembangan profesional dan hasil klinis;
\item
  Data monitor dan informasi dipergunakan untuk melakukan perbandingan eksternal dengan praktik berdasar bukti (evidence based practice) atau sumber rujukan tentang data dan informasi hasil klinis;
\item
  Dipimpin oleh ketua KSM/unit layanan, manajer medis, atau unit kajian tenaga medis; dan
\item
  Pemantauan dan evaluasi terhadap kepala bidang pelayanan dan kepala KSM oleh profesional yang kompeten.
\end{enumerate}

Kebijakan rumah sakit mengharuskan ada tinjauan (review) paling sedikit selama 12 (dua belas) bulan. Review dilakukan secara kolaborasi di antaranya oleh kepala KSM/unit layanan, kepala bidang pelayanan medis, Subkomite Mutu Profesi Komite Medik, dan bagian IT. Temuan, simpulan, dan tindakan yang dijatuhkan atau yang direkomendasikan dicatat di file praktisi serta tercermin di kewenangan kliniknya. Pemberitahuan diberikan kepada tempat di tempat praktisi memberikan layanan.
Informasi yang dibutuhkan untuk tinjauan ini dikumpulkan dari internal dan dari pemantauan serta evaluasi berkelanjutan setiap anggota staf termasuk juga dari sumber luar seperti organisasi profesi atau sumber instansi resmi.

File kredensial dari seorang anggota tenaga medis harus menjadi sumber informasi yang dinamis dan selalu ditinjau secara teratur. Contohnya, jika seorang anggota staf menyerahkan sertifikat kelulusan sebagai hasil dari pelatihan spesialisasi khusus maka kredensial baru ini harus diverifikasi segera dari sumber yang mengeluarkan sertifikat. Sama halnya, jika instansi dari luar (MKEK/MKDKI) menyelidiki kejadian sentinel terkait seorang anggota tenaga medis dan memberi sanksi maka informasi ini harus digunakan untuk evaluasi muatan kewenangan klinis anggota tenaga medis. Untuk menjamin bahwa file tenaga medis lengkap dan akurat, file diperiksa paling sedikit 3 (tiga) tahun sekali dan ada catatan di file tindakan yang diberikan atau tindakan yang tidak diperlukan sehingga penempatan tenaga medis dapat berlanjut.

Pertimbangan untuk merinci kewenangan klinis waktu penempatan kembali sebagai berikut:

\begin{enumerate}
\def\labelenumi{\arabic{enumi}.}
\tightlist
\item
  Anggota tenaga medis dapat diberikan kewenangan klinis tambahan berdasar atas pendidikan dan pelatihan lanjutan. Pendidikan dan pelatihan diverifikasi dari sumber utamanya. Pemberian penuh kewenangan klinis tambahan mungkin ditunda sampai proses verifikasi lengkap atau jika dibutuhkan waktu harus dilakukan supervisi sebelum kewenangan klinis diberikan.
  Contoh, jumlah kasus yang harus disupervisi dari kardiologi intervensi;
\item
  Kewenangan klinis anggota tenaga medis dapat dilanjutkan, dibatasi, atau dihentikan berdasar: hasil dari proses tinjauan praktik profesional berkelanjutan;
\item
  Pembatasan kewenangan klinik dari organisasi profesi, KKI, MKEK, MKDKI, atau badan resmi lainnya;
\item
  Temuan rumah sakit dari hasil evaluasi kejadian sentinel atau kejadian lain; kesehatan tenaga medis; dan/atau
\item
  Permintaan tenaga medis.
\end{enumerate}

\hypertarget{elemen-penilaian-kps-12}{%
\paragraph*{11. Elemen Penilaian KPS 12}\label{elemen-penilaian-kps-12}}
\addcontentsline{toc}{paragraph}{11. Elemen Penilaian KPS 12}

\begin{enumerate}
\def\labelenumi{\alph{enumi}.}
\tightlist
\item
  Rumah sakit telah menetapkan dan menerapkan proses penilaian kinerja untuk evaluasi mutu praktik profesional berkelanjutan, etik, dan disiplin (OPPE) tenaga medis
\item
  Penilaian OPPE tenaga medis memuat 3 (tiga) area umum 1) ? 3) dalam maksud dan tujuan.
\item
  Penilaian OPPE juga meliputi peran tenaga medis dalam pencapaian target indikator mutu yang diukur di unit tempatnya bekerja.
\item
  Data dan informasi hasil pelayanan klinis dari tenaga medis dikaji secara objektif dan berdasar atas bukti, jika memungkinkan dilakukan benchmarking dengan pihak eksternal rumah sakit.
\item
  Data dan informasi hasil pemantauan kinerja tenaga medis sekurang-kurangnya setiap 12 (dua belas) bulan dilakukan oleh kepala unit, kepala kelompok tenaga medis, Subkomite Mutu Profesi Komite Medik dan pimpinan pelayanan medis. Hasil, simpulan, dan tindakan didokumentasikan di dalam file kredensial tenaga medis tersebut
\item
  Jika terjadi kejadian insiden keselamatan pasien atau pelanggaran perilaku etik maka dilakukan tindakan terhadap tenaga medis tersebut secara adil (just culture) berdasarkan hasil analisisterkait kejadian tersebut.
\item
  Bila ada temuan yang berdampak pada pemberian kewenangan tenaga medis, temuan tersebut didokumentasi ke dalam file tenaga medis dan diinformasikan serta disimpan di unit tempat tenaga medis memberikan pelayanan
\end{enumerate}

\hypertarget{standar-kps-13}{%
\paragraph*{12. Standar KPS 13}\label{standar-kps-13}}
\addcontentsline{toc}{paragraph}{12. Standar KPS 13}

Rumah sakit paling sedikit setiap 3 (tiga) tahun melakukan rekredensial berdasarkan hasil penilaian praktik profesional berkelanjutan (OPPE) terhadap setiap semua tenaga medis rumah sakit untuk menentukan apabila tenaga medis dan kewenangan klinisnya dapat dilanjutkan dengan atau tanpa modifikasi.

\hypertarget{maksud-dan-tujuan-kps-13}{%
\paragraph*{13. Maksud dan Tujuan KPS 13}\label{maksud-dan-tujuan-kps-13}}
\addcontentsline{toc}{paragraph}{13. Maksud dan Tujuan KPS 13}

Penjelasan istilah dan ekspektasi yang ditemukan dalam standar-standar ini adalah sebagai berikut:

\begin{enumerate}
\def\labelenumi{\alph{enumi}.}
\tightlist
\item
  Rekredensial/penugasan kembali merupakan proses peninjauan, sedikitnya dilakukan setiap 3 (tiga) tahun, terhadap file tenaga medis untuk verifikasi:
\end{enumerate}

\begin{enumerate}
\def\labelenumi{\arabic{enumi}.}
\tightlist
\item
  Kelanjutan izin (license);
\item
  Apakah anggota tenaga medis tidak terkena tindakan etik dan disiplin dari MKEK dan MKDKI;
\item
  Apakah tersedia dokumen untuk mendukung penambahan kewenangan klinis atau tanggung jawab di rumah sakit; dan
\item
  Apakah anggota tenaga medis mampu secara fisik dan mental memberikan asuhan dan pengobatan tanpa supervisi.
\end{enumerate}

Informasi untuk peninjauan ini dikumpulkan dari sumber internal, penilaian praktik profesional berkelanjutan tenaga medis (OPPE), dan juga dari sumber eksternal seperti organisasi profesi atau sumber instansi resmi.
File kredensial dari seorang anggota tenaga medis harus menjadi sumber informasi yang dinamis dan selalu ditinjau secara teratur. Sebagai contoh, ketika anggota tenaga medis mendapatkan sertifikat pencapaian berkaitan dengan peningkatan gelar atau pelatihan spesialistis lanjutan, kredensial yang baru segera diverifikasi dari sumber yang mengeluarkan. Demikian pula ketika badan luar melakukan investigasi tentang kejadian sentinel yang berkaitan dengan anggota tenaga medis dan mengenakan sanksi, informasi ini harus segera digunakan untuk evaluasi ulang kewenangan klinis anggota tenaga medis tersebut. Untuk memastikan berkas tenaga medis lengkap dan akurat, berkas ditinjau sedikitnya setiap 3 (tiga) tahun, dan terdapat catatan dalam berkas yang menunjukkan tindakan yang telah dilakukan atau bahwa tidak diperlukan tindakan apa pun dan pengangkatan tenaga medis dilanjutkan.

Misalnya, jika seorang tenaga medis menyerahkan sertifikat kelulusan sebagai hasil dari pelatihan spesialisasi khusus, kredensial baru ini harus diverifikasi segera dari sumber yang mengeluarkan sertifikat. Sama halnya, jika instansi dari luar (MKEK/MKDKI) menyelidiki kejadian sentinel pada seorang tenaga medis dan memberi sanksi maka informasi ini harus digunakan untuk penilaian kewenangan klinis tenaga medis tersebut. Untuk menjamin bahwa file tenaga medis lengkap dan akurat, file diperiksa paling sedikit 3 (tiga) tahun sekali dan ada kesimpulan hasil peninjauan di file berupa tindakan yang akan dilakukan atau tindakan tidak diperlukan sehingga penempatan tenaga medis dapat dilanjutkan.

Pertimbangan untuk memberikan kewenangan klinis saat rekredensial/penugasan kembali mencakup hal- hal berikut:

\begin{enumerate}
\def\labelenumi{\arabic{enumi}.}
\tightlist
\item
  Tenaga medis dapat diberikan kewenangan tambahan berdasarkan pendidikan dan pelatihan lanjutan. Pendidikan dan pelatihan telah diverifikasi dari Badan/Lembaga/Institusi penyelenggara pendidikan atau pelatihan. Pelaksanaan kewenangan tambahan dapat ditunda sampai proses verifikasi selesai atau sesuai ketentuan rumah sakit terdapat periode waktu persyaratan untuk praktik di bawah supervisi sebelum pemberian kewenangan baru diberikan secara mandiri; misalnya jumlah kasus yang harus disupervisi dari kardiologi intervensi;
\item
  Kewenangan tenaga medis dapat dilanjutkan, dibatasi, dikurangi, atau dihentikan berdasarkan:
\end{enumerate}

\begin{enumerate}
\def\labelenumi{\alph{enumi}.}
\tightlist
\item
  Hasil evaluasi praktik profesional berkelanjutan (OPPE);
\item
  Batasan kewenangan yang dikenakan kepada staf oleh organisasi profesi, KKI, MKEK, MKDKI, atau badan resmi lainnya;
\item
  Temuan rumah sakit dari analisis terhadap kejadian sentinel atau kejadian lainnya;
\item
  Status kesehatan tenaga medis; dan/atau
\item
  Permintaan dari tenaga medis.
\end{enumerate}

\hypertarget{e.-tenaga-keperawatan}{%
\subsection*{e. Tenaga Keperawatan}\label{e.-tenaga-keperawatan}}
\addcontentsline{toc}{subsection}{e. Tenaga Keperawatan}

\hypertarget{standar-kps-14}{%
\paragraph*{1. Standar KPS 14}\label{standar-kps-14}}
\addcontentsline{toc}{paragraph}{1. Standar KPS 14}

Rumah sakit mempunyai proses yang efektif untuk melakukan kredensial tenaga perawat dengan mengumpulkan, verifikasi pendidikan, registrasi, izin, kewenangan, pelatihan, dan pengalamannya.

\hypertarget{maksud-dan-tujuan-kps-14}{%
\paragraph*{2. Maksud dan Tujuan KPS 14}\label{maksud-dan-tujuan-kps-14}}
\addcontentsline{toc}{paragraph}{2. Maksud dan Tujuan KPS 14}

Rumah sakit perlu memastikan untuk mempunyai tenaga perawat yang kompeten sesuai dengan misi, sumber daya, dan kebutuhan pasien. Tenaga perawat bertanggungjawab untuk memberikan asuhan keperawatan pasien secara langsung. Sebagai tambahan, asuhan keperawatan memberikan kontribusi terhadap outcome pasien secara keseluruhan. Rumah sakit harus memastikan bahwa perawat yang kompeten untuk memberikan asuhan keperawatan dan harus spesifik terhadap jenis asuhan keperawatan sesuai dengan peraturan perundang- undangan.

Rumah sakit memastikan bahwa setiap perawat yang kompeten untuk memberikan asuhan keperawatan, baik mandiri, kolaborasi, delegasi, serta mandat kepada pasien secara aman dan efektif dengan cara:

\begin{enumerate}
\def\labelenumi{\alph{enumi}.}
\tightlist
\item
  Memahami peraturan dan perundang-undangan terkait perawat dan praktik keperawatan;
\item
  Melakukan kredensial terhadap semua bukti pendidikan, pelatihan, pengalaman, informasi yang ada pada setiap perawat, sekurang-kurangnya meliputi:
\end{enumerate}

\begin{enumerate}
\def\labelenumi{\arabic{enumi}.}
\tightlist
\item
  Bukti pendidikan, registrasi, izin, kewenangan, pelatihan, serta pengalaman terbaru dan diverifikasi dari sumber aslinya;
\item
  Bukti kompetensi terbaru melalui informasi dari sumber lain di tempat perawat pernah bekerja sebelumnya; dan
\item
  Surat rekomendasi dan/atau informasi lain yang mungkin diperlukan rumahsakit, antara lain riwayat kesehatan dan sebagainya.
\end{enumerate}

\begin{enumerate}
\def\labelenumi{\alph{enumi}.}
\setcounter{enumi}{2}
\tightlist
\item
  Rumah sakit perlu melakukan setiap upaya untuk memverifikasi informasi penting dari berbagai sumber utama dengan jalan mengecek ke website resmi institusi pendidikan pelatihan melalui email dan surat tercatat.
\end{enumerate}

Pemenuhan standar mensyaratkan verifikasi sumber utama dilaksanakan untuk perawat yang akan dan sedang bekerja. Bila verifikasi tidak mungkin dilakukan seperti hilang karena bencana atau sekolahnya tutup maka hal ini didapatkan dari sumber resmi lain.

\hypertarget{elemen-penilaian-kps-14}{%
\paragraph*{3. Elemen Penilaian KPS 14}\label{elemen-penilaian-kps-14}}
\addcontentsline{toc}{paragraph}{3. Elemen Penilaian KPS 14}

\begin{enumerate}
\def\labelenumi{\alph{enumi}.}
\tightlist
\item
  Rumah sakit telah menetapkan dan menerapkan proses kredensial yang efektif terhadap tenaga perawat meliputi poin a) ? c) dalam maksud dan tujuan.
\item
  Tersedia bukti dokumentasi pendidikan, registrasi, sertifikasi, izin, pelatihan, dan pengalaman yang terbaharui di file tenaga perawat.
\item
  Terdapat pelaksanaan verifikasi ke sumber Badan/Lembaga/institusi penyelenggara pendidikan/ pelatihan yang seragam.
\item
  Terdapat bukti dokumen kredensial yang dipelihara pada setiap tenaga perawat.
\item
  Rumah sakit menerapkan proses untuk memastikan bahwa kredensial perawat kontrak lengkap sebelum penugasan.
\end{enumerate}

\hypertarget{standar-kps-15}{%
\paragraph*{4. Standar KPS 15}\label{standar-kps-15}}
\addcontentsline{toc}{paragraph}{4. Standar KPS 15}

Rumah sakit melakukan identifikasi tanggung jawab pekerjaan dan memberikan penugasan klinis berdasar atas hasil kredensial tenaga perawat sesuai dengan peraturan perundang-undangan.

\hypertarget{maksud-dan-tujuan-kps-15}{%
\paragraph*{5. Maksud dan Tujuan KPS 15}\label{maksud-dan-tujuan-kps-15}}
\addcontentsline{toc}{paragraph}{5. Maksud dan Tujuan KPS 15}

Hasil kredensial perawat berupa rincian kewenangan klinis menjadi landasan untuk membuat surat penugasan klinis kepada tenaga perawat.

\hypertarget{elemen-penilaian-kps-15}{%
\paragraph*{6. Elemen Penilaian KPS 15}\label{elemen-penilaian-kps-15}}
\addcontentsline{toc}{paragraph}{6. Elemen Penilaian KPS 15}

\begin{enumerate}
\def\labelenumi{\alph{enumi}.}
\tightlist
\item
  Rumah sakit telah menetapkan rincian kewenangan klinis perawat berdasar hasil kredensial terhadap perawat.
\item
  Rumah sakit telah menetapkan Surat Penugasan Klinis tenaga perawat sesuai dengan peraturan perundang- undangan.
\end{enumerate}

\hypertarget{standar-kps-16}{%
\paragraph*{7. Standar KPS 16}\label{standar-kps-16}}
\addcontentsline{toc}{paragraph}{7. Standar KPS 16}

Rumah sakit telah melakukan penilaian kinerja tenaga perawat termasuk perannya dalam kegiatan peningkatan mutu dan keselamatan pasien serta program manajemen risiko rumah sakit.

\hypertarget{maksud-dantujuan-kps-16}{%
\paragraph*{8. Maksud danTujuan KPS 16}\label{maksud-dantujuan-kps-16}}
\addcontentsline{toc}{paragraph}{8. Maksud danTujuan KPS 16}

Peran klinis tenaga perawat sangat penting dalam pelayanan pasien sehingga mengharuskan perawat berperan secara proaktif dalam program peningkatan mutu dan keselamatan pasien serta program manajemen risiko rumah sakit.
Rumah sakit melakukan penilaian kinerja tenaga perawat secara periodik menggunakan format dan metode sesuai ketentuan yang ditetapkan rumah sakit.
Bila ada temuan dalam kegiatan peningkatan mutu, laporan insiden keselamatan pasien atau manajemen risiko maka Pimpinan rumah sakit dan kepala unit akan mempertimbangkan secara adil (just culture) dengan melihat laporan mutu atau hasil Root Cause Analysis (RCA) sejauh mana peran perawat yang terkait kejadian tersebut.
Hasil kajian, tindakan yang diambil, dan setiap dampak atas tanggung jawab pekerjaan didokumentasikan dalam file kredensial perawat tersebut atau file lainnya.

\hypertarget{elemen-penilaian-kps-16}{%
\paragraph*{9. Elemen Penilaian KPS 16}\label{elemen-penilaian-kps-16}}
\addcontentsline{toc}{paragraph}{9. Elemen Penilaian KPS 16}

\begin{enumerate}
\def\labelenumi{\alph{enumi}.}
\tightlist
\item
  Rumah sakit telah melakukan penilaian kinerja tenaga perawat secara periodik menggunakan format dan metode sesuai ketentuan yang ditetapkan rumah sakit.
\item
  Penilaian kinerja tenaga perawat meliputi pemenuhan uraian tugasnya dan perannya dalam pencapaian target indikator mutu yang diukur di unit tempatnya bekerja.
\item
  Pimpinan rumah sakit dan kepala unit telah berlaku adil (just culture) ketika ada temuan dalam kegiatan peningkatan mutu, laporan insiden keselamatan pasien atau manajemen risiko.
\item
  Rumah sakit telah mendokumentasikan hasil kajian, tindakan yang diambil, dan setiap dampak atas tanggung jawab pekerjaan perawat dalam file kredensial perawat.
\end{enumerate}

\hypertarget{f.-tenaga-kesehatan-lainnya}{%
\subsection*{f.~Tenaga Kesehatan Lainnya}\label{f.-tenaga-kesehatan-lainnya}}
\addcontentsline{toc}{subsection}{f.~Tenaga Kesehatan Lainnya}

\hypertarget{standar-kps-17}{%
\paragraph*{1. Standar KPS 17}\label{standar-kps-17}}
\addcontentsline{toc}{paragraph}{1. Standar KPS 17}

Rumah sakit mempunyai proses yang efektif untuk melakukan kredensial tenaga kesehatan lain dengan mengumpulkan dan memverifikasi pendidikan, registrasi, izin, kewenangan, pelatihan, dan pengalamannya.

\hypertarget{maksud-dan-tujuan-kps-17}{%
\paragraph*{2. Maksud dan tujuan KPS 17}\label{maksud-dan-tujuan-kps-17}}
\addcontentsline{toc}{paragraph}{2. Maksud dan tujuan KPS 17}

Rumah sakit perlu memastikan bahwa tenaga kesehatan lainnya kompeten sesuai dengan misi, sumber daya, dan kebutuhan pasien. Profesional Pemberi Asuhan (PPA) yang bertanggungjawab memberikan asuhan pasien secara langsung termasuk bidan, nutrisionis, apoteker, fisioterapis, teknisi transfusi darah, penata anestesi, dan lainnya. sedangkan staf klinis adalah adalah staf yang menempuh pendidikan profesi maupun vokasi yang tidak memberikan pelayanan secara langsung kepada pasien.

Rumah sakit memastikan bahwa PPA dan staf klinis lainnya berkompeten dalam memberikan asuhan aman dan efektif kepada pasien sesuai dengan peraturan perundang- undangan dengan:

\begin{enumerate}
\def\labelenumi{\alph{enumi}.}
\tightlist
\item
  Memahami peraturan dan perundang-undangan terkait tenaga kesehatan lainnya.
\item
  Mengumpulkan semua kredensial yang ada untuk setiap profesional pemberi asuhan (PPA) lainnya dan staf klinis lainnya sekurang-kurangnya meliputi:
\end{enumerate}

\begin{enumerate}
\def\labelenumi{\arabic{enumi}.}
\tightlist
\item
  Bukti pendidikan, registrasi, izin, kewenangan, pelatihan, dan pengalaman terbaru serta diverifikasi dari sumber asli/website verifikasi ijazah Kementerian Pendidikan, Kebudayaan, Riset dan Teknologi;
\item
  Bukti kompetensi terbaru melalui informasi dari sumber lain di tempat tenaga kesehatan lainnya pernah bekerja sebelumnya; dan
\item
  Surat rekomendasi dan/atau informasi lain yang mungkin diperlukan rumah sakit, antara lain riwayat kesehatan dan sebagainya.
\end{enumerate}

\begin{enumerate}
\def\labelenumi{\alph{enumi}.}
\setcounter{enumi}{2}
\tightlist
\item
  Melakukan setiap upaya memverifikasi informasi penting dari berbagai sumber dengan jalan mengecek ke website resmi dari institusi pendidikan pelatihan melalui email dan surat tercatat. Pemenuhan standar mensyaratkan verifikasi sumber aslinya dilaksanakan untuk tenaga kesehatan lainnya yang akan dan sedang bekerja. Bila verifikasi tidak mungkin dilakukan seperti hilangnya dokumen karena bencana atau sekolahnya tutup maka hal ini dapat diperoleh dari sumber resmi lain. File kredensial setiap tenaga kesehatan lainnya harus tersedia dan dipelihara serta diperbaharui secara berkala sesuai dengan peraturan perundang- undangan.
\end{enumerate}

\hypertarget{elemen-penilaian-kps-17}{%
\paragraph*{3. Elemen Penilaian KPS 17}\label{elemen-penilaian-kps-17}}
\addcontentsline{toc}{paragraph}{3. Elemen Penilaian KPS 17}

\begin{enumerate}
\def\labelenumi{\alph{enumi}.}
\tightlist
\item
  Rumah sakit telah menetapkan dan menerapkan proses kredensial yang efektif terhadap tenaga Kesehatan lainnya meliputi poin a) - c) dalam maksud dan tujuan.
\item
  Tersedia bukti dokumentasi pendidikan, registrasi, sertifikasi, izin, pelatihan, dan pengalaman yang terbaharui di file tenaga Kesehatan lainnya.
\item
  Terdapat pelaksanaan verifikasi ke sumber Badan/Lembaga/institusi penyelenggara Pendidikan/pelatihan yang seragam.
\item
  Terdapat dokumen kredensial yang dipelihara dari setiap tenaga kesehatan lainnya.
\end{enumerate}

\hypertarget{standar-kps-18}{%
\paragraph*{4. Standar KPS 18}\label{standar-kps-18}}
\addcontentsline{toc}{paragraph}{4. Standar KPS 18}

Rumah sakit melakukan identifikasi tanggung jawab pekerjaan dan memberikan penugasan klinis berdasar atas hasil kredensial tenaga kesehatan lainnya sesuai dengan peraturan perundang-undangan.

\hypertarget{maksud-dan-tujuan-kps-18}{%
\paragraph*{5. Maksud dan Tujuan KPS 18}\label{maksud-dan-tujuan-kps-18}}
\addcontentsline{toc}{paragraph}{5. Maksud dan Tujuan KPS 18}

Rumah sakit mempekerjakan atau dapat mengizinkan tenaga kesehatan lainnya untuk memberikan asuhan dan pelayanan kepada pasien atau berpartisipasi dalam proses asuhan pasien.
Bila Tenaga kesehatan lainnya tersebut yang diizinkan bekerja atau berpraktik di rumah sakit maka rumah sakit bertanggungjawab untuk melakukan proses kredensialing.

\hypertarget{elemen-penilaian-kps-18}{%
\paragraph*{6. Elemen Penilaian KPS 18}\label{elemen-penilaian-kps-18}}
\addcontentsline{toc}{paragraph}{6. Elemen Penilaian KPS 18}

a Rumah sakit telah menetapkan rincian kewenangan klinis profesional pemberi asuhan (PPA) lainnya dan staf klinis lainnya berdasar atas hasil kredensial tenaga Kesehatan lainnya.
b. Rumah sakit telah menetapkan surat penugasan klinis kepada tenaga Kesehatan lainnya sesuai dengan peraturan perundang-undangan.

\hypertarget{standar-kps-19}{%
\paragraph*{7. Standar KPS 19}\label{standar-kps-19}}
\addcontentsline{toc}{paragraph}{7. Standar KPS 19}

Rumah sakit telah melakukan penilaian kinerja tenaga Kesehatan lainnya termasuk perannya dalam kegiatan peningkatan mutu dan keselamatan pasien serta program manajemen risiko rumah sakit.

\hypertarget{maksud-dan-tujuan-kps-19}{%
\paragraph*{8. Maksud dan Tujuan KPS 19}\label{maksud-dan-tujuan-kps-19}}
\addcontentsline{toc}{paragraph}{8. Maksud dan Tujuan KPS 19}

Peran klinis tenaga Kesehatan lainnya sangat penting dalam pelayanan pasien sehingga mengharuskan mereka berperan secara proaktif dalam program peningkatan mutu dan keselamatan pasien serta program manajemen risiko rumah sakit.
Rumah sakit melakukan penilaian kinerja tenaga Kesehatan lainnya secara periodik menggunakan format dan metode sesuai ketentuan yang ditetapkan rumah sakit.

Bila ada temuan dalam kegiatan peningkatan mutu, laporan insiden keselamatan pasien atau manajemen risiko maka Pimpinan rumah sakit dan kepala unit akan mempertimbangkan secara adil (just culture) dengan melihat laporan mutu atau hasil root cause analysis (RCA) sejauh mana peran tenaga Kesehatan lainnya yang terkait kejadian tersebut.
Hasil kajian, tindakan yang diambil, dan setiap dampak atas tanggung jawab pekerjaan didokumentasikan dalam file kredensial tenaga Kesehatan lainnya tersebut atau file lainnya.

\hypertarget{elemen-penilaian-kps-19}{%
\paragraph*{9. Elemen Penilaian KPS 19}\label{elemen-penilaian-kps-19}}
\addcontentsline{toc}{paragraph}{9. Elemen Penilaian KPS 19}

\begin{enumerate}
\def\labelenumi{\alph{enumi}.}
\tightlist
\item
  Rumah sakit telah melakukan penilaian kinerja tenaga Kesehatan lainnya secara periodik menggunakan format dan metode sesuai ketentuan yang ditetapkan rumah sakit.
\item
  Penilain kinerja tenaga Kesehatan lainnya meliputi pemenuhan uraian tugasnya dan perannya dalam pencapaian target indikator mutu yang diukur di unit tempatnya bekerja.
\item
  Pimpinan rumah sakit dan kepala unit telah berlaku adil (just culture) ketika ada temuan dalam kegiatan peningkatan mutu, laporan insiden keselamatan pasien atau manajemen risiko.
\item
  Rumah sakit telah mendokumentasikan hasil kajian, tindakan yang diambil, dan setiap dampak atas tanggung jawab pekerjaan tenaga kesehatan dalam file kredensial tenaga kesehatan lainnya.
\end{enumerate}

\hypertarget{manajemen-fasilitas-dan-keselamatan-mfk}{%
\section*{3. Manajemen Fasilitas dan Keselamatan (MFK)}\label{manajemen-fasilitas-dan-keselamatan-mfk}}
\addcontentsline{toc}{section}{3. Manajemen Fasilitas dan Keselamatan (MFK)}

\textbf{Gambaran Umum}

Fasilitas dan lingkungan dalam rumah sakit harus aman, berfungsi baik, dan memberikan lingkungan perawatan yang aman bagi pasien, keluarga, staf, dan pengunjung. Untuk mencapai tujuan itu maka fasilitas fisik, bangunan, prasarana dan peralatan kesehatan serta sumber daya lainnya harus dikelola secara efektif untuk mengurangi dan mengendalikan bahaya, risiko, mencegah kecelakaan, cidera dan penyakit akibat kerja. Dalam pengelolaan fasilitas dan lingkungan serta pemantauan keselamatan, rumah sakit menyusun program pengelolaan fasilitas dan lingkungan serta program pengelolaan risiko untuk pemantauan keselamatan di seluruh lingkungan rumah sakit.

Pengelolaan yang efektif mencakup perencanaan, pendidikan, dan pemantauan multidisiplin dimana pemimpin merencanakan ruang, peralatan, dan sumber daya yang diperlukan untuk mendukung layanan klinis yang disediakan secara aman dan efektif serta semua staf diedukasi mengenai fasilitas, cara mengurangi risiko, cara memantau dan melaporkan situasi yang berisiko termasuk melakukan penilaian risiko yang komprehensif di seluruh fasilitas yang dikembangkan dan dipantau berkala.

Bila di rumah sakit memiliki entitas non-rumah sakit atau tenant/penyewa lahan (seperti restoran, kantin, kafe, dan toko souvenir) maka rumah sakit wajib memastikan bahwa tenant/penyewa lahan tersebut mematuhi program pengelolaan fasilitas dan keselamatan, yaitu program keselamatan dan keamanan, program pengelolaan bahan berbahaya dan beracun, program penanganan bencana dan kedaruratan, serta proteksi kebakaran.

Rumah sakit perlu membentuk satuan kerja yang dapat mengelola, memantau dan memastikan fasilitas dan pengaturan keselamatan yang ada sehingga tidak menimbulkan potensi bahaya dan risiko yang akan berdampak buruk bagi pasien, staf dan pengunjung. Satuan kerja yang dibentuk dapat berupa Komite/Tim K3 RS yang disesuaikan dengan kebutuhan, ketersediaan sumber daya dan beban kerja rumah sakit. Rumah sakit harus memiliki program pengelolaan fasilitas dan keselamatan yang menjangkau seluruh fasilitas dan lingkungan rumah sakit.

Rumah sakit tanpa melihat ukuran dan sumber daya yang dimiliki harus mematuhi ketentuan dan peraturan perundangan yang berlaku sebagai bagian dari tanggung jawab mereka terhadap pasien, keluarga, staf, dan para pengunjung.
Fokus pada standar Manajemen Fasilitas dan Keamanan ini meliputi:

\begin{enumerate}
\def\labelenumi{\alph{enumi}.}
\tightlist
\item
  Kepemimpinan dan perencanaan;
\item
  Keselamatan;
\item
  Keamanan;
\item
  Pengelolaan Bahan Berbahaya dan Beracun (B3) dan Limbah B3;
\item
  Proteksi kebakaran;
\item
  Peralatan medis;
\item
  Sistim utilitas;
\item
  Penanganan kedaruratan dan bencana;
\item
  Konstruksi dan renovasi; dan
\item
  Pelatihan.
\end{enumerate}

\hypertarget{a.-kepemimpinan-dan-perencanaan}{%
\subsection*{a. Kepemimpinan dan perencanaan}\label{a.-kepemimpinan-dan-perencanaan}}
\addcontentsline{toc}{subsection}{a. Kepemimpinan dan perencanaan}

\hypertarget{standar-mfk-1}{%
\paragraph*{1. Standar MFK 1}\label{standar-mfk-1}}
\addcontentsline{toc}{paragraph}{1. Standar MFK 1}

Rumah sakit mematuhi persyaratan sesuai dengan peraturan perundang-undangan yang berkaitan dengan bangunan, prasarana dan peralatan medis rumah sakit.

\hypertarget{maksud-dan-tujuan-mfk-1}{%
\paragraph*{2. Maksud dan Tujuan MFK 1}\label{maksud-dan-tujuan-mfk-1}}
\addcontentsline{toc}{paragraph}{2. Maksud dan Tujuan MFK 1}

Rumah sakit harus mematuhi peraturan perundang- undangan termasuk mengenai bangunan dan proteksi kebakaran. Rumah sakit selalu menjaga fasilitas fisik dan lingkungan yang dimiliki dengan melakukan inspeksi fasilitas secara berkala dan secara proaktif mengumpulkan data serta membuat strategi untuk mengurangi risiko dan meningkatkan kualitas fasilitas keselamatan, kesehatan dan keamanan lingkungan pelayanan dan perawatan serta seluruh area rumah sakit.
Pimpinan rumah sakit dan penanggung jawab fasilitas keselamatan rumah sakit bertanggung jawab untuk mengetahui dan menerapkan hukum dan peraturan perundangan, keselamatan gedung dan kebakaran, dan persyaratan lainnya, seperti perizinan dan lisensi/sertifkat yang masih berlaku untuk fasilitas rumah sakit dan mendokumentasikan semua buktinya secara lengkap.
Perencanaan dan penganggaran untuk penggantian atau peningkatan fasilitas, sistem, dan peralatan yang diperlukan untuk memenuhi persyaratan yang berlaku atau seperti yang telah diidentifikasi berdasarkan pemantauan atau untuk memenuhi persyaratan yang berlaku dapat memberikan bukti pemeliharaan dan perbaikan.

\hypertarget{elemen-penilaian-mfk-1}{%
\paragraph*{3. Elemen Penilaian MFK 1}\label{elemen-penilaian-mfk-1}}
\addcontentsline{toc}{paragraph}{3. Elemen Penilaian MFK 1}

\begin{enumerate}
\def\labelenumi{\alph{enumi}.}
\tightlist
\item
  Rumah sakit menetapkan regulasi terkait Manajemen Fasilitas dan Keselamatan (MFK) yang meliputi poin a)
\end{enumerate}

\begin{itemize}
\item
  \begin{enumerate}
  \def\labelenumi{\alph{enumi})}
  \setcounter{enumi}{9}
  \tightlist
  \item
    pada gambaran umum.
  \end{enumerate}
\end{itemize}

\begin{enumerate}
\def\labelenumi{\alph{enumi}.}
\setcounter{enumi}{1}
\tightlist
\item
  Rumah sakit telah melengkapi izin-izin dan sertifikasi yang masih berlaku sesuai persyaratan peraturan perundang-undangan.
\item
  Pimpinan rumah sakit memenuhi perencanaan anggaran dan sumber daya serta memastikan rumah sakit memenuhi persyaratan perundang-undangan.
\end{enumerate}

\hypertarget{standar-mfk-2}{%
\paragraph*{4. Standar MFK 2}\label{standar-mfk-2}}
\addcontentsline{toc}{paragraph}{4. Standar MFK 2}

Rumah Sakit menetapkan penanggungjawab yang kompeten untuk mengawasi penerapan manajemen fasilitas dan keselamatan di rumah sakit.

\hypertarget{maksud-dan-tujuan-mfk-2}{%
\paragraph*{5. Maksud dan tujuan MFK 2}\label{maksud-dan-tujuan-mfk-2}}
\addcontentsline{toc}{paragraph}{5. Maksud dan tujuan MFK 2}

Untuk dapat mengelola fasilitas dan keselamatan di rumah sakit secara efektif, maka perlu di tetapkan penanggung jawab manajemen fasilitas dan keselamatan (MFK) yang bertanggungjawab langsung kepada Direktur. Penanggung jawab Manajemen Fasilitas dan Keselamatan (MFK) dapat berbentuk unit, tim, maupun komite sesuai dengan kondisi dan kompleksitas rumah sakit.

Penanggung jawab MFK harus memiliki kompetensi yang dibutuhkan serta berpengalaman untuk dapat melakukan pengelolaan dan pengawasan manajemen fasilitas dan keselamatan (MFK) seperti kesehatan dan keselamatan kerja, kesehatan lingkungan, farmasi, pengelolaan alat kesehatan, pengelolaan utilitas, dan unsur-unsur terkait lainnya sesuai kebutuhan rumah sakit.

Ruang lingkup tugas dan tanggung jawab penanggung jawab MFK meliputi:

\begin{enumerate}
\def\labelenumi{\alph{enumi}.}
\tightlist
\item
  Keselamatan: meliputi bangunan, prasarana, fasilitas, area konstruksi, lahan, dan peralatan rumah sakit tidak menimbulkan bahaya atau risiko bagi pasien, staf, atau pengunjung.
\item
  Keamanan: perlindungan dari kehilangan, kerusakan, gangguan, atau akses atau penggunaan yang tidak sah.
\item
  Bahan dan limbah berbahaya: Pengelolaan B3 termasuk penggunaan radioaktif serta bahan berbahaya lainnya dikontrol, dan limbah berbahaya dibuang dengan aman.
\item
  Proteksi kebakaran: Melakukan penilaian risiko yang berkelanjutan untuk meningkatkan perlindungan seluruh aset, properti dan penghuni dari kebakaran dan asap.
\item
  Penanganan kedaruratan dan bencana: Risiko diidentifikasi dan respons terhadap epidemi, bencana, dan keadaan darurat direncanakan dan efektif, termasuk evaluasi integritas struktural dan non struktural lingkungan pelayanan dan perawatan pasien.\\
\item
  Peralatan medis: Peralatan dipilih, dipelihara, dan digunakan dengan cara yang aman dan benar untuk mengurangi risiko.
\item
  Sistem utilitas: Listrik, air, gas medik dan sistem utilitas lainnya dipelihara untuk meminimalkan risiko kegagalan pengoperasian.
\item
  Konstruksi dan renovasi: Risiko terhadap pasien, staf, dan pengunjung diidentifikasi dan dinilai selama konstruksi, renovasi, pembongkaran, dan aktivitas pemeliharaan lainnya.
\item
  Pelatihan: Seluruh staf di rumah sakit dan para tenant/penyewa lahan dilatih dan memiliki pengetahuan tentang K3, termasuk penanggulangan kebakaran.
\item
  Pengawasan pada para tenant/penyewa lahan yang melakukan kegiatan di dalam area lingkungan rumah sakit.
\end{enumerate}

Penanggung jawab MFK menyusun Program Manajemen fasilitas dan keselamatan rumah sakit meliputi a) ? j) setiap tahun. Dalam program tersebut termasuk melakukan pengkajian dan penanganan risiko pada keselamatan, keamanan, pengelolaan B3, proteksi kebakaran, penanganan kedaruratan dan bencana, peralatan medis dan sistim utilitas.

Pengkajian dan penanganan risiko dimasukkan dalam daftar risiko manajemen fasilitas keselamatan (MFK). Berdasarkan daftar risiko tersebut, dibuat profil risiko MFK yang akan menjadi prioritas dalam pemantauan risiko di fasilitas dan lingkungan rumah sakit. Pengkajian, penanganan dan pemantauan risiko MFK tersebut akan diintegrasikan ke dalam daftar risiko rumah sakit untuk penyusunan program manajemen risiko rumah sakit.

Penanggung jawab MFK melakukan pengawasan terhadap manajemen fasilitas dan keselamatan yang meliputi:

\begin{enumerate}
\def\labelenumi{\alph{enumi}.}
\tightlist
\item
  Pengawasan semua aspek program manajemen fasilitas dan keselamatan seperti pengembangan rencana dan memberikan rekomendasi untuk ruangan, peralatan medis, teknologi, dan sumber daya;
\item
  Pengawasan pelaksanaan program secara konsisten dan berkesinambungan;
\item
  Pelaksanaan edukasi staf;
\item
  Pengawasan pelaksanaan pengujian/testing dan pemantauan program;
\item
  Penilaian ulang secara berkala dan merevisi program manajemen risiko fasilitas dan lingkungan jika dibutuhkan;
\item
  Penyerahan laporan tahunan kepada direktur rumah sakit;
\item
  Pengorganisasian dan pengelolaan laporan kejadian/insiden dan melakukan analisis, dan upaya perbaikan.
\end{enumerate}

\hypertarget{elemen-penilaian-mfk-2}{%
\paragraph*{6. Elemen Penilaian MFK 2}\label{elemen-penilaian-mfk-2}}
\addcontentsline{toc}{paragraph}{6. Elemen Penilaian MFK 2}

\begin{enumerate}
\def\labelenumi{\alph{enumi}.}
\tightlist
\item
  Rumah sakit telah menetapkan Penanggungjawab MFK yang memiliki kompetensi dan pengalaman dalam melakukan pengelolaan pada fasilitas dan keselamatan di lingkungan rumah sakit.
\item
  Penanggungjawab MFK telah menyusun Program Manajemen Fasilitas dan Keselamatan (MFK) yang meliputi poin a) ? j) dalam maksud dan tujuan.
\item
  Penanggungjawab MFK telah melakukan pengawasan dan evaluasi Manajemen Fasilitas dan Keselamatan (MFK) setiap tahunnya meliputi poin a) ? g) dalam maksud dan tujuan serta melakukan penyesuaian program apabila diperlukan.
\item
  Penerapan program Manajemen Fasilitas dan Keselamatan (MFK) pada tenant/penyewa lahan yang berada di lingkungan rumah sakit meliputi poin a) ? e) dalam maksud dan tujuan.
\end{enumerate}

\hypertarget{b.-keselamatan}{%
\subsection*{b. Keselamatan}\label{b.-keselamatan}}
\addcontentsline{toc}{subsection}{b. Keselamatan}

\hypertarget{standar-mfk-3}{%
\paragraph*{1. Standar MFK 3}\label{standar-mfk-3}}
\addcontentsline{toc}{paragraph}{1. Standar MFK 3}

Rumah sakit menerapkan Program Manajemen Fasilitas dan Keselamatan (MFK) terkait keselamatan di rumah sakit.

\hypertarget{maksud-dan-tujuan-mfk-3}{%
\paragraph*{2. Maksud dan tujuan MFK 3}\label{maksud-dan-tujuan-mfk-3}}
\addcontentsline{toc}{paragraph}{2. Maksud dan tujuan MFK 3}

Keselamatan di dalam standar ini adalah memberikan jaminan bahwa bangunan, prasarana, lingkungan, properti, teknologi medis dan informasi, peralatan, dan sistem tidak menimbulkan risiko fisik bagi pasien, keluarga, staf, dan pengunjung.
Program keselamatan dan Kesehatan kerja staf diintegrasikan dalam Program Manajemen fasilitas dan keselamatan terkait keselamatan sesuai ruang lingkup keselamatan yang telah dijelaskan diatas.

Pencegahan dan perencanaan penting untuk menciptakan fasilitas perawatan pasien termasuk area kerja staf yang aman. Perencanaan yang efektif membutuhkan kesadaran rumah sakit terhadap semua risiko yang ada di fasilitas. Tujuannya adalah untuk mencegah kecelakaan dan cedera serta untuk menjaga kondisi yang aman, dan menjamin keselamatan bagi pasien, staf, dan lainnya, seperti keluarga, kontraktor, vendor, relawan, pengunjung, peserta pelatihan, dan peserta didik.

Rumah sakit mengembangkan dan menerapkan program keselamatan serta mendokumentasikan hasil inspeksi fisik yang dilakukan. Penilaian risiko mempertimbangkan tinjauan proses dan evaluasi layanan baru dan terencana yang dapat menimbulkan risiko keselamatan. Penting untuk melibatkan tim multidisiplin saat melakukan inspeksi keselamatan di rumah sakit.

Rumah sakit menerapkan proses untuk mengelola dan memantau keselamatan (merupakan bagian dari program Manajemen Fasilitas Keselamatan/MFK pada standar MFK 1 yang meliputi:

\begin{enumerate}
\def\labelenumi{\alph{enumi}.}
\tightlist
\item
  Pengelolaan risiko keselamatan di lingkungan rumah sakit secara komprehensif
\item
  Penyediaan fasilitas pendukung yang aman untuk mencegah kecelakaan dan cedera, penyakit akibat kerja, mengurangi bahaya dan risiko, serta mempertahankan kondisi aman bagi pasien, keluarga, staf, dan pengunjung; dan
\item
  Pemeriksaan fasilitas dan lingkungan (ronde fasilitas) secara berkala dan dilaporkan sebagai dasar perencanaan anggaran untuk perbaikan, penggantian atau ?upgrading?.
\end{enumerate}

\hypertarget{elemen-penilaian-mfk-3}{%
\paragraph*{3. Elemen Penilaian MFK 3}\label{elemen-penilaian-mfk-3}}
\addcontentsline{toc}{paragraph}{3. Elemen Penilaian MFK 3}

\begin{enumerate}
\def\labelenumi{\alph{enumi}.}
\tightlist
\item
  Rumah sakit menerapkan proses pengelolaan keselamatan rumah sakit meliputi poin a) - c) pada maksud dan tujuan.
\item
  Rumah sakit telah mengintegrasikan program Kesehatan dan keselamatan kerja staf ke dalam program manajemen fasilitas dan keselamatan.
\item
  Rumah sakit telah membuat pengkajian risiko secara proaktif terkait keselamatan di rumah sakit setiap tahun yang didokumentasikan dalam daftar risiko/risk register.
\item
  Rumah sakit telah melakukan pemantauan risiko keselamatan dan dilaporkan setiap 6 (enam) bulan kepada piminan rumah sakit.
\end{enumerate}

\hypertarget{c.-keamanan}{%
\subsection*{c.~Keamanan}\label{c.-keamanan}}
\addcontentsline{toc}{subsection}{c.~Keamanan}

\hypertarget{standard-mfk-4}{%
\paragraph*{1. Standard MFK 4}\label{standard-mfk-4}}
\addcontentsline{toc}{paragraph}{1. Standard MFK 4}

Rumah sakit menerapkan Program Manajemen Fasilitas dan Keselamatan (MFK) terkait keamanan di rumah sakit.

\hypertarget{maksud-dan-tujuan-mfk-4}{%
\paragraph*{2. Maksud dan tujuan MFK 4}\label{maksud-dan-tujuan-mfk-4}}
\addcontentsline{toc}{paragraph}{2. Maksud dan tujuan MFK 4}

Keamanan adalah perlindungan terhadap properti milik rumah sakit, pasien, staf, keluarga, dan pengunjung dari bahaya kehilangan, kerusakan, atau pengrusakan oleh orang yang tidak berwenang. Contoh kerentanan dan ancaman yang terkait dengan risiko keamanan termasuk kekerasan di tempat kerja, penculikan bayi, pencurian, dan akses tidak terkunci/tidak aman ke area terlarang di rumah sakit. Insiden keamanan dapat disebabkan oleh individu baik dari luar maupun dalam rumah sakit.

Area yang berisiko seperti unit gawat darurat, ruangan neonatus/bayi, ruang operasi, farmasi, ruang rekam medik, ruangan IT harus diamankan dan dipantau. Anak-anak, orang dewasa, lanjut usia, dan pasien rentan yang tidak dapat melindungi diri mereka sendiri atau memberi isyarat untuk bantuan harus dilindungi dari bahaya. Area terpencil atau terisolasi dari fasilitas dan lingkungan misalnya tempat parkir, mungkin memerlukan kamera keamanan (CCTV).

Rumah sakit menerapkan proses untuk mengelola dan memantau keamanan (merupakan bagian dari program Manajemen Fasilitas dan Keselamatan (MFK) pada standar MFK 1 yang meliputi:

\begin{enumerate}
\def\labelenumi{\alph{enumi}.}
\tightlist
\item
  Menjamin lingkungan yang aman dengan memberikan identitas/tanda pengenal (badge nama sementara atau tetap) pada pasien, staf, pekerja kontrak, tenant/penyewa lahan, keluarga (penunggu pasien), atau pengunjung (pengunjung di luar jam besuk dan tamu rumah sakit) sesuai dengan regulasi rumah sakit;
\item
  Melakukan pemeriksaan dan pemantauan keamanan fasilitas dan lingkungan secara berkala dan membuat tindak lanjut perbaikan;
\item
  Pemantauan pada daerah berisiko keamanan sesuai penilaian risiko di rumah sakit. Pemantauan dapat dilakukan dengan penempatan petugas keamanan (sekuriti) dan atau memasang kamera sistem CCTV yang dapat dipantau oleh sekuriti;
\item
  Melindungi semua individu yang berada di lingkungan rumah sakit terhadap kekerasan, kejahatan dan ancaman; dan
\item
  Menghindari terjadinya kehilangan, kerusakan, atau pengrusakan barang milik pribadi maupun rumah sakit.
\end{enumerate}

\hypertarget{elemen-penilaian-mfk-4}{%
\paragraph*{3. Elemen Penilaian MFK 4}\label{elemen-penilaian-mfk-4}}
\addcontentsline{toc}{paragraph}{3. Elemen Penilaian MFK 4}

\begin{enumerate}
\def\labelenumi{\alph{enumi}.}
\tightlist
\item
  Rumah sakit menerapkan proses pengelolaan keamanan dilingkungan rumah sakit meliputi poin a) -
\item
  pada maksud dan tujuan.
\item
  Rumah sakit telah membuat pengkajian risiko secara proaktif terkait keamanan di rumah sakit setiap tahun yang didokumentasikan dalam daftar risiko/risk register.
\item
  Rumah sakit telah membuat pengkajian risiko secara proaktif terkait keselamatan di rumah sakit. (Daftar risiko/risk register).
\item
  Rumah sakit telah melakukan pemantauan risiko keamanan dan dilaporkan setiap 6 (enam) bulan kepada Direktur rumah sakit.
\end{enumerate}

\hypertarget{d.-pengelolaan-bahan-berbahaya-dan-beracun-b3-dan-limbah-b3}{%
\subsection*{d.~Pengelolaan Bahan Berbahaya dan Beracun (B3) dan Limbah B3}\label{d.-pengelolaan-bahan-berbahaya-dan-beracun-b3-dan-limbah-b3}}
\addcontentsline{toc}{subsection}{d.~Pengelolaan Bahan Berbahaya dan Beracun (B3) dan Limbah B3}

\hypertarget{standar-mfk-5}{%
\paragraph*{1. Standar MFK 5}\label{standar-mfk-5}}
\addcontentsline{toc}{paragraph}{1. Standar MFK 5}

Rumah sakit menetapkan dan menerapkan pengelolaan Bahan Berbahaya dan Beracun (B3) serta limbah B3 sesuai dengan peraturan perundang-undangan.

\hypertarget{maksud-dan-tujuan-mfk-5}{%
\paragraph*{2. Maksud dan tujuan MFK 5}\label{maksud-dan-tujuan-mfk-5}}
\addcontentsline{toc}{paragraph}{2. Maksud dan tujuan MFK 5}

Rumah sakit mengidentifikasi, menganalisis dan mengendalikan seluruh bahan berbahaya dan beracun dan limbahnya di rumah sakit sesuai dengan standar keamanan dan peraturan perundang-undangan.

Rumah sakit melakukan identifikasi menyeluruh untuk semua area di mana bahan berbahaya berada dan harus mencakup informasi tentang jenis setiap bahan berbahaya yang disimpan, jumlah (misalnya, perkiraan atau rata-rata) dan lokasinya di rumah sakit. Dokumentasi ini juga harus membahas jumlah maksimum yang diperbolehkan untuk menyimpan bahan berbahaya di area kerja (maximum quantity on hand). Misalnya, jika bahan sangat mudah terbakar atau beracun, ada batasan jumlah bahan yang dapat disimpan di area kerja. Inventarisasi bahan berbahaya dibuat dan diperbarui, setiap tahun, untuk memantau perubahan bahan berbahaya yang digunakan dan disimpan.

Kategori Bahan Berbahaya dan Beracun (B3) sesuai WHO meliputi:

\begin{enumerate}
\def\labelenumi{\alph{enumi}.}
\tightlist
\item
  Infeksius;
\item
  Patologis dan anatomi;
\item
  Farmasi;
\item
  Bahan kimia;
\item
  Logam berat;
\item
  Kontainer bertekanan;
\item
  Benda tajam;
\item
  Genotoksik/sitotoksik; dan
\item
  Radioaktif.
\end{enumerate}

Proses pengelolaan bahan berbahaya beracun dan limbahnya di rumah sakit (merupakan bagian dari program Manajemen Fasilitas dan Keselamatan (MFK) pada standar MFK 1 meliputi:

\begin{enumerate}
\def\labelenumi{\alph{enumi}.}
\tightlist
\item
  Inventarisasi B3 serta limbahnya yang meliputi jenis, jumlah, simbol dan lokasi;
\item
  Penanganan, penyimpanan, dan penggunaan B3 serta limbahnya;
\item
  Penggunaan alat pelindung diri (APD) dan prosedur penggunaan, prosedur bila terjadi tumpahan, atau paparan/pajanan;
\item
  Pelatihan yang dibutuhkan oleh staf yang menangani B3;
\item
  Pemberian label/rambu-rambu yang tepat pada B3 serta limbahnya;
\item
  Pelaporan dan investigasi dari tumpahan, eksposur (terpapar), dan insiden lainnya;
\item
  Dokumentasi, termasuk izin, lisensi, atau persyaratan peraturan lainnya; dan
\item
  Pengadaan/pembelian B3 dan pemasok (supplier) wajib melampirkan Lembar Data Keselamatan. Informasi yang tercantum di lembar data keselamatan diedukasi kepada staf rumah sakit, terutama kepada staf terdapat penyimpanan B3 di unitnya.
\end{enumerate}

Informasi mengenai prosedur penanganan bahan berbahaya dan limbah dengan cara yang aman harus segera tersedia setiap saat termasuk prosedur penanganan tumpahan. Jika terjadi tumpahan bahan berbahaya, rumah sakit memiliki prosedur untuk menanggapi dan mengelola tumpahan dan paparan yang termasuk menyediakan kit tumpahan untuk jenis dan ukuran potensi tumpahan serta proses pelaporan tumpahan dan paparan.

Rumah sakit menerapkan prosedur untuk menanggapi paparan bahan berbahaya, termasuk pertolongan pertama seperti akses ke tempat pencuci mata (eye washer) mungkin diperlukan untuk pembilasan segera dan terus menerus untuk mencegah atau meminimalkan cedera. Rumah sakit harus melakukan penilaian risiko untuk mengidentifikasi di mana saja lokasi pencuci mata diperlukan, dengan mempertimbangkan sifat fisik bahan kimia berbahaya yang digunakan, bagaimana bahan kimia ini digunakan oleh staf untuk melakukan aktivitas kerja mereka, dan penggunaan peralatan pelindung diri oleh staf. Alternatif untuk lokasi pencuci mata sesuai pada jenis risiko dan potensi eksposur. Rumah sakit harus memastikan pemeliharaan pencuci mata yang tepat, termasuk pembersihan mingguan dan pemeliharaan preventif.

\hypertarget{elemen-penilaian-mfk-5}{%
\paragraph*{3. Elemen Penilaian MFK 5}\label{elemen-penilaian-mfk-5}}
\addcontentsline{toc}{paragraph}{3. Elemen Penilaian MFK 5}

\begin{enumerate}
\def\labelenumi{\alph{enumi}.}
\tightlist
\item
  Rumah sakit telah melaksanakan proses pengelolaan B3 meliputi poin a) - h) pada maksud dan tujuan.
\item
  Rumah sakit telah membuat pengkajian risiko secara proaktif terkait pengelolaan B3 di rumah sakit setiap tahun yang didokumentasikan dalam daftar risiko/risk register.
\item
  Di area tertentu yang rawan terhadap pajanan telah dilengkapi dengan eye washer/body washer yang berfungsi dan terpelihara baik dan tersedia kit tumpahan/spill kit sesuai ketentuan.
\item
  Staf dapat menjelaskan dan atau memperagakan penanganan tumpahan B3.
\item
  Staf dapat menjelaskan dan atau memperagakan tindakan, kewaspadaan, prosedur dan partisipasi dalam penyimpanan, penanganan dan pembuangan limbah B3.
\end{enumerate}

\hypertarget{standar-mfk-5.1}{%
\paragraph*{4. Standar MFK 5.1}\label{standar-mfk-5.1}}
\addcontentsline{toc}{paragraph}{4. Standar MFK 5.1}

Rumah sakit mempunyai sistem pengelolaan limbah B3 cair dan padat sesuai dengan peraturan perundang-undangan.

\hypertarget{maksud-dan-tujuan-mfk-5.1}{%
\paragraph*{5. Maksud dan Tujuan MFK 5.1}\label{maksud-dan-tujuan-mfk-5.1}}
\addcontentsline{toc}{paragraph}{5. Maksud dan Tujuan MFK 5.1}

Rumah sakit juga menetapkan jenis limbah berbahaya yang dihasilkan oleh rumah sakit dan mengidentifikasi pembuangannya (misalnya, kantong/tempat sampah yang diberi kode warna dan diberi label).
Sistem penyimpanan dan pengelolaan limbah B3 mengikuti ketentuan peraturan perundangan-undangan.

Untuk pembuangan sementara limbah B-3, rumah sakit agar memenuhi persyaratan fasilitas pembuangan sementara limbah B-3 sebagai berikut:

\begin{enumerate}
\def\labelenumi{\alph{enumi}.}
\tightlist
\item
  Lantai kedap (impermeable), berlantai beton atau semen dengan sistem drainase yang baik, serta mudah dibersihkan dan dilakukan desinfeksi;
\item
  Tersedia sumber air atau kran air untuk pembersihan yang dilengkapi dengan sabun cair;
\item
  Mudah diakses untuk penyimpanan limbah;
\item
  Dapat dikunci untuk menghindari akses oleh pihak yang tidak berkepentingan;
\item
  Mudah diakses oleh kendaraan yang akan mengumpulkan atau mengangkut limbah;
\item
  Terlindungi dari sinar matahari, hujan, angin kencang, banjir, dan faktor lain yang berpotensi menimbulkan kecelakaan atau bencana kerja;
\item
  Terlindung dari hewan: kucing, serangga, burung, dan lain-lainnya;
\item
  Dilengkapi dengan ventilasi dan pencahayaan yang baik serta memadai;
\item
  Berjarak jauh dari tempat penyimpanan atau penyiapan makanan;
\item
  Peralatan pembersihan, alat pelindung diri/APD (antara lain masker, sarung tangan, penutup kepala, goggle, sepatu boot, serta pakaian pelindung) dan wadah atau kantong limbah harus diletakkan sedekat- dekatnya dengan lokasi fasilitas penyimpanan; dan
\item
  Dinding, lantai, dan juga langit-langit fasilitas penyimpanan senantiasa dalam keadaan bersih termasuk pembersihan lantai setiap hari.
\end{enumerate}

Untuk limbah berwujud cair dapat dilakukan di Instalasi Pengolahan Air Limbah (IPAL) dari fasilitas pelayanan kesehatan.
Tujuan pengolahan limbah medis adalah mengubah karakteristik biologis dan/atau kimia limbah sehingga potensi bahayanya terhadap manusia berkurang atau tidak ada.

Bila rumah sakit mengolah limbah B-3 sendiri maka wajib mempunyai izin mengolah limbah B-3. Namun, bila pengolahan B-3 dilaksanakan oleh pihak ketiga maka pihak ketiga tersebut wajib mempunyai izin sebagai pengolah B-3. Pengangkut/transporter dan pengolah limbah B3 dapat dilakukan oleh institusi yang berbeda.

\hypertarget{elemen-penilaian-mfk-5.1}{%
\paragraph*{6. Elemen Penilaian MFK 5.1}\label{elemen-penilaian-mfk-5.1}}
\addcontentsline{toc}{paragraph}{6. Elemen Penilaian MFK 5.1}

\begin{enumerate}
\def\labelenumi{\alph{enumi}.}
\tightlist
\item
  Rumah sakit melakukan penyimpanan limbah B3 sesuai poin a) ? k) pada maksud dan tujuan.
\item
  Rumah sakit mengolah limbah B3 padat secara mandiri atau menggunakan pihak ketiga yang berizin termasuk untuk pemusnahan limbah B3 cair yang tidak bisa dibuang ke IPAL.
\item
  Rumah sakit mengelola limbah B3 cair sesuai peraturan perundang-undangan.
\end{enumerate}

\hypertarget{e.-proteksi-kebakaran}{%
\subsection*{e. Proteksi kebakaran}\label{e.-proteksi-kebakaran}}
\addcontentsline{toc}{subsection}{e. Proteksi kebakaran}

\hypertarget{standar-mfk-6}{%
\paragraph*{1. Standar MFK 6}\label{standar-mfk-6}}
\addcontentsline{toc}{paragraph}{1. Standar MFK 6}

Rumah sakit menerapkan proses untuk pencegahan, penanggulangan bahaya kebakaran dan penyediaan sarana jalan keluar yang aman dari fasilitas sebagai respons terhadap kebakaran dan keadaan darurat lainnya.

\hypertarget{maksud-dan-tujuan-mfk-6}{%
\paragraph*{2. Maksud dan tujuan MFK 6}\label{maksud-dan-tujuan-mfk-6}}
\addcontentsline{toc}{paragraph}{2. Maksud dan tujuan MFK 6}

Rumah sakit harus waspada terhadap risiko kebakaran, karena kebakaran merupakan risiko yang selalu ada dalam lingkungan perawatan dan pelayanan kesehatan sehingga setiap rumah sakit perlu memastikan agar semua yang ada di rumah sakit aman dan selamat apabila terjadi kebakaran termasuk bahaya dari asap.

Proteksi kebakaran juga termasuk keadaan darurat non- kebakaran misalnya kebocoran gas beracun yang dapat mengancam sehingga perlu dievakuasi. Rumah sakit perlu melakukan penilaian terus menerus untuk memenuhi regulasi keamanan dan proteksi kebakaran sehingga secara efektif dapat mengidentifikasi, analisis, pengendalian risiko sehingga dapat dan meminimalkan risiko.

Pengkajian risiko kebakaran Fire Safety Risk Assessment (FSRA) merupakan salah satu upaya untuk menilai risiko keselamatan kebakaran.

Rumah sakit melakukan pengkajian risiko kebakaran meliputi:

\begin{enumerate}
\def\labelenumi{\alph{enumi}.}
\tightlist
\item
  Pemisah/kompartemen bangunan untuk mengisolasi asap/api.
\item
  Laundry/binatu, ruang linen, area berbahaya termasuk ruang di atas plafon.
\item
  Tempat pengelolaan sampah.
\item
  Pintu keluar darurat kebakaran (emergency exit).
\item
  Dapur termasuk peralatan memasak penghasil minyak.
\item
  Sistem dan peralatan listrik darurat/alternatif serta jalur kabel dan instalasi listrik.
\item
  Penyimpanan dan penanganan bahan yang berpotensi mudah terbakar (misalnya, cairan dan gas mudah terbakar, gas medis yang mengoksidasi seperti oksigen dan dinitrogen oksida), ruang penyimpanan oksigen dan komponennya dan vakum medis.
\item
  Prosedur dan tindakan untuk mencegah dan mengelola kebakaran akibat pembedahan.
\item
  Bahaya kebakaran terkait dengan proyek konstruksi, renovasi, atau pembongkaran.
\end{enumerate}

Berdasarkan hasil pengkajian risiko kebakaran, rumah sakit menerapkan proses proteksi kebakaran (yang merupakan bagian dari Manajemen Fasilitas dan Keamanan (MFK) pada standar MFK 1 untuk:

\begin{enumerate}
\def\labelenumi{\alph{enumi}.}
\tightlist
\item
  Pencegahan kebakaran melalui pengurangan risiko seperti penyimpanan dan penanganan bahan-bahan mudah terbakar secara aman, termasuk gas-gas medis yang mudah terbakar seperti oksigen, penggunaan bahan yang non combustible, bahan yang waterbase dan lainnya yang dapat mengurangi potensi bahaya kebakaran;
\item
  Pengendalian potensi bahaya dan risiko kebakaran yang terkait dengan konstruksi apapun di atau yang berdekatan dengan bangunan yang ditempati pasien;
\item
  Penyediaan rambu dan jalan keluar (evakuasi) yang aman serta tidak terhalang apabila terjadi kebakaran;
\item
  Penyediaan sistem peringatan dini secara pasif meliputi, detektor asap (smoke detector), detektor panas (heat detector), alarm kebakaran, dan lain- lainnya;
\item
  Penyediaan fasilitas pemadaman api secara aktif meliputi APAR, hidran, sistem sprinkler, dan lain- lainnya; dan
\item
  Sistem pemisahan (pengisolasian) dan kompartemenisasi pengendalian api dan asap.
\end{enumerate}

Risiko dapat mencakup peralatan, sistem, atau fitur lain untuk proteksi kebakaran yang rusak, terhalang, tidak berfungsi, atau perlu disingkirkan. Risiko juga dapat diidentifikasi dari proyek konstruksi, kondisi penyimpanan yang berbahaya, kerusakan peralatan dan sistem, atau pemeliharaan yang diperlukan yang berdampak pada sistem keselamatan kebakaran.

Rumah sakit harus memastikan bahwa semua yang di dalam faslitas dan lingkungannya tetap aman jika terjadi kebakaran, asap, dan keadaan darurat non-kebakaran.

Struktur dan desain fasilitas perawatan kesehatan dapat membantu mencegah, mendeteksi, dan memadamkan kebakaran serta menyediakan jalan keluar yang aman dari fasilitas tersebut.

\hypertarget{elemen-penilaian-mfk-6}{%
\paragraph*{3. Elemen Penilaian MFK 6}\label{elemen-penilaian-mfk-6}}
\addcontentsline{toc}{paragraph}{3. Elemen Penilaian MFK 6}

\begin{enumerate}
\def\labelenumi{\alph{enumi}.}
\tightlist
\item
  Rumah sakit telah melakukan pengkajian risiko kebakaran secara proaktif meliputi poin a) ? i) dalam maksud dan tujuan setiap tahun yang didokumentasikan dalam daftar risiko/risk register.
\item
  Rumah sakit telah menerapkan proses proteksi kebakaran yang meliputi poin a) ? f) pada maksud dan tujuan.
\item
  Rumah sakit menetapkan kebijakan dan melakukan pemantauan larangan merokok di seluruh area rumah sakit.
\item
  Rumah sakit telah melakukan pengkajian risiko proteksi kebakaran.
\item
  Rumah sakit memastikan semua staf memahami proses proteksi kebakaran termasuk melakukan pelatihan penggunaan APAR, hidran dan simulasi kebakaran setiap tahun.
\item
  Peralatan pemadaman kebakaran aktif dan sistem peringatan dini serta proteksi kebakaran secara pasif telah diinventarisasi, diperiksa, di ujicoba dan dipelihara sesuai dengan peraturan perundang- undangan dan didokumentasikan.
\end{enumerate}

\hypertarget{f.-peralatan-medis}{%
\subsection*{f.~Peralatan medis}\label{f.-peralatan-medis}}
\addcontentsline{toc}{subsection}{f.~Peralatan medis}

\hypertarget{standar-mfk-7}{%
\paragraph*{1. Standar MFK 7}\label{standar-mfk-7}}
\addcontentsline{toc}{paragraph}{1. Standar MFK 7}

Rumah sakit menetapkan dan menerapkan proses pengelolaan peralatan medik.

\hypertarget{maksud-dan-tujuan-mfk-7}{%
\paragraph*{2. Maksud dan tujuan MFK 7}\label{maksud-dan-tujuan-mfk-7}}
\addcontentsline{toc}{paragraph}{2. Maksud dan tujuan MFK 7}

Untuk menjamin peralatan medis dapat digunakan dan layak pakai maka rumah sakit perlu melakukan pengelolaan peralatan medis dengan baik dan sesuai standar serta peraturan perundangan yang berlaku.

Proses pengelolaan peralatan medis (yang merupakan bagian dari progam Manajemen Fasilitas dan Keselamatan/MFK pada standar MFK 1 meliputi:

a Identifikasi dan penilaian kebutuhan alat medik dan uji fungsi sesuai ketentuan penerimaan alat medik baru.
b. Inventarisasi seluruh peralatan medis yang dimiliki oleh rumah sakit dan peralatan medis kerja sama operasional (KSO) milik pihak ketiga; serta peralatan medik yang dimiliki oleh staf rumah sakit jika ada Inspeksi peralatan medis sebelum digunakan.
c.~Pemeriksaan peralatan medis sesuai dengan penggunaan dan ketentuan pabrik secara berkala.
d.~Pengujian yang dilakukan terhadap alat medis untuk memperoleh kepastian tidak adanya bahaya yang ditimbulkan sebagai akibat penggunaan alat.
e. Rumah sakit melakukan pemeliharaan preventif dan kalibrasi, dan seluruh prosesnya didokumentasikan.

Rumah Sakit menetapkan staf yang kompeten untuk melaksanakan kegiatan ini. Hasil pemeriksaan (inspeksi), uji fungsi, dan pemeliharaan serta kalibrasi didokumentasikan. Hal ini menjadi dasar untuk menyusun perencanaan dan pengajuan anggaran untuk penggantian, perbaikan, peningkatan (upgrade), dan perubahan lain.

Rumah sakit memiliki sistem untuk memantau dan bertindak atas pemberitahuan bahaya peralatan medis, penarikan kembali, insiden yang dapat dilaporkan, masalah, dan kegagalan yang dikirimkan oleh produsen, pemasok, atau badan pengatur. Rumah sakit harus mengidentifikasi dan mematuhi hukum dan peraturan yang berkaitan dengan pelaporan insiden terkait peralatan medis. Rumah sakit melakukan analisis akar masalah dalam menanggapi setiap kejadian sentinel.

Rumah sakit mempunyai proses identifikasi, penarikan (recall) dan pengembalian, atau pemusnahan produk dan peralatan medis yang ditarik kembali oleh pabrik atau pemasok. Ada kebijakan atau prosedur yang mengatur penggunaan setiap produk atau peralatan yang ditarik kembali (under recall).

\hypertarget{elemen-penilaian-mfk-7}{%
\paragraph*{3. Elemen Penilaian MFK 7}\label{elemen-penilaian-mfk-7}}
\addcontentsline{toc}{paragraph}{3. Elemen Penilaian MFK 7}

\begin{enumerate}
\def\labelenumi{\alph{enumi}.}
\tightlist
\item
  Rumah sakit telah menerapkan proses pengelolaan peralatan medik yang digunakan di rumah sakit meliputi poin a) - e) pada maksud dan tujuan.
\item
  Rumah sakit menetapkan penanggung jawab yang kompeten dalam pengelolaan dan pengawasan peralatan medik di rumah sakit.
\item
  Rumah sakit telah melakukan pengkajian risiko peralatan medik secara proaktif setiap tahun yang didokumentasikan dalam daftar risiko/risk register.
\item
  Terdapat bukti perbaikan yang dilakukan oleh pihak yang berwenang dan kompeten.
\item
  Rumah sakit telah menerapkan pemantauan, pemberitahuan kerusakan (malfungsi) dan penarikan (recall) peralatan medis yang membahayakan pasien.
\item
  Rumah sakit telah melaporkan insiden keselamatan pasien terkait peralatan medis sesuai dengan peraturan perundang-undangan.
\end{enumerate}

\hypertarget{g.-sistim-utilitas}{%
\subsection*{g. Sistim utilitas}\label{g.-sistim-utilitas}}
\addcontentsline{toc}{subsection}{g. Sistim utilitas}

\hypertarget{standar-mfk-8}{%
\paragraph*{1. Standar MFK 8}\label{standar-mfk-8}}
\addcontentsline{toc}{paragraph}{1. Standar MFK 8}

Rumah sakit menetapkan dan melaksanakan proses untuk memastikan semua sistem utilitas (sistem pendukung) berfungsi efisien dan efektif yang meliputi pemeriksaan, pemeliharaan, dan perbaikan sistem utilitas.

\hypertarget{maksud-dan-tujuan-mfk-8}{%
\paragraph*{2. Maksud dan Tujuan MFK 8}\label{maksud-dan-tujuan-mfk-8}}
\addcontentsline{toc}{paragraph}{2. Maksud dan Tujuan MFK 8}

Definisi utilitas adalah sistem dan peralatan untuk mendukung layanan penting bagi keselamatan pasien. Sistem utilitas disebut juga sistem penunjang yang mencakup jaringan listrik, air, ventilasi dan aliran udara, gas medik dan uap panas. Sistem utilitas yang berfungsi efektif akan menunjang lingkungan asuhan pasien yang aman. Selain sistim utilitas perlu juga dilakukan pengelolaan komponen kritikal terhadap listrik, air dan gas medis misalnya perpipaan, saklar, relay/penyambung, dan lain-lainnya.

Asuhan pasien rutin dan darurat berjalan selama 24 jam terus menerus, setiap hari, dalam waktu 7 (tujuh) hari dalam seminggu. Jadi, kesinambungan fungsi utilitas merupakan hal esensial untuk memenuhi kebutuhan pasien. Termasuk listrik dan air harus tersedia selama 24 jam terus menerus, setiap hari, dalam waktu 7 (tujuh) hari dalam seminggu.

Pengelolaan sistim utilitas yang baik dapat mengurangi potensi risiko pada pasien maupun staf. Sebagai contoh, kontaminasi berasal dari sampah di area persiapan makanan, kurangnya ventilasi di laboratorium klinik, tabung oksigen yang disimpan tidak terjaga dengan baik, kabel listrik bergelantungan, serta dapat menimbulkan bahaya. Untuk menghindari kejadian ini maka rumah sakit harus melakukan pemeriksaan berkala dan pemeliharan preventif.
Rumah sakit perlu menerapkan proses pengelolaan sistem utilitas dan komponen kritikal (yang merupakan bagian dari progam

Manajemen Fasilitas dan Keselamatan (MFK) pada standar MFK 1 sekurang- kurangnya meliputi:

a Ketersediaan air dan listrik 24 jam setiap hari dan dalam waktu 7 (tujuh) hari dalam seminggu secara terus menerus;
b. Membuat daftar inventaris komponen-komponen sistem utilitas, memetakan pendistribusiannya, dan melakukan update secara berkala;
c.~Pemeriksaan, pemeliharaan, serta perbaikan semua komponen utilitas yang ada di daftar inventaris;
d.~Jadwal pemeriksaan, uji fungsi, dan pemeliharaan semua sistem utilitas berdasar atas kriteria seperti rekomendasi dari pabrik, tingkat risiko, dan pengalaman rumah sakit; dan
e. Pelabelan pada tuas-tuas kontrol sistem utilitas untuk membantu pemadaman darurat secara keseluruhan atau sebagian saat terjadi kebakaran.

\hypertarget{elemen-penilaian-mfk-8}{%
\paragraph*{3. Elemen Penilaian MFK 8}\label{elemen-penilaian-mfk-8}}
\addcontentsline{toc}{paragraph}{3. Elemen Penilaian MFK 8}

\begin{enumerate}
\def\labelenumi{\alph{enumi}.}
\tightlist
\item
  Rumah sakit telah menerapkan proses pengelolaan sistem utilitas yang meliputi poin a) - e) dalam maksud dan tujuan.
\item
  Rumah sakit telah melakukan pengkajian risiko sistim utilitas dan komponen kritikalnya secara proaktif setiap tahun yang didokumentasikan dalam daftar risiko/risk register.
\end{enumerate}

\hypertarget{standar-mfk-8.1}{%
\paragraph*{4. Standar MFK 8.1}\label{standar-mfk-8.1}}
\addcontentsline{toc}{paragraph}{4. Standar MFK 8.1}

Dilakukan pemeriksaan, pemeliharaan, dan perbaikan sistem utilitas.

\hypertarget{maksud-dan-tujuan-mfk-8.1}{%
\paragraph*{5. Maksud dan Tujuan MFK 8.1}\label{maksud-dan-tujuan-mfk-8.1}}
\addcontentsline{toc}{paragraph}{5. Maksud dan Tujuan MFK 8.1}

Rumah sakit harus mempunyai daftar inventaris lengkap sistem utilitas dan menentukan komponen yang berdampak pada bantuan hidup, pengendalian infeksi, pendukung lingkungan, dan komunikasi. Proses menajemen utilitas menetapkan pemeliharaan utilitas untuk memastikan utilitas pokok/penting seperti air, listrik, sampah, ventilasi, gas medik, lift agar dijaga, diperiksa berkala, dipelihara, dan diperbaiki.

\hypertarget{elemen-penilaian-mfk-8.1}{%
\paragraph*{6. Elemen Penilaian MFK 8.1}\label{elemen-penilaian-mfk-8.1}}
\addcontentsline{toc}{paragraph}{6. Elemen Penilaian MFK 8.1}

\begin{enumerate}
\def\labelenumi{\alph{enumi}.}
\tightlist
\item
  Rumah sakit menerapkan proses inventarisasi sistim utilitas dan komponen kritikalnya setiap tahun.
\item
  Sistem utilitas dan komponen kritikalnya telah diinspeksi secara berkala berdasarkan ketentuan rumah sakit.
\item
  Sistem utilitas dan komponen kritikalnya diuji secara berkala berdasar atas kriteria yang sudah ditetapkan.
\item
  Sistem utilitas dan komponen kritikalnya dipelihara berdasar atas kriteria yang sudah ditetapkan.
\item
  Sistem utilitas dan komponen kritikalnya diperbaiki bila diperlukan.
\end{enumerate}

\hypertarget{standar-mfk-8.2}{%
\paragraph*{7. Standar MFK 8.2}\label{standar-mfk-8.2}}
\addcontentsline{toc}{paragraph}{7. Standar MFK 8.2}

Sistem utilitas rumah sakit menjamin tersedianya air bersih dan listrik sepanjang waktu serta menyediakan sumber cadangan/alternatif persediaan air dan tenaga listrik jika terjadi terputusnya sistem, kontaminasi, atau kegagalan.

\hypertarget{maksud-dan-tujuan-mfk-8.2}{%
\paragraph*{8. Maksud dan Tujuan MFK 8.2}\label{maksud-dan-tujuan-mfk-8.2}}
\addcontentsline{toc}{paragraph}{8. Maksud dan Tujuan MFK 8.2}

Pelayanan pasien dilakukan selama 24 jam terus menerus, setiap hari dalam seminggu di rumah sakit. Rumah sakit mempunyai kebutuhan sistem utilitas yang berbeda-beda bergantung pada misi rumah sakit, kebutuhan pasien, dan sumber daya. Walaupun begitu, pasokan sumber air bersih dan listrik terus menerus sangat penting untuk memenuhi kebutuhan pasien. Rumah sakit harus melindungi pasien dan staf dalam keadaan darurat seperti jika terjadi kegagalan sistem, pemutusan, dan kontaminasi.

Sistem tenaga listrik darurat dibutuhkan oleh semua rumah sakit yang ingin memberikan asuhan kepada pasien tanpa putus dalam keadaan darurat. Sistem darurat ini memberikan cukup tenaga listrik untuk mempertahankan fungsi yang esensial dalam keadaan darurat dan juga menurunkan risiko terkait terjadi kegagalan. Tenaga listrik cadangan dan darurat harus dites sesuai dengan rencana yang dapat membuktikan beban tenaga listrik memang seperti yang dibutuhkan. Perbaikan dilakukan jika dibutuhkan seperti menambah kapasitas listrik di area dengan peralatan baru.

Mutu air dapat berubah mendadak karena banyak sebab, tetapi sebagian besar karena terjadi di luar rumah sakit seperti ada kebocoran di jalur suplai ke rumah sakit. Jika terjadi suplai air ke rumah sakit terputus maka persediaan air bersih darurat harus tersedia segera.

Untuk mempersiapkan diri terhadap keadaan darurat seperti ini, rumah sakit agar mempunyai proses meliputi:
a Mengidentifikasi peralatan, sistem, serta area yang memiliki risiko paling tinggi terhadap pasien dan staf (sebagai contoh, rumah sakit mengidentifikasi area yang membutuhkan penerangan, pendinginan (lemari es), bantuan hidup/ventilator, serta air bersih untuk membersihkan dan sterilisasi alat);
b. Menyediakan air bersih dan listrik 24 jam setiap hari dan 7 (tujuh) hari seminggu;
c.~Menguji ketersediaan serta kehandalan sumber tenaga listrik dan air bersih darurat/pengganti/back-up;
d.~Mendokumentasikan hasil-hasil pengujian;
e. Memastikan bahwa pengujian sumber cadangan/alternatif air bersih dan listrik dilakukan setidaknya setiap 6 (enam) bulan atau lebih sering jika dipersyaratkan oleh peraturan perundang-undangan di daerah, rekomendasi produsen, atau kondisi sumber listrik dan air. Kondisi sumber listrik dan air yang mungkin dapat meningkatkan frekuensi pengujian mencakup:

\begin{enumerate}
\def\labelenumi{\arabic{enumi}.}
\tightlist
\item
  Perbaikan sistem air bersih yang terjadi berulang- ulang.
\item
  Sumber air bersih sering terkontaminasi.
\item
  Jaringan listrik yang tidak dapat diandalkan.
\item
  Pemadaman listrik yang tidak terduga dan berulang-ulang.
\end{enumerate}

\hypertarget{elemen-penilaian-mfk-8.2}{%
\paragraph*{9. Elemen Penilaian MFK 8.2}\label{elemen-penilaian-mfk-8.2}}
\addcontentsline{toc}{paragraph}{9. Elemen Penilaian MFK 8.2}

\begin{enumerate}
\def\labelenumi{\alph{enumi}.}
\tightlist
\item
  Rumah sakit mempunyai proses sistem utilitas terhadap keadaan darurat yang meliputi poin a)-c) pada maksud dan tujuan.
\item
  Air bersih harus tersedia selama 24 jam setiap hari, 7 (tujuh) hari dalam seminggu.
\item
  Listrik tersedia 24 jam setiap hari, 7 (tujuh) hari dalam seminggu.
\item
  Rumah sakit mengidentifikasi area dan pelayanan yang berisiko paling tinggi bila terjadi kegagalan listrik atau air bersih terkontaminasi atau terganggu dan melakukan penanganan untuk mengurangi risiko.
\item
  Rumah sakit mempunyai sumber listrik dan air bersih cadangan dalam keadaan darurat/emergensi.
\end{enumerate}

\hypertarget{standar-mfk-8.2.1}{%
\paragraph*{10. Standar MFK 8.2.1}\label{standar-mfk-8.2.1}}
\addcontentsline{toc}{paragraph}{10. Standar MFK 8.2.1}

Rumah sakit melakukan uji coba/uji beban sumber listrik dan sumber air cadangan/alternatif.

\hypertarget{maksud-dan-tujuan-mfk-8.2.1}{%
\paragraph*{11. Maksud dan Tujuan MFK 8.2.1}\label{maksud-dan-tujuan-mfk-8.2.1}}
\addcontentsline{toc}{paragraph}{11. Maksud dan Tujuan MFK 8.2.1}

Rumah sakit melakukan pengkajian risiko dan meminimalisasi risiko kegagalan sistem utilitas di area-area berisiko terutama area pelayanan pasien.

Rumah sakit merencanakan tenaga listrik cadangan darurat (dengan menyiapkan genset) dan penyediaan sumber air bersih darurat untuk area-area yang membutuhkan. Untuk memastikan kapasitas beban yang dapat dicapai oleh unit genset apakah benar-benar mampu mencapai beban tertinggi maka pada waktu pembelian unit genset, dilakukan test loading dengan menggunakan alat yang bernama dummy load.

Selain itu, rumah sakit melaksanakan uji coba sumber listrik cadangan/alternatif sekurangnya 6 (enam) bulan sekali atau lebih sering bila diharuskan oleh peraturan perundang-undangan atau oleh kondisi sumber listrik. Jika sistem listrik darurat membutuhkan sumber bahan bakar maka jumlah tempat penyimpanan bahan bakar perlu dipertimbangkan. Rumah sakit dapat menentukan jumlah bahan bakar yang disimpan, kecuali ada ketentuan lain dari pihak berwenang.

\hypertarget{elemen-penilaian-mfk-8.2.1}{%
\paragraph*{12. Elemen Penilaian MFK 8.2.1}\label{elemen-penilaian-mfk-8.2.1}}
\addcontentsline{toc}{paragraph}{12. Elemen Penilaian MFK 8.2.1}

\begin{enumerate}
\def\labelenumi{\alph{enumi}.}
\tightlist
\item
  Rumah sakit melaksanakan uji coba sumber air bersih dan listrik cadangan/alternatif sekurangnya 6 (enam) bulan sekali atau lebih sering bila diharuskan oleh peraturan perundang-undanganan yang berlaku atau oleh kondisi sumber air.
\item
  Rumah sakit mendokumentasi hasil uji coba sumber air bersih cadangan/alternatif tersebut.
\item
  Rumah sakit mendokumentasikan hasil uji sumber listrik/cadangan/alternatif tersebut.
\item
  Rumah sakit mempunyai tempat dan jumlah bahan bakar untuk sumber listrik cadangan/alternatif yang mencukupi.
\end{enumerate}

\hypertarget{standar-mfk-8.3}{%
\paragraph*{13. Standar MFK 8.3}\label{standar-mfk-8.3}}
\addcontentsline{toc}{paragraph}{13. Standar MFK 8.3}

Rumah sakit melakukan pemeriksaan air bersih dan air limbah secara berkala sesuai dengan peraturan dan perundang-undangan.

\hypertarget{maksud-dan-tujuan-mfk-8.3}{%
\paragraph*{14. Maksud dan Tujuan MFK 8.3}\label{maksud-dan-tujuan-mfk-8.3}}
\addcontentsline{toc}{paragraph}{14. Maksud dan Tujuan MFK 8.3}

Seperti dijelaskan di MFK 8.2 dan MFK 8.2.1, mutu air rentan terhadap perobahan yang mendadak, termasuk perobahan di luar kontrol rumah sakit. Mutu air juga kritikal di dalam proses asuhan klinik seperti pada dialisis ginjal. Jadi, rumah sakit menetapkan proses monitor mutu air termasuk tes (pemeriksaan) biologik air yang dipakai untuk dialisis ginjal. Tindakan dilakukan jika mutu air ditemukan tidak aman.

Monitor dilakukan paling sedikit 3 (tiga) bulan sekali atau lebih cepat mengikuti peraturan perundang-undangan, kondisi sumber air, dan pengalaman yang lalu dengan masalah mutu air. Monitor dapat dilakukan oleh perorangan yang ditetapkan rumah sakit seperti staf dari laboratorium klinik, atau oleh dinas kesehatan, atau pemeriksa air pemerintah di luar rumah sakit yang kompeten untuk melakukan pemeriksaan seperti itu. Apakah diperiksa oleh staf rumah sakit atau oleh otoritas di luar rumah sakit maka tanggung jawab rumah sakit adalah memastikan pemeriksaan (tes) dilakukan lengkap dan tercatat dalam dokumen.
Karena itu, rumah sakit perlu mempunyai proses meliputi:

\begin{enumerate}
\def\labelenumi{\alph{enumi}.}
\tightlist
\item
  Pelaksanaan pemantauan mutu air bersih paling sedikit 1 (satu) tahun sekali. Untuk pemeriksaan kimia minimal setiap 6 (enam) bulan atau lebih sering bergantung pada ketentuan peraturan perundang- undangan, kondisi sumber air, dan pengalaman sebelumnya dengan masalah mutu air. Hasil pemeriksaan didokumentasikan;
\item
  Pemeriksaan air limbah dilakukan setiap 3 (tiga) bulan atau lebih sering bergantung pada peraturan perundang-undangan, kondisi sumber air, dan hasil pemeriksaan air terakhir bermasalah. Hasil pemeriksaan didokumentasikan;
\item
  Pemeriksaan mutu air yang digunakan untuk dialisis ginjal setiap bulan untuk menilai pertumbuhan bakteri dan endotoksin. Pemeriksaan tahunan untuk menilai kontaminasi zat kimia. Hasil pemeriksaan didokumentasikan; dan
\item
  Melakukan pemantauan hasil pemeriksaan air dan perbaikan bila diperlukan.
\end{enumerate}

\hypertarget{elemen-penilaian-mfk-8.3}{%
\paragraph*{15. Elemen Penilaian MFK 8.3}\label{elemen-penilaian-mfk-8.3}}
\addcontentsline{toc}{paragraph}{15. Elemen Penilaian MFK 8.3}

\begin{enumerate}
\def\labelenumi{\alph{enumi}.}
\tightlist
\item
  Rumah sakit telah menerapkan proses sekurang- kurangnya meliputi poin a) - d) pada maksud dan tujuan.
\item
  Rumah sakit telah melakukan pemantauan dan evaluasi proses pada EP 1.
\item
  Rumah sakit telah menindaklanjuti hasil pemantauan dan evaluasi pada EP 2 dan didokumentasikan.
\end{enumerate}

\hypertarget{h.-penanganan-kedaruratan-dan-bencana}{%
\subsection*{h. Penanganan kedaruratan dan bencana}\label{h.-penanganan-kedaruratan-dan-bencana}}
\addcontentsline{toc}{subsection}{h. Penanganan kedaruratan dan bencana}

\hypertarget{standar-mfk-9}{%
\paragraph*{1. Standar MFK 9}\label{standar-mfk-9}}
\addcontentsline{toc}{paragraph}{1. Standar MFK 9}

Rumah sakit menerapkan proses penanganan bencana untuk menanggapi bencana yang berpotensi terjadi di wilayah rumah sakitnya.

\hypertarget{maksud-dan-tujuan-mfk-9}{%
\paragraph*{2. Maksud dan Tujuan MFK 9}\label{maksud-dan-tujuan-mfk-9}}
\addcontentsline{toc}{paragraph}{2. Maksud dan Tujuan MFK 9}

Keadaan darurat yang terjadi, epidemi, atau bencana alam akan berdampak pada rumah sakit. Proses penanganan bencana dimulai dengan mengidentifikasi jenis bencana yang mungkin terjadi di wilayah rumah sakit berada dan dampaknya terhadap rumah sakit yang dapat berupa kerusakan fisik, peningkatan jumlah pasien/korban yang signifikan, morbiditas dan mortalitas tenaga Kesehatan, dan gangguan operasionalisasi rumah sakit. Untuk menanggapi secara efektif maka rumah sakit perlu menetapkan proses pengelolaan bencana yang merupakan bagian dari progam Manajemen Fasilitas dan Keselamatan (MFK) pada standar MFK 1 meliputi:

\begin{enumerate}
\def\labelenumi{\alph{enumi}.}
\tightlist
\item
  Menentukan jenis yang kemungkinan terjadi dan konsekuensi bahaya, ancaman, dan kejadian;
\item
  Menentukan integritas struktural dan non struktural di lingkungan pelayanan pasien yang ada dan bagaimana bila terjadi bencana;
\item
  Menentukan peran rumah sakit dalam peristiwa/kejadian tersebut;
\item
  Menentukan strategi komunikasi pada waktu kejadian;
\item
  Mengelola sumber daya selama kejadian termasuk sumber-sumber alternatif;
\item
  Mengelola kegiatan klinis selama kejadian termasuk tempat pelayanan alternatif pada waktu kejadian;
\item
  Mengidentifikasi dan penetapan peran serta tanggung jawab staf selama kejadian dan; dan
\item
  Proses mengelola keadaan darurat ketika terjadi konflik antara tanggung jawab pribadi staf dan tanggung jawab rumah sakit untuk tetap menyediakan pelayanan pasien termasuk kesehatan mental dari staf.
\end{enumerate}

Rumah sakit yang aman adalah rumah sakit yang fasilitas layanannya tetap dapat diakses dan berfungsi pada kapasitas maksimum, serta dengan infrastruktur yang sama, sebelum, selama, dan segera setelah dampak keadaan darurat dan bencana.

Fungsi rumah sakit yang terus berlanjut bergantung pada berbagai faktor termasuk keamanan dan keselamatan bangunan, sistem dan peralatan pentingnya, ketersediaan persediaan,serta kapasitas penanganan darurat dan bencana di rumah sakit terutama tanggapan dan pemulihan dari bahaya atau kejadian yang mungkin terjadi.

Kunci pengembangan menuju keamanan dan keselamatan di rumah sakit adalah melakukan analisis kerentanan terhadap kemungkinan bencana (Hazard Vulnerability Analysis) yang dilakukan rumah sakit setiap tahun.

\hypertarget{elemen-penilaian-mfk-9}{%
\paragraph*{3. Elemen Penilaian MFK 9}\label{elemen-penilaian-mfk-9}}
\addcontentsline{toc}{paragraph}{3. Elemen Penilaian MFK 9}

\begin{enumerate}
\def\labelenumi{\alph{enumi}.}
\tightlist
\item
  Rumah sakit menerapkan proses pengelolaan bencana yang meliputi poin a) ? h) pada maksud dan tujuan di atas.
\item
  Rumah sakit telah mengidentifikasi risiko bencana internal dan eksternal dalam analisis kerentanan bahaya/Hazard Vulnerability Analysis (HVA) secara proaktif setiap tahun dan diintegrasikan ke dalam daftar risiko/risk register dan profil risiko.
\item
  Rumah sakit membuat program pengelolaan bencana di rumah sakit berdasarkan hasil analisis kerentanan bahaya/Hazard Vulnerability Analysis (HVA) setiap tahun.
\item
  Rumah sakit telah melakukan simulasi penanggulangan bencana (disaster drill) minimal setahun sekali termasuk debriefing.
\item
  Staf dapat menjelaskan dan atau memperagakan prosedur dan peran mereka dalam penanganan kedaruratan serta bencana internal dan external
\item
  Rumah sakit telah menyiapkan area dekontaminasi sesuai ketentuan pada instalasi gawat darurat.
\end{enumerate}

\hypertarget{i.-konstruksi-dan-renovasi}{%
\subsection*{i. Konstruksi dan renovasi}\label{i.-konstruksi-dan-renovasi}}
\addcontentsline{toc}{subsection}{i. Konstruksi dan renovasi}

\hypertarget{standar-mfk-10}{%
\paragraph*{1. Standar MFK 10}\label{standar-mfk-10}}
\addcontentsline{toc}{paragraph}{1. Standar MFK 10}

Rumah sakit melakukan penilaian risiko prakontruksi/Pre Contruction Risk Assessment (PCRA) pada waktu merencanakan pembangunan baru (proyek konstruksi), renovasi dan pembongkaran.

\hypertarget{maksud-dan-tujuan-mfk-10}{%
\paragraph*{2. Maksud dan tujuan MFK 10}\label{maksud-dan-tujuan-mfk-10}}
\addcontentsline{toc}{paragraph}{2. Maksud dan tujuan MFK 10}

Kegiatan konstruksi, renovasi, pembongkaran, dan pemeliharaan di rumah sakit dapat berdampak pada semua orang dalam area rumah sakit. Namun, pasien mungkin menderita dampak terbesar. Misalnya, kebisingan dan getaran yang terkait dengan aktivitas ini dapat memengaruhi tingkat kenyamanan pasien, dan debu serta bau dapat mengubah kualitas udara, yang dapat mengancam status pernapasan pasien. Risiko terhadap pasien, staf, pengunjung, badan usaha independen, dan lainnya di rumah sakit akan bervariasi tergantung pada sejauh mana aktivitas konstruksi, renovasi, pembongkaran, atau pemeliharaan dan dampaknya terhadap perawatan pasien, infrastruktur, dan utilitas.

Untuk menilai risiko yang terkait dengan konstruksi, renovasi, atau proyek pembongkaran, atau aktivitas pemeliharaan yang memengaruhi perawatan pasien maka rumah sakit melakukan koordinasi antar satuan kerja terkait, termasuk, sesuai kebutuhan, perwakilan dari desain proyek, pengelolaan proyek, teknik fasilitas, fasilitas keamanan/keselamatan, pencegahan dan pengendalian infeksi, keselamatan kebakaran, rumah tangga, layanan teknologi informasi, dan satuan kerja serta layanan klinis.

Penilaian risiko digunakan untuk mengevaluasi risiko secara komprehensif untuk mengembangkan rencana dan menerapkan tindakan pencegahan yang akan meminimalkan dampak proyek konstruksi terhadap kualitas, keselamatan dan keamanan perawatan pasien.
Proses penilaian risiko konstruksi meliputi:

\begin{enumerate}
\def\labelenumi{\alph{enumi}.}
\tightlist
\item
  Kualitas udara;
\item
  Pencegahan dan pengendalian infeksi;
\item
  Utilitas;
\item
  Kebisingan;
\item
  Getaran;
\item
  Bahan dan limbah berbahaya;
\item
  Keselamatan kebakaran;
\item
  Keamanan;
\item
  Prosedur darurat, termasuk jalur/keluar alternatif dan akses ke layanan darurat; dan
\item
  Bahaya lain yang mempengaruhi perawatan, pengobatan, dan layanan.
\end{enumerate}

Selain itu, rumah sakit memastikan bahwa kepatuhan kontraktor dipantau, ditegakkan, dan didokumentasikan. Sebagai bagian dari penilaian risiko, risiko infeksi pasien dari konstruksi dievaluasi melalui penilaian risiko pengendalian infeksi, juga dikenal sebagai ICRA.

Setiap ada kontruksi, renovasi dan demolisi harus dilakukan penilaian risiko prakontruksi termasuk dengan rencana/pelaksanaan pengurangan risiko dampak keselamatan serta keamanan bagi pasien, keluarga, pengunjung, dan staf. Hal ini berdampak memerlukan biaya maka rumah sakit dan pihak kontraktor juga perlu menyediakan anggaran untuk penerapan Pra Contruction Risk Assessment (PCRA) dan Infection Control Risk Assessment (ICRA).

\hypertarget{elemen-penilaian-mfk-10}{%
\paragraph*{3. Elemen Penilaian MFK 10}\label{elemen-penilaian-mfk-10}}
\addcontentsline{toc}{paragraph}{3. Elemen Penilaian MFK 10}

\begin{enumerate}
\def\labelenumi{\alph{enumi}.}
\tightlist
\item
  Rumah sakit menerapkan penilaian risiko prakonstruksi (PCRA) terkait rencana konstruksi, renovasi dan demolisi meliputi poin a) - j) seperti di maksud dan tujuan diatas.
\item
  Rumah sakit melakukan penilaian risiko prakontruksi (PCRA) bila ada rencana kontruksi, renovasi dan demolisi.
\item
  Rumah sakit melakukan tindakan berdasarkan hasil penilaian risiko untuk meminimalkan risiko selama pembongkaran, konstruksi, dan renovasi.
\item
  Rumah sakit memastikan bahwa kepatuhan kontraktor dipantau, dilaksanakan, dan didokumentasikan.
\end{enumerate}

\hypertarget{j.-pelatihan}{%
\subsection*{j. Pelatihan}\label{j.-pelatihan}}
\addcontentsline{toc}{subsection}{j. Pelatihan}

\hypertarget{standar-mfk-11}{%
\paragraph*{1. Standar MFK 11}\label{standar-mfk-11}}
\addcontentsline{toc}{paragraph}{1. Standar MFK 11}

Seluruh staf di rumah sakit dan yang lainnya telah dilatih dan memiliki pengetahuan tentang pengelolaan fasilitas rumah sakit, program keselamatan dan peran mereka dalam memastikan keamanan dan keselamatan fasilitas secara efektif.

\hypertarget{maksud-dan-tujuan-mfk-11}{%
\paragraph*{2. Maksud dan Tujuan MFK 11}\label{maksud-dan-tujuan-mfk-11}}
\addcontentsline{toc}{paragraph}{2. Maksud dan Tujuan MFK 11}

Staf adalah sumber kontak utama rumah sakit dengan pasien, keluarga, dan pengunjung. Oleh karena itu, mereka perlu dididik dan dilatih untuk menjalankan perannya dalam mengidentifikasi dan mengurangi risiko, melindungi orang lain dan diri mereka sendiri, serta menciptakan fasilitas yang aman, selamat dan terjamin.

Setiap rumah sakit harus memutuskan jenis dan tingkat pelatihan untuk staf dan kemudian melaksanakan dan mendokumentasikan program pelatihan. Program pelatihan dapat mencakup instruksi kelompok, modul pendidikan online, materi pendidikan tertulis, komponen orientasi staf baru, dan/atau beberapa mekanisme lain yang memenuhi kebutuhan rumah sakit. Pelatihan diberikan kepada semua staf di semua shift setiap tahun dan membahas semua program pengelolaan fasilitas dan keselamatan. Pelatihan mencakup instruksi tentang proses pelaporan potensi risiko dan pelaporan insiden dan cedera. Program pelatihan melibatkan pengujian pengetahuan staf. Staf dilatih dan diuji tentang prosedur darurat, termasuk prosedur keselamatan kebakaran.

Sebagaimana berlaku untuk peran dan tanggung jawab anggota staf, pelatihan dan pengujian membahas bahan berbahaya dan respons terhadap bahaya, seperti tumpahan bahan kimia berbahaya, dan penggunaan peralatan medis yang dapat menimbulkan risiko bagi pasien dan staf. Pengetahuan dapat diuji melalui berbagai cara, seperti demonstrasi individu atau kelompok, demonstrasi, peristiwa simulasi seperti epidemi di masyarakat, penggunaan tes tertulis atau komputer, atau cara lain yang sesuai dengan pengetahuan yang diuji. Dokumen rumah sakit yang diuji dan hasil pengujian.

\hypertarget{elemen-penilaian-mfk-11}{%
\paragraph*{3. Elemen Penilaian MFK 11}\label{elemen-penilaian-mfk-11}}
\addcontentsline{toc}{paragraph}{3. Elemen Penilaian MFK 11}

\begin{enumerate}
\def\labelenumi{\alph{enumi}.}
\tightlist
\item
  Semua staf telah diberikan pelatihan program manajemen fasilitas dan keselamatan (MFK) terkait keselamatan setiap tahun dan dapat menjelaskan dan/atau menunjukkan peran dan tanggung jawabnya dan didokumentasikan.
\item
  Semua staf telah diberikan pelatihan program manajemen fasilitas dan keselamatan (MFK) terkait keamanan setiap tahun dan dapat menjelaskan dan/atau menunjukkan peran dan tanggung jawabnya dan didokumentasikan.
\item
  Semua staf telah diberikan pelatihan program manajemen fasilitas dan keselamatan (MFK) terkait pengelolaan B3 dan limbahnya setiap tahun dan dapat menjelaskan dan/atau menunjukkan peran dan tanggung jawabnya dan didokumentasikan.
\item
  Semua staf telah diberikan pelatihan program manajemen fasilitas dan keselamatan (MFK) terkait proteksi kebakaran setiap tahun dan dapat menjelaskan dan/atau menunjukkan peran dan tanggung jawabnya dan didokumentasikan.
\item
  Semua staf telah diberikan pelatihan program manajemen fasilitas dan keselamatan (MFK) terkait peralatan medis setiap tahun dan dapat menjelaskan dan/atau menunjukkan peran dan tanggung jawabnya dan didokumentasikan.
\item
  Semua staf telah diberikan pelatihan program manajemen fasilitas dan keselamatan (MFK) terkait sistim utilitas setiap tahun dan dapat menjelaskan dan/atau menunjukkan peran dan tanggung jawabnya dan didokumentasikan.
\item
  Semua staf telah diberikan pelatihan program manajemen fasilitas dan keselamatan (MFK) terkait penanganan bencana setiap tahun dan dapat menjelaskan dan/atau menunjukkan peran dan tanggung jawabnya dan didokumentasikan.
\item
  Pelatihan tentang pengelolaan fasilitas dan program keselamatan mencakup vendor, pekerja kontrak, relawan, pelajar, peserta didik, peserta pelatihan, dan lainnya, sebagaimana berlaku untuk peran dan tanggung jawab individu, dan sebagaimana ditentukan oleh rumah sakit.
\end{enumerate}

\hypertarget{peningkatan-mutu-dan-keselamatan-pasien-pmkp}{%
\section*{4. Peningkatan Mutu dan Keselamatan Pasien (PMKP)}\label{peningkatan-mutu-dan-keselamatan-pasien-pmkp}}
\addcontentsline{toc}{section}{4. Peningkatan Mutu dan Keselamatan Pasien (PMKP)}

\textbf{Gambaran Umum}

Rumah sakit harus memiliki program peningkatan mutu dan keselamatan pasien (PMKP) yang menjangkau seluruh unit kerja dalam rangka meningkatkan mutu pelayanan dan menjamin keselamatan pasien. Direktur menetapkan Komite/Tim Penyelenggara Mutu untuk mengelola program peningkatan mutu dan keselamatan pasien, agar mekanisme koordinasi pelaksanaan program peningkatan mutu dan keselamatan pasien di rumah sakit dapat berjalan lebih baik.

Standar ini menjelaskan pendekatan yang komprehensif untuk peningkatan mutu dan keselamatan pasien yang berdampak pada semua aspek pelayanan, mencakup:

\begin{enumerate}
\def\labelenumi{\alph{enumi}.}
\tightlist
\item
  Peran serta dan keterlibatan setiap unit dalam program peningkatan mutu dan keselamatan pasien.
\item
  Pengukuran data objektif yang tervalidasi.
\item
  Penggunaan data yang objektif dan kaji banding untuk membuat program peningkatan mutu dan keselamatan pasien.
\end{enumerate}

Standar PMKP membantu profesional pemberi asuhan (PPA) untuk memahami bagaimana melakukan perbaikan dalam memberikan asuhan pasien yang aman dan menurunkan risiko. Staf non klinis juga dapat melakukan perbaikan agar proses menjadi lebih efektif dan efisien dalam penggunaan sumber daya dan risiko dapat dikurangi.

Standar PMKP ditujukan pada semua kegiatan di rumah sakit secara menyeluruh dalam spektrum yang luas berupa kerangka kerja untuk perbaikan kinerja dan menurunkan risiko akibat variasi dalam proses pelayanan. Kerangka kerja dalam standar PMKP ini juga dapat terintegrasi dengan kejadian yang tidak dapat dicegah (program manajemen risiko) dan pemanfaatan sumber daya (pengelolaan utilisasi).

Rumah sakit yang menerapkan kerangka kerja ini diharapkan akan:

\begin{enumerate}
\def\labelenumi{\alph{enumi}.}
\tightlist
\item
  Mengembangkan dukungan pimpinan yang lebih besar untuk program peningkatan mutu dan keselamatan pasien secara menyeluruh di rumah sakit;
\item
  Melatih semua staf tentang peningkatan mutu dan keselamatan pasien rumah sakit;
\item
  Menetapkan prioritas pengukuran data dan prioritas perbaikan;
\item
  Membuat keputusan berdasarkan pengukuran data; dan
\item
  Melakukan perbaikan berdasarkan perbandingan dengan rumah sakit setara atau data berbasis bukti lainnya, baik nasional dan internasional.
\end{enumerate}

Fokus standar peningkatan mutu dan keselamatan pasien adalah:

\begin{enumerate}
\def\labelenumi{\alph{enumi}.}
\tightlist
\item
  Pengelolaan kegiatan peningkatan mutu, keselamatan pasien dan manajemen risiko.
\item
  Pemilihan dan pengumpulan data indikator mutu.
\item
  Analisis dan validasi data indikator mutu.
\item
  Pencapaian dan upaya mempertahankan perbaikan mutu.
\item
  Sistem pelaporan dan pembelajaran keselamatan pasien rumah sakit (SP2KP-RS)
\item
  Penerapan manajemen risiko.
\end{enumerate}

\hypertarget{a.-pengelolaan-kegiatan-peningkatan-mutu-keselamatan-pasien-dan-manajemen-risiko}{%
\subsection*{a. Pengelolaan kegiatan peningkatan mutu, keselamatan pasien dan manajemen risiko}\label{a.-pengelolaan-kegiatan-peningkatan-mutu-keselamatan-pasien-dan-manajemen-risiko}}
\addcontentsline{toc}{subsection}{a. Pengelolaan kegiatan peningkatan mutu, keselamatan pasien dan manajemen risiko}

\hypertarget{standar-pmkp-1}{%
\paragraph*{1. Standar PMKP 1}\label{standar-pmkp-1}}
\addcontentsline{toc}{paragraph}{1. Standar PMKP 1}

Rumah sakit mempunyai Komite/Tim Penyelenggara Mutu yang kompeten untuk mengelola kegiatan Peningkatan Mutu dan Keselamatan Pasien (PMKP) sesuai dengan peraturan perundang-undangan.

\hypertarget{maksud-dan-tujuan-pmkp-1}{%
\paragraph*{2. Maksud dan Tujuan PMKP 1}\label{maksud-dan-tujuan-pmkp-1}}
\addcontentsline{toc}{paragraph}{2. Maksud dan Tujuan PMKP 1}

Peningkatan mutu dan keselamatan pasien merupakan proses kegiatan yang berkesinambungan (continuous improvement) yang dilaksanaan dengan koordinasi dan integrasi antara unit pelayanan dan komite-komite (Komite Medik, Komite Keperawatan, Komite/Tim PPI, Komite K3 dan fasilitas, Komite Etik, Komite PPRA, dan lain-lainnya). Oleh karena itu Direktur perlu menetapkan Komite/Tim Penyelenggara Mutu yang bertugas membantu Direktur atau Kepala Rumah Sakit dalam mengelola kegiatan peningkatan mutu, keselamatan pasien, dan manajemen risiko di rumah sakit.
Dalam melaksanakan tugasnya, Komite/ Tim Penyelenggara Mutu memiliki fungsi sesuai dengan peraturan perundang- undangan yang berlaku.

Dalam proses pengukuran data, Direktur menetapkan:

\begin{enumerate}
\def\labelenumi{\alph{enumi}.}
\tightlist
\item
  Kepala unit sebagai penanggung jawab peningkatan mutu dan keselamatan pasien (PMKP) di tingkat unit;
\item
  Staf pengumpul data; dan
\item
  Staf yang akan melakukan validasi data (validator).
\end{enumerate}

Bagi rumah sakit yang memiliki tenaga cukup, proses pengukuran data dilakukan oleh ketiga tenaga tersebut. Dalam hal keterbatasan tenaga, proses validasi data dapat dilakukan oleh penanggung jawab PMKP di unit kerja. Komite/Tim Penyelenggara Mutu, penanggung jawab mutu dan keselamatan pasien di unit, staf pengumpul data, validator perlu mendapat pelatihan peningkatan mutu dan keselamatan pasien termasuk pengukuran data mencakup pengumpulan data, analisis data, validasi data, serta perbaikan mutu.

Komite/ Tim Penyelenggara Mutu akan melaporkan hasil pelaksanaan program PMKP kepada Direktur setiap 3 (tiga) bulan.

Kemudian Direktur akan meneruskan laporan tersebut kepada Dewan Pengawas. Laporan tersebut mencakup:

\begin{enumerate}
\def\labelenumi{\alph{enumi}.}
\tightlist
\item
  Hasil pengukuran data meliputi: Pencapaian semua indikator mutu, analisis, validasi dan perbaikan yang telah dilakukan.
\item
  Laporan semua insiden keselamatan pasien meliputi jumlah, jenis (kejadian sentinel, KTD, KNC, KTC, KPCS), tipe insiden dan tipe harm, tindak lanjut yang dilakukan, serta tindakan perbaikan tersebut dapat dipertahankan.
\end{enumerate}

Di samping laporan hasil pelaksanaan program PMKP, Komite/ Tim Penyelenggara Mutu juga melaporkan hasil pelaksanaan program manajemen risiko berupa pemantauan penanganan risiko yang telah dilaksanakan setiap 6 (enam) bulan kepada Direktur yang akan diteruskan kepada Dewan Pengawas.

Rumah sakit membuat program peningkatan mutu dan keselamatan pasien yang akan diterapkan pada semua unit setiap tahun.

Program peningkatan mutu dan keselamatan pasien rumah sakit meliputi tapi tidak terbatas pada:
a. Pengukuran mutu indikator termasuk indikator nasional mutu (INM), indikator mutu prioritas rumah sakit (IMP RS) dan indikator mutu prioritas unit (IMP Unit).
b. Meningkatkan perbaikan mutu dan mempertahankan perbaikan berkelanjutan.
c.~Mengurangi varian dalam praktek klinis dengan menerapkan PPK/Algoritme/Protokol dan melakukan pengukuran dengan clinical pathway.
d.~Mengukur dampak efisiensi dan efektivitas prioritas perbaikan terhadap keuangan dan sumber daya misalnya SDM.
e. Pelaporan dan analisis insiden keselamatan pasien.
f.~Penerapan sasaran keselamatan pasien.
g. Evaluasi kontrak klinis dan kontrak manajemen.
h. Pelatihan semua staf sesuai perannya dalam program peningkatan mutu dan keselamatan pasien.
i. Mengkomunikasikan hasil pengukuran mutu meliputi masalah mutu dan capaian data kepada staf.

Hal-hal penting yang perlu dilakukan agar program peningkatan mutu dan keselamatan pasien dapat diterapkan secara menyeluruh di unit pelayanan, meliputi:

\begin{enumerate}
\def\labelenumi{\alph{enumi}.}
\tightlist
\item
  Dukungan Direktur dan pimpinan di rumah sakit:
\item
  Upaya perubahan budaya menuju budaya keselamatan pasien;
\item
  Secara proaktif melakukan identifikasi dan menurunkan variasi dalam pelayanan klinis;
\item
  Menggunakan hasil pengukuran data untuk fokus pada isu pelayanan prioritas yang akan diperbaiki atau ditingkatkan; dan
\item
  Berupaya mencapai dan mempertahankan perbaikan yang berkelanjutan.
\end{enumerate}

\hypertarget{elemen-penilaian-pmkp-1}{%
\paragraph*{3. Elemen Penilaian PMKP 1}\label{elemen-penilaian-pmkp-1}}
\addcontentsline{toc}{paragraph}{3. Elemen Penilaian PMKP 1}

\begin{enumerate}
\def\labelenumi{\alph{enumi}.}
\tightlist
\item
  Direktur telah menetapkan regulasi terkait peningkatan mutu dan keselamatan pasien serta manajemen risiko
\item
  Direktur rumah sakit telah membentuk Komite/Tim Penyelenggara Mutu untuk mengelola kegiatan PMKP serta uraian tugasnya sesuai dengan peraturan perundang-undangan.
\item
  Komite/ Tim Penyelenggara Mutu menyusun program PMKP rumah sakit meliputi poin a) ? i) yang telah ditetapkan Direktur rumah sakit dan disahkan oleh representatif pemilik/dewan pengawas.
\item
  Program PMKP dievaluasi dalam rapat koordinasi mellibatkan komite-komite, pimpinan rumah sakit dan kepala unit setiap triwulan untuk menjamin perbaikan mutu yang berkesinambungan.
\end{enumerate}

\hypertarget{b.-pemilihan-dan-pengumpulan-data-indikator-mutu}{%
\subsection*{b. Pemilihan dan pengumpulan data indikator mutu}\label{b.-pemilihan-dan-pengumpulan-data-indikator-mutu}}
\addcontentsline{toc}{subsection}{b. Pemilihan dan pengumpulan data indikator mutu}

\hypertarget{standar-pmkp-2}{%
\paragraph{1. Standar PMKP 2 \{=\}}\label{standar-pmkp-2}}

Komite/Tim Penyelenggara Mutu mendukung proses pemilihan indikator dan melaksanakan koordinasi serta integrasi kegiatan pengukuran data indikator mutu dan keselamatan pasien di rumah sakit

\hypertarget{maksud-dan-tujuan-pmkp-2}{%
\paragraph*{2. Maksud dan tujuan PMKP 2}\label{maksud-dan-tujuan-pmkp-2}}
\addcontentsline{toc}{paragraph}{2. Maksud dan tujuan PMKP 2}

Pemilihan indikator mutu prioritas rumah sakit adalah tanggung jawab pimpinan dengan mempertimbangkan prioritas untuk pengukuran yang berdampak luas/ menyeluruh di rumah sakit. Sedangkan kepala unit memilih indikator mutu prioritas di unit kerjanya. Semua unit klinis dan non klinis memilih indikator terkait dengan prioritasnya. Di rumah sakit yang besar harus diantisipasi jika ada indikator yang sama yang diukur di lebih dari satu unit. Misalnya, Unit Farmasi dan Komite/Tim PPI memilih prioritas pengukurannya adalah penurunan angka penggunaan antibiotik di rumah sakit. Program mutu dan keselamatan pasien berperan penting dalam membantu unit melakukan pengukuran indikator yang ditetapkan. Komite/Tim Penyelenggara Mutu juga bertugas untuk mengintegrasikan semua kegiatan pengukuran di rumah sakit, termasuk pengukuran budaya keselamatan dan sistem pelaporan insiden keselamatan pasien. Integrasi semua pengukuran ini akan menghasilkan solusi dan perbaikan yang terintegrasi.

\hypertarget{elemen-penilaian-pmkp-2}{%
\paragraph*{3. Elemen Penilaian PMKP 2}\label{elemen-penilaian-pmkp-2}}
\addcontentsline{toc}{paragraph}{3. Elemen Penilaian PMKP 2}

\begin{enumerate}
\def\labelenumi{\alph{enumi}.}
\tightlist
\item
  Komite/Tim Penyelenggara Mutu terlibat dalam pemilihan indikator mutu prioritas baik ditingkat rumah sakit maupun tingkat unit layanan.
\item
  Komite/Tim Penyelenggara Mutu melaksanakan koordinasi dan integrasi kegiatan pengukuran serta melakukan supervisi ke unit layanan.
\item
  Komite/Tim Penyelenggara Mutu mengintegrasikan laporan insiden keselamatan pasien, pengukuran budaya keselamatan, dan lainnya untuk mendapatkan solusi dan perbaikan terintegrasi.
\end{enumerate}

\hypertarget{standar-pmkp-3}{%
\paragraph*{4. Standar PMKP 3}\label{standar-pmkp-3}}
\addcontentsline{toc}{paragraph}{4. Standar PMKP 3}

Pengumpulan data indikator mutu dilakukan oleh staf pengumpul data yang sudah mendapatkan pelatihan tentang pengukuran data indikator mutu.

\hypertarget{maksud-dan-tujuan-pmkp-3}{%
\paragraph*{5. Maksud dan Tujuan PMKP 3}\label{maksud-dan-tujuan-pmkp-3}}
\addcontentsline{toc}{paragraph}{5. Maksud dan Tujuan PMKP 3}

Pengumpulan data indikator mutu berdasarkan peraturan yang berlaku yaitu pengukuran indikator nasional mutu (INM) dan prioritas perbaikan tingkat rumah sakit meliputi:

\begin{enumerate}
\def\labelenumi{\alph{enumi}.}
\tightlist
\item
  Indikator nasional mutu (INM) yaitu indikator mutu nasional yang wajib dilakukan pengukuran dan digunakan sebagai informasi mutu secara nasional.
\item
  Indikator mutu prioritas rumah sakit (IMP-RS) (TKRS 5) mencakup:
\end{enumerate}

\begin{enumerate}
\def\labelenumi{\arabic{enumi}.}
\tightlist
\item
  Indikator sasaran keselamatan pasien minimal 1 indikator setiap sasaran.
\item
  Indikator pelayanan klinis prioritas minimal 1 indikator.
\item
  Indikator sesuai tujuan strategis rumah sakit (KPI) minimal 1 indikator.
\item
  Indikator terkait perbaikan sistem minimal 1 indikator.
\item
  Indikator terkait manajemen risiko minimal 1 indikator.
\item
  Indikator terkait penelitian klinis dan program pendidikan kedokteran minimal 1 indikator. (apabila ada)
\end{enumerate}

\begin{enumerate}
\def\labelenumi{\alph{enumi}.}
\setcounter{enumi}{2}
\tightlist
\item
  Indikator mutu prioritas unit (IMP-Unit) adalah indikator prioritas yang khusus dipilih kepala unit terdiri dari minimal 1 indikator.
\end{enumerate}

Indikator mutu terpilih apabila sudah tercapai dan dapat dipertahankan selama 1 (satu) tahun, maka dapat diganti dengan indikator mutu yang baru. Setiap indikator mutu baik indikator mutu prioritas rumah sakit (IMP-RS) maupun indikator mutu prioritas unit (IMP-Unit) agar dilengkapi dengan profil indikator sebagai berikut:

\begin{enumerate}
\def\labelenumi{\alph{enumi}.}
\tightlist
\item
  Judul indikator.
\item
  Dasar pemikiran.
\item
  Dimensi mutu.
\item
  Tujuan.
\item
  Definisi operasional.
\item
  Jenis indikator.
\item
  Satuan pengukuran.
\item
  Numerator (pembilang).
\item
  Denominator (penyebut).
\item
  Target.
\item
  Kriteria inklusi dan eksklusi.
\item
  Formula.
\item
  Metode pengumpulan data.
\item
  Sumber data.
\item
  Instrumen pengambilan data.
\item
  Populasi/sampel (besar sampel dan cara pengambilan sampel).
\item
  Periode pengumpulan data.
\item
  Periode analisis dan pelaporan data.
\item
  Penyajian data.
\item
  Penanggung jawab.
\end{enumerate}

\hypertarget{elemen-penilaian-pmkp-3}{%
\paragraph*{6. Elemen Penilaian PMKP 3}\label{elemen-penilaian-pmkp-3}}
\addcontentsline{toc}{paragraph}{6. Elemen Penilaian PMKP 3}

\begin{enumerate}
\def\labelenumi{\alph{enumi}.}
\tightlist
\item
  Rumah sakit melakukan pengumpulan data mencakup (poin a) ? c)) dalam maksud dan tujuan.
\item
  Indikator mutu prioritas rumah sakit (IMP-RS) dan indikator mutu prioritas unit (IMP- Unit) telah dibuat profil indikator mencakup (poin a-t) dalam maksud dan tujuan.
\end{enumerate}

\hypertarget{c.-analisis-dan-validasi-data-indikator-mutu}{%
\subsection*{c.~Analisis dan validasi data indikator mutu}\label{c.-analisis-dan-validasi-data-indikator-mutu}}
\addcontentsline{toc}{subsection}{c.~Analisis dan validasi data indikator mutu}

\hypertarget{standar-pmkp-4}{%
\paragraph*{1. Standar PMKP 4}\label{standar-pmkp-4}}
\addcontentsline{toc}{paragraph}{1. Standar PMKP 4}

Agregasi dan analisis data dilakukan untuk mendukung program peningkatan mutu dan keselamatan pasien serta mendukung partisipasi dalam pengumpulan database eksternal.

\hypertarget{maksud-dan-tujuan-pmkp-4}{%
\paragraph*{2. Maksud dan Tujuan PMKP 4}\label{maksud-dan-tujuan-pmkp-4}}
\addcontentsline{toc}{paragraph}{2. Maksud dan Tujuan PMKP 4}

Data yang dikumpulkan akan diagregasi dan dianalisis menjadi informasi untuk pengambilan keputusan yang tepat dan akan membantu rumah sakit melihat pola dan tren capaian kinerjanya. Sekumpulan data tersebut misalnya data indikator mutu, data laporan insiden keselamatan pasien, data manajemen risiko dan data pencegahan dan pengendalian infeksi, Informasi ini penting untuk membantu rumah sakit memahami kinerjanya saat ini dan mengidentifikasi peluang-peluang untuk perbaikan kinerja rumah sakit.

Rumah sakit harus melaporkan data mutu dan keselamatan pasien ke eksternal sesuai dengan ketentuan yang ditetapkan meliputi:

\begin{enumerate}
\def\labelenumi{\alph{enumi}.}
\tightlist
\item
  Pelaporan indikator nasional mutu (INM) ke Kementrian Kesehatan melalui aplikasi mutu fasilitas pelayanan Kesehatan.
\item
  Pelaporan insiden keselamatan pasien (IKP) ke KNKP melalui aplikasi e-report.
\end{enumerate}

Dengan berpartisipasi dalam pelaporan data mutu dan keselamatan pasien ke eksternal rumah sakit dapat membandingkan kinerjanya dengan kinerja rumah sakit setara baik di skala lokal maupun nasional. Perbandingan kinerja merupakan pendekatan yang efektif untuk mencari peluang-peluang perbaikan.
Proses analisis data mencakup setidaknya satu dampak dari prioritas perbaikan rumah sakit secara keseluruhan terhadap biaya dan efisiensi sumber daya setiap tahun.

Program mutu dan keselamatan pasien mencakup analisis dampak prioritas perbaikan yang didukung oleh pimpinan. misalnya terdapat bukti yang mendukung pernyataan bahwa penggunaan panduan praktik klinis untuk mestandarkan perawatan memberikan dampak yang bermakna pada efisiensi perawatan dan pemendekan lama rawat, yang pada akhirnya menurunkan biaya. Staf program mutu dan keselamatan pasien mengembangkan instrumen untuk mengevaluasi penggunaan sumber daya untuk proses yang berjalan, kemudian untuk mengevaluasi kembali penggunaan sumber daya untuk proses yang telah diperbaiki. Sumber daya dapat berupa sumber daya manusia (misalnya, waktu yang digunakan untuk setiap langkah dalam suatu proses) atau melibatkan penggunaan teknologi dan sumber daya lainnya. Analisis ini akan memberikan informasi yang berguna terkait perbaikan yang memberikan dampak efisiensi dan biaya.

\hypertarget{elemen-penilaian-pmkp-4}{%
\paragraph*{3. Elemen Penilaian PMKP 4}\label{elemen-penilaian-pmkp-4}}
\addcontentsline{toc}{paragraph}{3. Elemen Penilaian PMKP 4}

\begin{enumerate}
\def\labelenumi{\alph{enumi}.}
\tightlist
\item
  Telah dilakukan agregasi dan analisis data menggunakan metode dan teknik statistik terhadap semua indikator mutu yang telah diukur oleh staf yang kompeten
\item
  Hasil analisis digunakan untuk membuat rekomendasi tindakan perbaikan dan serta menghasilkan efisiensi penggunaan sumber daya.
\item
  Memiliki bukti analisis data dilaporkan kepada Direktur dan reprentasi pemilik/dewan pengawas sebagai bagian dari program peningkatan mutu dan keselamatan pasien.
\item
  Memiliki bukti hasil analisis berupa informasi INM dan e-report IKP diwajibkan lapor kepada Kementrian kesehatan sesuai peraturan yang berlaku.
\item
  Terdapat proses pembelajaran dari database eksternal untuk tujuan perbandingan internal dari waktu ke waktu, perbandingan dengan rumah sakit yang setara, dengan praktik terbaik (best practices), dan dengan sumber ilmiah profesional yang objektik.
\item
  Keamanan dan kerahasiaan tetap dijaga saat berkontribusi pada database eksternal.
\item
  Telah menganalisis efisiensi berdasarkan biaya dan jenis sumber daya yang digunakan (sebelum dan sesudah perbaikan) terhadap satu proyek prioritas perbaikan yang dipilih setiap tahun.
\end{enumerate}

\hypertarget{standar-pmkp-4.1}{%
\paragraph*{4. Standar PMKP 4.1}\label{standar-pmkp-4.1}}
\addcontentsline{toc}{paragraph}{4. Standar PMKP 4.1}

Staf dengan pengalaman, pengetahuan, dan keterampilan yang bertugas mengumpulkan dan menganalisis data rumah sakit secara sistematis.

\hypertarget{maksud-dan-tujuan-pmkp-4.1}{%
\paragraph*{5. Maksud dan Tujuan PMKP 4.1}\label{maksud-dan-tujuan-pmkp-4.1}}
\addcontentsline{toc}{paragraph}{5. Maksud dan Tujuan PMKP 4.1}

Analisis data melibatkan staf yang memahami manajemen informasi, mempunyai keterampilan dalam metode-metode pengumpulan data, dan memahami teknik statistik. Hasil analisis data harus dilaporkan kepada Penanggung jawab indikator mutu (PIC) yang bertanggung jawab untuk menindaklanjuti hasil tersebut. Penanggung jawab tersebut bisa memiliki latar belakang klinis, non klinis, atau kombinasi keduanya. Hasil analisis data akan memberikan masukan untuk pengambilan keputusan dan memperbaiki proses klinis dan non klinis secara berkelanjutan. Run charts, diagram kontrol (control charts), histogram, dan diagram Pareto merupakan contoh dari alat-alat statistik yang sangat berguna dalam memahami tren dan variasi dalam pelayanan kesehatan.

Tujuan analisis data adalah untuk dapat membandingkan rumah sakit dengan empat cara. Perbandingan tersebut membantu rumah sakit dalam memahami sumber dan penyebab perubahan yang tidak diinginkan dan membantu memfokuskan upaya perbaikan.

\begin{enumerate}
\def\labelenumi{\alph{enumi}.}
\tightlist
\item
  Dengan rumah sakit sendiri dari waktu ke waktu, misalnya dari bulan ke bulan, dari tahun ke tahun.
\item
  Dengan rumah sakit setara, seperti melalui database referensi.
\item
  Dengan standar-standar, seperti yang ditentukan oleh badan akreditasi atau organisasi profesional ataupun standar-standar yang ditentukan oleh peraturan perundang-undangan yang berlaku.
\item
  Dengan praktik-praktik terbaik yang diakui dan menggolongkan praktik tersebut sebagai best practice (praktik terbaik) atau better practice (praktik yang lebih baik) atau practice guidelines (pedoman praktik).
\end{enumerate}

\hypertarget{elemen-penilaian-pmkp-4.1}{%
\paragraph*{6. Elemen Penilaian PMKP 4.1}\label{elemen-penilaian-pmkp-4.1}}
\addcontentsline{toc}{paragraph}{6. Elemen Penilaian PMKP 4.1}

\begin{enumerate}
\def\labelenumi{\alph{enumi}.}
\tightlist
\item
  Data dikumpulkan, dianalisis, dan diubah menjadi informasi untuk mengidentifikasi peluang-peluang untuk perbaikan.
\item
  Staf yang kompeten melakukan proses pengukuran menggunakan alat dan teknik statistik.
\item
  Hasil analisis data dilaporkan kepada penanggung jawab indikator mutu yang akan melakukan perbaikan.
\end{enumerate}

\hypertarget{standard-pmkp-5}{%
\paragraph*{7 Standard PMKP 5}\label{standard-pmkp-5}}
\addcontentsline{toc}{paragraph}{7 Standard PMKP 5}

Rumah sakit melakukan proses validasi data terhadap indikator mutu yang diukur.

\hypertarget{maksud-dan-tujuan-pmkp-5}{%
\paragraph*{8. Maksud dan Tujuan PMKP 5}\label{maksud-dan-tujuan-pmkp-5}}
\addcontentsline{toc}{paragraph}{8. Maksud dan Tujuan PMKP 5}

Validasi data adalah alat penting untuk memahami mutu dari data dan untuk menetapkan tingkat kepercayaan (confidence level) para pengambil keputusan terhadap data itu sendiri. Ketika rumah sakit mempublikasikan data tentang hasil klinis, keselamatan pasien, atau area lain, atau dengan cara lain membuat data menjadi publik, seperti di situs web rumah sakit, rumah sakit memiliki kewajiban etis untuk memberikan informasi yang akurat kepada publik. Pimpinan rumah sakit bertanggung jawab untuk memastikan bahwa data yang dilaporkan ke Direktur, Dewan Pengawas dan yang dipublikasikan ke masyarakat adalah valid. Keandalan dan validitas pengukuran dan kualitas data dapat ditetapkan melalui proses validasi data internal rumah sakit.

Kebijakan data yang harus divalidasi yaitu:

\begin{enumerate}
\def\labelenumi{\alph{enumi}.}
\tightlist
\item
  Pengukuran indikator mutu baru;
\item
  Bila data akan dipublikasi ke masyarakat baik melalui
  website rumah sakit atau media lain
\item
  Ada perubahan pada pengukuran yang selama ini sudah dilakukan, misalnya perubahan profil indikator, instrumen pengumpulan data, proses agregasi data, atau perubahan staf pengumpul data atau validator
\item
  Bila terdapat perubahan hasil pengukuran tanpa diketahui sebabnya
\item
  Bila terdapat perubahan sumber data, misalnya terdapat perubahan sistem pencatatan pasien dari manual ke elektronik;
\item
  Bila terdapat perubahan subjek data seperti perubahan umur rata rata pasien, perubahan protokol riset, panduan praktik klinik baru diberlakukan, serta adanya teknologi dan metodologi pengobatan baru.
\end{enumerate}

\hypertarget{elemen-penilaian-pmkp-5}{%
\paragraph*{9. Elemen Penilaian PMKP 5}\label{elemen-penilaian-pmkp-5}}
\addcontentsline{toc}{paragraph}{9. Elemen Penilaian PMKP 5}

\begin{enumerate}
\def\labelenumi{\alph{enumi}.}
\tightlist
\item
  Rumah sakit telah melakukan validasi yang berbasis bukti meliputi poin a) ? f) yang ada pada maksud dan tujuan.
\item
  Pimpinan rumah sakit bertanggung jawab atas validitas dan kualitas data serta hasil yang dipublikasikan.
\end{enumerate}

\hypertarget{d.-pencapaian-dan-upaya-mempertahankan-perbaikan-mutu}{%
\subsection*{d.~Pencapaian dan upaya mempertahankan perbaikan mutu}\label{d.-pencapaian-dan-upaya-mempertahankan-perbaikan-mutu}}
\addcontentsline{toc}{subsection}{d.~Pencapaian dan upaya mempertahankan perbaikan mutu}

\hypertarget{standar-pmkp-6}{%
\paragraph*{1. Standar PMKP 6}\label{standar-pmkp-6}}
\addcontentsline{toc}{paragraph}{1. Standar PMKP 6}

Rumah sakit mencapai perbaikan mutu dan dipertahankan.

\hypertarget{maksud-dan-tujuan-pmkp-6}{%
\paragraph*{2. Maksud dan Tujuan PMKP 6}\label{maksud-dan-tujuan-pmkp-6}}
\addcontentsline{toc}{paragraph}{2. Maksud dan Tujuan PMKP 6}

Hasil analisis data digunakan untuk mengidentifkasi potensi perbaikan atau untuk mengurangi atau mencegah kejadian yang merugikan. Khususnya, perbaikan yang direncanakan untuk prioritas perbaikan tingkat rumah sakit yang sudah ditetapkan Direktur rumah sakit.

Rencana perbaikan perlu dilakukan uji coba dan selama masa uji dan dilakukan evaluasi hasilnya untuk membuktikan bahwa perbaikan sudah sesuai dengan yang diharapkan. Proses uji perbaikan ini dapat menggunakan metode-metode perbaikan yang sudah teruji misalnya PDCA Plan-Do-Chek-Action (PDCA) atau Plan-Do-Study-Action (PDSA) atau metode lain. Hal ini untuk memastikan bahwa terdapat perbaikan berkelanjutan untuk meningkatkan mutu dan keselamatan pasien. Perubahan yang efektif tersebut distandardisasi dengan cara membuat regulasi di rumah sakit misalnya kebijakan, SPO, dan lain-lainnya, dan harus di sosialisasikan kepada semua staf.

Perbaikan-perbaikan yang dicapai dan dipertahankan oleh rumah sakit didokumentasikan sebagai bagian dari pengelolaan peningkatan mutu dan keselamatan pasien di rumah sakit.

\hypertarget{elemen-penilaian-pmkp-6}{%
\paragraph*{3. Elemen Penilaian PMKP 6}\label{elemen-penilaian-pmkp-6}}
\addcontentsline{toc}{paragraph}{3. Elemen Penilaian PMKP 6}

\begin{enumerate}
\def\labelenumi{\alph{enumi}.}
\tightlist
\item
  Rumah sakit telah membuat rencana perbaikan dan melakukan uji coba menggunakan metode yang telah teruji dan menerapkannya untuk meningkatkan mutu dan keselamatan pasien.
\item
  Tersedia kesinambungan data mulai dari pengumpulan data sampai perbaikan yang dilakukan dan dapat dipertahankan.
\item
  Memiliki bukti perubahan regulasi atau perubahan proses yang diperlukan untuk mempertahankan perbaikan.
\item
  Keberhasilan telah didokumentasikan dan dijadikan laporan PMKP.
\end{enumerate}

\hypertarget{standar-pmkp-7}{%
\paragraph*{4. Standar PMKP 7}\label{standar-pmkp-7}}
\addcontentsline{toc}{paragraph}{4. Standar PMKP 7}

Dilakukan evaluasi proses pelaksanaan standar pelayanan kedokteran di rumah sakit untuk menunjang pengukuran mutu pelayanan klinis prioritas.

\hypertarget{maksud-dan-tujuan-pmkp-7}{%
\paragraph*{5. Maksud dan Tujuan PMKP 7}\label{maksud-dan-tujuan-pmkp-7}}
\addcontentsline{toc}{paragraph}{5. Maksud dan Tujuan PMKP 7}

Penerapan standar pelayanan kedokteran di rumah sakit berdasarkan panduan praktik klinis (PPK) dievaluasi menggunakan alur klinis/clinical pathway (CP).
Terkait dengan pengukuran prioritas perbaikan pelayanan klinis yang ditetapkan Direktur, maka Direktur bersama- sama dengan pimpinan medis, ketua Komite Medik dan Kelompok tenaga medis terkait menetapkan paling sedikit 5 (lima) evaluasi pelayanan prioritas standar pelayanan kedokteran. Evaluasi pelayanan prioritas standar pelayanan kedokteran dilakukan sampai terjadi pengurangan variasi dari data awal ke target yang ditentukan ketentuan rumah sakit.

Tujuan pemantauan pelaksanaan evaluasi perbaikan pelayanan klinis berupa standar pelayanan kedokteran sebagai berikut:

\begin{enumerate}
\def\labelenumi{\alph{enumi}.}
\tightlist
\item
  Mendorong tercapainya standardisasi proses asuhan klinik.
\item
  Mengurangi risiko dalam proses asuhan, terutama yang berkaitan asuhan kritis.
\item
  Memanfaatkan sumber daya yang tersedia dengan efisien dalam memberikan asuhan klinik tepat waktu dan efektif.
\item
  Memanfaatkan indikator prioritas sebagai indikator dalam penilaian kepatuhan penerapan alur klinis di area yang akan diperbaiki di tingkat rumah sakit.
\item
  Secara konsisten menggunakan praktik berbasis bukti (evidence based practices) dalam memberikan asuhan bermutu tinggi.
\end{enumerate}

Evaluasi prioritas standar pelayanan kedokteran tersebut dipergunakan untuk mengukur keberhasilan dan efisensi peningkatan mutu pelayanan klinis prioritas rumah sakit.
Evaluasi perbaikan pelayanan klinis berupa standar pelayanan kedokteran dapat dilakukan melalui audit medis dan atau audit klinis serta dapat menggunakan indikator mutu.

Tujuan evaluasi adalah untuk menilai efektivitas penerapan standar pelayanan kedokteran di rumah sakit sehingga standar pelayanan kedokteran di rumah sakit dapat mengurangi a variasi dari proses dan hasil serta berdampak terhadap efisiensi (kendali biaya). Misalnya:

\begin{enumerate}
\def\labelenumi{\alph{enumi}.}
\tightlist
\item
  Dalam PPK disebutkan bahwa tata laksana stroke non- hemoragik harus dilakukan secara multidisiplin dan dengan pemeriksaan serta intervensi dari hari ke hari dengan urutan tertentu. Karakteristik penyakit stroke non-hemoragik sesuai untuk dibuat alur klinis (clinical pathway/CP); sehingga perlu dibuat CP untuk stroke non-hemoragik.
\item
  Dalam PPK disebutkan bahwa pada pasien gagal ginjal kronik perlu dilakukan hemodialisis. Uraian rinci tentang hemodialisis dimuat dalam protokol hemodialisis pada dokumen terpisah.
\item
  Dalam PPK disebutkan bahwa pada anak dengan kejang demam kompleks perlu dilakukan pungsi lumbal. Uraian pelaksanaan pungsi lumbal tidak dimuat dalam PPK melainkan dalam prosedur pungsi lumbal dalam dokumen terpisah.
\item
  Dalam tata laksana kejang demam diperlukan pemberian diazepam rektal dengan dosis tertentu yang harus diberikan oleh perawat bila dokter tidak ada; ini diatur dalam ?standing order?.
\end{enumerate}

\hypertarget{elemen-penilaian-pmkp-7}{%
\paragraph*{6. Elemen Penilaian PMKP 7}\label{elemen-penilaian-pmkp-7}}
\addcontentsline{toc}{paragraph}{6. Elemen Penilaian PMKP 7}

\begin{enumerate}
\def\labelenumi{\alph{enumi}.}
\tightlist
\item
  Rumah sakit melakukan evaluasi clinical pathway
  sesuai yang tercantum dalam maksud dan tujuan.
\item
  Hasil evaluasi dapat menunjukkan adanya perbaikan terhadap kepatuhan dan mengurangi variasi dalam penerapan prioritas standar pelayanan kedokteran di rumah sakit.
\item
  Rumah sakit telah melaksanakan audit klinis dan atau audit medis pada penerapan prioritas standar pelayanan kedokteran di rumah sakit.
\end{enumerate}

\hypertarget{e.-sistem-pelaporan-dan-pembelajaran-keselamatan-pasien-rumah-sakit-sp2kp-rs}{%
\subsection*{e. Sistem pelaporan dan pembelajaran keselamatan pasien rumah sakit (SP2KP-RS)}\label{e.-sistem-pelaporan-dan-pembelajaran-keselamatan-pasien-rumah-sakit-sp2kp-rs}}
\addcontentsline{toc}{subsection}{e. Sistem pelaporan dan pembelajaran keselamatan pasien rumah sakit (SP2KP-RS)}

\hypertarget{standar-pmkp-8}{%
\paragraph*{1. Standar PMKP 8}\label{standar-pmkp-8}}
\addcontentsline{toc}{paragraph}{1. Standar PMKP 8}

Rumah sakit mengembangkan Sistem pelaporan dan pembelajaran keselamatan pasien di rumah sakit (SP2KP- RS).

\hypertarget{maksud-dan-tujuan-pmkp-8}{%
\paragraph*{2. Maksud dan Tujuan PMKP 8}\label{maksud-dan-tujuan-pmkp-8}}
\addcontentsline{toc}{paragraph}{2. Maksud dan Tujuan PMKP 8}

Sistem pelaporan dan pembelajaran keselamatan pasien di rumah sakit (SP2KP-RS). tersebut meliputi definisi kejadian sentinel, kejadian yang tidak diharapkan (KTD), kejadian tidak cedera (KTC), dan kejadian nyaris cedera (KNC atau near-miss) dan Kondisi potensial cedera signifikan (KPCS), mekanisme pelaporan insiden keselamatan pasien baik internal maupun eksternal, grading matriks risiko serta investigasi dan analisis insiden berdasarkan hasil grading tersebut.
Rumah sakit berpartisipasi untuk melaporkan insiden keselamatan pasien yang telah dilakukan investigasi dan analisis serta dilakukan pembelajaran ke KNKP sesuai peraturan perundang-undangan yang berlaku.

Insiden keselamatan pasien merupakan suatu kejadian yang tidak disengaja ketika memberikan asuhan kepada pasien (care management problem (CMP) atau kondisi yang berhubungan dengan lingkungan di rumah sakit termasuk infrastruktur, sarana prasarana (service delivery problem (SDP), yang dapat berpotensi atau telah menyebabkan bahaya bagi pasien.
Kejadian keselamatan pasien dapat namun tidak selalu merupakan hasil dari kecacatan pada sistem atau rancangan proses, kerusakan sistem, kegagalan alat, atau kesalahan manusia.

Definisi kejadian yang tidak diharapkan (KTD), kejadian tidak cedera (KTC), kejadian nyaris cedera (KNC), dan kondisi potensial cedera signifikan (KPCS), yang didefinisikan sebagai berikut:

\begin{enumerate}
\def\labelenumi{\alph{enumi}.}
\tightlist
\item
  Kejadian tidak diharapkan (KTD) adalah insiden keselamatan pasien yang menyebabkan cedera pada pasien.
\item
  Kejadian tidak cedera (KTC) adalah insiden keselamatan pasien yang sudah terpapar pada pasien namun tidak menyebabkan cedera.
\item
  Kejadian nyaris cedera (near-miss atau hampir cedera) atau KNC adanya insiden keselamatan pasien yang belum terpapar pada pasien.
\item
  Suatu kondisi potensial cedera signifikan (KPCS) adalah suatu kondisi (selain dari proses penyakit atau kondisi pasien itu sendiri) yang berpotensi menyebabkan kejadian sentinel
\item
  Kejadian Sentinel adalah suatu kejadian yang tidak berhubungan dengan perjalanan penyakit pasien atau penyakit yang mendasarinya yang terjadi pada pasien.
\end{enumerate}

Kejadian sentinel merupakan salah satu jenis insiden keselamatan pasien yang harus dilaporkan yang menyebabkan terjadinya hal-hal berikut ini:

\begin{enumerate}
\def\labelenumi{\alph{enumi}.}
\tightlist
\item
  Kematian.
\item
  Cedera permanen.
\item
  Cedera berat yang bersifat sementara/reversible.
\end{enumerate}

Cedera permanen adalah dampak yang dialami pasien yang bersifat ireversibel akibat insiden yang dialaminya misalnya kecacadan, kelumpuhan, kebutaan, tuli, dan lain-lainnya. Cedera berat yang bersifat sementara adalah cedera yang bersifat kritis dan dapat mengancam nyawa yang berlangsung dalam suatu kurun waktu tanpa terjadi cedera permanen/gejala sisa, namun kondisi tersebut mengharuskan pemindahan pasien ke tingkat perawatan yang lebih tinggi /pengawasan pasien untuk jangka waktu yang lama, pemindahan pasien ke tingkat perawatan yang lebih tinggi karena adanya kondisi yang mengancam nyawa, atau penambahan operasi besar, tindakan, atau tata laksana untuk menanggulangi kondisi tersebut.

Kejadian juga dapat digolongkan sebagai kejadian sentinel jika terjadi salah satu dari berikut ini:

\begin{enumerate}
\def\labelenumi{\alph{enumi}.}
\tightlist
\item
  Bunuh diri oleh pasien yang sedang dirawat, ditatalaksana, menerima pelayanan di unit yang selalu memiliki staf sepanjang hari atau dalam waktu 72 jam setelah pemulangan pasien, termasuk dari Unit Gawat Darurat (UGD) rumah sakit;
\item
  Kematian bayi cukup bulan yang tidak diantisipasi;
\item
  Bayi dipulangkan kepada orang tua yang salah;
\item
  Penculikan pasien yang sedang menerima perawatan, tata laksana, dan pelayanan;
\item
  Kaburnya pasien (atau pulang tanpa izin) dari unit perawatan yang selalu dijaga oleh staf sepanjang hari (termasuk UGD), yang menyebabkan kematian, cedera permanen, atau cedera sementara derajat berat bagi pasien tersebut;
\item
  Reaksi transfusi hemolitik yang melibatkan pemberian darah atau produk darah dengan inkompatibilitas golongan darah mayor (ABO, Rh, kelompok darah lainnya);
\item
  Pemerkosaan, kekerasan (yang menyebabkan kematian, cedera permanen, atau cedera sementara derajat berat) atau pembunuhan pasien yang sedang menerima perawatan, tata laksana, dan layanan ketika berada dalam lingkungan rumah sakit;
\item
  Pemerkosaan, kekerasan (yang menyebabkan kematian, cedera permanen, atau cedera sementara derajat berat) atau pembunuhan anggota staf, praktisi mandiri berizin, pengunjung, atau vendor ketika berada dalam lingkungan rumah sakit
\item
  Tindakan invasif, termasuk operasi yang dilakukan pada pasien yang salah, pada sisi yang salah, atau menggunakan prosedur yang salah (secara tidak sengaja);
\item
  Tertinggalnya benda asing dalam tubuh pasien secara tidak sengaja setelah suatu tindakan invasif, termasuk operasi;
\item
  Hiperbilirubinemia neonatal berat (bilirubin \textgreater30 mg/dL);
\item
  Fluoroskopi berkepanjangan dengan dosis kumulatif
  \textgreater1.500 rad pada satu medan tunggal atau pemberian radioterapi ke area tubuh yang salah atau pemberian radioterapi \textgreater25\% melebihi dosis radioterapi yang direncanakan;
\item
  Kebakaran, lidah api, atau asap, uap panas, atau pijaran yang tidak diantisipasi selama satu episode perawatan pasien;
\item
  Semua kematian ibu intrapartum (terkait dengan proses persalinan); atau
\item
  Morbiditas ibu derajat berat (terutama tidak berhubungan dengan perjalanan alamiah penyakit pasien atau kondisi lain yang mendasari) terjadi pada pasien dan menyebabkan cedera permanen atau cedera sementara derajat berat.
\end{enumerate}

Definisi kejadian sentinel meliputi poin a) hingga o) di atas dan dapat meliputi kejadian-kejadian lainnya seperti yang disyaratkan dalam peraturan atau dianggap sesuai oleh rumah sakit untuk ditambahkan ke dalam daftar kejadian sentinel. Komite/ Tim Penyelenggara Mutu segera membentuk tim investigator segera setelah menerima laporan kejadian sentinel. Semua kejadian yang memenuhi definisi tersebut dianalisis akar masalahnya secara komprehensif (RCA) dengan waktu tidak melebihi 45 (empat puluh lima) hari.

Tidak semua kesalahan menyebabkan kejadian sentinel, dan tidak semua kejadian sentinel terjadi akibat adanya suatu kesalahan. Mengidentifikasi suatu insiden sebagai kejadian sentinel tidak mengindikasikan adanya tanggungan hukum.

\hypertarget{elemen-penilaian-pmkp-8}{%
\paragraph*{3. Elemen Penilaian PMKP 8}\label{elemen-penilaian-pmkp-8}}
\addcontentsline{toc}{paragraph}{3. Elemen Penilaian PMKP 8}

\begin{enumerate}
\def\labelenumi{\alph{enumi}.}
\tightlist
\item
  Direktur menetapkan sistem pelaporan dan pembelajaran keselamatan pasien rumah sakit (SP2KP RS) termasuk didalamnya definisi, jenis insiden kselamatan pasien meliputi kejadian sentinel (poin a ? o) dalam bagian maksud dan tujuan), KTD, KNC, KTC dan KPCS, mekanisme pelaporan dan analisisnya serta pembelajarannya,
\item
  Komite/ Tim Penyelenggara Mutu membentuk tim investigator sesegera mungkin untuk melakukan investigasi komprehensif/analisis akar masalah (root cause analysis) pada semua kejadian sentinel dalam kurun waktu tidak melebihi 45 (empat puluh lima) hari.
\item
  Pimpinan rumah sakit melakukan tindakan perbaikan korektif dan memantau efektivitasnya untuk mencegah atau mengurangi berulangnya kejadian sentinel tersebut.
\item
  Pimpinan rumah sakit menetapkan proses untuk menganalisis KTD, KNC, KTC, KPCS dengan melakukan investigasi sederhana dengan kurun waktu yaitu grading biru tidak melebihi 7 (tujuh) hari, grading hijau tidak melebihi 14 (empat belas) hari.
\item
  Pimpinan rumah sakit melakukan tindakan perbaikan korektif dan memantau efektivitasnya untuk mencegah atau mengurangi berulangnya KTD, KNC, KTC, KPCS tersebut.
\end{enumerate}

\hypertarget{standar-pmkp-9}{%
\paragraph*{4. Standar PMKP 9}\label{standar-pmkp-9}}
\addcontentsline{toc}{paragraph}{4. Standar PMKP 9}

Data laporan insiden keselamatan pasien selalu dianalisis setiap 3 (tiga) bulan untuk memantau ketika muncul tren atau variasi yang tidak diinginkan.

\hypertarget{maksud-dan-tujuan-pmkp-9}{%
\paragraph*{5. Maksud dan Tujuan PMKP 9}\label{maksud-dan-tujuan-pmkp-9}}
\addcontentsline{toc}{paragraph}{5. Maksud dan Tujuan PMKP 9}

Komite/ Tim Penyelenggara Mutu melakukan analisis dan memantau insiden keselamatan pasien yang dilaporkan setiap triwulan untuk mendeteksi pola, tren serta mungkin variasi berdasarkan frekuensi pelayanan dan/atau risiko terhadap pasien.

Laporan insiden dan hasil Investigasi baik investigasi komprehensif (RCA) maupun investigasi sederhana (simple RCA) harus dilakukan untuk setidaknya hal-hal berikut ini:

\begin{enumerate}
\def\labelenumi{\alph{enumi}.}
\tightlist
\item
  Semua reaksi transfusi yang sudah dikonfirmasi,
\item
  Semua kejadian serius akibat reaksi obat (adverse drug reaction) yang serius sesuai yang ditetapkan oleh rumah sakit
\item
  Semua kesalahan pengobatan (medication error) yang signifikan sesuai yang ditetapkan oleh rumah sakit
\item
  Semua perbedaan besar antara diagnosis pra- dan diagnosis pascaoperasi; misalnya diagnosis praoperasi adalah obstruksi saluran pencernaan dan diagnosis pascaoperasi adalah ruptur aneurisme aorta abdominalis (AAA)
\item
  Kejadian tidsk diharapkan atau pola kejadian tidak diharapkan selama sedasi prosedural tanpa memandang cara pemberian
\item
  Kejadian tidak diharapkan atau pola kejadian tidak diharapkan selama anestesi tanpa memandang cara pemberian
\item
  Kejadian tidak diharapkan yang berkaitan dengan identifikasi pasien
\item
  Kejadian-kejadian lain, misalnya infeksi yang berkaitan dengan perawatan kesehatan atau wabah penyakit menular
\end{enumerate}

\hypertarget{elemen-penilaian-pmkp-9}{%
\paragraph*{6. Elemen Penilaian PMKP 9}\label{elemen-penilaian-pmkp-9}}
\addcontentsline{toc}{paragraph}{6. Elemen Penilaian PMKP 9}

\begin{enumerate}
\def\labelenumi{\alph{enumi}.}
\tightlist
\item
  Proses pengumpulan data sesuai a) sampai h) dari maksud dan tujuan, analisis, dan pelaporan diterapkan untuk memastikan akurasi data.
\item
  Analisis data mendalam dilakukan ketika terjadi tingkat, pola atau tren yang tak diharapkan yang digunakan untuk meningkatkan mutu dan keselamatan pasien.
\item
  Data luaran (outcome) dilaporkan kepada direktur dan representatif pemilik/ dewan pengawas sebagai bagian dari program peningkatan mutu dan keselamatan pasien.
\end{enumerate}

\hypertarget{standar-pmkp-10}{%
\paragraph*{7. Standar PMKP 10}\label{standar-pmkp-10}}
\addcontentsline{toc}{paragraph}{7. Standar PMKP 10}

Rumah sakit melakukan pengukuran dan evaluasi budaya keselamatan pasien.

\hypertarget{maksud-dan-tujuan-pmkp-10}{%
\paragraph*{8. Maksud dan Tujuan PMKP 10}\label{maksud-dan-tujuan-pmkp-10}}
\addcontentsline{toc}{paragraph}{8. Maksud dan Tujuan PMKP 10}

Pengukuran budaya keselamatan pasien perlu dilakukan oleh rumah sakit dengan melakukan survei budaya keselamatan pasien setiap tahun. Budaya keselamatan pasien juga dikenal sebagai budaya yang aman, yakni sebuah budaya organisasi yang mendorong setiap individu anggota staf (klinis atau administratif) melaporkan hal-hal yang menghawatirkan tentang keselamatan atau mutu pelayanan tanpa imbal jasa dari rumah sakit.

Direktur rumah sakit melakukan evaluasi rutin terhadap hasil survei budaya keselamatan pasien dengan melakukan analisis dan tindak lanjutnya.

\hypertarget{elemen-penilaian-pmkp-10}{%
\paragraph*{9. Elemen Penilaian PMKP 10}\label{elemen-penilaian-pmkp-10}}
\addcontentsline{toc}{paragraph}{9. Elemen Penilaian PMKP 10}

\begin{enumerate}
\def\labelenumi{\alph{enumi}.}
\tightlist
\item
  Rumah sakit telah melaksanakan pengukuran budaya keselamatan pasien dengan survei budaya keselamatan pasien setiap tahun menggunakan metode yang telah terbukti.
\item
  Hasil pengukuran budaya sebagai acuan dalam menyusun program peningkatan budaya keselamatan di rumah sakit.
\end{enumerate}

\hypertarget{f.-penerapan-manajemen-risiko}{%
\subsection*{f.~Penerapan manajemen risiko}\label{f.-penerapan-manajemen-risiko}}
\addcontentsline{toc}{subsection}{f.~Penerapan manajemen risiko}

\hypertarget{standar-pmkp-11}{%
\paragraph*{1. Standar PMKP 11}\label{standar-pmkp-11}}
\addcontentsline{toc}{paragraph}{1. Standar PMKP 11}

Komite/ Tim Penyelenggara Mutu memandu penerapan program manajemen risiko di rumah sakit

\hypertarget{maksud-dan-tujuan-pmkp-11}{%
\paragraph*{2. Maksud dan Tujuan PMKP 11}\label{maksud-dan-tujuan-pmkp-11}}
\addcontentsline{toc}{paragraph}{2. Maksud dan Tujuan PMKP 11}

Komite/ Tim Penyelenggara Mutu membuat daftar risiko tingkat rumah sakit berdasarkan daftar risiko yang dibuat tiap unit setiap tahun. Berdasarkan daftar risiko tersebut ditentukan prioritas risiko yang dimasukkan dalam profil risiko rumah sakit. Profil risiko tersebut akan menjadi bahan dalam penyusunan Program manajemen risiko rumah sakit dan menjadi prioritas untuk dilakukan penanganan dan pemantauannya. Direktur rumah sakit juga berperan dalam memilih selera risiko yaitu tingkat risiko yang bersedia diambil rumah sakit dalam upayanya mewujudkan tujuan dan sasaran yang dikehendakinya.

Ada beberapa metode untuk melakukan analisis risiko secara proaktif yaitu failure mode effect analysis (analisis modus kegagalan dan dampaknya /FMEA/ AMKD), analisis kerentanan terhadap bahaya/hazard vulnerability analysis (HVA) dan infection control risk assessment (pengkajian risiko pengendalian infeksi/ICRA). Rumah sakit mengintegrasikan hasil analisis metode-metode tersebut dalam program manajemen risiko rumah sakit. Pimpinan rumah sakit akan mendesain ulang proses berisiko tinggi yang telah di analisis secara proaktif dengan melakukan tindakan untuk mengurangi risiko dalam proses tersebut. Proses analisis risiko proaktif ini dilaksanakan minimal sekali dalam setahun dan didokumentasikan pelaksanaannya.

\hypertarget{elemen-penilaian-pmkp-11}{%
\paragraph*{3. Elemen penilaian PMKP 11}\label{elemen-penilaian-pmkp-11}}
\addcontentsline{toc}{paragraph}{3. Elemen penilaian PMKP 11}

\begin{enumerate}
\def\labelenumi{\alph{enumi}.}
\tightlist
\item
  Komite/ Tim Penyelenggara Mutu memandu penerapan program manajemen risiko yang di tetapkan oleh Direktur
\item
  Komite/ Tim Penyelenggara Mutu telah membuat daftar risiko rumah sakit berdasarkan daftar risiko unit-unit di rumah sakit
\item
  Komite/ Tim Penyelenggara Mutu telah membuat profil risiko dan rencana penanganan
\item
  Komite/ Tim Penyelenggara Mutu telah membuat pemantauan terhadap rencana penanganan dan melaporkan kepada direktur dan representatif pemilik/dewan pengawas setiap 6 (enam) bulan
\item
  Komite/ Tim Penyelenggara Mutu telah menyusun Program manajemen risiko tingkat rumah sakit untuk ditetapkan Direktur
\item
  Komite/ Tim Penyelenggara Mutu telah memandu pemilihan minimal satu analisis secara proaktif proses berisiko tinggi yang diprioritaskan untuk dilakukan analisis FMEA setiap tahun.
\end{enumerate}

\hypertarget{manajemen-rekam-medis-dan-informasi-kesehatan-mrmik}{%
\section*{5. Manajemen Rekam Medis dan Informasi Kesehatan (MRMIK)}\label{manajemen-rekam-medis-dan-informasi-kesehatan-mrmik}}
\addcontentsline{toc}{section}{5. Manajemen Rekam Medis dan Informasi Kesehatan (MRMIK)}

\textbf{Gambaran Umum}

Setiap rumah sakit memiliki, mengelola, dan menggunakan informasi untuk meningkatkan luaran ( outcome ) bagi pasien, kinerja staf dan kinerja rumah sakit secara umum.

Dalam melakukan proses manajemen informasi, rumah sakit menggunakan metode pengembangan yang sesuai dengan sumber daya rumah sakit, dengan memperhatikan perkembangan teknologi informasi. Proses manajemen informasi tersebut juga mencakup:

\begin{enumerate}
\def\labelenumi{\alph{enumi}.}
\tightlist
\item
  Misi rumah sakit,
\item
  Layanan yang diberikan,
\item
  Sumber daya,
\item
  Akses ke teknologi informasi kesehatan, dan
\item
  Dukungan untuk menciptakan komunikasi efektif antar Professional Pemberi Asuhan (PPA).
\end{enumerate}

Untuk memberikan asuhan pasien yang terkoordinasi dan terintegrasi, rumah sakit bergantung pada informasi tentang perawatan pasien. Informasi merupakan salah satu sumber daya yang harus dikelola secara efektif oleh pimpinan rumah sakit.
Pelaksanaan asuhan pasien di rumah sakit adalah suatu proses yang kompleks yang sangat bergantung pada komunikasi dan informasi. Komunikasi dilakukan antara rumah sakit dengan pasien dan keluarga, antar Professional Pemberi Asuhan (PPA), serta komunitas di wilayah rumah sakit. Kegagalan dalam komunikasi adalah salah satu akar masalah pada insiden keselamatan pasien yang paling sering dijumpai. Sering kali, kegagalan komunikasi terjadi akibat tulisan yang tidak terbaca, penggunaan singkatan, simbol dan kode yang tidak seragam di dalam rumah sakit.

Seiring dengan perjalanan waktu dan perkembangannya, rumah sakit diharapkan mampu mengelola informasi secara lebih efektif dalam hal:

\begin{enumerate}
\def\labelenumi{\alph{enumi}.}
\tightlist
\item
  Mengidentifikasi kebutuhan informasi dan teknologi informasi;
\item
  Mengembangkan sistem informasi manajemen;
\item
  Menetapkan jenis informasi dan cara memperoleh data yang diperlukan;
\item
  Menganalisis data dan mengubahnya menjadi informasi;
\item
  Memaparkan dan melaporkan data serta informasi kepada publik;
\item
  Melindungi kerahasiaan, keamanan, dan integritas data dan informasi;
\item
  Mengintegrasikan dan menggunakan informasi untuk peningkatan kinerja.
\end{enumerate}

Walaupun komputerisasi dan teknologi lainnya dikembangkan untuk meningkatkan efisiensi, prinsip teknologi informasi yang baik harus diterapkan untuk seluruh metode dokumentasi. Standar ini dirancang untuk digunakan pada sistem informasi berbasis kertas serta elektronik.

Informasi rumah sakit terkait asuhan pasien sangat penting dalam komunikasi antar PPA, yang didokumentasikan dalam Rekam Medis. Rekam medis (RM) adalah bukti tertulis (kertas/elektronik) yang merekam berbagai informasi kesehatan pasien seperti hasil pengkajian, rencana dan pelaksanaan asuhan, pengobatan, catatan perkembangan pasien terintegrasi, serta ringkasan pasien pulang yang dibuat oleh Profesional Pemberi Asuhan (PPA). Penyelenggaraan rekam medis merupakan proses kegiatan yang dimulai saat pasien diterima di rumah sakit dan melaksanakan rencana asuhan dari PPA. Kegiatan dilanjutkan dengan penanganan rekam medis yang meliputi penyimpanan dan penggunaan untuk kepentingan pasien atau keperluan lainnya.
Dalam pemberian pelayanan kepada pasien, teknologi informasi kesehatan sangat dibutuhkan untuk meningkatkan efektifitas, efisiensi dan keamanan dalam proses komunikasi dan informasi.

Standar Manajemen Rekam Medis dan Informasi Kesehatan ini berfokus pada:

\begin{enumerate}
\def\labelenumi{\alph{enumi}.}
\tightlist
\item
  Manajemen informasi
\item
  Pengelolaan dokumen
\item
  Rekam medis pasien
\item
  Teknologi Informasi Kesehatan di Pelayanan Kesehatan
\end{enumerate}

\hypertarget{a.-manajemen-informasi}{%
\subsection*{a. Manajemen Informasi}\label{a.-manajemen-informasi}}
\addcontentsline{toc}{subsection}{a. Manajemen Informasi}

\hypertarget{standar-mrmik-1}{%
\paragraph*{1. Standar MRMIK 1}\label{standar-mrmik-1}}
\addcontentsline{toc}{paragraph}{1. Standar MRMIK 1}

Rumah sakit menetapkan proses manajemen informasi untuk memenuhi kebutuhan informasi internal maupun eksternal.

\hypertarget{maksud-dan-tujuan-mrmik-1}{%
\paragraph*{2. Maksud dan Tujuan MRMIK 1}\label{maksud-dan-tujuan-mrmik-1}}
\addcontentsline{toc}{paragraph}{2. Maksud dan Tujuan MRMIK 1}

Informasi yang diperoleh selama masa perawatan pasien harus dapat dikelola dengan aman dan efektif oleh rumah sakit. Kemampuan memperoleh dan menyediakan informasi tersebut memerlukan perencanaan yang efektif. Perencanaan ini melibatkan masukan dari berbagai sumber yang membutuhkan data dan informasi, termasuk:

\begin{enumerate}
\def\labelenumi{\alph{enumi}.}
\tightlist
\item
  Profesional Pemberi Asuhan (PPA) yang memberikan pelayanan kepada pasien
\item
  Pimpinan rumah sakit dan para kepala departemen/unit layanan
\item
  Staf, unit pelayanan, dan badan/individu di luar rumah sakit yang membutuhkan atau memerlukan data atau informasi tentang operasional dan proses perawatan rumah sakit
\end{enumerate}

Dalam menyusun perencanaan, ditentukan prioritas kebutuhan informasi dari sumber-sumber strategi manajemen informasi rumah sakit sesuai dengan ukuran rumah sakit, kompleksitas pelayanan, ketersediaan staf terlatih, dan sumber daya manusia serta teknikal lainnya. Perencanaan yang komprehensif meliputi seluruh unit kerja dan pelayanan yang ada di rumah sakit.
Rumah sakit melakukan pemantauan dan evaluasi secara berkala sesuai ketentuan rumah sakit terhadap perencanaan tersebut. Selanjutnya, rumah sakit melakukan upaya perbaikan berdasarkan hasil pemantauan dan evaluasi berkala yang telah dilakukan .
Apabila rumah sakit menyelenggarakan program penelitian dan atau pendidikan kesehatan maka pengelolaan terdapat data dan informasi yang mendukung asuhan pasien, pendidikan, serta riset telah tersedia tepat waktu dari sumber data terkini.

\hypertarget{elemen-penilaian-mrmik-1}{%
\paragraph*{3. Elemen Penilaian MRMIK 1}\label{elemen-penilaian-mrmik-1}}
\addcontentsline{toc}{paragraph}{3. Elemen Penilaian MRMIK 1}

\begin{enumerate}
\def\labelenumi{\alph{enumi}.}
\tightlist
\item
  Rumah sakit menetapkan regulasi pengelolaan informasi untuk memenuhi kebutuhan informasi sesuai poin a) ? g) yang terdapat dalam gambaran umum.
\item
  Terdapat bukti rumah sakit telah menerapkan proses pengelolaan informasi untuk memenuhi kebutuhan PPA, pimpinan rumah sakit, kepala departemen/unit layanan dan badan/individu dari luar rumah sakit.
\item
  Proses yang diterapkan sesuai dengan ukuran rumah sakit, kompleksitas layanan, ketersediaan staf terlatih, sumber daya teknis, dan sumber daya lainnya.
\item
  Rumah sakit melakukan pemantauan dan evaluasi secara berkala sesuai ketentuan rumah sakit serta upaya perbaikan terhadap pemenuhan informasi internal dan eksternal dalam mendukung asuhan, pelayanan, dan mutu serta keselamatan pasien.
\item
  Apabila terdapat program penelitian dan atau pendidikan Kesehatan di rumah sakit, terdapat bukti bahwa data dan informasi yang mendukung asuhan pasien, pendidikan, serta riset telah tersedia tepat waktu dari sumber data terkini.
\end{enumerate}

\hypertarget{standar-mrmik-2}{%
\paragraph*{4. Standar MRMIK 2}\label{standar-mrmik-2}}
\addcontentsline{toc}{paragraph}{4. Standar MRMIK 2}

Seluruh komponen dalam rumah sakit termasuk pimpinan rumah sakit, PPA, kepala unit klinis/non klinis dan staf dilatih mengenai prinsip manajemen dan penggunaan informasi.

\hypertarget{maksud-dan-tujuan-mrmik-2}{%
\paragraph*{5. Maksud dan Tujuan MRMIK 2}\label{maksud-dan-tujuan-mrmik-2}}
\addcontentsline{toc}{paragraph}{5. Maksud dan Tujuan MRMIK 2}

Seluruh komponen dalam rumah sakit termasuk pimpinan rumah sakit, PPA, kepala unit klinis/non klinis dan staf akan mengumpulkan dan menganalisis, serta menggunakan data dan informasi. Dengan demikian, mereka harus dilatih tentang prinsip pengelolaan dan penggunaan informasi agar dapat berpartisipasi secara efektif.

Pelatihan tersebut berfokus pada:

\begin{enumerate}
\def\labelenumi{\alph{enumi}.}
\tightlist
\item
  penggunakan sistem informasi, seperti sistem rekam medis elektronik, untuk melaksanakan tanggung jawab pekerjaan mereka secara efektif dan menyelenggarakan perawatan secara efisien dan aman;
\item
  Pemahaman terhadap kebijakan dan prosedur untuk memastikan keamanan dan kerahasiaan data dan informasi;
\item
  Pemahaman dan penerapan strategi untuk pengelolaan data, informasi, dan dokumentasi selama waktu henti (downtime ) yang direncanakan dan tidak terencana;
\item
  Penggunaan data dan informasi untuk membantu pengambilan keputusan;
\item
  Komunikasi yang mendukung partisipasi pasien dan keluarga dalam proses perawatan; dan
\item
  Pemantauan dan evaluasi untuk mengkaji dan meningkatkan proses kerja serta perawatan.
\end{enumerate}

Semua staf dilatih sesuai tanggung jawab, uraian tugas, serta kebutuhan data dan informasi. Rumah sakit yang menggunakan sistem rekam medis elektronik harus memastikan bahwa staf yang dapat mengakses, meninjau, dan/atau mendokumentasikan dalam rekam medis pasien telah mendapatkan edukasi untuk menggunakan sistem secara efektif dan efisien.

PPA, peneliti, pendidik, kepala unit klinis / non klinis sering kali membutuhkan informasi untuk membantu mereka dalam pelaksanaan tanggung jawab. Informasi demikian termasuk literatur ilmiah dan manajemen, panduan praktik klinis, hasil penelitian, metode pendidikan. Internet, materi cetakan di perpustakaan, sumber pencarian daring (on- line), dan materi pribadi yang semuanya merupakan sumber yang bernilai sebagai informasi terkini.

Proses manajemen informasi memungkinkan penggabungan informasi dari berbagai sumber dan menyusun laporan untuk menunjang pengambilan keputusan. Secara khusus, kombinasi informasi klinis dan non klinis membantu pimpinan departemen/pelayanan untuk menyusun rencana secara kolaboratif. Proses manajemen informasi mendukung para pimpinan departemen/pelayanan dengan data perbandingan dan data longitudinal terintegrasi.

\hypertarget{elemen-penilaian-mrmik-2}{%
\paragraph*{6. Elemen Penilaian MRMIK 2}\label{elemen-penilaian-mrmik-2}}
\addcontentsline{toc}{paragraph}{6. Elemen Penilaian MRMIK 2}

\begin{enumerate}
\def\labelenumi{\alph{enumi}.}
\tightlist
\item
  Terdapat bukti PPA, pimpinan rumah sakit, kepala departemen, unit layanan dan staf telah dilatih tentang prinsip pengelolaan dan penggunaan sistem informasi sesuai dengan peran dan tanggung jawab mereka.
\item
  Terdapat bukti bahwa data dan informasi klinis serta non klinis diintegrasikan sesuai kebutuhan dan digunakan dalam mendukung proses pengambilan keputusan.
\end{enumerate}

\hypertarget{standar-mrmik-2.1}{%
\paragraph*{7. Standar MRMIK 2.1}\label{standar-mrmik-2.1}}
\addcontentsline{toc}{paragraph}{7. Standar MRMIK 2.1}

Rumah sakit menjaga kerahasiaan, keamanan, privasi, integritas data dan informasi melalui proses untuk mengelola dan mengontrol akses.

\hypertarget{standar-mrmik-2.2}{%
\paragraph*{8. Standar MRMIK 2.2}\label{standar-mrmik-2.2}}
\addcontentsline{toc}{paragraph}{8. Standar MRMIK 2.2}

Rumah sakit menjaga kerahasiaan, keamanan, privasi, integritas data dan informasi melalui proses yang melindungi data dan informasi dari kehilangan, pencurian, kerusakan, dan penghancuran.

\hypertarget{maksud-dan-tujuan-mrmik-2.1-dan-mrmik-2.2}{%
\paragraph*{9. Maksud dan Tujuan MRMIK 2.1 dan MRMIK 2.2}\label{maksud-dan-tujuan-mrmik-2.1-dan-mrmik-2.2}}
\addcontentsline{toc}{paragraph}{9. Maksud dan Tujuan MRMIK 2.1 dan MRMIK 2.2}

Rumah sakit menjaga kerahasiaan, keamanan, integritas data dan informasi pasien yang bersifat sensitif. Keseimbangan antara keterbukaan dan kerahasiaan data harus diperhatikan. Tanpa memandang apakah rumah sakit menggunakan sistem informasi menggunakan kertas dan/atau elektronik, rumah sakit harus menerapkan langkah-langkah untuk mengamankan dan melindungi data dan informasi yang dimiliki.

Data dan informasi meliputi rekam medis pasien, data dari peralatan dan perangkat medis, data penelitian, data mutu, data tagihan, data sumber daya manusia, data operasional dan keuangan serta sumber lainnya, sebagaimana berlaku untuk rumah sakit. Langkah-langkah keamanan mencakup proses untuk mengelola dan mengontrol akses. Sebagai contoh, untuk menjaga kerahasiaan dan keamanan rekam medis pasien, rumah sakit menentukan siapa yang berwenang untuk mengakses rekam medis dan tingkat akses individu yang berwenang terhadap rekam medis tersebut. Jika menggunakan sistem informasi elektronik, rumah sakit mengimplementasikan proses untuk memberikan otorisasi kepada pengguna yang berwenang sesuai dengan tingkat akses mereka.

Bergantung pada tingkat aksesnya, pengguna yang berwenang dapat memasukkan data, memodifikasi, dan menghapus informasi, atau hanya memiliki akses untuk hanya membaca atau akses terbatas ke beberapa sistem/modul. Tingkat akses untuk sistem rekam medis elektronik dapat mengidentifikasi siapa yang dapat mengakses dan membuat entry dalam rekam medis, memasukkan instruksi untuk pasien, dan sebagainya. Rumah sakit juga menentukan tingkat akses untuk data lainnya seperti data peningkatan mutu, data laporan keuangan, dan data kinerja rumah sakit. Setiap staf memiliki tingkat akses dan kewenangan yang berbeda atas data dan informasi sesuai dengan kebutuhan, peran dan tanggung jawab staf tersebut.

Proses pemberian otorisasi yang efektif harus mendefinisikan:

\begin{enumerate}
\def\labelenumi{\alph{enumi}.}
\tightlist
\item
  Siapa yang memiliki akses terhadap data dan informasi, termasuk rekam medis pasien;
\item
  Informasi mana yang dapat diakses oleh staf tertentu (dan tingkat aksesnya);
\item
  Proses untuk memberikan hak akses kepada staf yang berwenang;
\item
  Kewajiban staf untuk menjaga kerahasiaan dan keamanan informasi;
\item
  Proses untuk menjaga integritas data (keakuratan, konsistensi, dan kelengkapannya); dan
\item
  Proses yang dilakukan apabila terjadi pelanggaran terhadap kerahasiaan, keamanan, ataupun integritas data.
\end{enumerate}

Untuk rumah sakit dengan sistem informasi elektronik, pemantauan terhadap data dan informasi pasien melalui audit keamanan terhadap penggunaan akses dapat membantu melindungi kerahasiaan dan keamanan. Rumah sakit menerapkan proses untuk secara proaktif memantau catatan penggunaan akses. Pemantauan keamanan dilakukan secara rutin sesuai ketentuan rumah sakit untuk mengidentifikasi kerentanan sistem dan pelanggaran terhadap kebijakan kerahasiaan dan keamanan.

Misalnya, sebagai bagian dari proses ini, rumah sakit dapat mengidentifikasi pengguna sistem yang telah mengubah,
mengedit, atau menghapus informasi dan melacak perubahan yang dibuat pada rekam medis elektronik. Hasil proses pemantauan tersebut dapat digunakan untuk melakukan validasi apakah penggunaan akses dan otorisasi telah diterapkan dengan tepat. Pemantauan keamanan juga efektif dalam mengidentifikasi kerentanan dalam keamanan, seperti adanya akses pengguna yang perlu diperbarui atau dihapus karena perubahan atau pergantian staf.

Saat menggunakan rekam medis elektronik, langkah- langkah keamanan tambahan untuk masuk/login ke dalam sistem harus diterapkan. Sebagai contoh, rumah sakit memiliki proses untuk memastikan bahwa staf mengakses sistem (login) menggunakan kredensial unik yang diberikan hanya untuk mereka dan kredensial tersebut tidak dipakai bersama orang lain. Selain proses untuk mengelola dan mengendalikan akses, rumah sakit memastikan bahwa seluruh data dan informasi rekam medis berbentuk cetak atau elektronik dilindungi dari kehilangan, pencurian, gangguan, kerusakan, dan penghancuran yang tidak diinginkan.

Penting bagi rumah sakit untuk menjaga dan memantau keamanan data dan informasi, baik yang disimpan dalam bentuk cetak maupun elektronik terhadap kehilangan, pencurian dan akses orang yang tidak berwenang. Rumah sakit menerapkan praktik terbaik untuk keamanan data dan memastikan penyimpanan catatan, data, dan informasi medis yang aman dan terjamin.

Contoh langkah-langkah dan strategi keamanan termasuk, tetapi tidak terbatas pada, berikut ini:

\begin{enumerate}
\def\labelenumi{\alph{enumi}.}
\tightlist
\item
  Memastikan perangkat lunak keamanan dan pembaruan sistem sudah menggunakan versi terkini dan terbaru
\item
  Melakukan enkripsi data, terutama untuk data yang disimpan dalam bentuk digital
\item
  Melindungi data dan informasi melalui strategi cadangan (back up) seperti penyimpanan di luar lokasi dan/atau layanan pencadangan cloud
\item
  Menyimpan dokumen fisik rekam medis di lokasi yang tidak terkena panas serta aman dari air dan api
\item
  Menyimpan dokumen rekam medis aktif di area yang hanya dapat diakses oleh staf yang berwenang.
\item
  Memastikan bahwa ruang server dan ruang untuk penyimpanan dokumen fisik rekam medis lainnya aman dan hanya dapat diakses oleh staf yang berwenang
\item
  Memastikan bahwa ruang server dan ruang untuk penyimpanan rekam medis fisik memiliki suhu dan tingkat kelembaban yang tepat.
\end{enumerate}

\hypertarget{elemen-penilaian-mrmik-2.1}{%
\paragraph*{10. Elemen Penilaian MRMIK 2.1}\label{elemen-penilaian-mrmik-2.1}}
\addcontentsline{toc}{paragraph}{10. Elemen Penilaian MRMIK 2.1}

\begin{enumerate}
\def\labelenumi{\alph{enumi}.}
\tightlist
\item
  Rumah sakit menerapkan proses untuk memastikan kerahasiaan, keamanan, dan integritas data dan informasi sesuai dengan peraturan perundangan.
\item
  Rumah sakit menerapkan proses pemberian akses kepada staf yang berwenang untuk mengakses data dan informasi, termasuk entry ke dalam rekam medis pasien.
\item
  Rumah sakit memantau kepatuhan terhadap proses ini dan mengambil tindakan ketika terjadi terjadi pelanggaran terhadap kerahasiaan, keamanan, atau integritas data.
\end{enumerate}

\hypertarget{elemen-penilaian-mrmik-2.2}{%
\paragraph*{11. Elemen Penilaian MRMIK 2.2}\label{elemen-penilaian-mrmik-2.2}}
\addcontentsline{toc}{paragraph}{11. Elemen Penilaian MRMIK 2.2}

\begin{enumerate}
\def\labelenumi{\alph{enumi}.}
\tightlist
\item
  Data dan informasi yang disimpan terlindung dari kehilangan, pencurian, kerusakan, dan penghancuran.
\item
  Rumah sakit menerapkan pemantauan dan evaluasi terhadap keamanan data dan informasi.
\item
  Terdapat bukti rumah sakit telah melakukan tindakan perbaikan untuk meningkatkan keamanan data dan informasi.
\end{enumerate}

\hypertarget{b.-pengelolaan-dokumen}{%
\subsection*{b. Pengelolaan dokumen}\label{b.-pengelolaan-dokumen}}
\addcontentsline{toc}{subsection}{b. Pengelolaan dokumen}

\hypertarget{standar-mrmik-3}{%
\paragraph*{1. Standar MRMIK 3}\label{standar-mrmik-3}}
\addcontentsline{toc}{paragraph}{1. Standar MRMIK 3}

Rumah Sakit menerapkan proses pengelolaan dokumen, termasuk kebijakan, pedoman, prosedur, dan program kerja secara konsisten dan seragam.

\hypertarget{maksud-dan-tujuan-mrmik-3}{%
\paragraph*{2. Maksud dan Tujuan MRMIK 3}\label{maksud-dan-tujuan-mrmik-3}}
\addcontentsline{toc}{paragraph}{2. Maksud dan Tujuan MRMIK 3}

Kebijakan dan prosedur bertujuan untuk memberikan acuan yang seragam mengenai fungsi klinis dan non-klinis di rumah sakit. Rumah Sakit dapat membuat Tata naskah untuk memandu cara menyusun dan mengendalikan dokumen misalnya kebijakan, prosedur, dan program rumah sakit. Dokumen pedoman tata naskah mencakup beberapa komponen kunci sebagai berikut:

\begin{enumerate}
\def\labelenumi{\alph{enumi}.}
\tightlist
\item
  Peninjauan dan persetujuan semua dokumen oleh pihak yang berwenang sebelum diterbitkan
\item
  Proses dan frekuensi peninjauan dokumen serta persetujuan berkelanjutan
\item
  Pengendalian untuk memastikan bahwa hanya dokumen versi terbaru/terkini dan relevan yang tersedia
\item
  Bagaimana mengidentifikasi adanya perubahan dalam dokumen
\item
  Pemeliharaan identitas dan keterbacaan dokumen
\item
  Proses pengelolaan dokumen yang berasal dari luar rumah sakit
\item
  Penyimpanan dokumen lama yang sudah tidak terpakai (obsolete) setidaknya selama waktu yang ditentukan oleh peraturan perundangan, sekaligus memastikan bahwa dokumen tersebut tidak akan salah digunakan
\item
  Identifikasi dan pelacakan semua dokumen yang beredar (misalnya, diidentifikasi berdasarkan judul, tanggal terbit, edisi dan/atau tanggal revisi terbaru, jumlah halaman, dan nama orang yang mensahkan pada saat penerbitan dan revisi dan/atau meninjau dokumen tersebut)
\end{enumerate}

Proses-proses tersebut diterapkan dalam menyusun serta memelihara dokumen termasuk kebijakan, prosedur, dan program kerja.
Dokumen internal rumah sakit terdiri dari regulasi dan dokumen pelaksanaan. Terdapat beberapa tingkat dokumen internal, yaitu:

\begin{enumerate}
\def\labelenumi{\alph{enumi}.}
\tightlist
\item
  dokumen tingkat pemilik/korporasi;
\item
  dokumen tingkat rumah sakit; dan
\item
  dokumen tingkat unit (klinis dan non klinis), mencakup:
\end{enumerate}

\begin{enumerate}
\def\labelenumi{\arabic{enumi}.}
\tightlist
\item
  Kebijakan di tingkat unit (klinis dan non klinis)
\item
  Pedoman pengorganisasian
\item
  Pedoman pelayanan/penyelenggaraan
\item
  Standar operasional prosedur (SOP)
\item
  Program kerja unit (tahunan)
\end{enumerate}

\hypertarget{elemen-penilaian-mrmik-3}{%
\paragraph*{3. Elemen Penilaian MRMIK 3}\label{elemen-penilaian-mrmik-3}}
\addcontentsline{toc}{paragraph}{3. Elemen Penilaian MRMIK 3}

\begin{enumerate}
\def\labelenumi{\alph{enumi}.}
\tightlist
\item
  Rumah sakit menerapkan pengelolaan dokumen sesuai dengan butir a) ? h) dalam maksud dan tujuan.
\item
  Rumah sakit memiliki dan menerapkan format yang seragam untuk semua dokumen sejenis sesuai dengan ketentuan rumah sakit.
\item
  Rumah sakit telah memiliki dokumen internal mencakup butir a) ? c) dalam maksud dan tujuan.
\end{enumerate}

\hypertarget{standar-mrmik-4}{%
\paragraph*{4. Standar MRMIK 4}\label{standar-mrmik-4}}
\addcontentsline{toc}{paragraph}{4. Standar MRMIK 4}

Kebutuhan data dan informasi dari pihak dalam dan luar rumah sakit dipenuhi secara tepat waktu dalam format yang memenuhi harapan pengguna dan dengan frekuensi yang diinginkan.

\hypertarget{maksud-dan-tujuan-mrmik-4}{%
\paragraph*{5. Maksud dan Tujuan MRMIK 4}\label{maksud-dan-tujuan-mrmik-4}}
\addcontentsline{toc}{paragraph}{5. Maksud dan Tujuan MRMIK 4}

Penyebaran data dan informasi untuk memenuhi kebutuhan pihak di dalam dan di luar rumah sakit merupakan aspek penting dari manajemen informasi. Rumah sakit menetapkan mekanisme untuk melakukan penyebaran data secara internal dan eksternal. Mekanisme tersebut mengatur agar data yang diberikan tepat waktu dan menggunakan format yang ditetapkan.

Secara internal, penyebaran data dan informasi dapat dilakukan antar Profesional Pemberi Asuhan (PPA) yang merawat pasien, termasuk dokter, perawat, dietisien, apoteker, dan staf klinis lainnya yang memerlukan akses ke informasi terbaru dan semua bagian dari rekam medis pasien.

Secara eksternal, rumah sakit dapat memberikan data dan informasi kepada Kementerian Kesehatan, dinas kesehatan, tenaga kesehatan (seperti dokter perawatan primer pasien di komunitas), layanan dan organisasi kesehatan luar (seperti laboratorium luar atau rumah sakit rujukan), dan individu (seperti pasien yang meminta rekam medis mereka setelah keluar dari rumah sakit).

Format dan kerangka waktu untuk menyebarkan data dan informasi dirancang untuk memenuhi harapan pengguna sesuai dengan layanan yang diberikan. Ketika data dan informasi dibutuhkan untuk perawatan pasien, data dan informasi tersebut harus disediakan pada waktu yang tepat guna mendukung kesinambungan perawatan dan keselamatan pasien.

Contoh penyebaran informasi untuk memenuhi harapan pengguna meliputi beberapa hal di bawah ini namun tidak terbatas pada :

\begin{enumerate}
\def\labelenumi{\alph{enumi}.}
\tightlist
\item
  Pelaporan dan pembaharuan data rumah sakit yang terdapat di aplikasi RS Online Kementerian Kesehatan;
\item
  Data kunjungan rumah sakit, data pelayanan rumah sakit seperti pelayanan laboratorium dan radiologi, data indikator layanan rumah sakit, morbiditas, mortalitas dan sepuluh besar penyakit di rawat jalan dan rawat inap dengan menggunakan kode diagnosis ICD 10 pada aplikasi SIRS Online Kementerian Kesehatan;
\item
  Memberikan data dan informasi spesifik yang diminta/dibutuhkan;
\item
  Menyediakan laporan dengan frekuensi yang dibutuhkan oleh staf atau rumah sakit;
\item
  Menyediakan data dan informasi dalam format yang memudahkan penggunaannya;
\item
  Menghubungkan sumber data dan informasi; dan
\item
  Menginterpretasi atau mengklarifikasi data.
\end{enumerate}

\hypertarget{elemen-penilaian-mrmik-4}{%
\paragraph*{6. Elemen Penilaian MRMIK 4}\label{elemen-penilaian-mrmik-4}}
\addcontentsline{toc}{paragraph}{6. Elemen Penilaian MRMIK 4}

\begin{enumerate}
\def\labelenumi{\alph{enumi}.}
\tightlist
\item
  Terdapat bukti bahwa penyebaran data dan informasi memenuhi kebutuhan internal dan eksternal rumah sakit sesuai dengan yang tercantum dalam maksud dan tujuan.
\item
  Terdapat proses yang memastikan bahwa data dan informasi yang dibutuhkan untuk perawatan pasien telah diterima tepat waktu dan sesuai format yang seragam dan sesuai dengan kebutuhan.
\end{enumerate}

\hypertarget{c.-rekam-medis-pasien}{%
\subsection*{c.~Rekam medis pasien}\label{c.-rekam-medis-pasien}}
\addcontentsline{toc}{subsection}{c.~Rekam medis pasien}

\hypertarget{standar-mrmik-5}{%
\paragraph*{1. Standar MRMIK 5}\label{standar-mrmik-5}}
\addcontentsline{toc}{paragraph}{1. Standar MRMIK 5}

Rumah sakit menetapkan penyelenggaraan dan pengelolaan rekam medis terkait asuhan pasien sesuai dengan peraturan perundang-undangan.

\hypertarget{maksud-dan-tujuan-mrmik-5}{%
\paragraph*{2. Maksud dan Tujuan MRMIK 5}\label{maksud-dan-tujuan-mrmik-5}}
\addcontentsline{toc}{paragraph}{2. Maksud dan Tujuan MRMIK 5}

Penyelenggaraan rekam medis merupakan proses kegiatan yang dimulai sejak saat pasien diterima rumah sakit dan mendapat asuhan medis, keperawatan, dan profesional pemberi asuhan lainnya. Proses penyelenggaraan rekam medis ini dilanjutkan sampai dengan pasien pulang, dirujuk, atau meninggal.

Kegiatan pengelolaan rekam medis yang meliputi: penerimaan pasien, asembling, analisis koding, indeksing, penyimpanan, pelaporan dan pemusnahan.

Rumah sakit menetapkan unit yang mengelola sistem rekam medis secara tepat, bernilai, dan dapat dipertanggungjawabkan. Unit kerja rekam medis memiliki struktur organisasi, uraian tugas, fungsi, tanggungjawab dan tata hubungan kerja dengan unit pelayanan lain.

Informasi kesehatan (rekam medis) baik kertas maupun elektronik harus dijaga keamanan dan kerahasiaannya dan disimpan sesuai dengan peraturan perundangan. Informasi kesehatan yang dikelola secara elektronik harus menjamin keamanan dan kerahasiaan dalam 3 (tiga) tempat, yaitu server di dalam rumah sakit, salinan (backup) data rutin, dan data virtual (cloud) atau salinan (backup) data di luar rumah sakit.

Penyimpanan dokumen fisik rekam medis mencakup lokasi yang tidak terkena panas serta aman dari air dan api, hanya dapat diakses oleh staf yang berwenang dan memastikan ruang penyimpanan rekam medis fisik memiliki suhu dan tingkat kelembaban yang tepat.

\hypertarget{elemen-penilaian-mrmik-5}{%
\paragraph*{3. Elemen Penilaian MRMIK 5}\label{elemen-penilaian-mrmik-5}}
\addcontentsline{toc}{paragraph}{3. Elemen Penilaian MRMIK 5}

\begin{enumerate}
\def\labelenumi{\alph{enumi}.}
\tightlist
\item
  Rumah sakit telah menetapkan regulasi tentang penyelenggaraan rekam medis di rumah sakit.
\item
  Rumah sakit menetapkan unit penyelenggara rekam medis dan 1 (satu) orang yang kompeten mengelola rekam medis.
\item
  Rumah Sakit menerapkan penyelenggaraan Rekam Medis yang dilakukan sejak pasien masuk sampai pasien pulang, dirujuk, atau meninggal.
\item
  Tersedia penyimpanan rekam medis yang menjamin keamanan dan kerahasiaan baik kertas maupun elektronik.
\end{enumerate}

\hypertarget{standar-mrmik-6}{%
\paragraph*{4. Standar MRMIK 6}\label{standar-mrmik-6}}
\addcontentsline{toc}{paragraph}{4. Standar MRMIK 6}

Setiap pasien memiliki rekam medis yang terstandardalam format yang seragam dan selalu diperbaharui (terkini) dan diisi sesuai dengan ketetapan rumah sakit dalam tatacara pengisian rekam medis.

\hypertarget{maksud-dan-tujuan-mrmik-6}{%
\paragraph*{5. Maksud dan Tujuan MRMIK 6}\label{maksud-dan-tujuan-mrmik-6}}
\addcontentsline{toc}{paragraph}{5. Maksud dan Tujuan MRMIK 6}

Setiap pasien memiliki rekam medis, baik dalam bentuk kertas maupun elektronik yang merupakan sumber informasi utama mengenai proses asuhan dan perkembangan pasien serta media komunikasi yang penting. Oleh karena itu, rekam medis harus selalu dievaluasi dan diperbaharui sesuai dengan kebutuhan dalam pelayanan pasien. Standardisasi dan identifikasi formulir rekam medis diperlukan untuk memberikan kemudahan PPA dalam melakukan pendokumentasian pada rekam medis pasien dan kemudahan dalam melakukan telusur isi rekam medis, serta kerapian dalam penyimpanan rekam medis.

Rekam medis pasien dipastikan selalu tersedia selama pemberian asuhan baik di rawat jalan, rawat inap maupun gawat darurat. Rumah sakit memastikan isi, format dan tata cara pengisian dalam rekam medis pasien sesuai dengan kebutuhan masing-masing PPA. Rumah sakit harus memiliki standar formulir rekam medis sebagai acuan bagi tenaga kesehatan/Profesional Pemberi Asuhan (PPA) dalam pelayanan pasien.

Pengelolaan rekam medis pasien harus mendukung terciptanya sistem yang baik sejak formulir dibuat atau direviu, dan dievaluasi penerapannya secara periodik, termasuk pengendalian rekam medis yang digunakan dan retensi formulir yang sudah tidak digunakan lagi.

\hypertarget{elemen-penilaian-mrmik-6}{%
\paragraph*{6. Elemen Penilaian MRMIK 6}\label{elemen-penilaian-mrmik-6}}
\addcontentsline{toc}{paragraph}{6. Elemen Penilaian MRMIK 6}

\begin{enumerate}
\def\labelenumi{\alph{enumi}.}
\tightlist
\item
  Terdapat bukti bahwa setiap pasien memiliki rekam medik dengan satu nomor RM sesuai sistem penomoran yang ditetapkan.
\item
  Rekam medis rawat jalan, rawat inap, gawat darurat dan pemeriksaan penunjang disusun dan diisi sesuai ketetapan rumah sakit.
\item
  Terdapat bukti bahwa formulir rekam medis dievaluasi dan diperbaharui (terkini) sesuai dengan kebutuhan dan secara periodik.
\end{enumerate}

\hypertarget{standar-mrmik-7}{%
\paragraph*{7. Standar MRMIK 7}\label{standar-mrmik-7}}
\addcontentsline{toc}{paragraph}{7. Standar MRMIK 7}

Rumah sakit menetapkan informasi yang akan dimuat pada rekam medis pasien.

\hypertarget{maksud-dan-tujuan-mrmik-7}{%
\paragraph*{8. Maksud dan Tujuan MRMIK 7}\label{maksud-dan-tujuan-mrmik-7}}
\addcontentsline{toc}{paragraph}{8. Maksud dan Tujuan MRMIK 7}

Rumah sakit menetapkan data dan informasi spesifik yang dicatat dalam rekam medis setiap pasien untuk melakukan penilaian/pengkajian dan mendapatkan pengobatan maupun tindakan oleh Profesional Pemberi Asuhan (PPA) sebagai pasien rawat jalan, rawat inap dan gawat darurat. Ketetapan ini sesuai dengan peraturan perundangan yang berlaku. Rekam medis memuat informasi yang memadai untuk:

\begin{enumerate}
\def\labelenumi{\alph{enumi}.}
\tightlist
\item
  Mengidentifikasi pasien;\\
\item
  Mendukung diagnosis;\\
\item
  Justifikasi/dasar pemberian pengobatan;\\
\item
  Mendokumentasikan hasil pemeriksaan dan hasil
  pengobatan;\\
\item
  Memuat ringkasan pasien pulang (discharge summary); dan
\item
  Meningkatkan kesinambungan pelayanan diantara Profesional Pemberi Asuhan (PPA).
\end{enumerate}

\hypertarget{elemen-penilaian-mrmik-7}{%
\paragraph*{9. Elemen Penilaian MRMIK 7}\label{elemen-penilaian-mrmik-7}}
\addcontentsline{toc}{paragraph}{9. Elemen Penilaian MRMIK 7}

\begin{enumerate}
\def\labelenumi{\alph{enumi}.}
\tightlist
\item
  Terdapat bukti rekam medis pasien telah berisi informasi yang sesuai dengan ketetapan rumah sakit dan peraturan perundangan yang berlaku.
\item
  Terdapat bukti rekam medis pasien mengandung informasi yang memadai sesuai butir a) ? f) pada maksud dan tujuan.
\end{enumerate}

\hypertarget{standar-mrmik-8}{%
\paragraph*{10. Standar MRMIK 8}\label{standar-mrmik-8}}
\addcontentsline{toc}{paragraph}{10. Standar MRMIK 8}

Setiap catatan (entry) pada rekam medis pasien mencantumkan identitas Profesional Pemberi Asuhan (PPA) yang menulis dan kapan catatan tersebut ditulis di dalam rekam medis.

\hypertarget{maksud-dan-tujuan-mrmik-8}{%
\paragraph*{11. Maksud dan Tujuan MRMIK 8}\label{maksud-dan-tujuan-mrmik-8}}
\addcontentsline{toc}{paragraph}{11. Maksud dan Tujuan MRMIK 8}

Rumah sakit memastikan bahwa setiap catatan dalam rekam medis dapat diidentifikasi dengan tepat, dimana setiap pengisian rekam medis ditulis tanggal, jam, serta indentitas Profesional Pemberi Asuhan ( PPA ) berupa nama jelas dan tanda tangan/paraf. Rumah sakit menetapkan proses pembenaran/koreksi terhadap kesalahan penulisan catatan dalam rekam medis. Selanjutnya dilakukan pemantauan dan evaluasi terhadap penulisan identitas, tanggal dan waktu penulisan catatan pada rekam medis pasien serta koreksi penulisan catatan dalam rekam medis.

\hypertarget{elemen-penilaian-mrmik-8}{%
\paragraph*{12. Elemen Penilaian MRMIK 8}\label{elemen-penilaian-mrmik-8}}
\addcontentsline{toc}{paragraph}{12. Elemen Penilaian MRMIK 8}

\begin{enumerate}
\def\labelenumi{\alph{enumi}.}
\tightlist
\item
  PPA mencantumkan identitas secara jelas pada saat mengisi RM.
\item
  Tanggal dan waktu penulisan setiap catatan dalam rekam medis pasien dapat diidentifikasi.
\item
  Terdapat prosedur koreksi penulisan dalam pengisian RM elektronik dan non elektronik.
\item
  Telah dilakukan pemantauan dan evaluasi terhadap penulisan identitas, tanggal dan waktu penulisan catatan pada rekam medis pasien serta koreksi penulisan catatan dalam rekam medis, dan hasil evaluasi yang ada telah digunakan sebagai dasar upaya perbaikan di rumah sakit.
\end{enumerate}

\hypertarget{standar-mrmik-9}{%
\paragraph*{13. Standar MRMIK 9}\label{standar-mrmik-9}}
\addcontentsline{toc}{paragraph}{13. Standar MRMIK 9}

Rumah sakit menggunakan kode diagnosis, kode prosedur, penggunaan simbol dan singkatan baku yang seragam dan terstandar.

\hypertarget{maksud-dan-tujuan-mrmik-9}{%
\paragraph*{14, Maksud dan Tujuan MRMIK 9}\label{maksud-dan-tujuan-mrmik-9}}
\addcontentsline{toc}{paragraph}{14, Maksud dan Tujuan MRMIK 9}

Penggunaan kode, simbol, dan singkatan yang terstandar berguna untuk mencegah terjadinya kesalahan komunikasi dan kesalahan pemberian asuhan kepada pasien. Penggunaan singkatan yang baku dan seragam menunjukkan bahwa singkatan, kode, simbol yang digunakan mempunyai satu arti/makna yang digunakan dan berlaku di semua lingkungan rumah sakit.
Rumah sakit menyusun dan menetapkan daftar atau penggunaan kode, simbol dan singkatan yang digunakan dan tidak boleh digunakan di rumah sakit. Penggunaan kode, simbol, dan singkatan baku yang seragam harus konsisten dengan standar praktik profesional. Prinsip penggunaan kode di rekam medis utamanya menggunakan ICD-10 untuk kode Penyakit dan dan ICD9 CM untuk kode Tindakan. Penggunaan kode di rekam medis sesuai dengan standar yang ditetapkan rumah sakit serta dilakukan evaluasi terkait penggunaan kode tersebut.

\hypertarget{elemen-penilaian-mrmik-9}{%
\paragraph*{15. Elemen Penilaian MRMIK 9}\label{elemen-penilaian-mrmik-9}}
\addcontentsline{toc}{paragraph}{15. Elemen Penilaian MRMIK 9}

\begin{enumerate}
\def\labelenumi{\alph{enumi}.}
\tightlist
\item
  Penggunaan kode diagnosis, kode prosedur, singkatan dan simbol sesuai dengan ketetapan rumah sakit.
\item
  Dilakukan evaluasi secara berkala penggunaan kode diagnosis, kode prosedur, singkatan dan simbol yang berlaku di rumah sakit dan hasilnya digunakan sebagai upaya tindak lanjut untuk perbaikan.
\end{enumerate}

\hypertarget{standar-mrmik-10}{%
\paragraph*{16. Standar MRMIK 10}\label{standar-mrmik-10}}
\addcontentsline{toc}{paragraph}{16. Standar MRMIK 10}

Rumah sakit menjamin keamanan, kerahasiaan dan kepemilikan rekam medis serta privasi pasien.

\hypertarget{maksud-dan-tujuan-mrmik-10}{%
\paragraph*{17. Maksud dan Tujuan MRMIK 10}\label{maksud-dan-tujuan-mrmik-10}}
\addcontentsline{toc}{paragraph}{17. Maksud dan Tujuan MRMIK 10}

Rekam medis adalah pusat informasi yang digunakan untuk tujuan klinis, penelitian, bukti hukum, administrasi, dan keuangan, sehingga harus dibatasi aksesibilitasnya. Pimpinan rumah sakit bertanggungjawab atas kehilangan, kerusakan pemalsuan dan/atau penggunaan oleh orang atau badan yang tidak berhak terhadap rekam medis.

Rekam medis, baik kertas atau elektronik, adalah alat komunikasi yang mendukung pengambilan keputusan klinis, koordinasi pelayanan, evaluasi mutu dan ketepatan perawatan, penelitian, perlindungan hukum, pendidikan, dan akreditasi serta proses manajemen. Dengan demikian, setiap pengisian rekam medis harus dapat dijamin otentifikasinya.

Menjaga kerahasiaan yang dimaksud termasuk adalah memastikan bahwa hanya individu yang berwenang yang memiliki akses ke informasi tersebut. Selain keamanan dan kerahasian maka dibutuhkan privasi sebagai hak ?untuk menjadi diri sendiri atau hak otonomi?, hak untuk ?menyimpan informasi tentang diri mereka sendiri dari yang diungkapkan kepada orang lain; hak untuk diketahui diri sendiri, maupun gangguan dari pihak yang tidak berkepentingan kecuali yang dimungkinkan atas perintah peraturan perundang-undangan.

\hypertarget{elemen-penilaian-mrmik-10}{%
\paragraph*{18. Elemen Penilaian MRMIK 10}\label{elemen-penilaian-mrmik-10}}
\addcontentsline{toc}{paragraph}{18. Elemen Penilaian MRMIK 10}

\begin{enumerate}
\def\labelenumi{\alph{enumi}.}
\tightlist
\item
  Rumah sakit menentukan otoritas pengisian rekam medis termasuk isi dan format rekam medis.
\item
  Rumah Sakit menentukan hak akses dalam pelepasan informasi rekam medis
\item
  Rumah sakit menjamin otentifikasi, keamanan dan kerahasiaan data rekam medis baik kertas maupun elektronik sebagai bagian dari hak pasien.
\end{enumerate}

\hypertarget{standar-mrmik-11}{%
\paragraph*{19. Standar MRMIK 11}\label{standar-mrmik-11}}
\addcontentsline{toc}{paragraph}{19. Standar MRMIK 11}

Rumah sakit mengatur lama penyimpanan rekam medis, data, dan informasi pasien.

\hypertarget{maksud-dan-tujuan-mrmik-11}{%
\paragraph*{20. Maksud dan Tujuan MRMIK 11}\label{maksud-dan-tujuan-mrmik-11}}
\addcontentsline{toc}{paragraph}{20. Maksud dan Tujuan MRMIK 11}

Rumah sakit menentukan jangka waktu penyimpanan rekam medis (kertas/elektronik), data, dan informasi lainnya terkait pasien sesuai dengan peraturan perundang- undangan untuk mendukung asuhan pasien, manajemen, dokumentasi yang sah secara hukum, serta pendidikan dan penelitian. Rumah sakit bertanggungjawab terhadap keamanan dan kerahasiaan data rekam medis selama proses penyimpanan sampai dengan pemusnahan.

Untuk rekam medis dalam bentuk kertas dilakukan pemilahan rekam medis aktif dan rekam medis yang tidak aktif serta disimpan secara terpisah. Penentuan jangka waktu penyimpanan rekam medis ditentukan atas dasar nilai manfaat setiap rekam medis yang konsisten dengan kerahasiaan dan keabsahan informasi.

Bila jangka waktu penyimpanan sudah habis maka rekam medis, serta data dan informasi yang terkait pasien dimusnahkan dengan prosedur yang tidak membahayakan keamanan dan kerahasiaan sesuai dengan peraturan perundang-undangan. Rumah sakit menetapkan dokumen, data dan/atau informasi tertentu terkait pasien yang memiliki nilaiguna untuk disimpan abadi (permanen).

\hypertarget{elemen-penilaian-mrmik-11}{%
\paragraph*{21. Elemen Penilaian MRMIK 11}\label{elemen-penilaian-mrmik-11}}
\addcontentsline{toc}{paragraph}{21. Elemen Penilaian MRMIK 11}

\begin{enumerate}
\def\labelenumi{\alph{enumi}.}
\tightlist
\item
  Rumah sakit memiliki regulasi jangka waktu penyimpanan berkas rekam medis (kertas/elektronik), serta data dan informasi lainnya terkait dengan pasien dan prosedur pemusnahannya sesuai dengan peraturan perundangan.
\item
  Dokumen, data dan/informasi terkait pasien dimusnahkan setelah melampaui periode waktu penyimpanan sesuai dengan peraturan perundang- undangan dengan prosedur yang tidak membahayakan keamanan dan kerahasiaan.
\item
  Dokumen, data dan/atau informasi tertentu terkait pasien yang bernilai guna, disimpan abadi (permanen) sesuai dengan ketetapan rumah sakit.
\end{enumerate}

\hypertarget{standar-mrmik-12}{%
\paragraph*{22. Standar MRMIK 12}\label{standar-mrmik-12}}
\addcontentsline{toc}{paragraph}{22. Standar MRMIK 12}

Dalam upaya perbaikan kinerja, rumah sakit secara teratur melakukan evaluasi atau pengkajian rekam medis.

\hypertarget{maksud-dan-tujuan-mrmik-12}{%
\paragraph*{23. Maksud dan tujuan MRMIK 12}\label{maksud-dan-tujuan-mrmik-12}}
\addcontentsline{toc}{paragraph}{23. Maksud dan tujuan MRMIK 12}

Setiap rumah sakit sudah menetapkan isi dan format rekam medis pasien dan mempunyai proses untuk melakukan pengkajian terhadap isi dan kelengkapan berkas rekam medis. Proses tersebut merupakan bagian dari kegiatan peningkatan kinerja rumah sakit yang dilaksanakan secara berkala. Pengkajian rekam medis berdasarkan sampel yang mewakili PPA yang memberikan pelayanan dan jenis pelayanan yang diberikan.

Proses pengkajian dilakukan oleh komite/tim rekam medis melibatkan tenaga medis, keperawatan, serta PPA lainnya yang relevan dan mempunyai otorisasi untuk mengisi rekam medis pasien. Pengkajian berfokus pada ketepatan waktu, kelengkapan, keterbacaan, keabsahan dan ketentuan lainnya seperti informasi klinis yang ditetapkan rumah sakit. Isi rekam medis yang dipersyaratkan oleh peraturan perundangan dimasukkan dalam proses evaluasi rekam medis. Pengkajian rekam medis di rumah sakit tersebut dilakukan terhadap rekam medis pasien yang sedang dalam perawatan dan pasien yang sudah pulang. Hasil pengkajian dilaporkan secara berkala kepada pimpinan rumah sakit dan selanjutnya dibuat upaya perbaikan.

\hypertarget{elemen-penilaian-mrmik-12}{%
\paragraph*{24. Elemen Penilaian MRMIK 12}\label{elemen-penilaian-mrmik-12}}
\addcontentsline{toc}{paragraph}{24. Elemen Penilaian MRMIK 12}

\begin{enumerate}
\def\labelenumi{\alph{enumi}.}
\tightlist
\item
  Rumah sakit menetapkan komite/tim rekam medis.
\item
  Komite/tim secara berkala melakukan pengkajian rekam medis pasien secara berkala setiap tahun dan menggunakan sampel yang mewakili (rekam medis pasien yang masih dirawat dan pasien yang sudah pulang).
\item
  Fokus pengkajian paling sedikit mencakup pada ketepatan waktu, keterbacaan, kelengkapan rekam medis dan isi rekam medis sesuai dengan peraturan perundangan.
\item
  Hasil pengkajian yang dilakukan oleh komite/tim rekam medis dilaporkan kepada pimpinan rumah sakit dan dibuat upaya perbaikan.
\end{enumerate}

\hypertarget{d.-teknologi-informasi-kesehatan-di-pelayanan-kesehatan}{%
\subsection*{d.~Teknologi Informasi Kesehatan di Pelayanan Kesehatan}\label{d.-teknologi-informasi-kesehatan-di-pelayanan-kesehatan}}
\addcontentsline{toc}{subsection}{d.~Teknologi Informasi Kesehatan di Pelayanan Kesehatan}

\hypertarget{standar-mrmik-13}{%
\paragraph*{1. Standar MRMIK 13}\label{standar-mrmik-13}}
\addcontentsline{toc}{paragraph}{1. Standar MRMIK 13}

Rumah sakit menerapkan sistem teknologi informasi kesehatan di pelayanan kesehatan untuk mengelola data dan informasi klinis serta non klinis sesuai peraturan perundang-undangan.

\hypertarget{maksud-dan-tujuan-mrmik-13}{%
\paragraph*{2. Maksud dan Tujuan MRMIK 13}\label{maksud-dan-tujuan-mrmik-13}}
\addcontentsline{toc}{paragraph}{2. Maksud dan Tujuan MRMIK 13}

Sistem teknologi informasi di pelayanan kesehatan merupakan seperangkat tatanan yang meliputi data, informasi, indikator, prosedur, teknologi, perangkat dan sumber daya manusia yang saling berkaitan dan dikelola secara terpadu untuk mengarahkan tindakan atau keputusan yang berguna dalam mendukung peningkatan mutu pelayanan dan pembangunan kesehatan. Untuk mendapatkan hasil yang optimal dalam pencapaian sistem informasi kesehatan diperlukan SIMRS yang menjadi media berupa sistem teknologi informasi komunikasi yang memproses dan mengintegrasikan seluruh alur proses pelayanan Rumah Sakit dalam bentuk jaringan koordinasi, pengumpulan data, pelaporan dan prosedur administrasi untuk memperoleh informasi secara tepat dan akurat.

Dalam pengembangan sistem informasi kesehatan, rumah sakit harus mampu meningkatkan dan mendukung proses pelayanan kesehatan yang meliputi:

a Kecepatan, akurasi, integrasi, peningkatan pelayanan, peningkatan efisiensi, kemudahan pelaporan dalam pelaksanaan operasional
b. Kecepatan mengambil keputusan, akurasi dan kecepatan identifikasi masalah dan kemudahan dalam penyusunan strategi dalam pelaksanaan manajerial; dan
c.~Budaya kerja, transparansi, koordinasi antar unit, pemahaman sistem dan pengurangan biaya adminstrasi dalam pelaksanaan organisasi

Apabila sistem informasi kesehatan yang dimiliki oleh rumah sakit sudah tidak sesuai dengan kebutuhan operasional dalam menunjang mutu pelayanan, maka dibutuhkan pengembangan sistem informasi kesehatan yang mendukung mutu pelayanan agar lebih optimal dengan memperhatikan peraturan yang ada. Sistem teknologi informasi rumah sakit harus dikelola secara efektif dan komprehensif serta terintegrasi.

Individu yang mengawasi sistem teknologi informasi kesehatan bertanggung jawab atas setidaknya hal-hal berikut:

\begin{enumerate}
\def\labelenumi{\alph{enumi}.}
\tightlist
\item
  Merekomendasikan ruang, peralatan, teknologi, dan sumber daya lainnya kepada pimpinan rumah sakit untuk mendukung sistem teknologi informasi di rumah sakit.
\item
  Mengkoordinasikan dan melakukan kegiatan pengkajian risiko untuk menilai risiko keamanan informasi, memprioritaskan risiko, dan mengidentifikasi perbaikan.
\item
  Memastikan bahwa staf di rumah sakit telah dilatih tentang keamanan informasi dan kebijakan serta prosedur yang berlaku.
\item
  Mengidentifikasi pengukuran untuk menilai sistem contohnya penilaian terhadap efektifitas sistem rekam medis elektronik bagi staf dan pasien.
\end{enumerate}

\hypertarget{elemen-penilaian-mrmik-13}{%
\paragraph*{3. Elemen Penilaian MRMIK 13}\label{elemen-penilaian-mrmik-13}}
\addcontentsline{toc}{paragraph}{3. Elemen Penilaian MRMIK 13}

\begin{enumerate}
\def\labelenumi{\alph{enumi}.}
\tightlist
\item
  Rumah sakit menetapkan regulasi tentang penyelenggaraan teknologi informasi kesehatan
\item
  Rumah sakit menerapkan SIMRS sesuai dengan ketetapan dan peraturan perundangan yang berlaku.
\item
  Rumah sakit menetapkan unit yang bertanggung jawab sebagai penyelenggara SIMRS dan dipimpim oleh staf kompeten.
\item
  Data serta informasi klinis dan non klinis diintegrasikan sesuai dengan kebutuhan untuk mendukung pengambilan keputusan
\item
  Rumah sakit telah menerapkan proses untuk menilai efektifitas sistem rekam medis elektronik dan melakukan upaya perbaikan terkait hasil penilaian yang ada.
\end{enumerate}

\hypertarget{standar-mrmik-13.1}{%
\paragraph*{4. Standar MRMIK 13.1}\label{standar-mrmik-13.1}}
\addcontentsline{toc}{paragraph}{4. Standar MRMIK 13.1}

Rumah sakit mengembangkan, memelihara, dan menguji program untuk mengatasi waktu henti (downtime) dari sistem data, baik yang terencana maupun yang tidak terencana.

\hypertarget{maksud-dan-tujuan-mrmik-13.1}{%
\paragraph*{5. Maksud dan tujuan MRMIK 13.1}\label{maksud-dan-tujuan-mrmik-13.1}}
\addcontentsline{toc}{paragraph}{5. Maksud dan tujuan MRMIK 13.1}

Sistem data adalah bagian yang penting dalam memberikan perawatan/ pelayanan pasien yang aman dan bermutu tinggi. Interupsi dan kegagalan sistem data adalah kejadian yang tidak bisa dihindari. Interupsi ini sering disebut sebagai waktu henti (down time), baik yang terencana maupun tidak terencana. Waktu henti, baik yang direncanakan atau tidak direncanakan, dapat memengaruhi seluruh sistem atau hanya memengaruhi satu aplikasi saja. Komunikasi adalah elemen penting dari strategi kesinambungan pelayanan selama waktu henti.

Pemberitahuan tentang waktu henti yang direncanakan memungkinkan dilakukannya persiapan yang diperlukan untuk memastikan bahwa operasional dapat berlanjut dengan cara yang aman dan efektif. Rumah sakit memiliki suatu perencanaan untuk mengatasi waktu henti (down time), baik yang terencana maupun tidak terencana dengan melatih staf tentang prosedur alternatif, menguji program pengelolaan gawat darurat yang dimiliki rumah sakit, melakukan pencadangan data terjadwal secara teratur, dan menguji prosedur pemulihan data

\hypertarget{elemen-penilaian-mrmik-13.1}{%
\paragraph*{6. Elemen Penilaian MRMIK 13.1}\label{elemen-penilaian-mrmik-13.1}}
\addcontentsline{toc}{paragraph}{6. Elemen Penilaian MRMIK 13.1}

\begin{enumerate}
\def\labelenumi{\alph{enumi}.}
\tightlist
\item
  Terdapat prosedur yang harus dilakukan jika terjadi waktu henti sistem data (down time) untuk mengatasi masalah pelayanan.
\item
  Staf dilatih dan memahami perannya di dalam prosedur penanganan waktu henti sistem data (down time), baik yang terencana maupun yang tidak terencana.
\item
  Rumah sakit melakukan evaluasi pasca terjadinya waktu henti sistem data (down time) dan menggunakan informasi dari data tersebut untuk persiapan dan perbaikan apabila terjadi waktu henti (down time) berikutnya.
\end{enumerate}

\hypertarget{pencegahan-dan-pengendalian-infeksi-ppi}{%
\section*{6. Pencegahan dan Pengendalian Infeksi (PPI)}\label{pencegahan-dan-pengendalian-infeksi-ppi}}
\addcontentsline{toc}{section}{6. Pencegahan dan Pengendalian Infeksi (PPI)}

\textbf{Gambaran Umum}

Tujuan program pencegahan dan pengendalian infeksi adalah untuk mengidentifikasi dan menurunkan risiko infeksi yang didapat dan ditularkan di antara pasien, staf, tenaga kesehatan, tenaga kontrak, sukarelawan, mahasiswa dan pengunjung. Risiko dan kegiatan dalam program PPI dapat berbeda dari satu rumah sakit ke rumah sakit yang lain, tergantung pada kegiatan dan pelayanan klinis rumah sakit, populasi pasien yang dilayani, lokasi geografis, jumlah pasien dan jumlah staf. Prioritas program sebaiknya mencerminkan risiko yang telah teridentifikasi tersebut, perkembangan global dan masyarakat setempat, serta kompleksitas dari pelayanan yang diberikan.

Penyelenggaraan program pencegahan dan pengendalian infeksi (PPI) dikelola oleh Komite / Tim PPI yang ditetapkan oleh Direktur rumah sakit. Agar kegiatan PPI dapat dilaksanakan secara efektif maka dibutuhkan kebijakan dan prosedur, pelatihan dan pendidikan staf, metode identifikasi risiko infeksi secara proaktif pada individu dan lingkungan serta koordinasi ke semua bagian di rumah sakit.

Fokus Standar Pencegahan dan pengendalian infeksi (PPI) meliputi:

\begin{enumerate}
\def\labelenumi{\alph{enumi}.}
\tightlist
\item
  Penyelenggaraan PPI di Rumah Sakit
\item
  Program PPI
\item
  Pengkajian Risiko
\item
  Peralatan medis dan/atau Bahan Medis Habis Pakai (BMHP)
\item
  Kebersihan lingkungan
\item
  Manajemen linen
\item
  Limbah infeksius
\item
  Pelayanan makanan
\item
  Risiko infeksi pada konstruksi dan renovasi
\item
  Penularan infeksi
\item
  Kebersihan Tangan
\item
  Peningkatan mutu dan program edukasi
\item
  Edukasi, Pendidikan dan Pelatihan
\end{enumerate}

\hypertarget{a.-penyelenggaraan-ppi-di-rumah-sakit}{%
\subsection*{a. Penyelenggaraan PPI di Rumah Sakit}\label{a.-penyelenggaraan-ppi-di-rumah-sakit}}
\addcontentsline{toc}{subsection}{a. Penyelenggaraan PPI di Rumah Sakit}

\hypertarget{standar-ppi-1}{%
\paragraph*{1. Standar PPI 1}\label{standar-ppi-1}}
\addcontentsline{toc}{paragraph}{1. Standar PPI 1}

Rumah sakit menetapkan Komite/Tim PPI untuk melakukan pengkajian, perencanaan, pelaksanaan, pemantauan, dan evaluasi kegiatan PPI di rumah sakit serta menyediakan sumber daya untuk mendukung program pencegahan dan pengendalian infeksi

\hypertarget{standar-ppi-1.1}{%
\paragraph*{2. Standar PPI 1.1}\label{standar-ppi-1.1}}
\addcontentsline{toc}{paragraph}{2. Standar PPI 1.1}

Direktur rumah sakit menetapkan Komite/Tim PPI untuk mengelola dan mengawasi kegiatan PPI disesuaikan dengan jenis pelayanan, kebutuhan, beban kerja, dan/atau klasifikasi rumah sakit sesuai sesuai peraturan perundang undangan. Komite/Tim PPI dipimpin oleh seorang tenaga medis yang mempunyai pengalaman klinis, pengalaman pencegahan dan pengendalian infeksi (PPI) serta kepemimpinan sehingga dapat mengarahkan, mengimplementasikan, dan mengukur perubahan. Kualifikasi Ketua Komite/Tim PPI dapat dipenuhi melalui pendidikan dan pelatihan, sertifikasi atau surat izin.

Komite/tim PPI melibatkan staf klinis dan non klinis, meliputi perawat PPI/IPCN, staf di bagian pemeliharaan fasilitas, dapur, kerumahtanggaan (tata graha), laboratorium, farmasi, ahli epidemiologi, ahli statistik, ahli mikrobiologi, staf sterilisasi (CSSD) serta staf bagian umum. Tergantung pada besar kecilnya ukuran rumah sakit dan kompleksitas layanan sesuai dengan peraturan perundang- undangan.

Komite/tim PPI menetapkan mekanisme dan koordinasi termasuk berkomunikasi dengan semua pihak di rumah sakit untuk memastikan program berjalan efektif dan berkesinambungan.
Mekanisme koordinasi ditetapkan secara priodik untuk melaksanakan program PPI dengan melibatkan pimpinan rumah sakit dan

Komite/Tim PPI. Koordinasi tersebut meliputi:

\begin{enumerate}
\def\labelenumi{\alph{enumi}.}
\tightlist
\item
  Menetapkan kriteria untuk mendefinisikan infeksi terkait pelayanan kesehatan;
\item
  Menetapkan metode pengumpulan data (surveilans);
\item
  Membuat strategi untuk menangani risiko PPI, dan pelaporannya; dan
\item
  Berkomunikasi dengan semua unit untuk memastikan bahwa program berkelanjutan dan proaktif.
\end{enumerate}

Hasil koordinasi didokumentasikan untuk meninjau efektivitas koordinasi program dan untuk memantau adanya perbaikan progresif.

Rumah sakit menetapkan perawat PPI/IPCN (perawat pencegah dan pengendali infeksi) yaitu perawat yang bekerja penuh waktu) dan IPCLN (perawat penghubung pencegah dan pengendali infeksi) berdasarkan jumlah dan kualifikasinya sesuai dengan ukuran rumah sakit, kompleksitas kegiatan, tingkat risiko, cakupan program dan peraturan perundang undangan. Kualifikasi pendidikan perawat tersebut minimal D-3 keperawatan dan sudah mengikuti pelatihan perawat PPI.

Dalam melaksanakan kegiatan program PPI yang berkesinambungan secara effektif dan effisien diperlukan dukungan sumber daya meliputi tapi tidak terbatas pada:

\begin{enumerate}
\def\labelenumi{\alph{enumi}.}
\tightlist
\item
  Ketersedian anggaran;
\item
  Sumber daya manusia yang terlatih;
\item
  Sarana prasarana dan perbekalan, untuk mencuci tangan berbasis alkohol (handrub), dan mencuci tangan dengan air mengalir (handwash), kantong pembuangan sampah infeksius dll;
\item
  Sistem manajemen informasi untuk mendukung penelusuran risiko, angka, dan tren infeksi yang terkait dengan pelayanan kesehatan; dan
\item
  Sarana penunjang lainnya untuk menunjang kegiatan PPI yang dapat mempermudah kegiatan PPI.
  Informasi dan data kegiatan PPI akan dintegrasikan ke Komite/ Tim Penyelenggara Mutu untuk peningkatan mutu dan keselamatan pasien rumah sakit oleh Komite / tim PPI setiap bulan.
\end{enumerate}

\hypertarget{elemen-penilaian-ppi-1}{%
\paragraph*{3. Elemen Penilaian PPI 1}\label{elemen-penilaian-ppi-1}}
\addcontentsline{toc}{paragraph}{3. Elemen Penilaian PPI 1}

\begin{enumerate}
\def\labelenumi{\alph{enumi}.}
\tightlist
\item
  Direktur rumah sakit telah menetapkan regulasi PPI meliputi a - m pada gambaran umum.
\item
  Direktur rumah sakit telah menetapkan komite/tim PPI untuk untuk mengelola dan mengawasi kegiatan PPI di rumah sakit.
\item
  Rumah sakit telah menerapkan mekanisme koordinasi yang melibatkan pimpinan rumah sakit dan komite/tim PPI untuk melaksanakan program PPI sesuai dalam maksud dan tujuan.
\item
  Direktur rumah sakit memberikan dukungan sumber daya terhadap penyelenggaraan kegiatan PPI meliputi namun tidak terbatas pada maksud dan tujuan.
\end{enumerate}

\hypertarget{elemen-penialian-ppi-1.1}{%
\paragraph*{4. Elemen Penialian PPI 1.1}\label{elemen-penialian-ppi-1.1}}
\addcontentsline{toc}{paragraph}{4. Elemen Penialian PPI 1.1}

\begin{enumerate}
\def\labelenumi{\alph{enumi}.}
\tightlist
\item
  Rumah sakit menetapkan perawat PPI/IPCN purna waktu dan IPCLN berdasarkan jumlah dan kualifikasi sesuai ukuran rumah sakit, kompleksitas kegiatan, tingkat risiko, cakupan program dan sesuai dengan peraturan perundang-undangan.
\item
  Ada bukti perawat PPI/IPCN melaksanakan supervisi pada semua kegiatan pencegahan dan pengendalian infeksi di rumah sakit.
\end{enumerate}

\hypertarget{b.-program-ppi}{%
\subsection*{b. Program PPI}\label{b.-program-ppi}}
\addcontentsline{toc}{subsection}{b. Program PPI}

\hypertarget{standar-ppi-2}{%
\paragraph*{1. Standar PPI 2}\label{standar-ppi-2}}
\addcontentsline{toc}{paragraph}{1. Standar PPI 2}

Rumah sakit menyusun dan menerapkan program PPI yang terpadu dan menyeluruh untuk mencegah penularan infeksi terkait pelayanan kesehatan berdasarkan pengkajian risiko secara proaktif setiap tahun.

\hypertarget{maksud-dan-tujuan-ppi-2}{%
\paragraph*{2. Maksud dan Tujuan PPI 2}\label{maksud-dan-tujuan-ppi-2}}
\addcontentsline{toc}{paragraph}{2. Maksud dan Tujuan PPI 2}

Secara prinsip, kejadian HAIs sebenarnya dapat dicegah bila fasilitas pelayanan kesehatan secara konsisten melaksanakan program PPI.

Pelaksanaan Pencegahan dan Pengendalian Infeksi di Fasilitas Pelayanan Kesehatan bertujuan untuk melindungi pasien, petugas kesehatan, pengunjung yang menerima pelayanan kesehatan serta masyarakat dalam lingkungannya dengan cara memutus siklus penularan penyakit infeksi melalui kewaspadaan Isolasi terdiri dari kewaspadaan standar dan berdasarkan transmisi.

\begin{enumerate}
\def\labelenumi{\alph{enumi}.}
\tightlist
\item
  Kesebelas kewaspadaan standar tersebut yang harus di terapkan di rumah sakit adalah:
\end{enumerate}

\begin{enumerate}
\def\labelenumi{\arabic{enumi}.}
\tightlist
\item
  Kebersihan tangan
\item
  Alat Pelindung diri
\item
  Dekontaminasi peralatan perawatan pasien
\item
  Pengendalian lingkungan
\item
  Pengelolaan limbah
\item
  Penatalaksanaan linen
\item
  Perlindungan kesehatan petugas
\item
  Penempatan pasien
\item
  Kebersihan pernafasan/etika batuk dan bersin
\item
  Praktik menyuntik yang aman
\item
  Praktik lumbal pungsi yang aman
\end{enumerate}

\begin{enumerate}
\def\labelenumi{\alph{enumi}.}
\setcounter{enumi}{1}
\tightlist
\item
  Kewaspadaan Transmisi
\end{enumerate}

Kewaspadaan berdasarkan transmisi sebagai tambahan Kewaspadaan Standar yang dilaksanakan sebelum pasien didiagnosis dan setelah terdiagnosis jenis infeksinya. Jenis kewaspadaan berdasarkan transmisi sebagai berikut:

\begin{enumerate}
\def\labelenumi{\arabic{enumi}.}
\tightlist
\item
  Melalui kontak
\item
  Melalui droplet
\item
  Melalui udara (Airborne Precautions)
\end{enumerate}

\hypertarget{elemen-penilaian-ppi-2}{%
\paragraph*{3. Elemen Penilaian PPI 2}\label{elemen-penilaian-ppi-2}}
\addcontentsline{toc}{paragraph}{3. Elemen Penilaian PPI 2}

\begin{enumerate}
\def\labelenumi{\alph{enumi}.}
\tightlist
\item
  Rumah sakit menetapkan kebijakan Program PPI yang terdiri dari kewaspadaan standar dan kewaspadaan transmisi sesuai maksud dan tujuan diatas.
\item
  Rumah sakit melakukan evaluasi pelaksanaan program PPI.
\end{enumerate}

\hypertarget{c.-pengkajian-risiko}{%
\subsection*{c.~Pengkajian Risiko}\label{c.-pengkajian-risiko}}
\addcontentsline{toc}{subsection}{c.~Pengkajian Risiko}

\hypertarget{standar-ppi-3}{%
\paragraph*{1. Standar PPI 3}\label{standar-ppi-3}}
\addcontentsline{toc}{paragraph}{1. Standar PPI 3}

Rumah sakit melakukan pengkajian proaktif setiap tahunnya sebagai dasar penyusunan program PPI terpadu untuk mencegah penularan infeksi terkait pelayanan kesehatan.

\hypertarget{maksud-dan-tujuan-ppi-3}{%
\paragraph*{2. Maksud dan Tujuan PPI 3}\label{maksud-dan-tujuan-ppi-3}}
\addcontentsline{toc}{paragraph}{2. Maksud dan Tujuan PPI 3}

Risiko infeksi dapat berbeda antara rumah sakit, tergantung ukuran rumah sakit, kompleksitas pelayanan dan kegiatan klinisnya, populasi pasien yang dilayani, lokasi geografis, volume pasien, dan jumlah staf yang dimiliki.

Rumah sakit secara proaktif setiap tahun melakukan pengkajian risiko pengendalian infeksi (ICRA) terhadap tingkat dan kecenderungan infeksi layanan kesehatan yang akan menjadi prioritas fokus Program PPI dalam upaya pencegahan dan penurunan risiko. Pengkajian risiko tersebut meliputi namun tidak terbatas pada:

\begin{enumerate}
\def\labelenumi{\alph{enumi}.}
\tightlist
\item
  Infeksi-infeksi yang penting secara epidemiologis yang merupakan data surveilans;
\item
  Proses kegiatan di area-area yang berisiko tinggi terjadinya infeksi;
\item
  Pelayanan yang menggunakan peralatan yang berisiko infeksi;
\item
  Prosedur/tindakan-tindakan berisiko tinggi;
\item
  Pelayanan distribusi linen bersih dan kotor;
\item
  Pelayanan sterilisasi alat;
\item
  Kebersihan permukaan dan lingkungan;
\item
  Pengelolaan linen/laundri;
\item
  Pengelolaan sampah;
\item
  Penyediaan makanan; dan
\item
  Pengelolaan kamar jenazah
\end{enumerate}

Data surveilans dikumpulkan di rumah sakit secara periodik dan dianalisis setiap triwulan. Data surveilans ini meliputi:

\begin{enumerate}
\def\labelenumi{\alph{enumi}.}
\tightlist
\item
  Saluran pernapasan seperti prosedur dan tindakan terkait intubasi, bantuan ventilasi mekanis, trakeostomi, dan lain-lain;
\item
  Saluran kemih seperti kateter, pembilasan urine, dan lain lain;
\item
  Alat invasif intravaskular, saluran vena verifer, saluran vena sentral, dan lain-lain
\item
  Lokasi operasi, perawatan, pembalutan luka, prosedur aseptik, dan lain-lain;
\item
  Penyakit dan organisme yang penting dari sudut epidemiologik seperti Multidrug Resistant Organism dan infeksi yang virulen; dan
\item
  Timbul nya penyakit infeksi baru atau timbul kembali penyakit infeksi di masyarakat (Emerging and or Re- Emerging Disease).
\end{enumerate}

Berdasarkan hasil pengkajian risiko pengendalian infeksi (ICRA), Komite/Tim PPI menyusun Program PPI rumah sakit setiap tahunnya.

Program pencegahan dan pengendalian infeksi harus komprehensif, mencakup risiko infeksi bagi pasien maupun staf yang meliputi:

\begin{enumerate}
\def\labelenumi{\alph{enumi}.}
\tightlist
\item
  Identifikasi dan penanganan:
\end{enumerate}

\begin{enumerate}
\def\labelenumi{\arabic{enumi}.}
\tightlist
\item
  Masalah infeksi yang penting secara epidemiologis seperti data surveilans
\item
  Infeksi yang dapat memberikan dampak bagi pasien, staf dan pengunjung:
\end{enumerate}

\begin{enumerate}
\def\labelenumi{\alph{enumi}.}
\setcounter{enumi}{1}
\tightlist
\item
  Strategi lintas unit: kegiatan di area-area yang berisiko tinggi terjadinya infeksi;
\item
  Kebersihan tangan;
\item
  Pengawasan untuk peningkatan penggunaan antimikroba yang aman serta memastikan penyiapan obat yang aman;
\item
  Investigasi wabah penyakit menular;
\item
  Penerapan program vaksinasi untuk staf dan pasien:
\item
  Pelayanan sterilisasi alat dan pelayanan yang menggunakan peralatan yang berisiko infeksi;
\item
  Pembersihan permukaan dan kebersihan lingkungan;
\item
  Pengelolaan linen/laundri;
\item
  Pengelolaan sampah;
\item
  Penyediaan makanan; dan
\item
  Pengelolaan di kamar jenazah.
\end{enumerate}

Rumah sakit juga melakukan kaji banding angka kejadian dan tren di rumah sakit lain yang setara. Ilmu pengetahuan terkait pengendalian infeksi melalui pedoman praktik klinik, program pengawasan antibiotik, program PPI dan pembatasan penggunaan peralatan invasif yang tidak diperlukan telah diterapkan untuk menurunkan tingkat infeksi secara signifikan.

Penanggung jawab program menerapkan intervensi berbasis bukti untuk meminimalkan risiko infeksi. Pemantauan yang berkelanjutan untuk risiko yang teridentifikasi dan intervensi pengurangan risiko dipantau efektivitasnya, termasuk perbaikan yang progresif dan berkelanjutan, serta apakah sasaran program perlu diubah berdasarkan keberhasilan dan tantangan yang muncul dari data pemantauan.

\hypertarget{elemen-penilaian-ppi-3}{%
\paragraph*{3. Elemen Penilaian PPI 3}\label{elemen-penilaian-ppi-3}}
\addcontentsline{toc}{paragraph}{3. Elemen Penilaian PPI 3}

\begin{enumerate}
\def\labelenumi{\alph{enumi}.}
\tightlist
\item
  Rumah sakit secara proaktif telah melaksanakan pengkajian risiko pengendalian infeksi (ICRA) setiap tahunnya terhadap tingkat dan kecenderungan infeksi layanan kesehatan sesuai poin a) ? k) pada maksud dan tujuan dan selanjutnya menggunakan data tersebut untuk membuat dan menentukan prioritas/fokus pada Program PPI.
\item
  Rumah sakit telah melaksanakan surveilans data secara periodik dan dianalisis setiap triwulan meliputi a) - f) dalam maksud dan tujuan.
\end{enumerate}

\hypertarget{d.-peralatan-medis-danatau-bahan-medis-habis-pakai-bmhp}{%
\subsection*{d.~Peralatan medis dan/atau Bahan Medis Habis Pakai (BMHP)}\label{d.-peralatan-medis-danatau-bahan-medis-habis-pakai-bmhp}}
\addcontentsline{toc}{subsection}{d.~Peralatan medis dan/atau Bahan Medis Habis Pakai (BMHP)}

\hypertarget{standar-ppi.-4}{%
\paragraph*{1. Standar PPI. 4}\label{standar-ppi.-4}}
\addcontentsline{toc}{paragraph}{1. Standar PPI. 4}

Rumah sakit mengurangi risiko infeksi terkait peralatan medis dan/atau bahan medis habis pakai (BMHP) dengan memastikan kebersihan, desinfeksi, sterilisasi, dan penyimpanan yang memenuhi syarat.

\hypertarget{maksud-dan-tujuan-ppi.-4}{%
\paragraph*{2. Maksud dan Tujuan PPI. 4}\label{maksud-dan-tujuan-ppi.-4}}
\addcontentsline{toc}{paragraph}{2. Maksud dan Tujuan PPI. 4}

Prosedur/tindakan yang menggunakan peralatan medis dan/atau bahan medis habis pakai (BMHP), dapat menjadi sumber utama patogen yang menyebabkan infeksi. Kesalahan dalam membersihkan, mendesinfeksi, maupun mensterilisasi, serta penggunaan maupun penyimpanan yang tidak layak dapat berisiko penularan infeksi. Tenaga Kesehatan harus mengikuti standar yang ditetapkan dalam melakukan kebersihan, desinfeksi, dan sterilisasi. Tingkat disinfeksi atau sterilisasi tergantung pada kategori peralatan medis dan/atau bahan medis habis pakai (BMHP):

\begin{enumerate}
\def\labelenumi{\alph{enumi}.}
\item
  Tingkat 1 - Kritikal: Benda yang dimasukkan ke jaringan yang normal steril atau ke sistem vaskular dan membutuhkan sterilisasi.
\item
\begin{verbatim}
 Tingkat 2 - Semi-kritikal: Benda yang menyentuh selaput lendir atau kulit yang tidak intak dan membutuhkan disinfeksi tingkat tinggi.
\end{verbatim}
\item
  Tingkat 3 - Non-kritikal: Benda yang menyentuh kulit intak tetapi tidak menyentuh selaput lendir, dan membutuhkan disinfeksi tingkat rendah.
\end{enumerate}

Pembersihan dan disinfeksi tambahan dibutuhkan untuk peralatan medis dan/atau bahan medis habis pakai (BMHP) yang digunakan pada pasien yang diisolasi sebagai bagian dari kewaspadaan berbasis transmisi.

Pembersihan, desinfeksi, dan sterilisasi dapat dilakukan di area CSSD atau, di area lain di rumah sakit dengan pengawasan. Metode pembersihan, desinfeksi, dan sterilisasi dilakukan sesuai standar dan seragam di semua area rumah sakit.

Staf yang memroses peralatan medis dan/atau BMHP harus mendapatkan pelatihan. Untuk mencegah kontaminasi, peralatan medis dan/atau BMHP bersih dan steril disimpan di area penyimpanan yang telah ditetapkan, bersih dan kering serta terlindung dari debu, kelembaban, dan perubahan suhu yang drastis. Idealnya, peralatan medis dan BMHP disimpan terpisah dan area penyimpanan steril memiliki akses terbatas.

\hypertarget{elemen-penilaian-ppi.-4}{%
\paragraph*{3. Elemen Penilaian PPI. 4}\label{elemen-penilaian-ppi.-4}}
\addcontentsline{toc}{paragraph}{3. Elemen Penilaian PPI. 4}

a Rumah sakit telah menerapkan pengolahan sterilisasi mengikuti peraturan perundang-undangan.
b. Staf yang memroses peralatan medis dan/atau BMHP telah diberikan pelatihan dalam pembersihan, desinfeksi, dan sterilisasi serta mendapat pengawasan.
c.~Metode pembersihan, desinfeksi, dan sterilisasi dilakukan secara seragam di semua area di rumah sakit.
d.~Penyimpanan peralatan medis dan/atau BMHP bersih dan steril disimpan dengan baik di area penyimpanan yang ditetapkan, bersih dan kering dan terlindungi dari debu, kelembaban, serta perubahan suhu yang ekstrem.
e. Bila sterilisasi dilaksanakan di luar rumah sakit harus dilakukan oleh lembaga yang memiliki sertifikasi mutu dan ada kerjasama yang menjamin kepatuhan proses sterilisasi sesuai dengan peraturan perundang- undangan.

\hypertarget{standar-ppi-4.1}{%
\paragraph*{4. Standar PPI 4.1}\label{standar-ppi-4.1}}
\addcontentsline{toc}{paragraph}{4. Standar PPI 4.1}

Rumah sakit mengidentifikasi dan menetapkan proses untuk mengelola peralatan medis dan/atau bahan medis habis pakai (BMHP) yang sudah kadaluwarsa dan penggunaan ulang (reuse) alat sekali-pakai apabila diizinkan.

\hypertarget{maksud-dan-tujuan-ppi.-4.1}{%
\paragraph*{5. Maksud dan Tujuan PPI. 4.1}\label{maksud-dan-tujuan-ppi.-4.1}}
\addcontentsline{toc}{paragraph}{5. Maksud dan Tujuan PPI. 4.1}

Rumah sakit menetapkan regulasi untuk melaksanakan proses mengelola peralatan medis dan/atau BMHP yang sudah habis waktu pakainya. Rumah sakit menetapkan penggunaan kembali peralatan medis sekali pakai dan/atau BMHP sesuai dengan peraturan perundang-undangan dan standar profesional. Beberapa alat medis sekali pakai dan/atau BMHP dapat digunakan lagi dengan persyaratan spesifik tertentu. Rumah sakit menetapkan ketentuan tentang penggunaan kembali alat medis sekali pakai sesuai dengan peraturan perundang-undangan dan standar profesional meliputi:

\begin{enumerate}
\def\labelenumi{\alph{enumi}.}
\tightlist
\item
  Alat dan material yang dapat dipakai kembali;
\item
  Jumlah maksimum pemakaian ulang dari setiap alat secara spesifik;
\item
  Identifikasi kerusakan akibat pemakaian dan keretakan yang menandakan alat tidak dapat dipakai;
\item
  Proses pembersihan setiap alat yang segera dilakukan sesudah pemakaian dan mengikuti protokol yang jelas;
\item
  Pencantuman identifikasi pasien pada bahan medis habis pakai untuk hemodialisis;
\item
  Pencatatan bahan medis habis pakai yang reuse di rekam medis; dan
\item
  Evaluasi untuk menurunkan risiko infeksi bahan medis habis pakai yang di-reuse.
\end{enumerate}

Ada 2 (dua) risiko jika menggunakan lagi (reuse) alat sekali pakai. Terdapat risiko tinggi terkena infeksi dan juga terdapat risiko kinerja alat tidak cukup atau tidak dapat terjamin sterilitas serta fungsinya Dilakukan pengawasan terhadap proses untuk memberikan atau mencabut persetujuan penggunaan kembali alat medis sekali pakai yang diproses ulang. Daftar alat sekali pakai yang disetujui untuk digunakan kembali diperiksa secara rutin untuk memastikan bahwa daftar tersebut akurat dan terkini.

\hypertarget{elemen-penilaian-ppi-.4.1}{%
\paragraph*{6. Elemen Penilaian PPI .4.1}\label{elemen-penilaian-ppi-.4.1}}
\addcontentsline{toc}{paragraph}{6. Elemen Penilaian PPI .4.1}

\begin{enumerate}
\def\labelenumi{\alph{enumi}.}
\tightlist
\item
  Rumah sakit menetapkan peralatan medis dan/atau BMHP yang dapat digunakan ulang meliputi a) ? g) dalam maksud dan tujuan.
\item
  Rumah sakit menggunakan proses terstandardisasi untuk menentukan kapan peralatan medis dan/atau BMHP yang digunakan ulang sudah tidak aman atau tidak layak digunakan ulang.
\item
  Ada bukti pemantauan, evaluasi, dan tindak lanjut pelaksanaan penggunaan kembali (reuse) peralatan medis dan/atau BMHP meliputi a) - g) dalam maksud dan tujuan.
\end{enumerate}

\hypertarget{e.-kebersihan-lingkungan}{%
\subsection*{e. Kebersihan lingkungan}\label{e.-kebersihan-lingkungan}}
\addcontentsline{toc}{subsection}{e. Kebersihan lingkungan}

\hypertarget{standar-ppi.-5}{%
\paragraph*{1. Standar PPI. 5}\label{standar-ppi.-5}}
\addcontentsline{toc}{paragraph}{1. Standar PPI. 5}

Rumah sakit mengidentifikasi dan menerapkan standar PPI yang diakui untuk pembersihan dan disinfeksi permukaan dan lingkungan.

\hypertarget{maksud-dan-tujuan-ppi.-5}{%
\paragraph*{2. Maksud dan Tujuan PPI. 5}\label{maksud-dan-tujuan-ppi.-5}}
\addcontentsline{toc}{paragraph}{2. Maksud dan Tujuan PPI. 5}

Patogen pada permukaan dan di seluruh lingkungan berperan terjadinya penyakit yang didapat di rumah sakit (hospital-acquired illness) pada pasien, staf, dan pengunjung. Proses pembersihan dan disinfeksi lingkungan meliputi pembersihan lingkungan rutin yaitu pembersihan harian kamar pasien dan area perawatan, ruang tunggu dan ruang publik lainnya, ruang kerja staf, dapur, dan lain sebagainya.

Rumah sakit menetapkan frekuensi pembersihan, peralatan dan cairan pembersih yang digunakan, staf yang bertanggung jawab untuk pembersihan, dan kapan suatu area membutuhkan pembersihan lebih sering. Pembersihan terminal dilakukan setelah pemulangan pasien; dan dapat ditingkatkan jika pasien diketahui atau diduga menderita infeksi menular sebagaimana diindikasikan oleh standar pencegahan dan pengendalian infeksi. Hasil pengkajian risiko akan menentukan area berisiko tinggi yang memerlukan pembersihan dan disinfeksi tambahan; misalnya area ruang operasi, CSSD, unit perawatan intensif neonatal, unit luka bakar, dan unit lainnya. Pembersihan dan disinfeksi lingkungan dipantau misalnya keluhan dan pujian dari pasien dan keluarga, menggunakan penanda fluoresens untuk memeriksa patogen residual.

\hypertarget{elemen-penilaian-ppi.-5}{%
\paragraph*{3. Elemen Penilaian PPI. 5}\label{elemen-penilaian-ppi.-5}}
\addcontentsline{toc}{paragraph}{3. Elemen Penilaian PPI. 5}

\begin{enumerate}
\def\labelenumi{\alph{enumi}.}
\tightlist
\item
  Rumah sakit menerapkan prosedur pembersihan dan disinfeksi permukaan dan lingkungan sesuai standar PPI
\item
  Rumah sakit melaksanakan pembersihan dan desinfeksi tambahan di area berisiko tinggi berdasarkan hasil pengkajian risiko
\item
  Rumah sakit telah melakukan pemantauan proses pembersihan dan disinfeksi lingkungan.
\end{enumerate}

\hypertarget{f.-manajemen-linen}{%
\subsection*{f.~Manajemen linen}\label{f.-manajemen-linen}}
\addcontentsline{toc}{subsection}{f.~Manajemen linen}

Rumah sakit menerapkan pengelolaan linen/laundry sesuai prinsipi PPI dan peraturan perundang undangan

\hypertarget{maksud-dan-tujuan-ppi-6}{%
\paragraph*{1. Maksud dan Tujuan PPI 6}\label{maksud-dan-tujuan-ppi-6}}
\addcontentsline{toc}{paragraph}{1. Maksud dan Tujuan PPI 6}

Penanganan linen, dan laundry di rumah sakit meliputi pengumpulan, pemilahan, pencucian, pengeringan, pelipatan, distribusi, dan penyimpanan. Rumah sakit mengidentifikasi area di mana staf harus untuk mengenakan APD sesuai prinsip PPI dan peraturan perundang undangan.

\hypertarget{elemen-penilaian-ppi.6}{%
\paragraph*{2. Elemen Penilaian PPI.6}\label{elemen-penilaian-ppi.6}}
\addcontentsline{toc}{paragraph}{2. Elemen Penilaian PPI.6}

\begin{enumerate}
\def\labelenumi{\alph{enumi}.}
\tightlist
\item
  Ada unit kerja pengelola linen/laundry yang menyelenggarakan penatalaksanaan sesuai dengan peraturan perundang-undangan.
\item
  Prinsip-prinsip PPI diterapkan pada pengelolaan linen/laundry, termasuk pemilahan, transportasi, pencucian, pengeringan, penyimpanan, dan distribusi
\item
  Ada bukti supervisi oleh IPCN terhadap pengelolaan linen/laundry sesuai dengan prinsip PPI termasuk bila dilaksanakan oleh pihak luar rumah sakit.
\end{enumerate}

\hypertarget{g.-limbah-infeksius}{%
\subsection*{g. Limbah infeksius}\label{g.-limbah-infeksius}}
\addcontentsline{toc}{subsection}{g. Limbah infeksius}

\hypertarget{standar-ppi.7}{%
\paragraph*{1. Standar PPI.7}\label{standar-ppi.7}}
\addcontentsline{toc}{paragraph}{1. Standar PPI.7}

Rumah sakit mengurangi risiko infeksi melalui pengelolaan limbah infeksius sesuai peraturan perundang undangan

\hypertarget{standar-ppi.7.1}{%
\paragraph*{2. Standar PPI.7.1}\label{standar-ppi.7.1}}
\addcontentsline{toc}{paragraph}{2. Standar PPI.7.1}

Rumah sakit menetapkan pengelolaan kamar mayat dan kamar bedah mayat sesuai dengan peraturan perundang- undangan

\hypertarget{standar-ppi-7.2}{%
\paragraph*{3. Standar PPI 7.2}\label{standar-ppi-7.2}}
\addcontentsline{toc}{paragraph}{3. Standar PPI 7.2}

Rumah sakit menetapkan pengelolaan limbah benda tajam dan jarum secara aman.

\hypertarget{maksud-dan-tujuan-ppi.7-ppi-71-ppi-72}{%
\paragraph*{4. Maksud dan Tujuan PPI.7 , PPI 7,1, PPI 7,2}\label{maksud-dan-tujuan-ppi.7-ppi-71-ppi-72}}
\addcontentsline{toc}{paragraph}{4. Maksud dan Tujuan PPI.7 , PPI 7,1, PPI 7,2}

Setiap hari rumah sakit banyak menghasilkan limbah, termasuk limbah infeksius. Pembuangan limbah infeksius dengan tidak benar dapat menimbulkan risiko infeksi di rumah sakit. Hal ini nyata terjadi pada pembuangan cairan tubuh dan material terkontaminasi dengan cairan tubuh, pembuangan darah dan komponen darah, serta pembuangan limbah dari lokasi kamar mayat dan kamar bedah mayat (post mortem). Pemerintah mempunyai regulasi terkait dengan penanganan limbah infeksius dan limbah cair, sedangkan rumah sakit diharapkan melaksanakan ketentuan tersebut sehingga dapat mengurangi risiko infeksi di rumah sakit.
Rumah sakit menyelenggaraan pengelolaan limbah dengan benar untuk meminimalkan risiko infeksi melalui kegiatan sebagai berikut:

\begin{enumerate}
\def\labelenumi{\alph{enumi}.}
\tightlist
\item
  Pengelolaan limbah cairan tubuh infeksius;
\item
  Penanganan dan pembuangan darah serta komponen darah;
\item
  Pemulasaraan jenazah dan bedah mayat;
\item
  Pengelolaan limbah cair;
\item
  Pelaporan pajanan limbah infeksius.
\end{enumerate}

Salah satu bahaya luka karena tertusuk jarum suntik adalah terjadi penularan penyakit melalui darah (blood borne diseases). Pengelolaan limbah benda tajam dan jarum yang tidak benar merupakan kekhawatiran staf terhadap keamanannya. Kebiasaan bekerja sangat memengaruhi timbulnya risiko menderita luka dan kemungkinan terpapar penyakit secara potensial. Identifikasi dan melaksanakan kegiatan praktik berdasar atas bukti sahih (evidence based) menurunkan risiko luka karena tertusuk jarum dan benda tajam. Rumah sakit perlu mengadakan edukasi kepada staf bagaimana mengelola dengan aman benda tajam dan jarum.

Pembuangan yang benar adalah dengan menggunakan wadah menyimpan khusus (safety box) yang dapat ditutup, antitertusuk, dan antibocor baik di dasar maupun di sisinya sesuai dengan peraturan perundangan. Wadah ini harus tersedia dan mudah dipergunakan oleh staf serta wadah tersebut tidak boleh terisi terlalu penuh.Pembuangan jarum yang tidak terpakai, pisau bedah (scalpel), dan limbah benda tajam lainnya jika tidak dilakukan dengan benar akan berisiko terhadap kesehatan masyarakat umumnya dan terutama pada mereka yang bekerja di pengelolaan sampah. Pembuangan wadah berisi limbah benda tajam di laut, misalnya akan menyebabkan risiko pada masyarakat karena wadah dapat rusak atau terbuka. Rumah sakit menetapkan regulasi yang memadai mencakup semua tahapan proses, termasuk identifikasi jenis dan penggunaan wadah secara tepat, pembuangan wadah, dan surveilans proses pembuangan

\hypertarget{elemen-penilaian-ppi.-7}{%
\paragraph*{5. Elemen Penilaian PPI. 7}\label{elemen-penilaian-ppi.-7}}
\addcontentsline{toc}{paragraph}{5. Elemen Penilaian PPI. 7}

\begin{enumerate}
\def\labelenumi{\alph{enumi}.}
\tightlist
\item
  Rumah sakit telah menerapkan pengelolaan limbah rumah sakit untuk meminimalkan risiko infeksi yang meliputi a) ? e) pada maksud dan tujuan.
\item
  Penanganan dan pembuangan darah serta komponen darah sesuai dengan regulasi, dipantau dan dievaluasi, serta di tindak lanjutnya.
\item
  Pelaporan pajanan limbah infeksius sesuai dengan regulasi dan dilaksanakan pemantauan, evaluasi, serta tindak lanjutnya.
\item
  Bila pengelolaan limbah dilaksanakan oleh pihak luar rumah sakit harus berdasar atas kerjasama dengan pihak yang memiliki izin dan sertifikasi mutu sesuai dengan peraturan perundang-undangan
\end{enumerate}

\hypertarget{elemen-penilaian-ppi-7.1}{%
\paragraph*{6. Elemen Penilaian PPI 7.1}\label{elemen-penilaian-ppi-7.1}}
\addcontentsline{toc}{paragraph}{6. Elemen Penilaian PPI 7.1}

\begin{enumerate}
\def\labelenumi{\alph{enumi}.}
\tightlist
\item
  Pemulasaraan jenazah dan bedah mayat sesuai dengan regulasi.
\item
  Ada bukti kegiatan kamar mayat dan kamar bedah mayat sudah dikelola sesuai dengan peraturan perundang-undangan.
\item
  Ada bukti pemantauan dan evaluasi, serta tindak lanjut kepatuhan prinsip-prinsip PPI sesuai dengan peraturan perundang-undangan.
\end{enumerate}

\hypertarget{standar-ppi-7.2-1}{%
\paragraph*{7. Standar PPI 7.2}\label{standar-ppi-7.2-1}}
\addcontentsline{toc}{paragraph}{7. Standar PPI 7.2}

Rumah sakit menetapkan pengelolaan limbah benda tajam dan jarum secara aman.

\hypertarget{maksud-dan-tujuan-ppi-7.2}{%
\paragraph*{8. Maksud dan Tujuan PPI 7.2}\label{maksud-dan-tujuan-ppi-7.2}}
\addcontentsline{toc}{paragraph}{8. Maksud dan Tujuan PPI 7.2}

Salah satu bahaya luka karena tertusuk jarum suntik adalah terjadi penularan penyakit melalui darah (blood borne diseases). Pengelolaan limbah benda tajam dan jarum yang tidak benar merupakan kekhawatiran staf terhadap keamanannya. Kebiasaan bekerja sangat memengaruhi timbulnya risiko menderita luka dan kemungkinan terpapar penyakit secara potensial. Identifikasi dan melaksanakan kegiatan praktik berdasar atas bukti sahih (evidence based) menurunkan risiko luka karena tertusuk jarum dan benda tajam.

Rumah sakit perlu mengadakan edukasi kepada staf bagaimana mengelola dengan aman benda tajam dan jarum. Pembuangan yang benar adalah dengan menggunakan wadah menyimpan khusus (safety box) yang dapat ditutup, antitertusuk, dan antibocor baik di dasar maupun di sisinya sesuai dengan peraturan perundangan. Wadah ini harus tersedia dan mudah dipergunakan oleh staf serta wadah tersebut tidak boleh terisi terlalu penuh.

Pembuangan jarum yang tidak terpakai, pisau bedah (scalpel), dan limbah benda tajam lainnya jika tidak dilakukan dengan benar akan berisiko terhadap kesehatan masyarakat umumnya dan terutama pada mereka yang bekerja di pengelolaan sampah. Pembuangan wadah berisi limbah benda tajam di laut, misalnya akan menyebabkan risiko pada masyarakat karena wadah dapat rusak atau terbuka. Rumah sakit menetapkan regulasi yang memadai mencakup:

\begin{enumerate}
\def\labelenumi{\alph{enumi}.}
\tightlist
\item
  Semua tahapan proses termasuk identifikasi jenis dan penggunaan wadah secara tepat, pembuangan wadah, dan surveilans proses pembuangan.
\item
  Laporan tertusuk jarum dan benda tajam.
\end{enumerate}

\hypertarget{elemen-penilaian-ppi-7.2}{%
\paragraph*{9. Elemen Penilaian PPI 7.2}\label{elemen-penilaian-ppi-7.2}}
\addcontentsline{toc}{paragraph}{9. Elemen Penilaian PPI 7.2}

\begin{enumerate}
\def\labelenumi{\alph{enumi}.}
\tightlist
\item
  Benda tajam dan jarum sudah dikumpulkan, disimpan di dalam wadah yang tidak tembus, tidak bocor, berwarna kuning, diberi label infeksius, dan dipergunakan hanya sekali pakai sesuai dengan peraturan perundangundangan.
\item
  Bila pengelolaan benda tajam dan jarum dilaksanakan oleh pihak luar rumah sakit harus berdasar atas kerjasama dengan pihak yang memiliki izin dan sertifikasi mutu sesuai dengan peraturan perundang- undangan.
\item
  Ada bukti data dokumen limbah benda tajam dan jarum.
\item
  Ada bukti pelaksanaan supervisi dan pemantauan oleh IPCN terhadap pengelolaan benda tajam dan jarum sesuai dengan prinsip PPI, termasuk bila dilaksanakan oleh pihak luar rumah sakit.
\item
  Ada bukti pelaksanaan pemantauan kepatuhan prinsip-prinsip PPI sesuai regulasi.
\end{enumerate}

\hypertarget{h.-pelayanan-makanan}{%
\subsection*{h. Pelayanan makanan}\label{h.-pelayanan-makanan}}
\addcontentsline{toc}{subsection}{h. Pelayanan makanan}

\hypertarget{standar-ppi-8}{%
\paragraph*{1. Standar PPI 8}\label{standar-ppi-8}}
\addcontentsline{toc}{paragraph}{1. Standar PPI 8}

Rumah sakit mengurangi risiko infeksi terkait penyelenggaraan pelayanan makanan.

\hypertarget{maksud-dan-tujuan-ppi-8}{%
\paragraph*{2. Maksud dan Tujuan PPI 8}\label{maksud-dan-tujuan-ppi-8}}
\addcontentsline{toc}{paragraph}{2. Maksud dan Tujuan PPI 8}

Penyimpanan dan persiapan makanan dapat menimbulkan penyaklit seperti keracunan makanan atau infeksi makanan. Penyakit yang berhubungan dengan makanan dapat sangat berbahaya bahkan mengancam jiwa pada pasien yang kondisi tubuhnya sudah lemah karena penyakit atau cedera. Rumah sakit harus memberikan makanan dan juga produk nutrisi dengan aman, yaitu melakukan peyimpanan dan penyiapan makanan pada suhu tertentu yang dapat mencegah perkembangan bakteri. Kontaminasi silang, terutama dari makanan mentah ke makanan yang sudah dimasak adalah salah satu sumber infeksi makanan. Kontaminasi silang dapat juga disebabkan oleh tangan yang terkontaminasi, permukaan meja, papan alas untuk memotong makanan, ataupun kain yang digunakan untuk mengelap permukaan meja atau mengeringkan piring. Selain itu, permukaan yang digunakan untuk menyiapkan makanan; alat makan, perlengkapan masak, panci, dan wajan yang digunakan untuk menyiapkan makanan; dan juga nampan, piring, serta alat makan yang digunakan untuk menyajikan makanan juga dapat menimbulkan risiko infeksi apabila tidak dibersihkan dan disanitasi secara tepat.

Bangunan dapur harus sesuai dengan ketentuan yang meliputi alur mulai bahan makanan masuk sampai makanan jadi keluar, tempat penyimpanan bahan makanan kering dan basah dengan temperatur yang dipersyaratkan, tempat persiapan pengolahan, tempat pengolahan,pembagian dan distribusi sesuai dengan peraturan dan perundangan termasuk kebersihan lantai.

Berdasar atas hal tersebut di atas maka rumah sakit agar menetapkan regulasi yang meliputi

\begin{enumerate}
\def\labelenumi{\alph{enumi}.}
\tightlist
\item
  pelayanan makanan di rumah sakit mulai dari pengelolaan bahan makanan, sanitasi dapur, makanan, alat masak, serta alat makan untuk mengurangi risiko infeksi dan kontaminasi silang;
\item
  standar bangunan, fasilitas dapur, dan pantry sesuai dengan peraturan perundangan termasuk bila makanan diambil dari sumber lain di luar rumah sakit.
\end{enumerate}

\hypertarget{elemen-penilaian-ppi-8}{%
\paragraph*{3. Elemen Penilaian PPI 8}\label{elemen-penilaian-ppi-8}}
\addcontentsline{toc}{paragraph}{3. Elemen Penilaian PPI 8}

\begin{enumerate}
\def\labelenumi{\alph{enumi}.}
\tightlist
\item
  Rumah sakit menetapkan regulasi tentang pelayanan makanan di rumah sakit yang meliputi a) ? b) pada maksud dan tujuan.
\item
  Ada bukti pelaksanaan yang penyimpanan bahan makanan, pengolahan, pembagian/pemorsian, dan distribusi makanan sudah sesuai dengan peraturan perundang-undangan.
\item
  Ada bukti pelaksanaan penyimpanan makanan dan produk nutrisi dengan memperhatikan kesehatan lingkungan meliputi sanitasi, suhu, pencahayaan, kelembapan, ventilasi, dan keamanan untuk mengurangi risiko infeksi.
\end{enumerate}

\hypertarget{i.-risiko-infeksi-pada-konstruksi-dan-renovasi}{%
\subsection*{i. Risiko infeksi pada konstruksi dan renovasi}\label{i.-risiko-infeksi-pada-konstruksi-dan-renovasi}}
\addcontentsline{toc}{subsection}{i. Risiko infeksi pada konstruksi dan renovasi}

\hypertarget{standar-ppi-9}{%
\paragraph*{1. Standar PPI 9}\label{standar-ppi-9}}
\addcontentsline{toc}{paragraph}{1. Standar PPI 9}

Rumah sakit menurunkan risiko infeksi pada fasilitas yang terkait dengan pengendalian mekanis dan teknis (mechanical dan enginering controls) serta pada saat melakukan pembongkaran, konstruksi, dan renovasi gedung.

\hypertarget{maksud-dan-tujuan-ppi-9}{%
\paragraph*{2. Maksud dan Tujuan PPI 9}\label{maksud-dan-tujuan-ppi-9}}
\addcontentsline{toc}{paragraph}{2. Maksud dan Tujuan PPI 9}

Pengendalian mekanis dan teknis (mechanical dan enginering controls) seperti sistem ventilasi bertekanan positif, biological safety cabinet, laminary airflow hood, termostat di lemari pendingin, serta pemanas air untuk sterilisasi piring dan alat dapur adalah contoh peran penting standar pengendalian lingkungan harus diterapkan agar dapat diciptakan sanitasi yang baik yang selanjutnya mengurangi risiko infeksi di rumah sakit. Pembongkaran, konstruksi, renovasi gedung di area mana saja di rumah sakit dapat merupakan sumber infeksi. Pemaparan terhadap debu dan kotoran konstruksi, kebisingan, getaran, kotoran, dan bahaya lain dapat merupakan bahaya potensial terhadap fungsi paru paru serta keamanan staf dan pengunjung.

Rumah sakit meggunakan kriteria risiko untuk menangani dampak renovasi dan pembangunan gedung baru, terhadap persyaratan mutu udara, pencegahan dan pengendalian infeksi, standar peralatan, syarat kebisingan, getaran, dan prosedur darurat. Untuk menurunkan risiko infeksi maka rumah sakit perlu mempunyai regulasi tentang penilaian risiko pengendalian infeksi (infection control risk assessment/ICRA) untuk pembongkaran, konstruksi, serta renovasi gedung di area mana saja di rumah sakit yang meliputi:

\begin{enumerate}
\def\labelenumi{\alph{enumi}.}
\tightlist
\item
  Identifikasi tipe/jenis konstruksi kegiatan proyek dengan kriteria;
\item
  Identifikasi kelompok risiko pasien;
\item
  Matriks pengendalian infeksi antara kelompok risiko pasien dan tipe kontruksi kegiatan;
\item
  Proyek untuk menetapkan kelas/tingkat infeksi;
\item
  Tindak pengendalian infeksi berdasar atas tingkat/kelas infeksi; dan
\item
  Pemantauan pelaksanaan.
\end{enumerate}

Karena itu, rumah sakit agar mempunyai regulasi pengendalian mekanis dan teknis (mechanical dan engineering controls) fasilitas yang antara lain meliputi

\begin{enumerate}
\def\labelenumi{\alph{enumi}.}
\tightlist
\item
  Sistem ventilasi bertekanan positif;
\item
  Biological safety cabinet;
\item
  Laminary airflow hood;
\item
  Termostat di lemari pendingin; dan
\item
  Pemanas air untuk sterilisasi piring dan alat dapur.
\end{enumerate}

\hypertarget{elemen-penilaian-ppi-9}{%
\paragraph*{3. Elemen Penilaian PPI 9}\label{elemen-penilaian-ppi-9}}
\addcontentsline{toc}{paragraph}{3. Elemen Penilaian PPI 9}

\begin{enumerate}
\def\labelenumi{\alph{enumi}.}
\tightlist
\item
  Rumah sakit menerapkan pengendalian mekanis dan teknis (mechanical dan engineering control) minimal untuk fasilitas yang tercantum pada a) - e) pada maksud dan tujuan.
\item
  Rumah sakit menerapkan penilaian risiko pengendalian infeksi (infection control risk assessment/ICRA) yang minimal meliputi a) - f) yang ada pada maksud dan tujuan.
\item
  Rumah sakit telah melaksanakan penilaian risiko pengendalian infeksi (infection control risk assessment/ICRA) pada semua renovasi, kontruksi dan demolisi sesuai dengan regulasi.
\end{enumerate}

\hypertarget{j.-penularan-infeksi}{%
\subsection*{j. Penularan infeksi}\label{j.-penularan-infeksi}}
\addcontentsline{toc}{subsection}{j. Penularan infeksi}

\hypertarget{standar-ppi-10}{%
\paragraph*{1. Standar PPI 10}\label{standar-ppi-10}}
\addcontentsline{toc}{paragraph}{1. Standar PPI 10}

Rumah sakit menyediakan APD untuk kewaspadaan (barrier precautions) dan prosedur isolasi untuk penyakit menular melindungi pasien dengan imunitas rendah (immunocompromised) dan mentransfer pasien dengan airborne diseases di dalam rumah sakit dan keluar rumah sakit serta penempatannya dalam waktu singkat jika rumah sakit tidak mempunyai kamar dengan tekanan negatif (ventilasi alamiah dan mekanik).

\hypertarget{standar-ppi-10.1}{%
\paragraph*{2. Standar PPI 10.1}\label{standar-ppi-10.1}}
\addcontentsline{toc}{paragraph}{2. Standar PPI 10.1}

Rumah sakit mengembangkan dan menerapkan sebuah proses untuk menangani lonjakan mendadak (outbreak) penyakit infeksi air borne.

\hypertarget{maksud-dan-tujuan-ppi-10-ppi-10.1}{%
\paragraph*{3. Maksud dan Tujuan PPI 10, PPI 10.1}\label{maksud-dan-tujuan-ppi-10-ppi-10.1}}
\addcontentsline{toc}{paragraph}{3. Maksud dan Tujuan PPI 10, PPI 10.1}

Rumah sakit menetapkan regulasi isolasi dan pemberian penghalang pengaman serta menyediakan fasilitasnya. Regulasi ditetapkan berdasar atas bagaimana penyakit menular dan cara menangani pasien infeksius atau pasien immuno-suppressed. Regulasi isolasi juga memberikan perlindungan kepada staf dan pengunjung serta lingkungan pasien. (lihat juga PP 3) Kewaspadaan terhadap udara penting untuk mencegah penularan bakteri infeksius yang dapat bertahan lama di udara. Pasien dengan infeksi ``airborne'' sebaiknya ditempatkan di kamar dengan tekanan negatif (negative pressure room).

Jika struktur bangunan tidak memungkinkan membangun ruangan dengan tekanan negatif maka rumah sakit dapat mengalirkan udara lewat sistem penyaring HEPA (high effieciency particulate air) pada tingkat paling sedikit 12 kali pertukaran udara per jam. Rumah sakit sebaiknya menetapkan program untuk menangani pasien infeksi ``air borne'' dalam waktu singkat jika sistem HEPA tidak ada, termasuk jika ada banyak pasien masuk menderita infeksi menular. Pembersihan kamar dengan benar setiap hari selama pasien tinggal di rumah sakit dan pembersihan kembali setelah pasien keluar pulang harus dilakukan sesuai dengan standar atau pedoman pengedalian infeksi.

\hypertarget{elemen-penilaian-ppi-10}{%
\paragraph*{4. Elemen Penilaian PPI 10}\label{elemen-penilaian-ppi-10}}
\addcontentsline{toc}{paragraph}{4. Elemen Penilaian PPI 10}

\begin{enumerate}
\def\labelenumi{\alph{enumi}.}
\tightlist
\item
  Rumah sakit menyediakan dan menempatkan ruangan untuk pasien dengan imunitas rendah (immunocompromised) sesuai dengan peraturan perundang undangan.
\item
  Rumah sakit melaksanakan proses transfer pasien airborne diseases di dalam rumah sakit dan keluar rumah sakit sesuai dengan peraturan perundang- undangan termasuk di ruang gawat darurat dan ruang lainnya
\item
  Rumah sakit telah menempatkan pasien infeksi ?air borne? dalam waktu singkat jika rumah sakit tidak mempunyai kamar dengan tekanan negatif sesuai dengan peraturan perundang-undangan termasuk di ruang gawat darurat dan ruang lainnya.
\item
  Ada bukti pemantauan ruang tekanan negatif dan penempatan pasien secara rutin.
\end{enumerate}

\hypertarget{elemen-penilaian-ppi-10.1}{%
\paragraph*{5. Elemen Penilaian PPI 10.1}\label{elemen-penilaian-ppi-10.1}}
\addcontentsline{toc}{paragraph}{5. Elemen Penilaian PPI 10.1}

\begin{enumerate}
\def\labelenumi{\alph{enumi}.}
\tightlist
\item
  Rumah sakit menerapkan proses pengelolaan pasien bila terjadi ledakan pasien (outbreak) penyakit infeksi air borne.
\item
  Rumah sakit menyediakan ruang isolasi dengan tekanan negatif bila terjadi ledakan pasien (outbreak) sesuai dengan peraturan perundangan.
\item
  Ada bukti dilakukan edukasi kepada staf tentang pengelolaan pasien infeksius jika terjadi ledakan pasien (outbreak) penyakit infeksi air borne.
\end{enumerate}

\hypertarget{k.-kebersihan-tangan}{%
\subsection*{k. Kebersihan Tangan}\label{k.-kebersihan-tangan}}
\addcontentsline{toc}{subsection}{k. Kebersihan Tangan}

\hypertarget{standar-ppi-11}{%
\paragraph*{1. Standar PPI 11}\label{standar-ppi-11}}
\addcontentsline{toc}{paragraph}{1. Standar PPI 11}

Kebersihan tangan menggunakan sabun dan desinfektan adalah sarana efektif untuk mencegah dan mengendalikan infeksi.

\hypertarget{standar-ppi-11.1}{%
\paragraph*{2. Standar PPI 11.1}\label{standar-ppi-11.1}}
\addcontentsline{toc}{paragraph}{2. Standar PPI 11.1}

Sarung tangan, masker, pelindung mata, serta alat pelindung diri lainnya tersedia dan digunakan secara tepat apabila disyaratkan.

\hypertarget{maksud-dan-tujuan-ppi-11-dan-ppi-11.1}{%
\paragraph*{3. Maksud dan Tujuan PPI 11 dan PPI 11.1}\label{maksud-dan-tujuan-ppi-11-dan-ppi-11.1}}
\addcontentsline{toc}{paragraph}{3. Maksud dan Tujuan PPI 11 dan PPI 11.1}

Kebersihan tangan, menggunakan alat pelindung diri, serta disinfektan adalah sarana efektif untuk mencegah dan mengendalikan infeksi. Oleh karena itu, harus tersedia di setiap tempat asuhan pasien yang membutuhkan barang ini. Rumah sakit menetapkan ketentuan tentang tempat di mana alat pelindung diri ini harus tersedia dan dilakukan pelatihan cara memakainya. Sabun, disinfektan, handuk/tissu, serta alat lainnya untuk mengeringkan ditempatkan di lokasi tempat cuci tangan dan prosedur disinfeksi tangan dilakukan

\hypertarget{elemen-penilaian-ppi-11}{%
\paragraph*{4. Elemen Penilaian PPI 11}\label{elemen-penilaian-ppi-11}}
\addcontentsline{toc}{paragraph}{4. Elemen Penilaian PPI 11}

\begin{enumerate}
\def\labelenumi{\alph{enumi}.}
\tightlist
\item
  Rumah sakit telah menerapkan hand hygiene yang mencakup kapan, di mana, dan bagaimana melakukan cuci tangan mempergunakan sabun (hand wash) dan atau dengan disinfektan (hand rubs) serta ketersediaan fasilitas hand hygiene.
\item
  Sabun, disinfektan, serta tissu/handuk sekali pakai tersedia di tempat cuci tangan dan tempat melakukan disinfeksi tangan.
\item
  Ada bukti pelaksanaan pelatihan hand hygiene kepada semua pegawai termasuk tenaga kontrak.
\end{enumerate}

\hypertarget{elemen-penilaian-ppi-11.1}{%
\paragraph*{5. Elemen Penilaian PPI 11.1}\label{elemen-penilaian-ppi-11.1}}
\addcontentsline{toc}{paragraph}{5. Elemen Penilaian PPI 11.1}

\begin{enumerate}
\def\labelenumi{\alph{enumi}.}
\tightlist
\item
  Rumah sakit menerapkan penggunaan alat pelindung diri, tempat yang harus menyediakan alat pelindung diri, dan pelatihan cara memakainya.
\item
  Alat pelindung diri sudah digunakan secara tepat dan benar.
\item
  Ketersediaan alat pelindung diri sudah cukup sesuai dengan regulasi.
\item
  Ada bukti pelatihan penggunaan alat pelindung diri kepada semua pegawai termasuk tenaga kontrak.
\end{enumerate}

\hypertarget{l.-peningkatan-mutu-dan-program-edukasi}{%
\subsection*{l. Peningkatan mutu dan program edukasi}\label{l.-peningkatan-mutu-dan-program-edukasi}}
\addcontentsline{toc}{subsection}{l. Peningkatan mutu dan program edukasi}

\hypertarget{standar-ppi-12}{%
\paragraph*{1. Standar PPI 12}\label{standar-ppi-12}}
\addcontentsline{toc}{paragraph}{1. Standar PPI 12}

Kegiatan PPI diintegrasikan dengan program PMKP (Peningkatan Mutu dan Keselamatan Pasien) dengan menggunakan indikator yang secara epidemiologik penting bagi rumah sakit.

\hypertarget{maksud-dan-tujuan-ppi-12}{%
\paragraph*{2. Maksud dan Tujuan PPI 12}\label{maksud-dan-tujuan-ppi-12}}
\addcontentsline{toc}{paragraph}{2. Maksud dan Tujuan PPI 12}

Rumah sakit menggunakan indikator sebagai informasi untuk memperbaiki kegiatan PPI dan mengurangi tingkat infeksi yang terkait layanan kesehatan sampai tingkat serendah-rendahnya. Rumah sakit dapat menggunakan data indikator dan informasi dan membandingkan dengan tingkat dan kecenderungan di rumah sakit lain. Semua departemen/unit layanan diharuskan ikut serta menentukan prioritas yang diukur di tingkat rumah sakit dan tingkat departemen/unit layanan program PPI.

\hypertarget{elemen-penilaian-ppi-12}{%
\paragraph*{3. Elemen Penilaian PPI 12}\label{elemen-penilaian-ppi-12}}
\addcontentsline{toc}{paragraph}{3. Elemen Penilaian PPI 12}

\begin{enumerate}
\def\labelenumi{\alph{enumi}.}
\tightlist
\item
  Ada regulasi sistem manajemen data terintegrasi antara data surveilans dan data indikator mutu di Komite/ Tim Penyelenggara Mutu.
\item
  Ada bukti pertemuan berkala antara Komite/ Tim Penyelenggara Mutu dan Komite/Tim PPI untuk berkoordinasi dan didokumentasikan.
\item
  Ada bukti penyampaian hasil analisis data dan rekomendasi Komite/Tim PPI kepada Komite/ Tim Penyelenggara Mutu setiap tiga bulan.
\end{enumerate}

\hypertarget{m.-edukasi-pendidikan-dan-pelatihan}{%
\subsection*{m. Edukasi, Pendidikan dan Pelatihan}\label{m.-edukasi-pendidikan-dan-pelatihan}}
\addcontentsline{toc}{subsection}{m. Edukasi, Pendidikan dan Pelatihan}

\hypertarget{standar-ppi-13}{%
\paragraph*{1. Standar PPI 13}\label{standar-ppi-13}}
\addcontentsline{toc}{paragraph}{1. Standar PPI 13}

Rumah sakit melakukan edukasi tentang PPI kepada staf klinis dan nonklinis, pasien, keluarga pasien, serta petugas lainnya yang terlibat dalam pelayanan pasien.

\hypertarget{maksud-dan-tujuan-ppi-13}{%
\paragraph*{2. Maksud dan Tujuan PPI 13}\label{maksud-dan-tujuan-ppi-13}}
\addcontentsline{toc}{paragraph}{2. Maksud dan Tujuan PPI 13}

Agar program PPI efektif harus dilakukan edukasi kepada staf klinis dan nonkliniks tentang program PPI pada waktu mereka baru bekerja di rumah sakit dan diulangi secara teratur. Edukasi diikuti oleh staf klinik dan staf nonklinik, pasien, keluarga pasien, pedagang, dan juga pengunjung. Pasien dan keluarga didorong untuk berpartisipasi dalam implementasi program PPI. Pelatihan diberikan sebagai bagian dari orientasi kepada semua staf baru dan dilakukan pelatihan kembali secara berkala, atau paling sedikit jika ada perubahan kebijakan, prosedur, dan praktik yang menjadi panduan program PPI. Dalam pendidikan juga disampaikan temuan dan kecenderungan ukuran kegiatan. Berdasar atas hal di atas maka rumah sakit agar menetapkan program pelatihan PPI yang meliputi pelatihan untuk

\begin{enumerate}
\def\labelenumi{\alph{enumi}.}
\tightlist
\item
  orientasi pegawai baru baik staf klinis maupun nonklinis di tingkat rumah sakit maupun di unit pelayanan;
\item
  staf klinis (profesional pemberi asuhan) secara berkala;
\item
  staf nonklinis;
\item
  pasien dan keluarga; dan
\item
  pengunjung.
\end{enumerate}

\hypertarget{elemen-penilaian-ppi-13}{%
\paragraph*{3. Elemen Penilaian PPI 13}\label{elemen-penilaian-ppi-13}}
\addcontentsline{toc}{paragraph}{3. Elemen Penilaian PPI 13}

\begin{enumerate}
\def\labelenumi{\alph{enumi}.}
\tightlist
\item
  Rumah sakit menetapkan program pelatihan dan edukasi tentang PPI yang meliputi a) ? e) yang ada pada maksud dan tujuan.
\item
  Ada bukti pelaksanaan pelatihan untuk semua staf klinik dan nonklinik sebagai bagian dari orientasi pegawai baru tentang regulasi dan praktik program PPI.
\item
  Ada bukti pelaksanaan edukasi untuk pasien, keluarga, dan pengunjung
\end{enumerate}

\hypertarget{pendidikan-dalam-pelayanan-kesehatan-ppk}{%
\section*{7. Pendidikan Dalam Pelayanan Kesehatan (PPK)}\label{pendidikan-dalam-pelayanan-kesehatan-ppk}}
\addcontentsline{toc}{section}{7. Pendidikan Dalam Pelayanan Kesehatan (PPK)}

\textbf{Gambaran Umum}

Rumah sakit pendidikan harus mempunyai mutu dan keselamatan pasien yang lebih tinggi daripada rumah sakit non pendidikan. Agar mutu dan keselamatan pasien di rumah sakit pendidikan tetap terjaga maka perlu ditetapkan standar akreditasi untuk rumah sakit pendidikan. Rumah sakit pendidikan memiliki keunikan dengan adanya peserta didik yang terlibat dalam upaya pelayanan pasien. Keberadaan peserta didik ini dapat membantu proses pelayanan namun juga berpotensi untuk mempengaruhi mutu pelayanan dan keselamatan pasien. Ini disebabkan peserta didik masih dalam tahap belajar dan tidak memahami secara penuh protokol yang ditetapkan oleh rumah sakit. Untuk itu perlu pengaturan khusus bagi rumah sakit yang mengadakan pendidikan kesehatan.

\hypertarget{a.-kebijakan-penyelenggaraan-pendidikan}{%
\subsection*{a. Kebijakan Penyelenggaraan Pendidikan}\label{a.-kebijakan-penyelenggaraan-pendidikan}}
\addcontentsline{toc}{subsection}{a. Kebijakan Penyelenggaraan Pendidikan}

\hypertarget{standar-ppk-1}{%
\paragraph*{1. Standar PPK 1}\label{standar-ppk-1}}
\addcontentsline{toc}{paragraph}{1. Standar PPK 1}

Rumah sakit menetapkan regulasi tentang persetujuan dan pemantauan pemilik pimpinan dalam kerja sama penyelenggaraan pendidikan kesehatan di rumah sakit.

\hypertarget{maksud-dan-tujuan-ppk-1}{%
\paragraph*{2. Maksud dan Tujuan PPK 1}\label{maksud-dan-tujuan-ppk-1}}
\addcontentsline{toc}{paragraph}{2. Maksud dan Tujuan PPK 1}

Keputusan penetapan rumah sakit pendidikan merupakan kewenangan kementerian yang membidangi masalah kesehatan berdasarkan keputusan bersama yang dilanjutkan dengan pembuatan perjanjian kerja sama pemilik dan pimpinan rumah sakit dengan pimpinan institusi pendidikan. Hal tersebut penting karena mengintegrasikan penyelenggaraan pendidikan klinis ke dalam operasional rumah sakit memerlukan komitmen dalam pengaturanwaktu, tenaga, dan sumber daya.

Peserta pendidikan klinis termasuk trainee, fellow, peserta pendidikan dokter spesialis, dokter, dokter gigi, dan peserta pendidikan tenaga kesehatan profesional lainnya. Keputusan untuk mengintegrasikan operasional rumah sakit dan pendidikan klinis paling baik dibuat oleh jenjang pimpinan tertinggi yang berperan sebagai pengambil keputusan di suatu rumah sakit bersama institusi pendidikan kedokteran, kedokteran gigi, dan profesi kesehatan lainnya yang didelegasikan kepada organisasi yang mengoordinasi pendidikan klinis.

Untuk penyelenggaraan pendidikan klinis di rumah sakit maka semua pihak harus mendapat informasi lengkap tentang hubungan dan tanggung jawab masing-masing. Pemilik dan/atau representasi pemilik memberikan persetujuan terhadap keputusan tentang visi-misi, rencana strategis, alokasi sumber daya, dan program mutu rumah sakit sehingga dapat ikut bertanggung jawab terhadap seluruh proses penyelenggaraan pendidikan klinis di rumah sakit yang harus konsisten dengan regulasi yang berlaku, visi-misi rumah sakit, komitmen pada mutu, keselamatan pasien, serta kebutuhan pasien. Rumah sakit mendapatkan informasi tentang output dengan kriteria-kriteria yang diharapkan dari institusi pendidikan dari pendidikan klinis yang dilaksanakan di rumah sakit untuk mengetahui mutu pelayanan dalam penyelenggaraan pendidikan klinis di rumah sakit.

Rumah sakit menyetujui output serta kriteria penilaian pendidikan dan harus dimasukkan dalam perjanjian kerja sama. Organisasi yang mengoordinasi pendidikan klinis bertanggung jawab untuk merencanakan, memonitor, dan mengevaluasi penyelenggaraan program pendidikan klinis di rumah sakit. Organisasi yang mengoordinasi pendidikan klinis melakukan penilaian berdasar atas kriteria yang sudah disetujui bersama. Organisasi yang mengoordinasi pendidikan klinis harus melaporkan hasil evaluasi penerimaan, pelaksanaan, dan penilaian output dari program pendidikan kepada pimpinan rumah sakit dan pimpinan institusi pendidikan. (lihat PPK 6)

\hypertarget{elemen-penilaian-ppk-1}{%
\paragraph*{3. Elemen Penilaian PPK 1}\label{elemen-penilaian-ppk-1}}
\addcontentsline{toc}{paragraph}{3. Elemen Penilaian PPK 1}

\begin{enumerate}
\def\labelenumi{\alph{enumi}.}
\tightlist
\item
  Rumah sakit memilki kerjasama resmi rumah sakit dengan institusi pendidikan yang masih berlaku.
\item
  Kerja sama antara rumah sakit dengan institusi pendidikan yang sudah terakreditasi.
\item
  Kriteria penerimaan peserta didik sesuai dengan kapasitas RS harus dicantumkan dalam perjanjian Kerjasama.
\item
  Pemilik, pimpinan rumah sakit dan pimpinan institusi pendidikan membuat kajian tertulis sedikitnya satu kali setahun terhadap hasil evaluasi program pendidikan kesehatan yang dijalankan di rumah sakit.
\end{enumerate}

\hypertarget{standar-ppk-2}{%
\paragraph*{4. Standar PPK 2}\label{standar-ppk-2}}
\addcontentsline{toc}{paragraph}{4. Standar PPK 2}

Pelaksanaan pelayanan dalam pendidikan klinis yang diselenggarakan di rumah sakit mempunyai akuntabilitas manajemen, koordinasi, dan prosedur yang jelas.

\hypertarget{maksud-dan-tujuan-ppk-2}{%
\paragraph*{5. Maksud dan Tujuan PPK 2}\label{maksud-dan-tujuan-ppk-2}}
\addcontentsline{toc}{paragraph}{5. Maksud dan Tujuan PPK 2}

Organisasi yang mengoordinasi pendidikan di rumah sakit menetapkan kewenangan, perencanaan, pemantauan implementasi program pendidikan klinis, serta evaluasi dan analisisnya.
Kesepakatan antara rumah sakit dan institusi pendidikan kedokteran, kedokteran gigi, dan pendidikan tenaga kesehatan professional lainnya harus tercermin dalam organisasi dan kegiatan organisasi yang mengoordinasi pendidikan di rumah sakit.
Rumah sakit memiliki regulasi yang mengatur:

\begin{enumerate}
\def\labelenumi{\alph{enumi}.}
\tightlist
\item
  Kapasitas penerimaan peserta didik sesuai dengan kapasitas rumah sakit yang dicantumkan dalam perjanjian kerja sama;
\item
  Persyaratan kualifikasi pendidik/dosen klinis; dan
\item
  Peserta pendidikan klinis di rumah sakit yang dipertimbangkan berdasarkan masa pendidikan dan level kompetensi.
\end{enumerate}

Rumah sakit mendokumentasikan daftar akurat yang memuat semua peserta pendidikan klinis di rumah sakit. Untuk setiap peserta pendidikan klinis dilakukan pemberian kewenangan klinis untuk menentukan sejauh mana kewenangan yang diberikan secara mandiri atau di bawah supervisi. Rumah sakit harus mempunyai dokumentasi yang paling sedikit meliputi:

\begin{enumerate}
\def\labelenumi{\alph{enumi}.}
\tightlist
\item
  Surat keterangan peserta didik dari institusi pendidikan;
\item
  Ijazah, surat tanda registrasi, dan surat izin praktik yang menjadi persyaratan sesuai dengan peraturan perundang-undangan;
\item
  Klasifikasi akademik;
\item
  Identifikasi kompetensi peserta pendidikan klinis; dan
\item
  Laporan pencapaian kompetensi.
\end{enumerate}

\hypertarget{elemen-penilaian-ppk-2}{%
\paragraph*{6. Elemen Penilaian PPK 2}\label{elemen-penilaian-ppk-2}}
\addcontentsline{toc}{paragraph}{6. Elemen Penilaian PPK 2}

\begin{enumerate}
\def\labelenumi{\alph{enumi}.}
\tightlist
\item
  Rumah sakit menetapkan regulasi tentang pengelolaan dan pengawasan pelaksanaan pendidikan klinis yang telah disepakati bersama meliputi poin a) sampai dengan c) pada maksud dan tujuan.
\item
  Rumah sakit memiliki daftar lengkap memuat nama semua peserta pendidikan klinis yang saat ini ada di rumah sakit.
\item
  Untuk setiap peserta pendidikan klinis terdapat dokumentasi yang meliputi poin a) ? e) pada maksud dan tujuan
\end{enumerate}

\hypertarget{standar-ppk-3}{%
\paragraph*{7. Standar PPK 3}\label{standar-ppk-3}}
\addcontentsline{toc}{paragraph}{7. Standar PPK 3}

Tujuan dan sasaran program pendidikan klinis di rumah sakit disesuaikan dengan jumlah staf yang memberikan pendidikan klinis, variasi dan jumlah pasien, teknologi, serta fasilitas rumah sakit.

\hypertarget{maksud-dan-tujuan-ppk-3}{%
\paragraph*{8. Maksud dan Tujuan PPK 3}\label{maksud-dan-tujuan-ppk-3}}
\addcontentsline{toc}{paragraph}{8. Maksud dan Tujuan PPK 3}

Pendidikan klinis di rumah sakit harus mengutamakan keselamatan pasien serta memperhatikan kebutuhan pelayanan sehingga pelayanan rumah sakit tidak terganggu, akan tetapi justru menjadi lebih baik dengan terdapat program pendidikan klinis ini. Pendidikan harus dilaksanakan secara terintegrasi dengan pelayanan dalam rangka memperkaya pengalaman dan kompetensi peserta didik, termasuk juga pengalaman pendidik klinis untuk selalu memperhatikan prinsip pelayanan berfokus pada pasien.

\begin{enumerate}
\def\labelenumi{\alph{enumi}.}
\tightlist
\item
  Variasi dan jumlah pasien harus selaras dengan kebutuhan untuk berjalannya program, demikian juga fasilitas pendukung pembelajaran harus disesuaikan dengan teknologi berbasis bukti yang harus tersedia.
\item
  Jumlah peserta pendidikan klinis di rumah sakit harus memperhatikan jumlah staf pendidik klinis serta ketersediaan sarana dan prasarana.
\end{enumerate}

\hypertarget{elemen-penilaian-ppk-3}{%
\paragraph*{9. Elemen Penilaian PPK 3}\label{elemen-penilaian-ppk-3}}
\addcontentsline{toc}{paragraph}{9. Elemen Penilaian PPK 3}

\begin{enumerate}
\def\labelenumi{\alph{enumi}.}
\tightlist
\item
  Terdapat bukti perhitungan rasio peserta pendidikan dengan staf pendidik klinis untuk seluruh peserta dari setiap program pendidikan profesi yang disepakati oleh rumah sakit dan institusi pendidikan sesuai dengan peraturan perundang-undangan.
\item
  Terdapat bukti perhitungan peserta didik yang diterima di rumah sakit per periode untuk proses pendidikan disesuaikan dengan jumlah pasien untuk menjamin mutu dan keselamatan pasien.
\item
  Terdapat bukti bahwa sarana prasarana, teknologi, dan sumber daya lain di rumah sakit tersedia untuk mendukung pendidikan peserta didik.
\end{enumerate}

\hypertarget{b.-kompetensi-dan-supervisi}{%
\subsection*{b. Kompetensi dan Supervisi}\label{b.-kompetensi-dan-supervisi}}
\addcontentsline{toc}{subsection}{b. Kompetensi dan Supervisi}

\hypertarget{standar-ppk-4}{%
\paragraph*{1. Standar PPK 4}\label{standar-ppk-4}}
\addcontentsline{toc}{paragraph}{1. Standar PPK 4}

Seluruh staf yang memberikan pendidikan klinis mempunyai kompetensi sebagai pendidik klinis dan mendapatkan kewenangan dari institusi pendidikan dan rumah sakit.

\hypertarget{maksud-dan-tujuan-ppk-4}{%
\paragraph*{2. Maksud dan Tujuan PPK 4}\label{maksud-dan-tujuan-ppk-4}}
\addcontentsline{toc}{paragraph}{2. Maksud dan Tujuan PPK 4}

Seluruh staf yang memberikan pendidikan klinis telah mempunyai kompetensi dan kewenangan klinis untuk dapat mendidik dan memberikan pembelajaran klinis kepada peserta pendidikan klinis di rumah sakit sesuai dengan peraturan perundang-undangan. Daftar staf yang memberikan pendidikan klinis dengan seluruh gelar akademis dan profesinya tersedia di rumah sakit.

Seluruh staf yang memberikan pendidikan klinis harus memenuhi persyaratan kredensial dan memiliki kewenangan klinis untuk melaksanakan pendidikan klinis yang sesuai dengan tuntutan tanggung jawabnya.

\hypertarget{elemen-penilaian-ppk-4}{%
\paragraph*{3. Elemen Penilaian PPK 4}\label{elemen-penilaian-ppk-4}}
\addcontentsline{toc}{paragraph}{3. Elemen Penilaian PPK 4}

\begin{enumerate}
\def\labelenumi{\alph{enumi}.}
\tightlist
\item
  Rumah sakit menetapkan staf klinis yang memberikan pendidikan klinis dan penetapan penugasan klinis serta rincian kewenangan klinis dari rumah sakit.
\item
  Rumah sakit memiliki daftar staf klinis yang memberikan pendidikan klinis secara lengkap (akademik dan profesi) sesuai dengan jenis pendidikan yang dilaksanakan di rumah sakit.
\item
  Rumah sakit memiliki bukti staf klinis yang memberikan pendidikan klinis telah mengikuti pendidikan sebagai pendidikan dan keprofesian berkelanjutan.
\end{enumerate}

\hypertarget{standar-ppk-5}{%
\paragraph*{4. Standar PPK 5}\label{standar-ppk-5}}
\addcontentsline{toc}{paragraph}{4. Standar PPK 5}

Rumah sakit memastikan pelaksanaan pendidikan yang dijalankan untuk setiap jenis dan jenjang pendidikan staf klinis di rumah sakit aman bagi pasien dan peserta didik.

\hypertarget{maksud-dan-tujuan-ppk-5}{%
\paragraph*{5. Maksud dan Tujuan PPK 5}\label{maksud-dan-tujuan-ppk-5}}
\addcontentsline{toc}{paragraph}{5. Maksud dan Tujuan PPK 5}

Supervisi dalam pendidikan menjadi tanggung jawab staf klinis yang memberikan pendidikan klinis untuk menjadi acuan pelayanan rumah sakit agar pasien, staf, dan peserta didik terlindungi secara hukum. Supervisi diperlukan untuk memastikan asuhan pasien yang aman dan merupakan bagian proses belajar bagi peserta pendidikan klinis. Tingkat supervise ditentukan oleh rumah sakit sesuai dengan jenjang pembelajaran dan level kompetensi peserta pendidikan klinis.

Setiap peserta pendidikan klinis di rumah sakit mengerti proses supervisi klinis, meliputi siapa saja yang melakukan supervisi dan frekuensi supervisi oleh staf klinis yang memberikan pendidikan klinis. Pelaksanaan supervisi didokumentasikan dalam log book atau sistem dokumentasi lain untuk peserta didik dan staf klinis yang memberikan pendidikan klinis sesuai dengan ketetapan yang berlaku.

\hypertarget{elemen-penilaian-ppk-5}{%
\paragraph*{6. Elemen Penilaian PPK 5}\label{elemen-penilaian-ppk-5}}
\addcontentsline{toc}{paragraph}{6. Elemen Penilaian PPK 5}

\begin{enumerate}
\def\labelenumi{\alph{enumi}.}
\tightlist
\item
  Rumah sakit telah memiliki tingkat supervisi yang diperlukan oleh setiap peserta pendidikan klinis di rumah sakit untuk setiap jenjang pendidikan.
\item
  Setiap peserta pendidikan klinis mengetahui tingkat, frekuensi, dan dokumentasi untuk supervisinya.
\item
  Rumah sakit telah memiliki format spesifik untuk mendokumentasikan proses supervisi yang sesuai dengan kebijakan rumah sakit, tujuan program pendidikan, serta mutu dan keselamatan asuhan pasien.
\item
  Rumah sakit telah memiliki proses pengkajian rekam medis untuk memastikan kepatuhan batasan kewenangan dan proses supervisi peserta pendidikan yang mempunyai akses pengisian rekam medis.
\end{enumerate}

\hypertarget{c.-mutu-dan-keselamatan-dalam-pelaksanaan-pendidikan}{%
\subsection*{c.~Mutu dan Keselamatan Dalam Pelaksanaan Pendidikan}\label{c.-mutu-dan-keselamatan-dalam-pelaksanaan-pendidikan}}
\addcontentsline{toc}{subsection}{c.~Mutu dan Keselamatan Dalam Pelaksanaan Pendidikan}

\hypertarget{standar-ppk-6}{%
\paragraph*{1. Standar PPK 6}\label{standar-ppk-6}}
\addcontentsline{toc}{paragraph}{1. Standar PPK 6}

Pelaksanaan pendidikan klinis di rumah sakit harus mematuhi regulasi rumah sakit dan pelayanan yang diberikan berada dalam upaya mempertahankan atau meningkatkan mutu dan keselamatan pasien.

\hypertarget{maksud-dan-tujuan-ppk-6}{%
\paragraph*{2. Maksud dan Tujuan PPK 6}\label{maksud-dan-tujuan-ppk-6}}
\addcontentsline{toc}{paragraph}{2. Maksud dan Tujuan PPK 6}

Dalam pelaksanaannya program pendidikan klinis tersebut senantiasa menjamin mutu dan keselamatan pasien. Rumah sakit memiliki rencana dan melaksanakan program orientasi terkait penerapan konsep mutu dan keselamatan pasien yang harus diikuti oleh seluruh peserta pendidikan klinis serta mengikutsertakan peserta didik dalam semua pemantauan mutu dan keselamatan pasien. Orientasi peserta pendidikan klinis minimal mencakup:

\begin{enumerate}
\def\labelenumi{\alph{enumi}.}
\tightlist
\item
  Program rumah sakit tentang mutu dan keselamatan pasien;
\item
  Program pengendalian infeksi;
\item
  Program keselamatan penggunaan obat; dan
\item
  Sasaran keselamatan pasien.
\end{enumerate}

Peserta pendidikan klinis seyogyanya diikutsertakan dalam pelaksanaan program peningkatan mutu dan keselamatan pasien di rumah sakit, yang disesuaikan dengan jenis dan jenjang pendidikannya. Penugasan peserta didik dalam pelaksanaan program mutu dan keselamatan pasien diatur bersama antara organisasi pengelola pendidikan, pengelola mutu dan keselamatan pasien, serta kepala unit pelayanan. Rumah sakit harus dapat membuktikan bahwa adanya peserta didik di rumah sakit tidak menurunkan mutu pelayanan dan tidak membahayakan keselamatan pasien di rumah sakit. Hasil survei kepuasan pasien atas pelayanan rumah sakit harus memasukkan unsur kepuasan atas keterlibatan peserta didik dalam pelayanan kepada pasien.

\hypertarget{elemen-penilaian-ppk-6}{%
\paragraph*{3. Elemen Penilaian PPK 6}\label{elemen-penilaian-ppk-6}}
\addcontentsline{toc}{paragraph}{3. Elemen Penilaian PPK 6}

\begin{enumerate}
\def\labelenumi{\alph{enumi}.}
\tightlist
\item
  Rumah sakit menetapkan unit yang bertanggung jawab untuk mengelola pelaksanaan pendidikan klinis di rumah sakit.
\item
  Rumah sakit menetapkan program orientasi peserta pendidikan klinis.
\item
  Rumah sakit telah memiliki bukti pelaksanaan dan sertifikat program orientasi peserta pendidikan klinis.
\item
  Rumah sakit telah memiliki bukti pelaksanaan dan dokumentasi peserta didik diikutsertakan dalam semua program peningkatan mutu dan keselamatan pasien di rumah sakit.
\item
  Rumah sakit telah memantau dan mengevaluasi bahwa pelaksanaan program pendidikan kesehatan tidak menurunkan mutu dan keselamatan pasien yang dilaksanakan sekurang-kurangnya sekali setahun yang terintegrasi dengan program mutu dan keselamatan pasien.
\item
  Rumah sakit telah melakukan survei mengenai kepuasan pasien terhadap pelayanan rumah sakit atas dilaksanakannya pendidikan klinis sekurang- kurangnya sekali setahun.
\end{enumerate}

\hypertarget{b.-kelompok-pelayanan-berfokus-pada-pasien}{%
\chapter*{B. Kelompok Pelayanan Berfokus Pada Pasien}\label{b.-kelompok-pelayanan-berfokus-pada-pasien}}
\addcontentsline{toc}{chapter}{B. Kelompok Pelayanan Berfokus Pada Pasien}

\hypertarget{akses-dan-kesinambungan-pelayanan-akp}{%
\section*{1. Akses dan Kesinambungan Pelayanan (AKP)}\label{akses-dan-kesinambungan-pelayanan-akp}}
\addcontentsline{toc}{section}{1. Akses dan Kesinambungan Pelayanan (AKP)}

\textbf{Gambaran Umum}

Rumah sakit mempertimbangkan bahwa asuhan di rumah sakit merupakan bagian dari suatu sistem pelayanan yang terintegrasi dengan para profesional pemberi asuhan (PPA) dan tingkat pelayanan yang akan membangun suatu kesinambungan pelayanan. Dimulai dengan skrining, yang tidak lain adalah memeriksa pasien secara cepat, untuk mengidentifikasi kebutuhan pasien.

Tujuan sistem pelayanan yang terintegrasi adalah menyelaraskan kebutuhan asuhan pasien dengan pelayanan yang tersedia di rumah sakit, mengkoordinasikan pelayanan, merencanakan pemulangan dan tindakan selanjutnya. Hasil yang diharapkan dari proses asuhan di rumah sakit adalah meningkatkan mutu asuhan pasien dan efisiensi penggunaan sumber daya yang tersedia di rumah sakit.

Fokus pada standar mencakup:

\begin{enumerate}
\def\labelenumi{\alph{enumi}.}
\tightlist
\item
  Skrining pasien di rumah sakit;
\item
  Registrasi dan admisi di rumah sakit;
\item
  Kesinambungan pelayanan;
\item
  Transfer pasien internal di dalam rumah sakit;
\item
  Pemulangan, rujukan dan tindak lanjut; dan
\item
  Transportasi.
\end{enumerate}

\hypertarget{a.-skrining-pasien-di-rumah-sakit}{%
\subsection*{a. Skrining pasien di rumah sakit}\label{a.-skrining-pasien-di-rumah-sakit}}
\addcontentsline{toc}{subsection}{a. Skrining pasien di rumah sakit}

\hypertarget{standar-akp-1}{%
\subsubsection*{1. Standar AKP 1}\label{standar-akp-1}}
\addcontentsline{toc}{subsubsection}{1. Standar AKP 1}

Rumah sakit menetapkan proses skrining baik pasien rawat inap maupun rawat jalan untuk mengidentifikasi pelayanan Kesehatan yang dibutuhkan sesuai dengan misi serta sumber daya rumah sakit.

\hypertarget{maksud-dan-tujuan-akp-1}{%
\subsubsection*{2. Maksud dan Tujuan AKP 1}\label{maksud-dan-tujuan-akp-1}}
\addcontentsline{toc}{subsubsection}{2. Maksud dan Tujuan AKP 1}

Menyesuaikan kebutuhan pasien dengan misi dan sumber daya rumah sakit bergantung pada informasi yang diperoleh tentang kebutuhan pasien dan kondisinya lewat skrining pada kontak pertama. Skrining penerimaan pasien dilaksanakan melalui jalur cepat (fast track) kriteria triase, evaluasi visual atau pengamatan, atau hasil pemeriksaan fisis, psikologis, laboratorium klinis, atau diagnostik imajing sebelumnya. Skrining dapat dilakukan di luar rumah sakit seperti ditempat pasien berada, di ambulans, atau saat pasien tiba di rumah sakit.

Keputusan untuk mengobati, mentransfer atau merujuk dilakukan setelah hasil hasil skrining selesai dievaluasi. Bila rumah sakit mempunyai kemampuan memberikan pelayanan yang dibutuhkan serta konsisten dengan misi dan kemampuan pelayanannya maka dipertimbangkan untuk menerima pasien rawat inap atau pasien rawat jalan. Skirining khusus dapat dilakukan oleh RS sesuai kebutuhan seperti skrining infeksi (TBC, PINERE, COVID- 19, dll), skrining nyeri, skrining geriatri, skrining jatuh atau skrining lainnya

\hypertarget{elemen-penilaian-akp-1}{%
\subsubsection*{3. Elemen Penilaian AKP 1}\label{elemen-penilaian-akp-1}}
\addcontentsline{toc}{subsubsection}{3. Elemen Penilaian AKP 1}

\begin{enumerate}
\def\labelenumi{\alph{enumi}.}
\tightlist
\item
  Rumah sakit telah menetapkan regulasi akses dan kesinambungan pelayanan (AKP) meliputi poin a) -- f) pada gambaran umum.
\item
  Rumah sakit telah menerapkan proses skrining baik di dalam maupun di luar rumah sakit dan terdokumentasi.
\item
  Ada proses untuk memberikan hasil pemeriksaan diagnostik kepada tenaga kesehatan yang kompeten/terlatih untuk bertanggung jawab menentukan apakah pasien akan diterima, ditransfer, atau dirujuk.
\item
  Bila kebutuhan pasien tidak dapat dipenuhi sesuai misi dan sumber daya yang ada, maka rumah sakit akan merujuk atau membantu pasien ke fasilitas pelayanan yang sesuai kebutuhannya.
\end{enumerate}

\hypertarget{standar-akp-1.1}{%
\subsubsection*{4. Standar AKP 1.1}\label{standar-akp-1.1}}
\addcontentsline{toc}{subsubsection}{4. Standar AKP 1.1}

Pasien dengan kebutuhan darurat, sangat mendesak, atau yang membutuhkan pertolongan segera diberikan prioritas untuk pengkajian dan tindakan.

\hypertarget{maksud-dan-tujuan-akp-1.1}{%
\subsubsection*{5. Maksud dan Tujuan AKP 1.1}\label{maksud-dan-tujuan-akp-1.1}}
\addcontentsline{toc}{subsubsection}{5. Maksud dan Tujuan AKP 1.1}

Pasien dengan kebutuhan gawat dan/atau darurat, atau pasien yang membutuhkan pertolongan segera diidentifikasi menggunakan proses triase berbasis bukti untuk memprioritaskan kebutuhan pasien, dengan mendahulukan dari pasien yang lain. Pada kondisi bencana, dapat menggunakan triase bencana. Sesudah dinyatakan pasien darurat, mendesak dan membutuhkan pertolongan segera, dilakukan pengkajian dan memberikan pelayanan sesegera mungkin. Kriteria psikologis berbasis bukti dibutuhkan dalam proses triase untuk kasus kegawatdaruratan psikiatris. Pelatihan bagi staf diadakan agar staf mampu menerapkan kriteria triase berbasis bukti dan memutuskan pasien yang membutuhkan pertolongan segera serta pelayanan yang dibutuhkan.

\hypertarget{elemen-penilaian-akp-1.1}{%
\subsubsection*{6. Elemen Penilaian AKP 1.1}\label{elemen-penilaian-akp-1.1}}
\addcontentsline{toc}{subsubsection}{6. Elemen Penilaian AKP 1.1}

\begin{enumerate}
\def\labelenumi{\alph{enumi}.}
\tightlist
\item
  Proses triase dan pelayanan kegawatdaruratan telah diterapkan oleh staf yang kompeten dan bukti dokumen kompetensi dan kewenangan klinisnya tersedia.
\item
  Staf telah menggunakan kriteria triase berbasis bukti untuk memprioritaskan pasien sesuai dengan kegawatannya.
\item
  Pasien darurat dinilai dan distabilkan sesuai kapasitas rumah sakit sebelum ditransfer ke ruang rawat atau dirujuk dan didokumentasikan dalam rekam medik.
\end{enumerate}

\hypertarget{standar-akp-1.2}{%
\subsubsection*{7. Standar AKP 1.2}\label{standar-akp-1.2}}
\addcontentsline{toc}{subsubsection}{7. Standar AKP 1.2}

Rumah sakit melakukan skrining kebutuhan pasien saat admisi rawat inap untuk menetapkan pelayanan preventif, paliatif, kuratif, rehabilitatif, pelayanan khusus/spesialistik atau pelayanan intensif.

\hypertarget{maksud-dan-tujuan-akp-1.2}{%
\subsubsection*{8. Maksud dan Tujuan AKP 1.2}\label{maksud-dan-tujuan-akp-1.2}}
\addcontentsline{toc}{subsubsection}{8. Maksud dan Tujuan AKP 1.2}

Ketika pasien diputuskan diterima untuk masuk rawat inap, maka proses skrining akan membantu staf mengidentifikasi pelayanan preventif, kuratif, rehabilitatif, paliatif yang dibutuhkan pasien kemudian menentukan pelayanan yang paling sesuai dan mendesak atau yang paling diprioritaskan.

Setiap rumah sakit harus menetapkan kriteria prioritas untuk menentukan pasien yang membutuhkan pelayanan di unit khusus/spesialistik (misalnya unit luka bakar atau transplantasi organ) atau pelayanan di unit intensif (misalnya ICU, ICCU, NICU, PICU, pascaoperasi). Kriteria prioritas meliputi kriteria masuk dan kriteria keluar menggunakan parameter diagnostik dan atau parameter objektif termasuk kriteria berbasis fisiologis.

Dengan mempertimbangkan bahwa pelayanan di unit khusus/spesialistik dan di unit intensif menghabiskan banyak sumber daya, maka rumah sakit dapat membatasi hanya pasien dengan kondisi medis yang reversibel yang dapat diterima dan pasien kondisi khusus termasuk menjelang akhir kehidupan yang sesuai dengan peraturan perundangundangan.
Staf di unit khusus/spesialistik atau unit intensif berpartisipasi dalam menentukan kriteria masuk dan kriteria keluar dari unit tersebut. Kriteria dipergunakan untuk menentukan apakah pasien dapat diterima di unit tersebut, baik dari dalam atau dari luar rumah sakit.

Pasien yang diterima di unit tersebut harus dilakukan pengkajian ulang untuk menentukan apakah kondisi pasien berubah sehingga tidak memerlukan lagi pelayanan khusus/intensif misalnya, jika status fisiologis sudah stabil dan pemantauan intensif baik sehingga tindakan lain tidak diperlukan lagi maka pasien dapat dipindah ke unit layanan yang lebih rendah (seperti unit rawat inap atau unit pelayanan paliatif).

Apabila rumah sakit melakukan penelitian atau menyediakan pelayanan spesialistik atau melaksanakan program, penerimaan pasien di program tersebut harus melalui kriteria tertentu atau ketentuan protokol. Mereka yang terlibat dalam riset atau program lain harus terlibat dalam menentukan kriteria atau protokol. Penerimaan ke dalam program tercatat di rekam medis pasien termasuk kriteria atau protokol yang diberlakukan terhadap pasien yang diterima masuk.

\hypertarget{elemen-penilaian-akp-1.2}{%
\subsubsection*{9. Elemen Penilaian AKP 1.2}\label{elemen-penilaian-akp-1.2}}
\addcontentsline{toc}{subsubsection}{9. Elemen Penilaian AKP 1.2}

\begin{enumerate}
\def\labelenumi{\alph{enumi}.}
\tightlist
\item
  Rumah sakit telah melaksanakan skrining pasien masuk rawat inap untuk menetapkan kebutuhan pelayanan preventif, paliatif, kuratif, dan rehabilitatif, pelayanan khusus/spesialistik atau pelayanan intensif.
\item
  Rumah sakit telah menetapkan kriteria masuk dan kriteria keluar di unit pelayanan khusus/spesialistik menggunakan parameter diagnostik dan atau parameter objektif termasuk kriteria berbasis fisiologis dan terdokumentasikan di rekam medik.
\item
  Rumah sakit telah menerapkan kriteria masuk dan kriteria keluar di unit pelayanan intensif menggunakan parameter diagnostik dan atau parameter objektif termasuk kriteria berbasis fisiologis dan terdokumentasikan di rekam medik
\item
  Staf yang kompeten dan berwenang di unit pelayanan khusus dan unit pelayanan intensif terlibat dalam penyusunan kriteria masuk dan kriteria keluar di unitnya.
\end{enumerate}

\hypertarget{standar-akp-1.3}{%
\subsubsection*{10. Standar AKP 1.3}\label{standar-akp-1.3}}
\addcontentsline{toc}{subsubsection}{10. Standar AKP 1.3}

Rumah Sakit mempertimbangkan kebutuhan klinis pasien dan memberikan informasi kepada pasien jika terjadi penundaan dan kelambatan pelaksanaan tindakan/pengobatan dan atau pemeriksaan penunjang diagnostik.

\hypertarget{maksud-dan-tujuan-akp-1.3}{%
\subsubsection*{11. Maksud dan Tujuan AKP 1.3}\label{maksud-dan-tujuan-akp-1.3}}
\addcontentsline{toc}{subsubsection}{11. Maksud dan Tujuan AKP 1.3}

Pasien diberitahu jika ada penundaan dan kelambatan pelayanan antara lain akibat kondisi pasien atau jika pasien harus masuk dalam daftar tunggu. Pasien diberi informasi alasan mengapa terjadi penundaan/kelambatan pelayanan dan alternatif yang tersedia. Ketentuan ini berlaku bagi pasien rawat inap dan rawat jalan serta pemeriksaan penunjang diagnostik. Untuk beberapa pelayanan, seperti onkologi atau transplan tidak berlaku ketentuan tentang penundaan/kelambatan pelayanan atau pemeriksaan.

Hal ini tidak berlaku untuk keterlambatan staf medis di rawat jalan atau bila unit gawat darurat terlalu ramai dan ruang tunggunya penuh. (Lihat juga ACC.2). Untuk layanan tertentu, seperti onkologi atau transplantasi, penundaan mungkin sesuai dengan norma nasional yang berlaku untuk pelayanan tersebut.

\hypertarget{elemen-penilaian-akp-1.3}{%
\subsubsection*{12. Elemen Penilaian AKP 1.3}\label{elemen-penilaian-akp-1.3}}
\addcontentsline{toc}{subsubsection}{12. Elemen Penilaian AKP 1.3}

\begin{enumerate}
\def\labelenumi{\alph{enumi}.}
\tightlist
\item
  Pasien dan atau keluarga diberi informasi jika ada penundaan dan atau keterlambatan pelayanan beserta alasannya dan dicatat di rekam medis.
\item
  Pasien dan atau keluarga diberi informasi tentang alternatif yang tersedia sesuai kebutuhan klinis pasien dan dicatat di rekam medis.
\end{enumerate}

\hypertarget{b.-registrasi-dan-admisi-di-rumah-sakit}{%
\subsection*{b. Registrasi dan admisi di rumah sakit}\label{b.-registrasi-dan-admisi-di-rumah-sakit}}
\addcontentsline{toc}{subsection}{b. Registrasi dan admisi di rumah sakit}

\hypertarget{standar-akp-2}{%
\subsubsection*{1. Standar AKP 2}\label{standar-akp-2}}
\addcontentsline{toc}{subsubsection}{1. Standar AKP 2}

Rumah Sakit menetapkan proses penerimaan dan pendaftaran pasien rawat inap, rawat jalan, dan pasien gawat darurat.

\hypertarget{maksud-dan-tujuan-akp-2}{%
\subsubsection*{2. Maksud dan Tujuan AKP 2}\label{maksud-dan-tujuan-akp-2}}
\addcontentsline{toc}{subsubsection}{2. Maksud dan Tujuan AKP 2}

Rumah sakit melaksanakan proses penerimaan pasien rawat inap dan pendaftaran pasien rawat jalan dan gawat darurat sesuai peraturan perundang-undangan. Staf memahami dan mampu melaksanakan proses penerimaan pasien. Proses tersebut antara lain meliputi:

\begin{enumerate}
\def\labelenumi{\alph{enumi}.}
\tightlist
\item
  Pendaftaran pasien gawat darurat;
\item
  Penerimaan langsung pasien dari IGD ke rawat inap;
\item
  Admisi pasien rawat inap;
\item
  Pendaftaran pasien rawat jalan;
\item
  Observasi pasien; dan
\item
  Mengelola pasien bila tidak tersedia tempat tidur. Rumah Sakit sering melayani berbagai pasien misalnya pasien lansia, disabilitas (fisik, mental, intelektual), berbagai bahasa dan dialek, budaya yang berbeda atau hambatan yang lainnya, sehingga dibutuhkan sistem pendaftaran dan admisi secara online. Sistim tersbut diharapkan dapat mengurangi hambatan pada saat penerimaan pasien.
\end{enumerate}

Saat pasien diputuskan untuk rawat inap, maka staf medis yang memutuskan tersebut memberi informasi tentang rencana asuhan yang diberikan dan hasil asuhan yang diharapkan. Informasi juga harus diberikan oleh petugas admisi/pendaftaran rawat inap tentang perkiraan biaya selama perawatan. Pemberian informasi tersebut didokumentasikan.

Keselamatan pasien adalah salah satu aspek perawatan pasien yang penting. Orientasi lingkungan di bangsal rawat inap dan peralatan yang terkait dalam pemberian perawatan dan pelayanan yang diberikan merupakan salahsatukomponenpentingdarikeselamatanpasien.

\hypertarget{elemen-penilaian-akp-2}{%
\subsubsection*{3. Elemen Penilaian AKP 2}\label{elemen-penilaian-akp-2}}
\addcontentsline{toc}{subsubsection}{3. Elemen Penilaian AKP 2}

\begin{enumerate}
\def\labelenumi{\alph{enumi}.}
\tightlist
\item
  Rumah sakit telah menerapkan proses penerimaan pasien meliputi poin a) -- f) pada maksud dan tujuan.
\item
  Rumah sakit telah menerapkan sistim pendaftaran pasien rawat jalan dan rawat inap baik secara offline maupun secara online dan dilakukan evaluasi dan tindak lanjutnya.
\item
  Rumah sakit telah memberikan informasi tentang rencana asuhan yang akan diberikan, hasil asuhan yang diharapkan serta perkiraan biaya yang harus dibayarkan oleh pasien/keluarga.
\item
  Saat diterima sebagai pasien rawat inap, pasien dan keluarga mendapat edukasi dan orientasi tentang ruang rawat inap.
\end{enumerate}

\hypertarget{standar-akp-2.1}{%
\subsubsection*{4. Standar AKP 2.1}\label{standar-akp-2.1}}
\addcontentsline{toc}{subsubsection}{4. Standar AKP 2.1}

Rumah sakit menetapkan proses untuk mengelola alur pasien di seluruh area rumah sakit.

\hypertarget{maksud-dan-tujuan-akp-2.1}{%
\subsubsection*{5. Maksud dan Tujuan AKP 2.1}\label{maksud-dan-tujuan-akp-2.1}}
\addcontentsline{toc}{subsubsection}{5. Maksud dan Tujuan AKP 2.1}

Rumah sakit menetapkan pengelolaan alur pasien saat terjadi penumpukan pasien di UGD sementara tempat tidur di rawat inap sedang terisi penuh. Pengelolaan alur tersebut harus dilakukan secara efektif mulai dari penerimaan, pengkaijan, tindakan, transfer pasien sampai pemulangan untuk mengurangi penundaan asuhan kepada pasien. Komponen pengelolaan alur pasien tersebut meliputi:

\begin{enumerate}
\def\labelenumi{\alph{enumi}.}
\tightlist
\item
  Ketersediaan tempat tidur di tempat sementara/transit/intermediate sebelum mendapatkan tempat tidur di rawat inap;
\item
  Perencanaan fasilitas, peralatan, utilitas, teknologi medis, dan kebutuhan lain untuk mendukung penempatan sementara pasien;
\item
  Perencanaan tenaga untuk memberikan asuhan pasien di tempat sementara/transit termasuk pasien yang diobservasi di unit gawat darurat;
\item
  Alur pelayanan pasien di tempat sementara/transit meliputi pemberian asuhan, tindakan, pemeriksaan laboratorium, pemeriksaan radiologi, tindakan di kamar operasi, dan unit pascaanestes harus sama seperti yang diberikan dirawat inap;
\item
  Efisiensi pelayanan nonklinis penunjang asuhan dan tindakan kepada pasien (seperti kerumahtanggaan dan transportasi);
\item
  Memberikan asuhan pasien yang sama kepada pasien yang dirawat di tempat sementara/transit/intermediate seperti perawatan kepada pasien yang dirawat di ruang rawat inap; dan
\item
  Akses pelayanan yang bersifat mendukung (seperti pekerja sosial, keagamaan atau bantuan spiritual, dan sebagainya).
\end{enumerate}

Pemantauan dan perbaikan proses ini bermanfaat untuk mengatasi masalah penumpukan pasien. Semua staf rumah sakit, mulai dari unit gawat darurat, unit rawat inap, staf medis, keperawatan, administrasi, lingkungan, dan manajemen risiko dapat ikut berperan serta menyelesaikan masalah alur pasien ini. Koordinasi dapat dilakukan oleh Manajer Pelayanan Pasien (MPP)/Case Manager.

Rumah sakit harus menetapkan standar waktu berapa lama pasien dapat diobservasi di unit gawat darurat dan kapan harus di transfer ke di lokasi sementara/transit/intermediate sebelum ditransfer ke unit rawat inap di rumah sakit. Diharapkan rumah sakit dapat mengatur dan menyediakan tempat tersebut bagi pasien.

\hypertarget{elemen-penilaian-akp-2.1}{%
\subsubsection*{6. Elemen Penilaian AKP 2.1}\label{elemen-penilaian-akp-2.1}}
\addcontentsline{toc}{subsubsection}{6. Elemen Penilaian AKP 2.1}

\begin{enumerate}
\def\labelenumi{\alph{enumi}.}
\tightlist
\item
  Rumah sakit telah melaksanakan pengelolaan alur pasien untuk menghindari penumpukan. mencakup poin a) -- g) pada maksud dan tujuan.
\item
  Manajer pelayanan pasien (MPP)/case manager bertanggung jawab terhadap pelaksanaan pengaturan alur pasien untuk menghindari penumpukan.
\item
  Rumah sakit telah melakukan evaluasi terhadap pengelolaan alur pasien secara berkala dan melaksanakan upaya perbaikannya.
\item
  Ada sistem informasi tentang ketersediaan tempat tidur secara online kepada masyarakat.
\end{enumerate}

\hypertarget{c.-kesinambungan-pelayanan}{%
\subsection*{c.~Kesinambungan pelayanan}\label{c.-kesinambungan-pelayanan}}
\addcontentsline{toc}{subsection}{c.~Kesinambungan pelayanan}

\hypertarget{standar-akp-3}{%
\subsubsection*{1. Standar AKP 3}\label{standar-akp-3}}
\addcontentsline{toc}{subsubsection}{1. Standar AKP 3}

Rumah sakit memiliki proses untuk melaksanakan kesinambungan pelayanan di rumah sakit dan integrasi antara profesional pemberi asuhan (PPA) dibantu oleh manajer pelayanan pasien (MPP)/case manager.

\hypertarget{maksud-dan-tujuan-akp-3}{%
\subsubsection*{2. Maksud dan Tujuan AKP 3}\label{maksud-dan-tujuan-akp-3}}
\addcontentsline{toc}{subsubsection}{2. Maksud dan Tujuan AKP 3}

Pelayanan berfokus pada pasien diterapkan dalam bentuk Asuhan Pasien Terintegrasi yang bersifat integrasi horizontal dan vertikal. Pada integrasi horizontal kontribusi profesi tiap-tiap profesional pemberi asuhan (PPA) adalah sama pentingnya atau sederajat. Pada integrasi vertikal pelayanan berjenjang oleh/melalui berbagai unit pelayanan ke tingkat pelayanan yang berbeda maka peranan manajer pelayanan pasien (MPP) penting untuk integrasi tersebut dengan komunikasi yang memadai terhadap profesional pemberi asuhan (PPA).

Pelaksanaan asuhan pasien secara terintegrasi fokus pada pasien mencakup:

\begin{enumerate}
\def\labelenumi{\alph{enumi}.}
\item
  Keterlibatan dan pemberdayaan pasien dan keluarga;
\item
  Dokter penanggung jawab pelayanan (DPJP) sebagai Ketua tim asuhan pasien oleh profesional pemberi asuhan (PPA) (clinical leader);
\item
  Profesional pemberi asuhan (PPA) bekerja sebagai tim interdisiplin dengan kolaborasi interprofesional dibantu antara lain oleh Panduan Praktik Klinis (PPK), Panduan Asuhan Profesional Pemberi Asuhan (PPA) lainnya, Alur Klinis/clinical pathway terintegrasi, Algoritme, Protokol, Prosedur, Standing Order dan CPPT (Catatan Perkembangan Pasien Terintegrasi);
\item
  Perencanaan pemulangan pasien (P3)/discharge planning terintegrasi;
\item
  Asuhan gizi terintegrasi; dan
\item
  Manajer pelayanan pasien/case manager.
\end{enumerate}

Manajer Pelayanan Pasien (MPP) bukan merupakan profesional pemberi asuhan (PPA) aktif dan dalam menjalankan manajemen pelayanan pasien mempunyai peran minimal adalah sebagai berikut:

\begin{enumerate}
\def\labelenumi{\alph{enumi}.}
\tightlist
\item
  Memfasilitasi pemenuhan kebutuhan asuhan pasien;
\item
  Mengoptimalkan terlaksananya pelayanan berfokus pada pasien;
\item
  Mengoptimalkan proses reimbursemen; dan dengan fungsi sebagai berikut;
\item
  Asesmen untuk manajemen pelayanan pasien;
\item
  Perencanaan untuk manajemen pelayanan pasien;
\item
  Komunikasi dan koordinasi;
\item
  Edukasi dan advokasi; dan
\item
  Kendali mutu dan biaya pelayanan pasien.
\end{enumerate}

Keluaran yang diharapkan dari kegiatan manajemen pelayanan pasien antara lain adalah:

\begin{enumerate}
\def\labelenumi{\alph{enumi}.}
\tightlist
\item
  Pasien mendapat asuhan sesuai dengan kebutuhannya;
\item
  Terpelihara kesinambungan pelayanan;
\item
  Pasien memahami/mematuhi asuhan dan peningkatan kemandirian pasien;
\item
  Kemampuan pasien mengambil keputusan;
\item
  Keterlibatan serta pemberdayaan pasien dan keluarga;
\item
  Optimalisasi sistem pendukung pasien;
\item
  Pemulangan yang aman; dan
\item
  Kualitas hidup dan kepuasan pasien.
\end{enumerate}

Oleh karenanya, dalam pelaksanaan manajemen pelayanan pasien, manajer pelayanan pasien (MPP) mencatat pada lembar formulir A yang merupakan evaluasi awal manajemen pelayanan pasien dan formulir B yang merupakan catatan implementasi manajemen pelayanan pasien. Kedua formulir tersebut merupakan bagian rekam medis.

Pada formulir A dicatat antara lain identifikasi/skrining pasien untuk kebutuhan pengelolaan manajer pelayanan pasien (MPP) dan asesmen untuk manajemen pelayanan pasien termasuk rencana, identifikasi masalah -- risiko -- kesempatan, serta perencanaan manajemen pelayanan pasien, termasuk memfasiltasi proses perencanaan pemulangan pasien (discharge planning). Pada formulir B dicatat antara lain pelaksanaan rencana manajemen pelayanan pasien, pemantauan, fasilitasi, koordinasi, komunikasi dan kolaborasi, advokasi, hasil pelayanan, serta terminasi manajemen pelayanan pasien.

Agar kesinambungan asuhan pasien tidak terputus, rumah sakit harus menciptakan proses untuk melaksanakan kesinambungan dan koordinasi pelayanan di antara profesional pemberi asuhan (PPA), manajer pelayanan pasien (MPP), pimpinan unit, dan staf lain sesuai dengan regulasi rumah sakit di beberapa tempat.

\begin{enumerate}
\def\labelenumi{\alph{enumi}.}
\tightlist
\item
  Pelayanan darurat dan penerimaan rawat inap;
\item
  Pelayanan diagnostik dan tindakan;
\item
  Pelayanan bedah dan nonbedah;
\item
  Pelayanan rawat jalan; dan
\item
  Organisasi lain atau bentuk pelayanan lainnya.
\end{enumerate}

Proses koordinasi dan kesinambungan pelayanan dibantu oleh penunjang lain seperti panduan praktik klinis, alur klinis/clinical pathways, rencana asuhan, format rujukan, daftar tilik/check list lain, dan sebagainya. Diperlukan regulasi untuk proses koordinasi tersebut.

\hypertarget{elemen-penilaian-akp-3}{%
\subsubsection*{3. Elemen Penilaian AKP 3}\label{elemen-penilaian-akp-3}}
\addcontentsline{toc}{subsubsection}{3. Elemen Penilaian AKP 3}

\begin{enumerate}
\def\labelenumi{\alph{enumi}.}
\tightlist
\item
  Para PPA telah memberikan asuhan pasien secara terintegrasi berfokus pada pasien meliputi poin a) - f) pada maksud dan tujuan.
\item
  Ada penunjukkan MPP dengan uraian tugas meliputi poin a) - h) pada maksud dan tujuan.
\item
  Para profesional pemberi asuhan (PPA) dan manajer pelayanan pasien (MPP) telah melaksanakan kesinambungan dan koordinasi pelayanan meliputi poin a) - e) pada maksud dan tujuan.
\item
  Pencatatan perkembangan pasien didokumentasikan para PPA di formulir catatan pasien terintegrasi (CPPT).
\item
  Pencatatan di unit intensif atau unit khusus menggunakan lembar pemantauan pasien khusus, pencatatan perkembangan pasien dilakukan pada lembar tersebut oleh DPJP di unit tersebut, PPA lain dapat melakukan pencatatan perkembangan pasien di formulir catatan pasien terintegrasi (CPPT).
\item
  Perencanaan dan pelayanan pasien secara terintegrasi diinformasikan kepada pasien dan atau keluarga secara berkala sesuai ketentuan Rumah Sakit.
\end{enumerate}

\hypertarget{standar-akp-3.1}{%
\subsubsection*{4. Standar AKP 3.1}\label{standar-akp-3.1}}
\addcontentsline{toc}{subsubsection}{4. Standar AKP 3.1}

Rumah sakit menetapkan bahwa setiap pasien harus memiliki dokter penanggung jawab pelayanan (DPJP) untuk memberikan asuhan kepada pasien.

\hypertarget{maksud-dan-tujuan-akp-3.1}{%
\subsubsection*{5. Maksud dan Tujuan AKP 3.1}\label{maksud-dan-tujuan-akp-3.1}}
\addcontentsline{toc}{subsubsection}{5. Maksud dan Tujuan AKP 3.1}

Asuhan pasien diberikan oleh profesional pemberi asuhan (PPA) yang bekerja sebagai tim interdisiplin dengan kolaborasi interprofesional dan dokter penanggung jawab pelayanan (DPJP) berperan sebagai ketua tim asuhan pasien oleh profesional pemberi asuhan (PPA) (clinical leader).

Untuk mengatur kesinambungan asuhan selama pasien berada di rumah sakit, harus ada dokter penanggung jawab pelayanan (DPJP) sebagai individu yang bertanggung jawab mengelola pasien sesuai dengan kewenangan klinisnya, serta melakukan koordinasi dan kesinambungan asuhan. Dokter penanggung jawab pelayanan (DPJP) yang ditunjuk ini tercatat namanya di rekam medis pasien.

Dokter penanggung jawab pelayanan (DPJP)/para DPJP memberikan keseluruhan asuhan selama pasien berada di RS dapat meningkatkan antara lain kesinambungan, koordinasi, kepuasan pasien, mutu, keselamatan, dan termasuk hasil asuhan. Individu ini membutuhkan kolaborasi dan komunikasi dengan profesional pemberi asuhan (PPA) lainnya.

Bila seorang pasien dikelola oleh lebih satu dokter penanggung jawab pelayanan (DPJP) maka harus ditetapkan DPJP utama. Sebagai tambahan, rumah sakit menetapkan kebijakan dan proses perpindahan tanggung jawab dari satu dokter penanggung jawab pelayanan (DPJP) ke DPJP lain.

\hypertarget{elemen-penilaian-akp-3.1}{%
\subsubsection*{6. Elemen Penilaian AKP 3.1}\label{elemen-penilaian-akp-3.1}}
\addcontentsline{toc}{subsubsection}{6. Elemen Penilaian AKP 3.1}

\begin{enumerate}
\def\labelenumi{\alph{enumi}.}
\tightlist
\item
  Rumah sakit telah menetapkan bahwa setiap pasien memiliki dokter penanggung jawab pelayanan (DPJP) dan telah melakukan asuhan pasien secara terkoordinasi dan terdokumentasi dalam rekam medis pasien.
  b Rumah sakit juga menetapkan proses perpindahan tanggung jawab koordinasi asuhan pasien dari satu dokter penanggung jawab pelayanan (DPJP) ke DPJP lain, termasuk bila terjadi perubahan DPJP utama.
\item
  Bila dilaksanakan rawat bersama ditetapkan DPJP utama sebagai koordinator asuhan pasien.
\end{enumerate}

\hypertarget{d.-transfer-pasien-internal-di-dalam-rumah-sakit}{%
\subsection*{d.~Transfer pasien internal di dalam rumah sakit}\label{d.-transfer-pasien-internal-di-dalam-rumah-sakit}}
\addcontentsline{toc}{subsection}{d.~Transfer pasien internal di dalam rumah sakit}

\hypertarget{standar-akp-4}{%
\subsubsection*{1. Standar AKP 4}\label{standar-akp-4}}
\addcontentsline{toc}{subsubsection}{1. Standar AKP 4}

Rumah sakit menetapkan informasi tentang pasien disertakan pada proses transfer internal antar unit di dalam rumah sakit.

\hypertarget{maksud-dan-tujuan-akp-4}{%
\subsubsection*{2. Maksud dan Tujuan AKP 4}\label{maksud-dan-tujuan-akp-4}}
\addcontentsline{toc}{subsubsection}{2. Maksud dan Tujuan AKP 4}

Selama dirawat inap di rumah sakit, pasien mungkin dipindah dari satu pelayanan atau dari satu unit rawat inap ke berbagai unit pelayanan lain atau unit rawat inap lain. Jika profesional pemberi asuhan (PPA) berubah akibat perpindahan ini maka informasi penting terkait asuhan harus mengikuti pasien. Pemberian obat dan tindakan lain dapat berlangsung tanpa halangan dan kondisi pasien dapat dimonitor. Untuk memastikan setiap tim asuhan menerima informasi yang diperlukan maka rekam medis pasien ikut pindah atau ringkasan informasi yang ada di rekam medis disertakan waktu pasien pindah dan menyerahkan kepada tim asuhan yang menerima pasien.

Formulir transfer pasien internal meliputi:

\begin{enumerate}
\def\labelenumi{\alph{enumi}.}
\tightlist
\item
  Alasan admisi;
\item
  Temuan signifikan;
\item
  Diagnosis;
\item
  Prosedur yang telah dilakukan;
\item
  Obat-obatan;
\item
  Perawatan lain yang diterima pasien; dan
\item
  Kondisi pasien saat transfer.
\end{enumerate}

Bila pasien dalam pengelolaan manajer pelayanan pasien (MPP) maka kesinambungan proses tersebut di atas dipantau, diikuti, dan transfernya disupervisi oleh manajer pelayanan pasien (MPP).

\hypertarget{elemen-penilaian-akp-4}{%
\subsubsection*{3. Elemen penilaian AKP 4}\label{elemen-penilaian-akp-4}}
\addcontentsline{toc}{subsubsection}{3. Elemen penilaian AKP 4}

\begin{enumerate}
\def\labelenumi{\alph{enumi}.}
\tightlist
\item
  Rumah sakit telah menerapkan proses transfer pasien antar unit pelayanan di dalam rumah sakit dilengkapi dengan formulir transfer pasien.
\item
  Formulir transfer internal meliputi poin a) - g) pada maksud dan tujuan.
\end{enumerate}

\hypertarget{e.-pemulangan-rujukan-dan-tindak-lanjut}{%
\subsection*{e. Pemulangan, rujukan dan tindak lanjut}\label{e.-pemulangan-rujukan-dan-tindak-lanjut}}
\addcontentsline{toc}{subsection}{e. Pemulangan, rujukan dan tindak lanjut}

\hypertarget{standar-akp-5}{%
\subsubsection*{1. Standar AKP 5}\label{standar-akp-5}}
\addcontentsline{toc}{subsubsection}{1. Standar AKP 5}

Rumah sakit menetapkan dan melaksanakan proses pemulangan pasien dari rumah sakit berdasarkan kondisi kesehatan pasien dan kebutuhan kesinambungan asuhan atau tindakan.

\hypertarget{maksud-dan-tujuan-akp-5}{%
\subsubsection*{2. Maksud dan Tujuan AKP 5}\label{maksud-dan-tujuan-akp-5}}
\addcontentsline{toc}{subsubsection}{2. Maksud dan Tujuan AKP 5}

Merujuk atau mengirim pasien ke fasilitas pelayanan Kesehatan, maupun perorangan di luar rumah sakit didasarkan atas kondisi kesehatan pasien dan kebutuhannya untuk memperoleh kesinambungan asuhan. Dokter penanggung jawab pelayanan (DPJP) dan profesional pemberi asuhan (PPA) lainnya yang bertanggung jawab atas asuhan pasien berkordinasi menentukan kesiapan pasien untuk pulang dari rumah sakit berdasarkan kriteria atau indikasi rujukan yang ditetapkan rumah sakit.

Rujukan ke dokter spesialis, rehabilitasi fisik atau kebutuhan upaya preventif di rumah dikoordinasikan dengan keluarga pasien. Diperlukan proses yang terorganisir untuk memastikan bahwa kesinambungan asuhan dikelola oleh tenaga kesehatan atau oleh sebuah fasilitas pelayanan kesehatan di luar rumah sakit.

Pasien yang memerlukan perencanaan pemulangan pasien (discharge planning) maka rumah sakit mulai merencanakan hal tersebut sejak awal dan mencatatnya di pengkajian awal pasien. Untuk menjaga kesinambungan asuhan dilakukan secara terintegrasi melibatkan semua profesional pemberi asuhan (PPA) terkait difasilitasi oleh manajer pelayanan pasien (MPP). Keluarga dilibatkan sesuai dengan kebutuhan .

Rumah sakit dapat menetapkan kemungkinan pasien diizinkan keluar rumah sakit dalam jangka waktu tertentu untuk keperluan penting.

\hypertarget{elemen-penilaian-akp-5}{%
\subsubsection*{3. Elemen Penilaian AKP 5}\label{elemen-penilaian-akp-5}}
\addcontentsline{toc}{subsubsection}{3. Elemen Penilaian AKP 5}

\begin{enumerate}
\def\labelenumi{\alph{enumi}.}
\tightlist
\item
  Rumah sakit telah menetapkan kriteria pemulangan pasien sesuai dengan kondisi kesehatan dan kebutuhan pelayanan pasien beserta edukasinya.
\item
  Rumah sakit telah menetapkan kemungkinan pasien diizinkan keluar rumah sakit dalam jangka waktu tertentu untuk keperluan penting.
\item
  Penyusunan rencana dan instruksi pemulangan didokumentasikan dalam rekam medis pasien dan diberikan kepada pasien secara tertulis.
\item
  Tindak lanjut pemulangan pasien bila diperlukan dapat ditujukan kepada fasilitas pelayanan kesehatan baik perorangan ataupun dimana pasien untuk memberikan pelayanan berkelanjutan.
\end{enumerate}

\hypertarget{standar-akp-5.1}{%
\subsubsection*{4. Standar AKP 5.1}\label{standar-akp-5.1}}
\addcontentsline{toc}{subsubsection}{4. Standar AKP 5.1}

Ringkasan pasien pulang (discharge summary) dibuat untuk semua pasien rawat inap yang keluar dari rumah sakit.

\hypertarget{maksud-dan-tujuan-akp-5.1}{%
\subsubsection*{5. Maksud dan Tujuan AKP 5.1}\label{maksud-dan-tujuan-akp-5.1}}
\addcontentsline{toc}{subsubsection}{5. Maksud dan Tujuan AKP 5.1}

Ringkasan pasien pulang memberikan gambaran tentang pasien yang dirawat di rumah sakit. Ringkasan dapat digunakan oleh tenaga kesehatan yang bertanggung jawab memberikan tindak lanjut asuhan.

Ringkasan pasien pulang (discharge summary) meliputi:

\begin{enumerate}
\def\labelenumi{\alph{enumi}.}
\tightlist
\item
  Indikasi pasien masuk dirawat, diagnosis, dan komorbiditas lain;
\item
  Temuan fisik penting dan temuan-temuan lain;
\item
  Tindakan diagnostik dan prosedur terapi yang telah dikerjakan;
\item
  Obat yang diberikan selama dirawat inap dengan potensi akibat efek residual setelah obat tidak diteruskan dan semua obat yang harus digunakan di rumah;
\item
  Kondisi pasien (status present); dan
\item
  Instruksi tindak lanjut.
\end{enumerate}

Ringkasan pasien pulang dijelaskan dan ditandatangani oleh pasien/keluarga karena memuat instruksi tindak lanjut.

Ringkasan pasien pulang dibuat sebelum pasien keluar dari rumah sakit oleh dokter penanggung jawab pelayanan (DPJP). Satu salinan/copy dari ringkasan diberikan kepada tenaga kesehatan yang bertanggung jawab memberikan tindak lanjut asuhan kepada pasien. Satu salinan diberikan kepada pasien sesuai dengan regulasi rumah sakit yang mengacu pada peraturan perundangan yang berlaku. Satu salinan diberikan kepada penjamin. Salinan ringkasan berada di rekam medis pasien.

\hypertarget{elemen-penilaian-akp-5.1}{%
\subsubsection*{6. Elemen Penilaian AKP 5.1}\label{elemen-penilaian-akp-5.1}}
\addcontentsline{toc}{subsubsection}{6. Elemen Penilaian AKP 5.1}

\begin{enumerate}
\def\labelenumi{\alph{enumi}.}
\tightlist
\item
  Rumah sakit telah menetapkan Ringkasan pasien pulang meliputi a) -- f) pada maksud dan tujuan.
\item
  Rumah sakit memberikan salinan ringkasan pasien pulang kepada pihak yang berkepentingan dan tersimpan di dalam rekam medik.
\item
  Formulir Ringkasan pasien pulang dijelaskan kepada pasien dan atau keluarga.
\end{enumerate}

\hypertarget{standar-akp-5.2}{%
\subsubsection*{7. Standar AKP 5.2}\label{standar-akp-5.2}}
\addcontentsline{toc}{subsubsection}{7. Standar AKP 5.2}

Rumah sakit menetapkan proses untuk mengelola dan melakukan tindak lanjut pasien dan memberitahu staf rumah sakit bahwa mereka berniat keluar rumah sakit serta menolak rencana asuhan medis.

\hypertarget{standar-akp-5.3}{%
\subsubsection*{8. Standar AKP 5.3}\label{standar-akp-5.3}}
\addcontentsline{toc}{subsubsection}{8. Standar AKP 5.3}

Rumah sakit menetapkan proses untuk mengelola pasien yang menolak rencana asuhan medis yang melarikan diri.

\hypertarget{maksud-dan-tujuan-akp-5.2-dan-akp-5.3}{%
\subsubsection*{9. Maksud dan Tujuan AKP 5.2 dan AKP 5.3}\label{maksud-dan-tujuan-akp-5.2-dan-akp-5.3}}
\addcontentsline{toc}{subsubsection}{9. Maksud dan Tujuan AKP 5.2 dan AKP 5.3}

Jika seorang pasien rawat inap atau rawat jalan telah selesai menjalani pemeriksaan lengkap dan sudah ada rekomendasi tindakan yang akan dilakukan, kemudian pasien memutuskan meninggalkan rumah sakit maka pasien ini dianggap sebagai pasien keluar dan menolak rencana asuhan medis. Pasien rawat inap dan rawat jalan (termasuk pasien dari unit gawat darurat) berhak menolak tindakan medis dan keluar rumah sakit. Pasien ini menghadapi risiko karena menerima pelayanan atau tindakan tidak lengkap yang berakibat terjadi kerusakan permanen atau kematian. Jika seorang pasien rawat inap atau rawat jalan minta untuk keluar dari rumah sakit tanpa persetujuan dokter maka pasien harus diberitahu tentang risiko medis oleh dokter yang membuat rencana asuhan atau tindakan dan proses keluarnya pasien sesuai dengan regulasi rumah sakit. Jika pasien mempunyai dokter keluarga maka dokter keluarga tersebut harus diberitahu tentang keputusan pasien. Bila tidak ada dokter keluarga maka pasien dimotivasi untuk mendapat/mencari pelayanan kesehatan lebih lanjut.

Harus diupayakan agar mengetahui alasan mengapa pasien keluar menolak rencana asuhan medis. Rumah sakit perlu mengetahui alasan ini agar dapat melakukan komunikasi lebih baik dengan pasien dan atau keluarga pasien dalam rangka memperbaiki proses.

Jika pasien menolak rencana asuhan medis tanpa memberi tahu siapapun di dalam rumah sakit atau ada pasien rawat jalan yang menerima pelayanan kompleks atau pelayanan untuk menyelamatkan jiwa, seperti kemoterapi atau terapi radiasi, tidak kembali ke rumah sakit maka rumah sakit harus berupaya menghubungi pasien untuk memberi tahu tentang potensi risiko bahaya yang ada. Rumah sakit menetapkan regulasi untuk proses ini sesuai dengan peraturan perundangan yang berlaku, termasuk rumah sakit membuat laporan ke dinas kesehatan atau kementerian kesehatan tentang kasus infeksi dan memberi informasi tentang pasien yang mungkin mencelakakan dirinya atau orang lain.

\hypertarget{elemen-penilaian-akp-5.2}{%
\subsubsection*{10. Elemen Penilaian AKP 5.2}\label{elemen-penilaian-akp-5.2}}
\addcontentsline{toc}{subsubsection}{10. Elemen Penilaian AKP 5.2}

\begin{enumerate}
\def\labelenumi{\alph{enumi}.}
\tightlist
\item
  Rumah sakit telah menetapkan proses untuk mengelola pasien rawat jalan dan rawat inap yang menolak rencana asuhan medis termasuk keluar rumah sakit atas permintaan sendiri dan pasien yang menghendaki penghentian pengobatan.
\item
  Ada bukti pemberian edukasi kepada pasien tentang risiko medis akibat asuhan medis yang belum lengkap.
\item
  Pasien keluar rumah sakit atas permintaan sendiri, tetapi tetap mengikuti proses pemulangan pasien.
\item
  Dokter keluarga (bila ada) atau dokter yang memberi asuhan berikutnya kepada pasien diberitahu tentang kondisi tersebut.
\item
  Ada dokumentasi rumah sakit melakukan pengkajian untuk mengetahui alasan pasien keluar rumah sakit apakah permintaan sendiri, menolak asuhan medis, atau tidak melanjutkan program pengobatan.
\end{enumerate}

\hypertarget{elemen-penilaian-akp-5.3}{%
\subsubsection*{11. Elemen Penilaian AKP 5.3}\label{elemen-penilaian-akp-5.3}}
\addcontentsline{toc}{subsubsection}{11. Elemen Penilaian AKP 5.3}

\begin{enumerate}
\def\labelenumi{\alph{enumi}.}
\tightlist
\item
  Ada regulasi yang mengatur pasien rawat inap dan rawat jalan yang meninggalkan rumah sakit tanpa pemberitahuan (melarikan diri).
\item
  Rumah sakit melakukan identifikasi pasien menderita penyakit yang membahayakan dirinya sendiri atau lingkungan.
\item
  Rumah sakit melaporkan kepada pihak yang berwenang bila ada indikasi kondisi pasien yang membahayakan dirinya sendiri atau lingkungan.
\end{enumerate}

\hypertarget{standar-akp-5.4}{%
\subsubsection*{12. Standar AKP 5.4}\label{standar-akp-5.4}}
\addcontentsline{toc}{subsubsection}{12. Standar AKP 5.4}

Pasien dirujuk ke fasilitas pelayanan kesehatan lain berdasar atas kondisi pasien untuk memenuhi kebutuhan asuhan berkesinambungan dan sesuai dengan kemampuan fasilitas kesehatan penerima untuk memenuhi kebutuhan pasien.

\hypertarget{maksud-dan-tujuan-akp-5.4}{%
\subsubsection*{13. Maksud dan Tujuan AKP 5.4}\label{maksud-dan-tujuan-akp-5.4}}
\addcontentsline{toc}{subsubsection}{13. Maksud dan Tujuan AKP 5.4}

Pasien dirujuk ke fasilitas kesehatan lain didasarkan atas kondisi pasien dan kebutuhan untuk memperoleh asuhan berkesinambungan. Rujukan pasien antara lain untuk memenuhi kebutuhan pasien atau konsultasi spesialistik dan tindakan, serta penunjang diagnostik. Jika pasien dirujuk ke rumah sakit lain, yang merujuk harus memastikan fasilitas kesehatan penerima menyediakan pelayanan yang dapat memenuhi kebutuhan pasien dan mempunyai kapasitas menerima pasien.

Diperoleh kepastian terlebih dahulu dan kesediaan menerima pasien serta persyaratan rujukan diuraikan dalam kerja sama formal atau dalam bentuk perjanjian. Ketentuan seperti ini dapat memastikan kesinambungan asuhan tercapai dan kebutuhan pasien terpenuhi. Rujukan terjadi juga ke fasilitas kesehatan lain dengan atau tanpa ada perjanjian formal.

\hypertarget{elemen-penilaian-akp-5.4}{%
\subsubsection*{14. Elemen Penilaian AKP 5.4}\label{elemen-penilaian-akp-5.4}}
\addcontentsline{toc}{subsubsection}{14. Elemen Penilaian AKP 5.4}

\begin{enumerate}
\def\labelenumi{\alph{enumi}.}
\tightlist
\item
  Ada regulasi tentang rujukan sesuai dengan peraturan perundang-undangan.
\item
  Rujukan pasien dilakukan sesuai dengan kebutuhan kesinambungan asuhan pasien.
\item
  Rumah sakit yang merujuk memastikan bahwa fasilitas kesehatan yang menerima dapat memenuhi kebutuhan pasien yang dirujuk.
\item
  Ada kerjasama rumah sakit yang merujuk dengan rumah sakit yang menerima rujukan yang sering dirujuk.
\end{enumerate}

\hypertarget{standar-akp-5.5}{%
\subsubsection*{15. Standar AKP 5.5}\label{standar-akp-5.5}}
\addcontentsline{toc}{subsubsection}{15. Standar AKP 5.5}

Rumah sakit menetapkan proses rujukan untuk memastikan pasien pindah dengan aman.

\hypertarget{maksud-dan-tujuan-akp-5.5}{%
\subsubsection*{16. Maksud dan Tujuan AKP 5.5}\label{maksud-dan-tujuan-akp-5.5}}
\addcontentsline{toc}{subsubsection}{16. Maksud dan Tujuan AKP 5.5}

Rujukan pasien sesuai dengan kondisi pasien, menentukan kualifikasi staf pendamping yang memonitor dan menentukan jenis peralatan medis khusus. Selain itu, harus dipastikan fasilitas pelayanan kesehatan penerima menyediakan pelayanan yang dapat memenuhi kebutuhan pasien dan mempunyai kapasitas pasien dan jenis teknologi medis. Diperlukan proses konsisten melakukan rujukan pasien untuk memastikan keselamatan pasien. Proses ini menangani:

\begin{enumerate}
\def\labelenumi{\alph{enumi}.}
\tightlist
\item
  Ada staf yang bertanggung jawab dalam pengelolaan rujukan termasuk untuk memastikan pasien diterima di rumah sakit rujukan yang dapat memenuhi kebutuhan pasien;
\item
  Selama dalam proses rujukan ada staf yang kompeten sesuai dengan kondisi pasien yang selalu memonitor dan mencatatnya dalam rekam medis;
\item
  Dilakukan identifikasi kebutuhan obat, bahan medis habis pakai, alat kesehatan dan peralatan medis yang dibutuhkan selama proses rujukan; dan
\item
  Dalam proses pelaksanaan rujukan, ada proses serah terima pasien antara staf pengantar dan yang menerima. Rumah sakit melakukan evaluasi terhadap mutu dan keamanan proses rujukan untuk memastikan pasien telah ditransfer dengan staf yang kompeten dan dengan peralatan medis yang tepat.
\end{enumerate}

\hypertarget{elemen-penilaian-akp-5.5}{%
\subsubsection*{17. Elemen Penilaian AKP 5.5}\label{elemen-penilaian-akp-5.5}}
\addcontentsline{toc}{subsubsection}{17. Elemen Penilaian AKP 5.5}

\begin{enumerate}
\def\labelenumi{\alph{enumi}.}
\tightlist
\item
  Rumah sakit memiliki staf yang bertanggung jawab dalam pengelolaan rujukan termasuk untuk memastikan pasien diterima di rumah sakit rujukan yang dapat memenuhi kebutuhan pasien.
\item
  Selama proses rujukan ada staf yang kompeten sesuai dengan kondisi pasien yang selalu memantau dan mencatatnya dalam rekam medis.
\item
  Selama proses rujukan tersedia obat, bahan medis habis pakai, alat kesehatan, dan peralatan medis sesuai dengan kebutuhan kondisi pasien.
\item
  Rumah sakit memiliki proses serah terima pasien antara staf pengantar dan yang menerima.
\item
  Pasien dan keluarga dijelaskan apabila rujukan yang dibutuhkan tidak dapat dilaksanakan.
\end{enumerate}

\hypertarget{standar-akp-5.6}{%
\subsubsection*{18. Standar AKP 5.6}\label{standar-akp-5.6}}
\addcontentsline{toc}{subsubsection}{18. Standar AKP 5.6}

Rumah sakit menetapkan regulasi untuk mengatur proses rujukan dan dicatat di rekam medis pasien.

\hypertarget{maksud-dan-tujuan-akp-5.6}{%
\subsubsection*{19. Maksud dan Tujuan AKP 5.6}\label{maksud-dan-tujuan-akp-5.6}}
\addcontentsline{toc}{subsubsection}{19. Maksud dan Tujuan AKP 5.6}

Informasi tentang pasien yang dirujuk disertakan bersama dengan pasien untuk menjamin kesinambungan asuhan.

Formulir rujukan berisi:

\begin{enumerate}
\def\labelenumi{\alph{enumi}.}
\tightlist
\item
  Identitas pasien;
\item
  Hasil pemeriksaan (anamnesis, pemeriksaan fisis, dan pemeriksaan penunjang) yang telah dilakukan;
\item
  Diagnosis kerja;
\item
  Terapi dan/atau tindakan yang telah diberikan;
\item
  Tujuan rujukan; dan
\item
  Nama dan tanda tangan tenaga kesehatan yang memberikan pelayanan rujukan.
\end{enumerate}

Dokumentasi juga memuat nama fasilitas pelayanan kesehatan dan nama orang di fasilitas pelayanan kesehatan yang menyetujui menerima pasien, kondisi khusus untuk rujukan (seperti kalau ruangan tersedia di penerima rujukan atau tentang status pasien). Juga dicatat jika kondisi pasien atau kondisi pasien berubah selama ditransfer (misalnya, pasien meninggal atau membutuhkan resusitasi).

Dokumen lain yang diminta sesuai dengan kebijakan rumah sakit (misalnya, tanda tangan perawat atau dokter yang menerima serta nama orang yang memonitor pasien dalam perjalanan rujukan) masuk dalam catatan. Dokumen rujukan diberikan kepada fasilitas pelayanan kesehatan penerima bersama dengan pasien.
Catatan setiap pasien yang dirujuk ke fasilitas pelayanan kesehatan lainnya memuat juga dokumentasi selama proses rujukan.

Jika proses rujukan menggunakan transportasi dan tenaga pendamping dari pihak ketiga, rumah sakit memastikan ketersediaan kebutuhan pasien selama perjalanan dan melakukan serah terima dengan petugas tersebut.

\hypertarget{elemen-penilaian-akp-5.6}{%
\subsubsection*{20. Elemen Penilaian AKP 5.6}\label{elemen-penilaian-akp-5.6}}
\addcontentsline{toc}{subsubsection}{20. Elemen Penilaian AKP 5.6}

\begin{enumerate}
\def\labelenumi{\alph{enumi}.}
\tightlist
\item
  Dokumen rujukan berisi nama dari fasilitas pelayanan kesehatan yang menerima dan nama orang yang menyetujui menerima pasien.
\item
  Dokumen rujukan berisi alasan pasien dirujuk, memuat kondisi pasien, dan kebutuhan pelayanan lebih lanjut.
\item
  Dokumen rujukan juga memuat prosedur dan intervensi yang sudah dilakukan.
\item
  Proses rujukan dievaluasi dalam aspek mutu dan keselamatan pasien.
\end{enumerate}

\hypertarget{standar-akp-5.7}{%
\subsubsection*{21. Standar AKP 5.7}\label{standar-akp-5.7}}
\addcontentsline{toc}{subsubsection}{21. Standar AKP 5.7}

Untuk pasien rawat jalan yang membutuhkan asuhan yang kompleks atau diagnosis yang kompleks dibuat catatan tersendiri profil ringkas medis rawat jalan (PRMRJ) dan tersedia untuk PPA.

\hypertarget{maksud-dan-tujuan-akp-5.7}{%
\subsubsection*{22. Maksud dan Tujuan AKP 5.7}\label{maksud-dan-tujuan-akp-5.7}}
\addcontentsline{toc}{subsubsection}{22. Maksud dan Tujuan AKP 5.7}

Jika rumah sakit memberikan asuhan dan tindakan berlanjut kepada pasien dengan diagnosis kompleks dan atau yang membutuhkan asuhan kompleks (misalnya pasien yang datang beberapa kali dengan masalah kompleks, menjalani tindakan beberapa kali, datang di beberapa unit klinis, dan sebagainya) maka kemungkinan dapat bertambahnya diagnosis dan obat, perkembangan riwayat penyakit, serta temuan pada pemeriksaan fisis. Oleh karena itu, untuk kasus seperti ini harus dibuat ringkasannya. Sangat penting bagi setiap PPA yang berada di berbagai unit yang memberikan asuhan kepada pasien ini mendapat akses ke informasi profil ringkas medis rawat jalan (PRMRJ) tersebut.

Profil ringkas medis rawat jalan (PRMRJ) memuat informasi, termasuk:

\begin{enumerate}
\def\labelenumi{\alph{enumi}.}
\tightlist
\item
  Identifikasi pasien yang menerima asuhan kompleks atau dengan diagnosis kompleks (seperti pasien di klinis jantung dengan berbagai komorbiditas antara lain DM tipe 2, total knee replacement, gagal ginjal tahap akhir, dan sebagainya. Atau pasien di klinis neurologik dengan berbagai komorbiditas).
\item
  Identifikasi informasi yang dibutuhkan oleh para dokter penanggung jawab pelayanan (DPJP) yang menangani pasien tersebut
\item
  Menentukan proses yang digunakan untuk memastikan bahwa informasi medis yang dibutuhkan dokter penanggung jawab pelayanan (DPJP) tersedia dalam format mudah ditelusur (easy-to-retrieve) dan mudah direvieu.
\item
  Evaluasi hasil implementasi proses untuk mengkaji bahwa informasi dan proses memenuhi kebutuhan dokter penanggung jawab pelayanan (DPJP) dan meningkatkan mutu serta keselamatan pasien.
\end{enumerate}

\hypertarget{elemen-penilaian-akp-5.7}{%
\subsubsection*{23. Elemen Penilaian AKP 5.7}\label{elemen-penilaian-akp-5.7}}
\addcontentsline{toc}{subsubsection}{23. Elemen Penilaian AKP 5.7}

\begin{enumerate}
\def\labelenumi{\alph{enumi}.}
\tightlist
\item
  Rumah sakit telah menetapkan kriteria pasien rawat jalan dengan asuhan yang kompleks atau yang diagnosisnya kompleks diperlukan Profil Ringkas Medis Rawat Jalan (PRMRJ) meliputi poin a-d dalam maksud tujuan.
\item
  Rumah sakit memiliki proses yang dapat dibuktikan bahwa PRMRJ mudah ditelusur dan mudah di-review.
\item
  Proses tersebut dievaluasi untuk memenuhi kebutuhan para DPJP dan meningkatkan mutu serta keselamatan pasien.
\end{enumerate}

\hypertarget{f.-transportasi}{%
\subsection*{f.~Transportasi}\label{f.-transportasi}}
\addcontentsline{toc}{subsection}{f.~Transportasi}

\hypertarget{standar-akp-6}{%
\subsubsection*{1. Standar AKP 6}\label{standar-akp-6}}
\addcontentsline{toc}{subsubsection}{1. Standar AKP 6}

Rumah sakit menetapkan proses transportasi dalam merujuk, memindahkan atau pemulangan, pasien rawat inap dan rawat jalan utk memenuhi kebutuhan pasien.

\hypertarget{maksud-dan-tujuan-akp-6}{%
\subsubsection*{2. Maksud dan Tujuan AKP 6}\label{maksud-dan-tujuan-akp-6}}
\addcontentsline{toc}{subsubsection}{2. Maksud dan Tujuan AKP 6}

Proses merujuk, memindahkan, dan memulangkan pasien membutuhkan pemahaman tentang kebutuhan transpor pasien. Jenis kendaraan untuk transportasi berbagai macam, mungkin ambulans atau kendaraan lain milik rumah sakit atau berasal dari sumber yang diatur oleh keluarga atau kerabat. Jenis kendaraan yang diperlukanvbergantung pada kondisi dan status pasien. Kendaraan transportasi milik rumah sakit harus tunduk pada peraturan perundangan yang mengatur tentang kegiatan operasionalnya, kondisi, dan perawatan kendaraan. Rumah sakit mengidentifikasi kegiatan transportasi yang berisiko terkena infeksi dan menentukan strategi mengurangi risiko infeksi. Persediaan obat dan perbekalan medis yang harus tersedia dalam kendaraan bergantung pada pasien yang dibawa. Jika rumah sakit membuat kontrak layanan transportasi maka rumah sakit harus dapat menjamin bahwa kontraktor harus memenuhi standar untuk mutu dan keselamatan pasien dan kendaraan. Jika layanan transpor diberikan oleh Kementerian Kesehatan atau Dinas Kesehatan, perusahaan asuransi, atau
organisasi lain yang tidak berada dalam pengawasan rumah sakit maka masukan dari rumah sakit tentang keselamatan dan mutu transpor dapat memperbaiki kinerja penyedia pelayanan transpor. Dalam semua hal, rumah sakit melakukan evaluasi terhadap mutu dan keselamatan pelayanan transportasi. Hal ini termasuk penerimaan, evaluasi, dan tindak lanjut keluhan terkait pelayanan transportasi.

\hypertarget{elemen-penilaian-akp-6}{%
\subsubsection*{3. Elemen Penilaian AKP 6}\label{elemen-penilaian-akp-6}}
\addcontentsline{toc}{subsubsection}{3. Elemen Penilaian AKP 6}

\begin{enumerate}
\def\labelenumi{\alph{enumi}.}
\tightlist
\item
  Rumah sakit memiliki proses transportasi pasien sesuai dengan kebutuhannya yang meliputi pengkajian kebutuhan transportasi, SDM, obat, bahan medis habis pakai, alat kesehatan, peralatan medis dan persyaratan PPI yang sesuai dengan kebutuhan pasien.
\item
  Bila rumah sakit memiliki kendaraan transport sendiri, ada bukti pemeliharan kendaraan tersebut sesuai dengan peraturan perundang-undangan.
\item
  Bila rumah sakit bekerja sama dengan jasa transportasi pasien mandiri, ada bukti kerja sama tersebut dan evaluasi berkala dari rumah sakit mengenai kelayakan kendaraan transportasi, memenuhi aspek mutu, keselamatan pasien dan keselamatan transportasi.
\item
  Kriteria alat transportasi yang digunakan untuk merujuk, memindahkan, atau memulangkan pasien ditentukan oleh rumah sakit (staf yang kompeten), harus sesuai dengan Program PPI, memenuhi aspek mutu, keselamatan pasien dan keselamatan transportasi.
\end{enumerate}

\hypertarget{hak-pasien-dan-keterlibatan-keluarga-hpk}{%
\section*{2. Hak Pasien dan Keterlibatan Keluarga (HPK)}\label{hak-pasien-dan-keterlibatan-keluarga-hpk}}
\addcontentsline{toc}{section}{2. Hak Pasien dan Keterlibatan Keluarga (HPK)}

\textbf{Gambaran Umum}

Hak pasien dalam pelayanan kesehatan dilindungi oleh undang- undang. Dalam memberikan pelayanan, rumah sakit menjamin hak pasien yang dilindungi oleh peraturan perundangan tersebut dengan mengupayakan agar pasien mendapatkan haknya di rumah sakit.

Dalam memberikan hak pasien, rumah sakit harus memahami bahwa pasien dan keluarganya memiliki sikap, perilaku, kebutuhan pribadi, agama, keyakinan, budaya dan nilai-nilai yang dianut.

Hasil pelayanan pada pasien akan meningkat bila pasien dan keluarga atau mereka yang berhak mengambil keputusan diikutsertakan dalam pengambilan keputusan pelayanan dan proses yang sesuai dengan harapan, nilai, serta budaya yang dimiliki. Pendidikan pasien dan keluarga membantu pasien lebih memahami dan berpartisipasi dalam perawatan mereka untuk membuat keputusan perawatan yang lebih baik.

Standar ini akan membahas proses-proses untuk:

\begin{enumerate}
\def\labelenumi{\alph{enumi}.}
\tightlist
\item
  Mengidentifikasi, melindungi, dan mempromosikan hak-hak pasien;
\item
  Menginformasikan pasien tentang hak-hak mereka;
\item
  Melibatkan keluarga pasien, bila perlu, dalam keputusan tentang perawatan pasien;
\item
  Mendapatkan persetujuan (informed consent); dan
\item
  Mendidik staf tentang hak pasien.
\end{enumerate}

Proses-proses ini terkait dengan bagaimana sebuah organisasi menyediakan perawatan kesehatan dengan cara yang adil dan sesuai dengan peraturan perundangan yang berlaku.
Lebih lanjut, standar Hak Pasien dan Keterlibatan Keluarga akan berfokus pada:

\begin{enumerate}
\def\labelenumi{\alph{enumi}.}
\tightlist
\item
  Hak pasien dan keluarga; dan
\item
  Permintaan persetujuan pasien.
\end{enumerate}

\hypertarget{a.-hak-pasien-dan-keluarga}{%
\subsection*{a. Hak pasien dan keluarga}\label{a.-hak-pasien-dan-keluarga}}
\addcontentsline{toc}{subsection}{a. Hak pasien dan keluarga}

\hypertarget{standar-hpk-1}{%
\subsubsection*{1. Standar HPK 1}\label{standar-hpk-1}}
\addcontentsline{toc}{subsubsection}{1. Standar HPK 1}

Rumah sakit menerapkan proses yang mendukung hak-hak pasien dan keluarganya selama pasien mendapatkan pelayanan dan perawatan di rumah sakit.

\hypertarget{maksud-dan-tujuan-hpk-1}{%
\subsubsection*{2. Maksud dan Tujuan HPK 1}\label{maksud-dan-tujuan-hpk-1}}
\addcontentsline{toc}{subsubsection}{2. Maksud dan Tujuan HPK 1}

Pimpinan rumah sakit harus mengetahui dan memahami hak-hak pasien dan keluarganya serta tanggung jawab organisasi sebagaimana tercantum dalam peraturan perundangan. Pimpinan memberikan arahan untuk memastikan bahwa seluruh staf ikut berperan aktif dalam melindungi hak pasien tersebut.

Hak pasien dan keluarga merupakan unsur dasar dari seluruh hubungan antara organisasi, staf, pasien dan keluarga. Rumah sakit menggunakan proses kolaboratif untuk mengembangkan kebijakan dan prosedur, dan apabila diperlukan, melibatkan para pasien dan keluarganya selama proses tersebut.

Sering kali, pasien ingin agar keluarga dapat berpartisipasi dalam pengambilan keputusan terkait perawatan mereka. Pasien memiliki hak untuk mengidentifikasi siapa yang mereka anggap sebagai keluarga dan diizinkan untuk melibatkan orang-orang tersebut dalam perawatan. Agar keluarga dapat berpartisipasi, mereka harus diizinkan hadir. Pasien diberi kesempatan untuk memutuskan apakah mereka ingin keluarga ikut terlibat dan sejauh mana keluarga akan terlibat dalam perawatan pasien, informasi apa mengenai perawatan yang dapat diberikan kepada keluarga/pihak lain, serta dalam keadaan apa.

\hypertarget{elemen-penilaian-hpk-1}{%
\subsubsection*{3. Elemen Penilaian HPK 1}\label{elemen-penilaian-hpk-1}}
\addcontentsline{toc}{subsubsection}{3. Elemen Penilaian HPK 1}

\begin{enumerate}
\def\labelenumi{\alph{enumi}.}
\tightlist
\item
  Rumah sakit menerapkan regulasi hak pasien dan keluarga sebagaimana tercantum dalam poin a) -- d) pada gambaran umum dan peraturan perundang- undangan.
\item
  Rumah sakit memiliki proses untuk mengidentifikasi siapa yang diinginkan pasien untuk berpartisipasi dalam pengambilan keputusan terkait perawatannya.
\item
  Rumah sakit memiliki proses untuk menentukan preferensi pasien, dan pada beberapa keadaan preferensi keluarga pasien, dalam menentukan informasi apa mengenai perawatan pasien yang dapat diberikan kepada keluarga/pihak lain, dan dalam situasi apa.
\item
  Semua staff dilatih tentang proses dan peran mereka dalam mendukung hak-hak serta partisipasi pasien dan keluarga dalam perawatan.
\end{enumerate}

\hypertarget{standar-hpk-1.1}{%
\subsubsection*{4. Standar HPK 1.1}\label{standar-hpk-1.1}}
\addcontentsline{toc}{subsubsection}{4. Standar HPK 1.1}

Rumah sakit berupaya mengurangi hambatan fisik, bahasa, budaya, dan hambatan lainnya dalam mengakses dan memberikan layanan serta memberikan informasi dan edukasi kepada pasien dan keluarga dalam bahasa dan cara yang dapat mereka pahami.

\hypertarget{maksud-dan-tujuan-hpk-1.1}{%
\subsubsection*{5. Maksud dan Tujuan HPK 1.1}\label{maksud-dan-tujuan-hpk-1.1}}
\addcontentsline{toc}{subsubsection}{5. Maksud dan Tujuan HPK 1.1}

Rumah sakit mengidentifikasi hambatan, menerapkan proses untuk menghilangkan atau mengurangi hambatan, dan mengambil tindakan untuk mengurangi dampak hambatan bagi pasien yang memerlukan pelayanan dan perawatan. Sebagai contoh: tersedia akses yang aman ke unit perawatan/pelayanan, tersedia rambu-rambu disabilitas dan rambu-rambu lain seperti penunjuk arah atau alur evakuasi yang mencakup penggunaan rambu multi bahasa dan/atau simbol internasional, dan disediakan penerjemah yang dapat digunakan untuk pasien dengan kendala bahasa.

Rumah sakit menyiapkan pernyataan tertulis tentang hak dan tanggung jawab pasien dan keluarga yang tersedia bagi pasien ketika mereka dirawat inap atau terdaftar sebagai pasien rawat jalan. Pernyataan tersebut terpampang di area rumah sakit atau dalam bentuk brosur atau dalam metode lain seperti pemberian informasi staf pada saat diperlukan. Pernyataan tersebut sesuai dengan usia, pemahaman, bahasa dan cara yang dipahami pasien.

\hypertarget{elemen-penilaian-hpk-1.1}{%
\subsubsection*{6. Elemen Penilaian HPK 1.1}\label{elemen-penilaian-hpk-1.1}}
\addcontentsline{toc}{subsubsection}{6. Elemen Penilaian HPK 1.1}

\begin{enumerate}
\def\labelenumi{\alph{enumi}.}
\tightlist
\item
  Rumah mengidentifikasi hambatan serta menerapkan proses untuk mengurangi hambatan bagi pasien dalam mendapatkan akses, proses penerimaan dan pelayanan perawatan.
\item
  Informasi terkait aspek perawatan dan tata laksana medis pasien diberikan dengan cara dan bahasa yang dipahami pasien.
\item
  Informasi mengenai hak dan tanggung jawab pasien terpampang di area rumah sakit atau diberikan kepada setiap pasien secara tertulis atau dalam metode lain dalam bahasa yang dipahami pasien.
\end{enumerate}

\hypertarget{standar-hpk-1.2}{%
\subsubsection*{7. Standar HPK 1.2}\label{standar-hpk-1.2}}
\addcontentsline{toc}{subsubsection}{7. Standar HPK 1.2}

Rumah sakit memberikan pelayanan yang menghargai martabat pasien, menghormati nilai-nilai dan kepercayaan pribadi pasien serta menanggapi permintaan yang terkait dengan keyakinan agama dan spiritual.

\hypertarget{maksud-dan-tujuan-hpk-1.2}{%
\subsubsection*{8. Maksud dan Tujuan HPK 1.2}\label{maksud-dan-tujuan-hpk-1.2}}
\addcontentsline{toc}{subsubsection}{8. Maksud dan Tujuan HPK 1.2}

Salah satu kebutuhan manusia yang paling penting adalah keinginan untuk dihargai dan memiliki martabat. Pasien memiliki hak untuk dirawat dengan penuh rasa hormat dan tenggang rasa, dalam berbagai keadaan, serta perawatan yang menjaga harkat dan martabat pasien.

Setiap pasien membawa nilai-nilai dan kepercayaan masing-masing ke dalam proses perawatan. Sebagian nilai dan kepercayaan yang umumnya dimiliki oleh semua pasien sering kali berasal dari budaya atau agamanya.

Nilai-nilai dan kepercayaan lainnya dapat berasal dari diri pasien itu sendiri. Semua pasien dapat menjalankan kepercayaannya masing-masing dengan cara yang menghormati kepercayaan orang lain. Semua staf harus berusaha memahami perawatan dan pelayanan yang mereka berikan dalam konteks dari nilai-nilai dan kepercayaan pasien.

\hypertarget{elemen-penilaian-hpk-1.2}{%
\subsubsection*{9. Elemen Penilaian HPK 1.2}\label{elemen-penilaian-hpk-1.2}}
\addcontentsline{toc}{subsubsection}{9. Elemen Penilaian HPK 1.2}

\begin{enumerate}
\def\labelenumi{\alph{enumi}.}
\tightlist
\item
  Staf memberikan perawatan yang penuh penghargaan dengan memerhatikan harkat dan martabat pasien.
\item
  Rumah sakit menghormati keyakinan spiritual dan budaya pasien serta nilai-nilai yang dianut pasien.
\item
  Rumah sakit memenuhi kebutuhan pasien terhadap bimbingan rohani.
\end{enumerate}

\hypertarget{standar-hpk-1.3}{%
\subsubsection*{10. Standar HPK 1.3}\label{standar-hpk-1.3}}
\addcontentsline{toc}{subsubsection}{10. Standar HPK 1.3}

Rumah sakit menjaga privasi pasien dan kerahasiaan informasi dalam perawatan, serta memberikan hak kepada pasien untuk memperoleh akses dalam informasi kesehatan mereka sesuai perundang-undangan yang berlaku.

\hypertarget{maksud-dan-tujuan-hpk-1.3}{%
\subsubsection*{11. Maksud dan Tujuan HPK 1.3}\label{maksud-dan-tujuan-hpk-1.3}}
\addcontentsline{toc}{subsubsection}{11. Maksud dan Tujuan HPK 1.3}

Hak privasi pasien, terutama ketika diwawancara, diperiksa, dirawat dan dipindahkan adalah hal yang sangat penting. Pasien mungkin menginginkan privasinya terlindung dari para karyawan, pasien lain, dan bahkan dari anggota keluarga atau orang lain yang ditentukan oleh pasien. Oleh karena itu staf rumah sakit yang melayani dan merawat pasien harus menanyakan tentang kebutuhan privasi pasien dan harapan yang terkait dengan pelayanan yang dimaksud serta meminta persetujuan terhadap pelepasan informasi medik yang diperlukan.

Informasi medis serta informasi kesehatan lainnya yang didokumentasikan dan dikumpulkan harus dijaga kerahasiannya. Rumah sakit menghargai kerahasiaan informasi tersebut dan menerapkan prosedur yang melindungi informasi tersebut dari kehilangan atau penyalahgunaan. Kebijakan dan prosedur mencakup informasi yang dapat diberikan sesuai ketentuan peraturan dan undang-undang lainnya.

Pasien juga memiliki hak untuk mengakses informasi kesehatan mereka sendiri. Ketika mereka memiliki akses terhadap informasi kesehatan mereka, pasien dapat lebih terlibat di dalam keputusan perawatan dan membuat keputusan yang lebih baik tentang perawatan mereka.

\hypertarget{elemen-penilaian-hpk-1.3}{%
\subsubsection*{12. Elemen Penilaian HPK 1.3}\label{elemen-penilaian-hpk-1.3}}
\addcontentsline{toc}{subsubsection}{12. Elemen Penilaian HPK 1.3}

\begin{enumerate}
\def\labelenumi{\alph{enumi}.}
\tightlist
\item
  Rumah sakit menjamin kebutuhan privasi pasien selama perawatan dan pengobatan di rumah sakit.
\item
  Kerahasiaan informasi pasien dijaga sesuai dengan peraturan perundangan.
\item
  Rumah sakit memiliki proses untuk meminta persetujuan pasien terkait pemberian informasi.
\item
  Rumah sakit memiliki proses untuk memberikan pasien akses terhadap informasi kesehatan mereka.
\end{enumerate}

\hypertarget{standar-hpk-1.4}{%
\subsubsection*{13. Standar HPK 1.4}\label{standar-hpk-1.4}}
\addcontentsline{toc}{subsubsection}{13. Standar HPK 1.4}

Rumah sakit melindungi harta benda pasien dari pencurian atau kehilangan.

\hypertarget{maksud-dan-tujuan-hpk-1.4}{%
\subsubsection*{14. Maksud dan Tujuan HPK 1.4}\label{maksud-dan-tujuan-hpk-1.4}}
\addcontentsline{toc}{subsubsection}{14. Maksud dan Tujuan HPK 1.4}

Rumah sakit bertanggung jawab melindungi terhadap harta benda pasien dari pencurian atau kehilangan. Terdapat proses untuk mencatat dan membuat daftar harta benda yang dibawa pasien dan memastikan agar harta benda tersebut tidak dicuri atau hilang. Proses ini dilakukan di ODC (Pelayanan Satu Hari), pasien rawat inap, serta untuk pasien yang tidak mampu mengambil keputusan untuk menjaga keamanan harta benda mereka karena tidak sadarkan diri atau tidak didampingi penunggu.

\hypertarget{elemen-penilaian-hpk-1.4}{%
\subsubsection*{15. Elemen Penilaian HPK 1.4}\label{elemen-penilaian-hpk-1.4}}
\addcontentsline{toc}{subsubsection}{15. Elemen Penilaian HPK 1.4}

\begin{enumerate}
\def\labelenumi{\alph{enumi}.}
\tightlist
\item
  Rumah sakit menetapkan proses untuk mencatat dan melindungi pertanggungjawaban harta benda pasien.
\item
  Pasien mendapat informasi mengenai tanggung jawab rumah sakit untuk melindungi harta benda pribadi mereka.
\end{enumerate}

\hypertarget{standar-hpk-1.5}{%
\subsubsection*{16. Standar HPK 1.5}\label{standar-hpk-1.5}}
\addcontentsline{toc}{subsubsection}{16. Standar HPK 1.5}

Rumah sakit melindungi pasien dari serangan fisik dan verbal, dan populasi yang berisiko diidentifikasi serta dilindungi dari kerentanan.

\hypertarget{maksud-dan-tujuan-hpk-1.5}{%
\subsubsection*{17. Maksud dan Tujuan HPK 1.5}\label{maksud-dan-tujuan-hpk-1.5}}
\addcontentsline{toc}{subsubsection}{17. Maksud dan Tujuan HPK 1.5}

Rumah sakit bertanggung jawab untuk melindungi pasien dari penganiayaan fisik dan verbal yang dilakukan pengunjung, pasien lain, dan petugas. Tanggung jawab ini sangat penting terutama bagi bayi dan anak-anak, lansia, dan kelompok yang tidak mampu melindungi dirinya sendiri. Rumah sakit berupaya mencegah penganiayaan melalui berbagai proses seperti memeriksa orang-orang yang berada dilokasi tanpa identifikasi yang jelas, memantau wilayah yang terpencil atau terisolasi, dan cepat tanggap dalam membantu mereka yang berada dalam bahaya atau dianiaya.

\hypertarget{elemen-penilaian-hpk-1.5}{%
\subsubsection*{18. Elemen Penilaian HPK 1.5}\label{elemen-penilaian-hpk-1.5}}
\addcontentsline{toc}{subsubsection}{18. Elemen Penilaian HPK 1.5}

\begin{enumerate}
\def\labelenumi{\alph{enumi}.}
\tightlist
\item
  Rumah sakit mengembangkan dan menerapkan proses untuk melindungi semua pasien dari serangan fisik dan verbal.
\item
  Rumah sakit mengidentifikasi populasi yang memiliki risiko lebih tinggi untuk mengalami serangan.
\item
  Rumah sakit memantau area fasilitas yang terisolasi dan terpencil.
\end{enumerate}

\hypertarget{standar-hpk-2}{%
\subsubsection*{19. Standar HPK 2}\label{standar-hpk-2}}
\addcontentsline{toc}{subsubsection}{19. Standar HPK 2}

Pasien dan keluarga pasien dilibatkan dalam semua aspek perawatan dan tata laksana medis melalui edukasi, dan diberikan kesempatan untuk berpartisipasi dalam proses pengambilan keputusan mengenai perawatan serta tata laksananya.

\hypertarget{maksud-dan-tujuan-hpk-2}{%
\subsubsection*{20. Maksud dan Tujuan HPK 2}\label{maksud-dan-tujuan-hpk-2}}
\addcontentsline{toc}{subsubsection}{20. Maksud dan Tujuan HPK 2}

Pasien dan keluarganya ikut berperan serta dalam proses asuhan dengan membuat keputusan mengenai perawatan, mengajukan pertanyaan tentang perawatan, dan bahkan menolak prosedur diagnostik dan tata laksana. Agar pasien dan keluarga dapat berpartisipasi dalam keputusan perawatan, mereka memerlukan informasi mengenai kondisi medis, hasil pemeriksaan, diagnosis, rencana pengobatan dan rencana tindakan serta perawatan, dan alternatif tindakan bila ada. Rumah sakit memastikan mereka dapat berpartisipasi dalam pengambilan keputusan terkait perawatan termasuk untuk melakukan perawatan sendiri di rumah.

Selama proses asuhan, pasien juga memiliki hak untuk diberitahu mengenai kemungkinan hasil yang tidak dapat diantisipasi dari terapi dan perawatan, serta ketika suatu peristiwa atau kejadian yang tidak terduga terjadi selama perawatan dilakukan.

Pasien dan keluarga pasien memahami jenis keputusan yang harus diambil terkait asuhan dan bagaimana mereka berpartisipasi dalam pengambilan keputusan tersebut. Ketika pasien meminta pendapat kedua, rumah sakit tidak boleh menghambat, mencegah ataupun menghalangi upaya pasien yang mencari pendapat kedua, namun sebaliknya, rumah sakit harus memfasilitasi permintaan akan pendapat kedua tersebut dan membantu menyediakan informasi mengenai kondisi pasien, seperti informasi hasil pemeriksaan, diagnosis, rekomendasi terapi, dan sebagainya.

Rumah sakit mendukung dan menganjurkan keterlibatan pasien dan keluarga dalam semua aspek perawatan. Seluruh staf diajarkan mengenai kebijakan dan prosedur serta peranan mereka dalam mendukung hak pasien dan keluarga untuk berpartisipasi dalam proses perawatan.

\hypertarget{elemen-penilaian-hpk-2}{%
\subsubsection*{21. Elemen Penilaian HPK 2}\label{elemen-penilaian-hpk-2}}
\addcontentsline{toc}{subsubsection}{21. Elemen Penilaian HPK 2}

\begin{enumerate}
\def\labelenumi{\alph{enumi}.}
\tightlist
\item
  Rumah sakit menerapkan proses untuk mendukung pasien dan keluarga terlibat dan berpartisipasi dalam proses asuhan dan dalam pengambilan keputusan.
\item
  Rumah sakit menerapkan proses untuk memberikan edukasi kepada pasien dan keluarganya mengenai kondisi medis, diagnosis, serta rencana perawatan dan terapi yang diberikan.
\item
  Pasien diberikan informasi mengenai hasil asuhan dan tata laksana yang diharapkan.
\item
  Pasien diberikan informasi mengenai kemungkinan hasil yang tidak dapat diantisipasi dari terapi dan perawatan.
\item
  Rumah sakit memfasilitasi permintaan pasien untuk mencari pendapat kedua tanpa perlu khawatir akan mempengaruhi perawatannya selama di dalam atau luar rumah sakit.
\end{enumerate}

\hypertarget{standar-hpk-2.1}{%
\subsubsection*{22. Standar HPK 2.1}\label{standar-hpk-2.1}}
\addcontentsline{toc}{subsubsection}{22. Standar HPK 2.1}

Rumah sakit memberikan informasi kepada pasien dan keluarga mengenai hak dan kewajibannya untuk menolak atau menghentikan terapi, menolak diberikan pelayanan resusitasi, serta melepaskan atau menghentikan terapi penunjang kehidupan.

\hypertarget{maksud-dan-tujuan-hpk-2.1}{%
\subsubsection*{23. Maksud dan Tujuan HPK 2.1}\label{maksud-dan-tujuan-hpk-2.1}}
\addcontentsline{toc}{subsubsection}{23. Maksud dan Tujuan HPK 2.1}

Pasien atau keluarga yang mengambil keputusan atas nama pasien, dapat memutuskan untuk tidak melanjutkan rencana perawatan atau terapi ataupun menghentikan perawatan atau terapi setelah proses tersebut dimulai.

Salah satu keputusan yang paling sulit untuk pasien dan keluarga dan juga untuk staf RS adalah keputusan untuk menghentikan layanan resusitasi atau perawatan yang menunjang kehidupan. Oleh karena itu, penting bagi rumah sakit untuk mengembangkan sebuah proses dalam pengambilan keputusan-keputusan sulit.

Untuk memastikan proses pengambilan keputusan yang terkait dengan keinginan pasien dilakukan secara konsisten, rumah sakit mengembangkan proses yang melibatkan berbagai profesional dan sudut pandang dalam proses pengembangannya. Proses tersebut mencakup pemberian informasi secara jelas dan lengkap mengenai kondisi pasien, konsekuensi dari keputusan yang diambil, serta pilihan atau alternatif lain yang dapat di jadikan pertimbangan. Selain itu, proses tersebut mengidentifikasi garis akuntabilitas serta bagaimana proses tersebut dapat di integrasikan di dalam rekam medis pasien.

\hypertarget{elemen-penilaian-hpk-2.1}{%
\subsubsection*{24. Elemen Penilaian HPK 2.1}\label{elemen-penilaian-hpk-2.1}}
\addcontentsline{toc}{subsubsection}{24. Elemen Penilaian HPK 2.1}

\begin{enumerate}
\def\labelenumi{\alph{enumi}.}
\tightlist
\item
  Rumah sakit menerapkan proses mengenai pemberian pelayanan resusitasi dan penghentian terapi penunjang kehidupan untuk pasien.
\item
  Rumah sakit memberi informasi kepada pasien dan keluarga mengenai hak mereka untuk menolak atau menghentikan terapi, konsekuensi dari keputusan yang dibuat serta terapi dan alternatif lain yang dapat dijadikan pilihan.
\end{enumerate}

\hypertarget{standar-hpk-2.2}{%
\subsubsection*{25. Standar HPK 2.2}\label{standar-hpk-2.2}}
\addcontentsline{toc}{subsubsection}{25. Standar HPK 2.2}

Rumah sakit mendukung hak pasien untuk mendapat pengkajian dan tata laksana nyeri serta perawatan yang penuh kasih menjelang akhir hayatnya.

\hypertarget{maksud-dan-tujuan-hpk-2.2}{%
\subsubsection*{26. Maksud dan Tujuan HPK 2.2}\label{maksud-dan-tujuan-hpk-2.2}}
\addcontentsline{toc}{subsubsection}{26. Maksud dan Tujuan HPK 2.2}

Nyeri adalah hal yang sering dialami pasien di dalam proses perawatan. Pasien merespons rasa nyeri sesuai dengan nilai, tradisi, budaya serta agama yang dianut. Nyeri yang tidak dapat diatasi dapat memiliki efek fisiologis yang negatif. Oleh karena itu, pasien perlu didukung dan diberi edukasi agar melaporkan nyeri yang mereka rasakan.

Menjelang akhir hayat, pasien memiliki kebutuhan khas yang juga dapat dipengaruhi oleh tradisi budaya dan agama. Perhatian terhadap kenyamanan dan martabat pasien memandu semua aspek perawatan di akhir hayat mereka. Untuk memberikan perawatan yang terbaik pada pasien yang sedang memasuki fase menjelang akhir hayat, semua staf harus rumah sakit menyadari kebutuhan yang unik dan spesifik dari seorang pasien di akhir hayatnya.

Kebutuhan-kebutuhan unik tersebut meliputi tata laksana terhadap keluhan utama dan keluhan tambahan; tata laksana nyeri; tanggapan terhadap kekhawatiran psikologis, sosial, emosional, agama, dan kultural pasien serta keluarganya serta keterlibatan dalam keputusan perawatan. Proses perawatan yang diberikan rumah sakit harus menjunjung tinggi dan mencerminkan hak dari semua pasien untuk mendapatkan pengkajian dan tata laksana nyeri serta pengkajian dan pengelolaan kebutuhan pasien yang unik dan spesifik di akhir hayatnya.

\hypertarget{elemen-penilaian-hpk-2.2}{%
\subsubsection*{27. Elemen Penilaian HPK 2.2}\label{elemen-penilaian-hpk-2.2}}
\addcontentsline{toc}{subsubsection}{27. Elemen Penilaian HPK 2.2}

\begin{enumerate}
\def\labelenumi{\alph{enumi}.}
\tightlist
\item
  Rumah sakit menerapkan proses untuk menghargai dan mendukung hak pasien mendapatkan pengkajian dan pengelolaan nyeri.
\item
  Rumah sakit menerapkan proses untuk menghargai dan mendukung hak pasien untuk mendapatkan pengkajian dan pengelolaan terhadap kebutuhan pasien menjelang akhir hayat.
\end{enumerate}

\hypertarget{standar-hpk-3}{%
\subsubsection*{28. Standar HPK 3}\label{standar-hpk-3}}
\addcontentsline{toc}{subsubsection}{28. Standar HPK 3}

Rumah sakit memberitahu pasien dan keluarganya mengenai proses untuk menerima dan menanggapi keluhan, tindakan rumah sakit bila terdapat konflik/perbedaan pendapat di dalam asuhan pasien, serta hak pasien untuk berperan dalam semua proses ini.

\hypertarget{maksud-dan-tujuan-hpk-3}{%
\subsubsection*{29. Maksud dan Tujuan HPK 3}\label{maksud-dan-tujuan-hpk-3}}
\addcontentsline{toc}{subsubsection}{29. Maksud dan Tujuan HPK 3}

Pasien memiliki hak untuk menyampaikan keluhan tentang asuhan mereka dan keluhan tersebut harus ditanggapi dan diselesaikan. Di samping itu, keputusan terkait perawatan kadang kala menimbulkan pertanyaan, konflik atau dilema lain bagi rumah sakit, pasien dan keluarga atau pengambil keputusan lain. Dilema ini mungkin timbul sejak pasien mengakses pelayanan, selama menjalani masa perawatan, dan pada proses pemulangan. Rumah sakit menetapkan penanggung jawab dan proses untuk menyelesaikan keluhan tersebut.

Rumah sakit mengidentifikasi kebijakan dan prosedur bagi mereka yang perlu dilibatkan dalam menyelesaikan keluhan dan bagaimana pasien dan keluarganya dapat ikut berperan serta.

\hypertarget{elemen-penilaian-hpk-3}{%
\subsubsection*{30. Elemen Penilaian HPK 3}\label{elemen-penilaian-hpk-3}}
\addcontentsline{toc}{subsubsection}{30. Elemen Penilaian HPK 3}

\begin{enumerate}
\def\labelenumi{\alph{enumi}.}
\tightlist
\item
  Pasien diberikan informasi mengenai proses untuk menyampaikan keluhan dan proses yang harus dilakukan pada saat terjadi konflik/perbedaan pendapat pada proses perawatan.
\item
  Keluhan, konflik, dan perbedaan pendapat tersebut dikaji dan diselesaikan oleh unit/petugas yang bertanggung jawab melalui sebuah alur/proses spesifik.
\item
  Pasien dan keluarga berpartisipasi dalam proses penyelesaian keluhan, konflik, dan perbedaan pendapat.
\end{enumerate}

\hypertarget{b.-permintaan-persetujuan-pasien}{%
\subsection*{b. Permintaan persetujuan pasien}\label{b.-permintaan-persetujuan-pasien}}
\addcontentsline{toc}{subsection}{b. Permintaan persetujuan pasien}

\hypertarget{standar-hpk-4}{%
\subsubsection*{1. Standar HPK 4}\label{standar-hpk-4}}
\addcontentsline{toc}{subsubsection}{1. Standar HPK 4}

Rumah sakit menetapkan batasan yang jelas untuk persetujuan umum yang diperoleh pasien pada saat akan menjalani rawat inap atau didaftarkan pertama kalinya sebagai pasien rawat jalan.

\hypertarget{maksud-dan-tujuan-hpk-4}{%
\subsubsection*{2. Maksud dan Tujuan HPK 4}\label{maksud-dan-tujuan-hpk-4}}
\addcontentsline{toc}{subsubsection}{2. Maksud dan Tujuan HPK 4}

Rumah sakit meminta persetujuan umum untuk pengobatan ketika pasien di terima rawat inap di rumah sakit atau ketika pasien didaftarkan untuk pertama kalinya sebagai pasien rawat jalan. Pada saat persetujuan umum itu diperoleh, pasien telah diberi informasi mengenai lingkup persetujuan umum tersebut. Selanjutnya, rumah sakit menentukan bagaimana persetujuan umum didokumentasikan dalam rekam medis pasien.

Selain general consent (persetujuan umum), semua pasien diberikan informasi mengenai pemeriksaan, tindakan dan pengobatan di mana informed consent (persetujuan tindakan) terpisah akan dibuat. Selain itu, pasien juga harus menerima informasi mengenai kemungkinan adanya peserta didik, seperti peserta didik perawat, peserta didik fisioterapi, mahasiswa kedokteran, dokter yang sedang menjalani pendidikan spesialis/trainee/fellowship, serta peserta didik lainnyayang terlibat dalam proses asuhan.

\hypertarget{elemen-penilaian-hpk-4}{%
\subsubsection*{3. Elemen Penilaian HPK 4}\label{elemen-penilaian-hpk-4}}
\addcontentsline{toc}{subsubsection}{3. Elemen Penilaian HPK 4}

\begin{enumerate}
\def\labelenumi{\alph{enumi}.}
\tightlist
\item
  Rumah sakit menerapkan proses bagaimana persetujuan umum didokumentasikan dalam rekam medis pasien.
\item
  Pasien dan keluarga diberikan informasi mengenai pemeriksaan, tindakan dan pengobatan yang memerlukan informed consent.
\item
  Pasien menerima informasi mengenai kemungkinan keterlibatan peserta didik, mahasiswa, residen traine dan fellow yang berpartisipasi dalam proses perawatan.
\end{enumerate}

\hypertarget{standar-hpk-4.1}{%
\subsubsection*{4. Standar HPK 4.1}\label{standar-hpk-4.1}}
\addcontentsline{toc}{subsubsection}{4. Standar HPK 4.1}

Persetujuan tindakan (informed consent) pasien diperoleh melalui cara yang telah ditetapkan rumah sakit dan dilaksanakan oleh petugas terlatih dengan cara dan bahasa yang mudah dipahami pasien.

\hypertarget{maksud-dan-tujuan-hpk-4.1}{%
\subsubsection*{5. Maksud dan Tujuan HPK 4.1}\label{maksud-dan-tujuan-hpk-4.1}}
\addcontentsline{toc}{subsubsection}{5. Maksud dan Tujuan HPK 4.1}

Salah satu proses penting di mana pasien dapat terlibat dalam pengambilan keputusan tentang perawatan mereka adalah dengan memberikan informed consent. Untuk memberikan persetujuan ini, pasien harus di informasikan terlebih dahulu mengenai faktor-faktor yang terkait dengan rencana perawatan yang dibutuhkan untuk mengambil keputusan. Proses persetujuan harus didefinisikan secara jelas oleh rumah sakit dalam kebijakan dan prosedur sesuai perundang-udangan yang berlaku.

Pasien dan keluarga diberikan informasi mengenai pemeriksaan, tindakan, dan pengobatan mana yang memerlukan persetujuan dan bagaimana mereka dapat memberikan persetujuan. Edukasi diberikan oleh staf rumah sakit yang kompeten dan merupakian bagian dari proses untuk mendapatkan informed consent (sebagai contoh, untuk pembedahan dan anestesi).

Jika perawatan yang direncanakan meliputi prosedur pembedahan atau tindakan invasif, anestesi, sedasi, penggunaan darah dan produk darah, perawatan atau tindakan berisiko tinggi, maka diperlukan persetujuan tindakan secara terpisah. Rumah sakit mengidentifikasi perawatan dan prosedur berisiko tinggiatau prosedur dan perawatan lainnya yang membutuhkan persetujuan. Rumah sakit membuat daftar perawatan dan prosedur ini serta mengedukasi petugas untuk memastikan bahwa proses untuk memperoleh persetujuan itu harus diterapkan secara konsisten. Daftar tersebut dikembangkan bersama- sama oleh para dokter dan orang lain yang memberikan perawatan atau melakukan tindakan. Daftar ini meliputi semua tindakan dan perawatan yang disiapkan bagi pasien rawat jalan dan pasien rawat inap.

\hypertarget{elemen-penilaian-hpk-4.1}{%
\subsubsection*{6. Elemen Penilaian HPK 4.1}\label{elemen-penilaian-hpk-4.1}}
\addcontentsline{toc}{subsubsection}{6. Elemen Penilaian HPK 4.1}

\begin{enumerate}
\def\labelenumi{\alph{enumi}.}
\tightlist
\item
  Rumah sakit menerapkan proses bagi pasien untuk mendapatkan informed consent.
\item
  Pemberian informed consent dilakukan oleh staf yang kompeten dan diberikan dengan cara dan bahasa yang mudah dipahami pasien.
\item
  Rumah sakit memiliki daftar tindakan invasif, pemeriksaan dan terapi tambahan yang memerlukaninformed consent.
\end{enumerate}

\hypertarget{standar-hpk-4.2}{%
\subsubsection*{7. Standar HPK 4.2}\label{standar-hpk-4.2}}
\addcontentsline{toc}{subsubsection}{7. Standar HPK 4.2}

Rumah sakit menerapkan proses untuk pemberian persetujuan oleh orang lain, sesuai dengan peraturan perundangan yang berlaku.

\hypertarget{maksud-dan-tujuan-hpk-4.2}{%
\subsubsection*{8. Maksud dan Tujuan HPK 4.2}\label{maksud-dan-tujuan-hpk-4.2}}
\addcontentsline{toc}{subsubsection}{8. Maksud dan Tujuan HPK 4.2}

Ada kalanya terdapat kondisi dimana orang lain selain pasien (baik sendiri maupun bersama pasien) ikut terlibat dalam keputusan mengenai perawatan pasien dalam proses pemberian informed consent untuk perawatan. Hal ini terutama berlaku ketika pasien tidak memiliki kemampuan mental atau fisik untuk mengambil keputusan tentang perawatannya sendiri, ketika latar belakang budaya atau kebiasaan mengharuskan orang lain yang mengambil keputusan tentang perawatan atau ketika pasien masih kanak-kanak. Ketika pasien tidak dapat membuat keputusan tentang perawatannya, maka ditentukan perwakilan untuk mengambil keputusan tersebut. Ketika ada orang lain selain pasien itu yang memberi persetujuan, nama individu itu dicatat dalam rekam medis pasien.

\hypertarget{elemen-penilaian-hpk-4.2}{%
\subsubsection*{9. Elemen Penilaian HPK 4.2}\label{elemen-penilaian-hpk-4.2}}
\addcontentsline{toc}{subsubsection}{9. Elemen Penilaian HPK 4.2}

\begin{enumerate}
\def\labelenumi{\alph{enumi}.}
\tightlist
\item
  Rumah sakit menerapkan proses untuk pemberian informed consent oleh orang lain selain pasien sesuai peraturan perundangan yang berlaku.
\item
  Rekam medis pasien mencantumkan (satu atau lebih) nama individu yang menyatakan persetujuan.
\end{enumerate}

\hypertarget{pengkajian-pasien-pp}{%
\section*{3. Pengkajian Pasien (PP)}\label{pengkajian-pasien-pp}}
\addcontentsline{toc}{section}{3. Pengkajian Pasien (PP)}

\textbf{Gambaran Umum}

Tujuan dari pengkajian adalah untuk menentukan perawatan, pengobatan dan pelayanan yang akan memenuhi kebutuhan awal dan kebutuhan berkelanjutan pasien. Pengkajian pasien merupakan proses yang berkelanjutan dan dinamis yang berlangsung di layanan rawat jalan serta rawat inap. Pengkajian pasien terdiri atas tiga proses utama:

\begin{enumerate}
\def\labelenumi{\alph{enumi}.}
\tightlist
\item
  Mengumpulkan informasi dan data terkait keadaan fisik, psikologis, status sosial, dan riwayat kesehatan pasien.
\item
  Menganalisis data dan informasi, termasuk hasil pemeriksaan laboratorium, pencitraan diagnostik, dan pemantauan fisiologis, untuk mengidentifikasi kebutuhan pasien akan layanan kesehatan.
\item
  Membuat rencana perawatan untuk memenuhi kebutuhan pasien yang telah teridentifikasi.
\end{enumerate}

Pengkajian pasien yang efektif akan menghasilkan keputusan tentang kebutuhan asuhan, tata laksana pasien yang harus segera dilakukan dan pengobatan berkelanjutan untuk emergensi atau elektif/terencana, bahkan ketika kondisi pasien berubah.

Asuhan pasien di rumah sakit diberikan dan dilaksanakan berdasarkan konsep pelayanan berfokus pada pasien (Patient/Person Centered Care) Pola ini dipayungi oleh konsep WHO dalam Conceptual framework integrated people-centred health services. Penerapan konsep pelayanan berfokus pada pasien adalah dalam bentuk Asuhan Pasien Terintegrasi yang bersifat integrasi horizontal dan vertikal dengan elemen:

\begin{enumerate}
\def\labelenumi{\alph{enumi}.}
\tightlist
\item
  Dokter penanggung jawab pelayanan (DPJP) sebagai ketua tim asuhan/Clinical Leader;
\item
  Profesional Pemberi Asuhan bekerja sebagai tim intra dan interdisiplin dengan kolaborasi interprofesional, dibantu antara lain dengan Panduan Praktik Klinis (PPK), Panduan Asuhan PPA lainnya, Alur Klinis/Clinical Pathway terintegrasi, Algoritma, Protokol, Prosedur, Standing Order dan CPPT (Catatan Perkembangan Pasien Terintegrasi);
\item
  Manajer Pelayanan Pasien/Case Manager; dan
\item
  Keterlibatan dan pemberdayaan pasien dan keluarga.
\end{enumerate}

Pengkajian ulang harus dilakukan selama asuhan, pengobatan dan pelayanan untuk mengidentifikasi kebutuhan pasien. Pengkajian ulang adalah penting untuk memahami respons pasien terhadap pemberian asuhan, pengobatan dan pelayanan, serta juga penting untuk menentukan apakah keputusan asuhan memadai dan efektif. Proses-proses ini paling efektif dilaksanakan bila berbagai profesional kesehatan yang bertanggung jawab atas pasien bekerja sama.

Standar Pengkajian Pasien ini berfokus kepada:

\begin{enumerate}
\def\labelenumi{\alph{enumi}.}
\tightlist
\item
  Pengkajian awal pasien;
\item
  Pengkajian ulang pasien;
\item
  Pelayanan laboratorium dan pelayanan darah; dan
\item
  Pelayanan radiologi klinik.
\end{enumerate}

\hypertarget{a.-pengkajian-awal-pasien}{%
\subsection*{a. Pengkajian awal pasien}\label{a.-pengkajian-awal-pasien}}
\addcontentsline{toc}{subsection}{a. Pengkajian awal pasien}

\hypertarget{standar-pp-1}{%
\subsubsection*{1. Standar PP 1}\label{standar-pp-1}}
\addcontentsline{toc}{subsubsection}{1. Standar PP 1}

Semua pasien yang dirawat di rumah sakit diidentifikasi kebutuhan perawatan kesehatannya melalui suatu proses pengkajian yang telah ditetapkan oleh rumah sakit.

\hypertarget{standar-pp-1.1}{%
\subsubsection*{2. Standar PP 1.1}\label{standar-pp-1.1}}
\addcontentsline{toc}{subsubsection}{2. Standar PP 1.1}

Kebutuhan medis dan keperawatan pasien diidentifikasi berdasarkan pengkajian awal.

\hypertarget{standar-pp-1.2}{%
\subsubsection*{3. Standar PP 1.2}\label{standar-pp-1.2}}
\addcontentsline{toc}{subsubsection}{3. Standar PP 1.2}

Pasien dilakukan skrining risiko nutrisi, skrining nyeri, kebutuhan fungsional termasuk risiko jatuh dan kebutuhan khusus lainnya

\hypertarget{maksud-dan-tujuan-pp1-pp-1.1-dan-pp-1.2}{%
\subsubsection*{4. Maksud dan Tujuan PP1, PP 1.1 dan PP 1.2}\label{maksud-dan-tujuan-pp1-pp-1.1-dan-pp-1.2}}
\addcontentsline{toc}{subsubsection}{4. Maksud dan Tujuan PP1, PP 1.1 dan PP 1.2}

Proses pengkajian pasien yang efektif menghasilkan keputusan tentang kebutuhan pasien untuk mendapatkan tata laksana segera dan berkesinambungan untuk pelayanan gawat darurat, elektif atau terencana, bahkan ketika kondisi pasien mengalami perubahan. Pengkajian pasien adalah sebuah proses berkesinambungan dan dinamis yang dilakukan di unit gawat darurat, rawat inap dan rawat jalan serta unit lainnya. Pengkajian pasien terdiri dari tiga proses primer:

\begin{enumerate}
\def\labelenumi{\alph{enumi}.}
\tightlist
\item
  Pengumpulan informasi dan data mengenai kondisi fisik, psikologis, dan status sosial serta riwayat kesehatan pasien sebelumnya.
\item
  Analisis data dan informasi, termasuk hasil pemeriksaan laboratorium dan uji diagnostik pencitraan, untuk mengidentifikasi kebutuhan perawatan pasien.
\item
  Pengembangan rencana perawatan pasien untuk memenuhi kebutuhan yang telah diidentifikasi.
\end{enumerate}

Pengkajian disesuaikan dengan kebutuhan pasien, sebagai contoh, rawat inap atau rawat jalan. Bagaimana pengkajian ini dilakukan dan informasi apa yang perlu dikumpulkan serta didokumentasikan ditetapkan dalam kebijakan dan prosedur rumah sakit.
Isi minimal pengkajian awal antara lain:

\begin{enumerate}
\def\labelenumi{\alph{enumi}.}
\tightlist
\item
  Keluhan saat ini
\item
  Status fisik;
\item
  Psiko-sosio-spiritual;
\item
  Ekonomi;
\item
  Riwayat kesehatan pasien;
\item
  Riwayat alergi;
\item
  Riwayat penggunaan obat;
\item
  Pengkajian nyeri;
\item
  Risiko jatuh;
\item
  Pengkajian fungsional;
\item
  Risiko nutrisional;
\item
  Kebutuhan edukasi; dan
\item
  Perencanaan pemulangan pasien (Discharge Planning). Pada kelompok pasien tertentu, misalnya dengan risiko jatuh, nyeri dan status nutrisi maka dilakukan skrining sebagai bagian dari pengkajian awal, kemudian dilanjutkan dengan pengkajian lanjutan.
\end{enumerate}

Agar pengkajian kebutuhan pasien dilakukan secara konsisten, rumah sakit harus mendefinisikan dalam kebijakan, isi minimum dari pengkajian yang dilakukan oleh para dokter, perawat, dan disiplin klinis lainnya. Pengkajian dilakukan oleh setiap disiplin dalam ruang lingkup praktiknya, perizinan, perundangundangan. Hanya PPA yang kompeten dan di izinkan oleh rumah sakit yang akan melakukan pengkajian.

Rumah sakit mendefinisikan elemen-elemen yang akan digunakan pada seluruh pengkajian dan mendefinisikan perbedaan-perbedaan yang ada terutama dalam ruang lingkup kedokteran umum dan layanan spesialis. Pengkajian yang didefinisikan dalam kebijakan dapat dilengkapi oleh lebih dari satu individu yang kompeten dan dilakukan pada beberapa waktu yang berbeda. Semua pengkajian tersebut harus sudah terisi lengkap dan memiliki informasi terkini (kurang dari atau sama dengan 30 (tiga puluh) hari) pada saat tata laksana dimulai.

\hypertarget{elemen-penilaian-pp-1}{%
\subsubsection*{5. Elemen Penilaian PP 1}\label{elemen-penilaian-pp-1}}
\addcontentsline{toc}{subsubsection}{5. Elemen Penilaian PP 1}

\begin{enumerate}
\def\labelenumi{\alph{enumi}.}
\tightlist
\item
  Rumah sakit menetapkan regulasi tentang pengkajian awal dan pengkajian ulang medis dan keperawatan di unit gawat darurat, rawat inap dan rawat jalan.
\item
  Rumah sakit menetapkan isi minimal pengkajian awal meliputi poin a) -- l) pada maksud dan tujuan.
\item
  Hanya PPA yang kompeten, diperbolehkan untuk melakukan pengkajian sesuai dengan ketentuan rumah sakit.
\item
  Perencanaanan pulang yang mencakup identifikasi kebutuhan khusus dan rencana untuk memenuhi kebutuhan tersebut, disusun sejak pengkajian awal.
\end{enumerate}

\hypertarget{elemen-penilaian-pp-1.1}{%
\subsubsection*{6. Elemen Penilaian PP 1.1}\label{elemen-penilaian-pp-1.1}}
\addcontentsline{toc}{subsubsection}{6. Elemen Penilaian PP 1.1}

\begin{enumerate}
\def\labelenumi{\alph{enumi}.}
\tightlist
\item
  Pengkajian awal medis dan keperawatandilaksanakan dan didokumentasikan dalam kurun waktu 24 jam pertama sejak pasien masuk rawat inap, atau lebih awal bila diperlukan sesuai dengan kondisi pasien.
\item
  Pengkajian awal medis menghasilkan diagnosis medis yang mencakup kondisi utama dan kondisi lainnya yang membutuhkan tata laksana dan pemantauan.
\item
  Pengkajian awal keperawatan menghasilkan diagnosis keperawatan untuk menentukan kebutuhan asuhan keperawatan, intervensi atau pemantauan pasien yang spesifik.
\item
  Sebelum pembedahan pada kondisi mendesak, minimal terdapat catatan singkat dan diagnosis praoperasi yang didokumentasikan di dalam rekam medik.
\item
  Pengkajian medis yang dilakukan sebelum masuk rawat inap atau sebelum pasien menjalani prosedur di layanan rawat jalan rumah sakit harus dilakukan dalam waktu kurang atau sama dengan 30 (tiga puluh) hari sebelumnya. Jika lebih dari 30 (tiga puluh) hari, maka harus dilakukan pengkajian ulang.
\item
  Hasil dari seluruh pengkajian yang dikerjakan di luar rumah sakit ditinjau dan/atau diverifikasi pada saat masuk rawat inap atau sebelum tindakan di unit rawat jalan.
\end{enumerate}

\hypertarget{elemen-penilaian-pp-1.2}{%
\subsubsection*{7. Elemen Penilaian PP 1.2}\label{elemen-penilaian-pp-1.2}}
\addcontentsline{toc}{subsubsection}{7. Elemen Penilaian PP 1.2}

\begin{enumerate}
\def\labelenumi{\alph{enumi}.}
\tightlist
\item
  Rumah sakit menetapkan kriteria risiko nutrisional yang dikembangkan bersama staf yang kompeten dan berwenang.
\item
  Pasien diskrining untuk risiko nutrisi sebagai bagian dari pengkajian awal.
\item
  Pasien dengan risiko nutrisional dilanjutkan dengan pengkajian gizi.
\item
  Pasien diskrining untuk kebutuhan fungsional termasuk risiko jatuh.
\end{enumerate}

\hypertarget{standar-pp-1.3}{%
\subsubsection*{8. Standar PP 1.3}\label{standar-pp-1.3}}
\addcontentsline{toc}{subsubsection}{8. Standar PP 1.3}

Rumah sakit melakukan pengkajian awal yang telah dimodifikasi untuk populasi khusus yang dirawat di rumah sakit.

\hypertarget{maksud-dan-tujuan-pp-1.3}{%
\subsubsection*{9. Maksud dan Tujuan PP 1.3}\label{maksud-dan-tujuan-pp-1.3}}
\addcontentsline{toc}{subsubsection}{9. Maksud dan Tujuan PP 1.3}

Pengkajian tambahan untuk pasien tertentu atau untuk populasi pasien khusus mengharuskan proses pengkajian tambahan sesuai dengan kebutuhan populasi pasien tertentu. Setiap rumah sakit menentukan kelompok populasi pasien khusus dan menyesuaikan proses pengkajian untuk memenuhi kebutuhan khusus mereka.
Pengkajian tambahan dilakukan antara lain namun tidak terbatas untuk:

\begin{enumerate}
\def\labelenumi{\alph{enumi}.}
\item
  Neonatus.
\item
  Anak.
\item
  Remaja.
\item
  Obsteri / maternitas.
\item
  Geriatri.
\item
  Sakit terminal / menghadapi kematian.
\item
  Pasien dengan nyeri kronik atau nyeri (intense).
\item
  \begin{enumerate}
  \def\labelenumii{\alph{enumii})}
  \setcounter{enumii}{7}
  \tightlist
  \item
    Pasien dengan gangguan emosional atau pasien psikiatris.
  \end{enumerate}
\item
  Pasien kecanduan obat terlarang atau alkohol.
\item
  Korban kekerasan atau kesewenangan.
\item
  Pasien dengan penyakit menular atau infeksius.
\item
  Pasien yang menerima kemoterapi atau terapi radiasi.
\item
  Pasien dengan sistem imunologi terganggu.
\end{enumerate}

Tambahan pengkajian terhadap pasien ini memperhatikan kebutuhan dan kondisi mereka berdasarkan budaya dan nilai yang dianut pasien. Proses pengkajian disesuaikan dengan peraturan perundangan dan standar profesional.

\hypertarget{elemen-penilaian-pp-1.3}{%
\subsubsection*{10. Elemen Penilaian PP 1.3}\label{elemen-penilaian-pp-1.3}}
\addcontentsline{toc}{subsubsection}{10. Elemen Penilaian PP 1.3}

\begin{enumerate}
\def\labelenumi{\alph{enumi}.}
\tightlist
\item
  Rumah sakit menetapkan jenis populasi khusus yang akan dilakukan pengkajian meliputi poin a) - m) pada maksud dan tujuan.
\item
  Rumah sakit telah melaksanakan pengkajian tambahan terhadap populasi pasien khusus sesuai ketentuan rumah sakit.
\end{enumerate}

\hypertarget{b.-pengkajian-ulang-pasien}{%
\subsection*{b. Pengkajian ulang pasien}\label{b.-pengkajian-ulang-pasien}}
\addcontentsline{toc}{subsection}{b. Pengkajian ulang pasien}

\hypertarget{standard-pp-2}{%
\subsubsection*{1. Standard PP 2}\label{standard-pp-2}}
\addcontentsline{toc}{subsubsection}{1. Standard PP 2}

Rumah sakit melakukan pengkajian ulang bagi semua pasien dengan interval waktu yang ditentukan untuk kemudian dibuat rencana asuhan lanjutan.

\hypertarget{maksud-dan-tujuan-pp-2}{%
\subsubsection*{2. Maksud dan Tujuan PP 2}\label{maksud-dan-tujuan-pp-2}}
\addcontentsline{toc}{subsubsection}{2. Maksud dan Tujuan PP 2}

Pengkajian ulang dilakukan oleh semua PPA untuk menilai apakah asuhan yang diberikan telah berjalan dengan efektif. Pengkajian ulang dilakukan dalam interval waktu yang didasarkan atas kebutuhan dan rencana asuhan, dan digunakan sebagai dasar rencana pulang pasien sesuai dengan regulasi rumah sakit. Hasil pengkajian ulang dicatat di rekam medik pasien/CPPT sebagai informasi untuk di gunakan oleh semua PPA.

Pengkajian ulang oleh DPJP dibuat dibuat berdasarkan asuhan pasien sebelumnya. DPJP melakukan pengkajian terhadap pasien sekurang-kurangnya setiap hari, termasuk di akhir minggu/hari libur, dan jika ada perubahan kondisi pasien. Perawat melakukan pengkajian ulang minimal satu kali pershift atau sesuai perkembangan pasien, dan setiap hari DPJP akan mengkoordinasi dan melakukan verifikasi ulang perawat untuk asuhan keperawatan selanjutnya.

Penilaian ulang dilakukan dan hasilnya dimasukkan ke dalam rekam medis pasien:

\begin{enumerate}
\def\labelenumi{\alph{enumi}.}
\tightlist
\item
  Secara berkala selama perawatan (misalnya, staf perawat secara berkala mencatat tanda-tanda vital, nyeri, penilaian dan suara paru-paru dan jantung, sesuai kebutuhan berdasarkan kondisi pasien);
\item
  Setiap hari oleh dokter untuk pasien perawatan akut;
\item
  Dalam menanggapi perubahan signifikan dalam kondisi pasien; (Juga lihat PP 3.2)
\item
  Jika diagnosis pasien telah berubah dan kebutuhan perawatan memerlukan perencanaan yang direvisi; dan
\item
  Untuk menentukan apakah pengobatan dan perawatan lain telah berhasil dan pasien dapat dipindahkan atau dipulangkan.
\end{enumerate}

Temuan pada pengkajian digunakan sepanjang proses pelayanan untuk mengevaluasi kemajuan pasien dan untuk memahami kebutuhan untuk pengkajian ulang. Oleh karena itu pengkajian medis, keperawatan dan PPA lain dicatat di rekam medik untuk digunakan oleh semua PPA yang memberikan asuhan ke pasien.

\hypertarget{elemen-penilaian-pp-2}{%
\subsubsection*{3. Elemen Penilaian PP 2}\label{elemen-penilaian-pp-2}}
\addcontentsline{toc}{subsubsection}{3. Elemen Penilaian PP 2}

\begin{enumerate}
\def\labelenumi{\alph{enumi}.}
\tightlist
\item
  Rumah sakit melaksanakan pengkajian ulang oleh DPJP, perawat dan PPA lainnya untuk menentukan rencana asuhan lanjutan.
\item
  Terdapat bukti pelaksanaan pengkajian ulang medis dilaksanakan minimal satu kali sehari, termasuk akhir minggu/libur untuk pasien akut.
\item
  Terdapat bukti pelaksanaan pengkajian ulang oleh perawat minimal satu kali per shift atau sesuai dengan perubahan kondisi pasien.
\item
  Terdapat bukti pengkajian ulang oleh PPA lainnya dilaksanakan dengan interval sesuai regulasi rumah sakit.
\end{enumerate}

\hypertarget{c.-pelayanan-laboratorium-dan-pelayanan-darah}{%
\subsection*{c.~Pelayanan laboratorium dan pelayanan darah}\label{c.-pelayanan-laboratorium-dan-pelayanan-darah}}
\addcontentsline{toc}{subsection}{c.~Pelayanan laboratorium dan pelayanan darah}

\hypertarget{standar-pp-3}{%
\subsubsection*{1. Standar PP 3}\label{standar-pp-3}}
\addcontentsline{toc}{subsubsection}{1. Standar PP 3}

Pelayanan laboratorium tersedia untuk memenuhi kebutuhan pasien sesuai peraturan perundangan.

\hypertarget{maksud-dan-tujuan-pp-3}{%
\subsubsection*{2. Maksud dan Tujuan PP 3}\label{maksud-dan-tujuan-pp-3}}
\addcontentsline{toc}{subsubsection}{2. Maksud dan Tujuan PP 3}

Rumah Sakit mempunyai sistem untuk menyediakan pelayanan laboratorium, meliputi pelayanan patologi klinis, dapat juga tersedia patologi anatomi dan pelayanan laboratorium lainnya, yang dibutuhkan populasi pasiennya, dan PPA. Organisasi pelayanan laboratorium yang di bentuk dan diselenggarakan sesuai peraturan perundangan Di Rumah Sakit dapat terbentuk pelayanan laboratorium utama (induk), dan juga pelayanan laboratorium lain, misalnya laboratorium Patologi Anatomi, laboratorium Mikrobiologi maka harus diatur secara organisatoris pelayanan laboratorium terintegrasi, dengan pengaturan tentang kepala pelayanan laboratorium terintegrasi yang membawahi semua jenis pelayanan laboratorium di Rumah Sakit.

Salah satu pelayanan laboratorium di ruang rawat (Point of Care Testing) yang dilakukan oleh perawat ruangan harus memenuhi persyaratan kredensial. Pelayanan laboratorium, tersedia 24 jam termasuk pelayanan darurat, diberikan di dalam rumah sakit dan rujukan sesuai dengan peraturan perundangan. Rumah sakit dapat juga menunjuk dan menghubungi para spesialis di bidang diagnostik khusus, seperti parasitologi, virologi, atau toksikologi. Jika diperlukan, rumah sakit dapat melakukan pemeriksaan rujukan dengan memilih sumber dari luar berdasarkan rekomendasi dari pimpinan laboratorum rumah sakit. Sumber dari luar tersebut dipilih oleh Rumah Sakit karena memenuhi peraturan perundangan dan mempunyai sertifikat mutu. Bila melakukan pemeriksaan rujukan keluar, harus melalui laboratorium Rumah Sakit.

\hypertarget{elemen-penilaian-pp-3}{%
\subsubsection*{3. Elemen Penilaian PP 3}\label{elemen-penilaian-pp-3}}
\addcontentsline{toc}{subsubsection}{3. Elemen Penilaian PP 3}

\begin{enumerate}
\def\labelenumi{\alph{enumi}.}
\tightlist
\item
  Rumah sakit menetapkan regulasi tentang pelayanan laboratorium di rumah sakit.
\item
  Pelayanan laboratorium buka 24 jam, 7 (tujuh) hari seminggu, sesuai dengan kebutuhan pasien.
\end{enumerate}

\hypertarget{standar-pp-3.1}{%
\subsubsection*{4. Standar PP 3.1}\label{standar-pp-3.1}}
\addcontentsline{toc}{subsubsection}{4. Standar PP 3.1}

Rumah sakit menetapkan bahwa seorang yang kompeten dan berwenang, bertanggung jawab mengelola pelayanan laboratorium.

\hypertarget{maksud-dan-tujuan-pp-3.1}{%
\subsubsection*{5. Maksud dan Tujuan PP 3.1}\label{maksud-dan-tujuan-pp-3.1}}
\addcontentsline{toc}{subsubsection}{5. Maksud dan Tujuan PP 3.1}

Pelayanan laboratorium berada dibawah pimpinan seorang yang kompeten dan memenuhi persyaratan peraturan perundang-undangan. Pimpinan laboratorium bertanggung jawab mengelola fasilitas dan pelayanan laboratorium, termasuk pemeriksaan Point-of-care testing (POCT), juga tanggung jawabnya dalam melaksanakan regulasi RS secara konsisten, seperti pelatihan, manajemen logistik dan sebagainya.

Tanggung jawab pimpinan laboratorium antara lain:

\begin{enumerate}
\def\labelenumi{\alph{enumi}.}
\tightlist
\item
  Menyusun dan evaluasi regulasi.
\item
  Pengawasan pelaksanaan administrasi.
\item
  Melaksanakan program kendali mutu (PMI dan PME) dan mengintegrasikan program mutu laboratorium dengan program Manajemen Fasilitas dan Keamanan serta program Pencegahan dan Pengendalian Infeksi di rumah sakit.
\item
  Melakukan pemantauan dan evaluasi semua jenis pelayanan laboratorium.
\item
  Mereview dan menindak lanjuti hasil pemeriksaan laboratorium rujukan.
\end{enumerate}

\hypertarget{elemen-penilaian-pp-3.1}{%
\subsubsection*{6. Elemen Penilaian PP 3.1}\label{elemen-penilaian-pp-3.1}}
\addcontentsline{toc}{subsubsection}{6. Elemen Penilaian PP 3.1}

\begin{enumerate}
\def\labelenumi{\alph{enumi}.}
\tightlist
\item
  Direktur rumah sakit menetapkan penanggung jawab laboratorium yang memiliki kompetensi sesuai ketentuan perundang-undangan.
\item
  Terdapat bukti pelaksanaan tanggung jawab pimpinan laboratorium sesuai poin a) - e) pada maksud dan tujuan.
\end{enumerate}

\hypertarget{standar-pp-3.2}{%
\subsubsection*{7. Standar PP 3.2}\label{standar-pp-3.2}}
\addcontentsline{toc}{subsubsection}{7. Standar PP 3.2}

Staf laboratorium mempunyai pendidikan, pelatihan, kualifikasi dan pengalaman yang dipersyaratkan untuk mengerjakan pemeriksaan.

\hypertarget{maksud-dan-tujuan-pp-3.2}{%
\subsubsection*{8. Maksud dan Tujuan PP 3.2}\label{maksud-dan-tujuan-pp-3.2}}
\addcontentsline{toc}{subsubsection}{8. Maksud dan Tujuan PP 3.2}

Syarat pendidikan, pelatihan, kualifikasi dan pengalaman ditetapkan rumah sakit bagi mereka yang memiliki kompetensi dan kewenangan diberi ijin mengerjakan pemeriksaan laboratorium, termasuk yang mengerjakan pemeriksaan di tempat tidur pasien (POCT). Interpretasi hasil pemeriksaan dilakukan oleh dokter yang kompeten dan berwenang. Pengawasan terhadap staf yang mengerjakan pemeriksaan diatur oleh regulasi RS. Staf pengawas dan staf pelaksana diberi orientasi tugas mereka. Staf pelaksana diberi tugas sesuai latar belakang pendidikan dan pengalaman. Unit kerja laboratorium menyusun dan melaksanakan pelatihan staf yang memungkinkan staf mampu melakukan tugas sesuai dengan uraian tugasnya.

\hypertarget{elemen-penilaian-pp-3.2}{%
\subsubsection{9. Elemen Penilaian PP 3.2}\label{elemen-penilaian-pp-3.2}}

\begin{enumerate}
\def\labelenumi{\alph{enumi}.}
\tightlist
\item
  Staf laboratorium yang membuat interpretasi telah memenuhi persyaratan kredensial.
\item
  Staf laboratorium dan staf lain yang melaksanakan pemeriksaan termasuk yang mengerjakan Point-of-care testing (POCT), memenuhi persyaratan kredensial.
\end{enumerate}

\hypertarget{standar-pp-3.3}{%
\subsubsection*{10. Standar PP 3.3}\label{standar-pp-3.3}}
\addcontentsline{toc}{subsubsection}{10. Standar PP 3.3}

Rumah Sakit menetapkan kerangka waktu penyelesaian pemeriksaan regular dan pemeriksaan segera (cito).

\hypertarget{maksud-dan-tujuan-pp-3.3}{%
\subsubsection*{11. Maksud dan Tujuan PP 3.3}\label{maksud-dan-tujuan-pp-3.3}}
\addcontentsline{toc}{subsubsection}{11. Maksud dan Tujuan PP 3.3}

Rumah sakit menetapkan kerangka waktu penyelesaian pemeriksaan laboratorium. Penyelesaian pemeriksaan laboratorium dilaporkan sesuai kebutuhan pasien. Hasil pemeriksaan segera (cito), antara lain dari unit gawat darurat, kamar operasi, unit intensif diberi perhatian khusus terkait kecepatan hasil pemeriksaan. Jika pemeriksaan dilakukan melalui kontrak (pihak ketiga) atau laboratorium rujukan, kerangka waktu melaporkan hasil pemeriksaan juga mengikuti ketentuan rumah sakit.

\hypertarget{elemen-penilaian-pp-3.3}{%
\subsubsection*{12. Elemen Penilaian PP 3.3}\label{elemen-penilaian-pp-3.3}}
\addcontentsline{toc}{subsubsection}{12. Elemen Penilaian PP 3.3}

\begin{enumerate}
\def\labelenumi{\alph{enumi}.}
\tightlist
\item
  Rumah sakit menetapkan dan menerapkan kerangka waktu penyelesaian pemeriksaan laboratorium regular dan cito.
\item
  Terdapat bukti pencatatan dan evaluasi waktu penyelesaian pemeriksaan laboratorium.
\item
  Terdapat bukti pencatatan dan evaluasi waktu penyelesaian pemeriksaan cito.
\item
  Terdapat bukti pencatatan dan evaluasi pelayanan laboratorium rujukan.
\end{enumerate}

\hypertarget{standar-pp-3.4}{%
\subsubsection*{13. Standar PP 3.4}\label{standar-pp-3.4}}
\addcontentsline{toc}{subsubsection}{13. Standar PP 3.4}

Rumah sakit memiliki prosedur pengelolaan semua reagensia esensial dan di evaluasi secara berkala pelaksaksanaannya.

\hypertarget{maksud-dan-tujuan-pp-3.4}{%
\subsubsection*{14. Maksud dan Tujuan PP 3.4}\label{maksud-dan-tujuan-pp-3.4}}
\addcontentsline{toc}{subsubsection}{14. Maksud dan Tujuan PP 3.4}

Rumah sakit menetapkan reagensia dan bahan-bahan lain yang selalu harus ada untuk pelayanan laboratorium bagi pasien. Suatu proses yang efektif untuk pemesanan atau menjamin ketersediaan reagensia esensial dan bahan lain yang diperlukan. Semua reagensia disimpan dan didistribusikan sesuai prosedur yang ditetapkan. Dilakukan audit secara periodik untuk semua reagensia esensial untuk memastikan akurasi dan presisi hasil pemeriksaan, antara lain untuk aspek penyimpanan, label, kadaluarsa dan fisik. Prosedur tertulis memastikan pemberian label secara lengkap dan akurat untuk reagensia dan larutan dan akurasi serta presisi dari hasil.

\hypertarget{elemen-penilaian-pp-3.4}{%
\subsubsection*{15. Elemen Penilaian PP 3.4}\label{elemen-penilaian-pp-3.4}}
\addcontentsline{toc}{subsubsection}{15. Elemen Penilaian PP 3.4}

\begin{enumerate}
\def\labelenumi{\alph{enumi}.}
\tightlist
\item
  Terdapat bukti pelaksanaan semua reagensia esensial disimpan dan diberi label, serta didistribusi sesuai prosedur dari pembuatnya atau instruksi pada kemasannya
\item
  Terdapat bukti pelaksanaan evaluasi/audit semua reagen.
\end{enumerate}

\hypertarget{standar-pp-3.5}{%
\subsubsection*{16. Standar PP 3.5}\label{standar-pp-3.5}}
\addcontentsline{toc}{subsubsection}{16. Standar PP 3.5}

Rumah sakit memiliki prosedur untuk cara pengambilan, pengumpulan, identifikasi, pengerjaan, pengiriman, penyimpanan, dan pembuangan spesimen.

\hypertarget{maksud-dan-tujuan-pp-3.5}{%
\subsubsection*{17. Maksud dan Tujuan PP 3.5}\label{maksud-dan-tujuan-pp-3.5}}
\addcontentsline{toc}{subsubsection}{17. Maksud dan Tujuan PP 3.5}

Prosedur pelayanan laboratorium meliputi minimal tapi tidak terbatas pada:

\begin{enumerate}
\def\labelenumi{\alph{enumi}.}
\tightlist
\item
  Permintaan pemeriksaan.
\item
  Pengambilan, pengumpulan dan identifikasi spesimen.
\item
  Pengiriman, pembuangan, penyimpanan dan pengawetan spesimen.
\item
  Penerimaan, penyimpanan, telusur spesimen (tracking).
\end{enumerate}

\hypertarget{elemen-penilaian-pp-3.5}{%
\subsubsection*{18. Elemen Penilaian PP 3.5}\label{elemen-penilaian-pp-3.5}}
\addcontentsline{toc}{subsubsection}{18. Elemen Penilaian PP 3.5}

\begin{enumerate}
\def\labelenumi{\alph{enumi}.}
\tightlist
\item
  Pengelolaan spesimen dilaksanakan sesuai poin a) - d) pada maksud dan tujuan.
\item
  Terdapat bukti pemantauan dan evaluasi terhadap pengelolaan spesimen.
\end{enumerate}

\hypertarget{standar-pp-3.6}{%
\subsubsection*{19. Standar PP 3.6}\label{standar-pp-3.6}}
\addcontentsline{toc}{subsubsection}{19. Standar PP 3.6}

Rumah sakit menetapkan nilai normal dan rentang nilai untuk interpretasi dan pelaporan hasil laboratorium klinis.

\hypertarget{maksud-dan-tujuan-pp-3.6}{%
\subsubsection*{20. Maksud dan Tujuan PP 3.6}\label{maksud-dan-tujuan-pp-3.6}}
\addcontentsline{toc}{subsubsection}{20. Maksud dan Tujuan PP 3.6}

Rumah sakit menetapkan rentang nilai normal/rujukan setiap jenis pemeriksaan. Rentang nilai dilampirkan di dalam laporan klinik, baik sebagai bagian dari pemeriksaan atau melampirkan daftar terkini, nilai ini yang ditetapkan pimpinan laboratorium. Jika pemeriksaan dilakukan oleh laboratorium rujukan, rentang nilai diberikan. Selalu harus dievaluasi dan direvisi apabila metode pemeriksaan berubah.

\hypertarget{elemen-penilaian-pp-3.6}{%
\subsubsection*{21. Elemen Penilaian PP 3.6}\label{elemen-penilaian-pp-3.6}}
\addcontentsline{toc}{subsubsection}{21. Elemen Penilaian PP 3.6}

\begin{enumerate}
\def\labelenumi{\alph{enumi}.}
\tightlist
\item
  Rumah sakit menetapkan dan mengevaluasi rentang nilai normal untuk interpretasi, pelaporan hasil laboratorium klinis.
\item
  Setiap hasil pemeriksaan laboratorium dilengkapi dengan rentang nilai normal.
\end{enumerate}

\hypertarget{standar-pp-3.7}{%
\subsubsection*{22. Standar PP 3.7}\label{standar-pp-3.7}}
\addcontentsline{toc}{subsubsection}{22. Standar PP 3.7}

Rumah sakit melaksanakan prosedur kendali mutu pelayanan laboratorium, di evaluasi dan dicatat sebagai dokumen.

\hypertarget{maksud-dan-tujuan-pp-3.7}{%
\subsubsection*{23. Maksud dan Tujuan PP 3.7}\label{maksud-dan-tujuan-pp-3.7}}
\addcontentsline{toc}{subsubsection}{23. Maksud dan Tujuan PP 3.7}

Kendali mutu yang baik sangat esensial bagi pelayanan laboratorium agar laboratorium dapat memberikan layanan prima. Program kendali mutu di laboratorium mencakup pemantapan mutu internal (PMI) dan pemantauan mutu eksternal (PME). Tahapan PMI praanalitik, analitik dan pascaanalitik yang memuat antara lain:

\begin{enumerate}
\def\labelenumi{\alph{enumi}.}
\tightlist
\item
  Validasi tes yang digunakan untuk tes akurasi, presisi, hasil rentang nilai;
\item
  Dilakukan surveilans hasil pemeriksaan oleh staf yang kompeten;
\item
  Reagensia di tes;
\item
  Koreksi cepat jika ditemukan kekurangan;
\item
  Dokumentasi hasil dan tindakan koreksi; dan
\item
  Pemantapan Mutu Eksternal.
\end{enumerate}

\hypertarget{elemen-penilaian-pp-3.7}{%
\subsubsection*{24. Elemen Penilaian PP 3.7}\label{elemen-penilaian-pp-3.7}}
\addcontentsline{toc}{subsubsection}{24. Elemen Penilaian PP 3.7}

\begin{enumerate}
\def\labelenumi{\alph{enumi}.}
\tightlist
\item
  Terdapat bukti bahwa unit laboratorium telah melakukan Pemantapan Mutu Internal (PMI) secara rutin yang meliputi poin a-e pada maksud dan tujuan.
\item
  Terdapat bukti bahwa unit laboratorium telah melakukan Pemantapan Mutu Eksternal (PME) secara rutin.
\end{enumerate}

\hypertarget{standar-pp-3.8}{%
\subsubsection*{25. Standar PP 3.8}\label{standar-pp-3.8}}
\addcontentsline{toc}{subsubsection}{25. Standar PP 3.8}

Rumah sakit bekerjasama dengan laboratorium rujukan yang terakreditasi.

\hypertarget{maksud-dan-tujuan-pp-3.8}{%
\subsubsection*{26. Maksud dan Tujuan PP 3.8}\label{maksud-dan-tujuan-pp-3.8}}
\addcontentsline{toc}{subsubsection}{26. Maksud dan Tujuan PP 3.8}

Untuk memastikan pelayanan yang aman dan bermutu rumah sakit memiliki perjanjian kerjasama dengan laboratorium rujukan. Perjanjian kerjasama ini bertujuan agar rumah sakit memastikan bahwa laboratorium rujukan telah memenhi persyaratan dan terakreditasi. Perjanjian kerjasama mencantumkan hal hal yang harus ditaati kedua belah pihak dan perjanjian dievaluasi secara berkala oleh pimpinan rumah sakit.

\hypertarget{elemen-penilaian-dari-pp-3.8}{%
\subsubsection*{27. Elemen Penilaian dari PP 3.8}\label{elemen-penilaian-dari-pp-3.8}}
\addcontentsline{toc}{subsubsection}{27. Elemen Penilaian dari PP 3.8}

\begin{enumerate}
\def\labelenumi{\alph{enumi}.}
\tightlist
\item
  Unit laboratorium memiliki bukti sertifikat akreditasi laboratorium rujukan yang masih berlaku.
\item
  Telah dilakukan pemantauan dan evaluasi kerjasama pelayanan kontrak sesuai dengan kesepakatan kedua belah pihak.
\end{enumerate}

\hypertarget{standar-pp-3.9}{%
\subsubsection*{28. Standar PP 3.9}\label{standar-pp-3.9}}
\addcontentsline{toc}{subsubsection}{28. Standar PP 3.9}

Rumah Sakit menetapkan regulasi tentang penyelenggara pelayanan darah dan menjamin pelayanan yang diberikan sesuai peraturan dan perundang-undangan dan standar pelayanan.

\hypertarget{maksud-dan-tujuan-pp-3.9}{%
\subsubsection*{29. Maksud dan Tujuan PP 3.9}\label{maksud-dan-tujuan-pp-3.9}}
\addcontentsline{toc}{subsubsection}{29. Maksud dan Tujuan PP 3.9}

Jika terdapat pelayanan yang direncanakan untuk penggunaan darah dan produk darah, maka dalam hal ini diperlukan persetujuan tindakan khusus. Rumah sakit mengidentifikasi prosedur berisiko tinggi di dalam perawatan yang membutuhkan persetujuan, diantaranya adalah pemberian darah dan produk darah.

\hypertarget{elemen-penilaian-pp-3.9}{%
\subsubsection*{30. Elemen Penilaian PP 3.9}\label{elemen-penilaian-pp-3.9}}
\addcontentsline{toc}{subsubsection}{30. Elemen Penilaian PP 3.9}

\begin{enumerate}
\def\labelenumi{\alph{enumi}.}
\tightlist
\item
  Rumah sakit menerapkan regulasi tentang penyelenggaraan pelayanan darah di rumah sakit.
\item
  Penyelenggaraan pelayanan darah dibawah tanggung jawab seorang staf yang kompeten.
\end{enumerate}

\begin{enumerate}
\def\labelenumi{\alph{enumi})}
\setcounter{enumi}{2}
\tightlist
\item
  Rumah sakit telah melakukan pemantauan dan evaluasi mutu terhadap penyelenggaran pelayanan darah di rumah sakit.
\item
  Rumah sakit menerapkan proses persetujuan tindakan pasien untuk pemberian darah dan produk darah.
\end{enumerate}

\hypertarget{d.-pelayanan-radiologi-klinik}{%
\subsection*{d.~Pelayanan radiologi klinik}\label{d.-pelayanan-radiologi-klinik}}
\addcontentsline{toc}{subsection}{d.~Pelayanan radiologi klinik}

\hypertarget{standar-pp-4}{%
\subsubsection*{1. Standar PP 4}\label{standar-pp-4}}
\addcontentsline{toc}{subsubsection}{1. Standar PP 4}

Pelayanan radiologi klinik menetapkan regulasi pelayanan radiologi klinis di rumah sakit.

\hypertarget{maksud-dan-tujuan-pp-4}{%
\subsubsection*{2. Maksud dan Tujuan PP 4}\label{maksud-dan-tujuan-pp-4}}
\addcontentsline{toc}{subsubsection}{2. Maksud dan Tujuan PP 4}

Pelayanan radiodiagnostik, imajing dan radiologi intervensional (RIR) meliputi:

\begin{enumerate}
\def\labelenumi{\alph{enumi}.}
\tightlist
\item
  Pelayanan radiodiagnostik;
\item
  Pelayanan diagnostik Imajing; dan
\item
  Pelayanan radiologi intervensional.
\end{enumerate}

Rumah sakit menetapkan sistem yang terintegrasi untuk menyelenggarakan pelayanan radiodiagnostik, imajing dan radiologi intervensional yang dibutuhkan pasien, asuhan klinis dan Profesional Pemberi Asuhan (PPA). Pelayanan radiologi klinik buka 24 jam, 7 (tujuh) hari seminggu sesuai dengan kebutuhan pasien.

\hypertarget{elemen-penilaian-pp-4}{%
\subsubsection*{3. Elemen Penilaian PP 4}\label{elemen-penilaian-pp-4}}
\addcontentsline{toc}{subsubsection}{3. Elemen Penilaian PP 4}

\begin{enumerate}
\def\labelenumi{\alph{enumi}.}
\tightlist
\item
  Rumah Sakit menetapkan dan melaksanakan regulasi pelayanan radiologi klinik.
\item
  Terdapat pelayanan radiologi klinik selama 24 jam, 7 (tujuh) hari seminggu, sesuai dengan kebutuhan pasien.
\end{enumerate}

\hypertarget{standar-pp-4.1}{%
\subsubsection*{4. Standar PP 4.1}\label{standar-pp-4.1}}
\addcontentsline{toc}{subsubsection}{4. Standar PP 4.1}

Rumah Sakit menetapkan seorang yang kompeten dan berwenang, bertanggung jawab mengelola pelayanan RIR.

\hypertarget{maksud-dan-tujuan-pp-4.1}{%
\subsubsection*{5. Maksud dan Tujuan PP 4.1}\label{maksud-dan-tujuan-pp-4.1}}
\addcontentsline{toc}{subsubsection}{5. Maksud dan Tujuan PP 4.1}

Pelayanan Radiodiagnostik, Imajing dan Radiologi Intervensional berada dibawah pimpinan seorang yang kompeten dan berwenang memenuhi persyaratan peraturan perundangan.

Pimpinan radiologi klinik bertanggung jawab mengelola fasilitas dan pelayanan RIR, termasuk pemeriksaan yang dilakukan di tempat tidur pasien (POCT), juga tanggung jawabnya dalam melaksanakan regulasi RS secara konsisten, seperti pelatihan, manajemen logistik, dan sebagainya.
Tanggung jawab pimpinan pelayanan radiologi diagnostik imajing, dan radiologi intervensional antara lain:

\begin{enumerate}
\def\labelenumi{\alph{enumi}.}
\tightlist
\item
  Menyusun dan evaluasi regulasi.
\item
  Pengawasan pelaksanaan administrasi.
\item
  Melaksanakan program kendali mutu (PMI dan PME) dan mengintegrasikan program mutu radiologi dengan program Manajemen Fasilitas dan Keamanan serta program Pencegahan dan Pengendalian Infeksi di rumah sakit.
\item
  Memonitor dan evaluasi semua jenis pelayanan RIR.
\item
  Mereviu dan menindak lanjuti hasil pemeriksaan pelayanan RIR rujukan.
\end{enumerate}

\hypertarget{elemen-penilaian-pp-4.1}{%
\subsubsection*{6. Elemen Penilaian PP 4.1}\label{elemen-penilaian-pp-4.1}}
\addcontentsline{toc}{subsubsection}{6. Elemen Penilaian PP 4.1}

\begin{enumerate}
\def\labelenumi{\alph{enumi}.}
\tightlist
\item
  Direktur menetapkankan penanggung jawab radiologi klinik yang memiliki kompetensi sesuai ketentuan dengan peraturan perundang-undangan.
\item
  Terdapat bukti pengawasan pelayanan radiologi klinik oleh penanggung jawab radiologi klinik sesuai poin a) -- e) pada maksud dan tujuan.
\end{enumerate}

\hypertarget{standar-pp-4.2}{%
\subsubsection*{7. Standar PP 4.2}\label{standar-pp-4.2}}
\addcontentsline{toc}{subsubsection}{7. Standar PP 4.2}

Semua staf radiologi klinik mempunyai pendidikan, pelatihan, kualifikasi dan pengalaman yang dipersyaratkan untuk mengerjakan pemeriksaan.

\hypertarget{maksud-dan-tujuan-pp-4.2}{%
\subsubsection*{8. Maksud dan Tujuan PP 4.2}\label{maksud-dan-tujuan-pp-4.2}}
\addcontentsline{toc}{subsubsection}{8. Maksud dan Tujuan PP 4.2}

Rumah sakit menetapkan mereka yang bekerja sebagai staf radiologi dan diagnostik imajing yang kompeten dan berwenang melakukan pemeriksaan radiodiagnostik, imajing dan radiologi intervensional, pembacaan diagnostik imajing, pelayanan pasien di tempat tidur (POCT), membuat interpretasi, melakukan verifikasi dan serta melaporkan hasilnya, serta mereka yang mengawasi prosesnya.

Staf pengawas dan staf pelaksana teknikal mempunyai latar belakang pelatihan, pengalaman, ketrampilan dan telah menjalani orientasi tugas pekerjaannya. Staf teknikal diberi tugas pekerjaan sesuai latar belakang pendidikan dan pengalaman mereka. Sebagai tambahan, jumlah staf cukup tersedia untuk melakukan tugas, membuat interpretasi, dan melaporkan segera hasilnya untuk layanan darurat.

\hypertarget{elemen-penilaian-pp-4.2}{%
\subsubsection*{9. Elemen Penilaian PP 4.2}\label{elemen-penilaian-pp-4.2}}
\addcontentsline{toc}{subsubsection}{9. Elemen Penilaian PP 4.2}

\begin{enumerate}
\def\labelenumi{\alph{enumi}.}
\tightlist
\item
  Staf radiologi klinik yang membuat interpretasi telah memenuhi persyaratan kredensial
\item
  Staf radiologi klinik dan staf lain yang melaksanakan pemeriksaan termasuk yang mengerjakan tindakan di Ruang Rawat pasien, memenuhi persyaratan kredensial.
\end{enumerate}

\hypertarget{standar-pp-4.3}{%
\subsubsection*{10. Standar PP 4.3}\label{standar-pp-4.3}}
\addcontentsline{toc}{subsubsection}{10. Standar PP 4.3}

Rumah sakit menetapkan kerangka waktu penyelesaian pemeriksaan radiologi klinik regular dan cito.

\hypertarget{maksud-dan-tujuan-pp-4.3}{%
\subsubsection*{11. Maksud dan Tujuan PP 4.3}\label{maksud-dan-tujuan-pp-4.3}}
\addcontentsline{toc}{subsubsection}{11. Maksud dan Tujuan PP 4.3}

Rumah sakit menetapkan kerangka waktu penyelesaian pemeriksaan radiologi dan diagnostik imajing. Penyelesaian pemeriksaan radiodiagnostik, imajing dan radiologi intervensional (RIR) dilaporkan sesuai kebutuhan pasien. Hasil pemeriksaan cito, antara lain dari unit darurat, kamar operasi, unit intensif diberi perhatian khusus terkait kecepatan hasil pemeriksaan. Jika pemeriksaan dilakukan melalui kontrak (pihak ketiga) atau radiologi rujukan, kerangka waktu melaporkan hasil pemeriksaan mengikuti ketentuan rumah sakit dan MOU dengan radiodiagnostik, imajing dan radiologi intervensional (RIR) rujukan.

\hypertarget{elemen-penilaian-pp-4.3}{%
\subsubsection*{12. Elemen Penilaian PP 4.3}\label{elemen-penilaian-pp-4.3}}
\addcontentsline{toc}{subsubsection}{12. Elemen Penilaian PP 4.3}

\begin{enumerate}
\def\labelenumi{\alph{enumi}.}
\tightlist
\item
  Rumah sakit menetapkan kerangka waktu penyelesaian pemeriksaan radiologi klinik.
\item
  Dilakukan pencatatan dan evaluasi waktu penyelesaian pemeriksaan radiologi klinik.
\item
  Dilakukan pencatatan dan evaluasi waktu penyelesaian pemeriksaan cito.
\item
  Terdapat bukti pencatatan dan evaluasi pelayanan radiologi rujukan.
\end{enumerate}

\hypertarget{standar-pp-4.4}{%
\subsubsection*{13. Standar PP 4.4}\label{standar-pp-4.4}}
\addcontentsline{toc}{subsubsection}{13. Standar PP 4.4}

Film X-ray dan bahan lainnya tersedia secara teratur.

\hypertarget{maksud-dan-tujuan-pp-4.4}{%
\subsubsection*{14. Maksud dan tujuan PP 4.4}\label{maksud-dan-tujuan-pp-4.4}}
\addcontentsline{toc}{subsubsection}{14. Maksud dan tujuan PP 4.4}

Untuk menjamin pelayanan radiologi dapat berjalan dengan baik maka pimpinan rumah sakit harus memastikan ketersediaan sarana dan prasarana pelayanan radiologi. Perencanaan kebutuhan dan pengelolaan bahan habis pakai dilakukan sesuai ketentuan yang berlaku.

\hypertarget{elemen-penilaian-pp-4.4}{%
\subsubsection*{15. Elemen Penilaian PP 4.4}\label{elemen-penilaian-pp-4.4}}
\addcontentsline{toc}{subsubsection}{15. Elemen Penilaian PP 4.4}

\begin{enumerate}
\def\labelenumi{\alph{enumi}.}
\tightlist
\item
  Rumah sakit menetapkan proses pengelolaan logistik film x-ray, reagens, dan bahan lainnya, termasuk kondisi bila terjadi kekosongan.
\item
  Semua film x-ray disimpan dan diberi label, serta didistribusi sesuai pedoman dari pembuatnya atau instruksi pada kemasannya.
\end{enumerate}

\hypertarget{standar-pp-4.5}{%
\subsubsection*{16. Standar PP 4.5}\label{standar-pp-4.5}}
\addcontentsline{toc}{subsubsection}{16. Standar PP 4.5}

Rumah sakit menetapkan program kendali mutu, dilaksanakan, divalidasi dan didokumentasikan.

\hypertarget{maksud-dan-tujuan-pp-4.5}{%
\subsubsection*{17. Maksud dan Tujuan PP 4.5}\label{maksud-dan-tujuan-pp-4.5}}
\addcontentsline{toc}{subsubsection}{17. Maksud dan Tujuan PP 4.5}

Kendali mutu dalam pelayanan radiodiagnostik terdiri dari Pemantapan Mutu Internal dan Pemantaoan Mutu Eksternal. Kedua hal tersebut dilakukan sesuai ketentuan peraturan perundangan.

\hypertarget{elemen-penilaian-pp-4.5}{%
\subsubsection*{18. Elemen Penilaian PP 4.5}\label{elemen-penilaian-pp-4.5}}
\addcontentsline{toc}{subsubsection}{18. Elemen Penilaian PP 4.5}

\begin{enumerate}
\def\labelenumi{\alph{enumi}.}
\tightlist
\item
  Terdapat bukti bahwa unit radiologi klinik telah melaksanakan Pemantapan Mutu Internal (PMI).
\item
  Terdapat bukti bahwa unit radiologi klinik melaksanakan Pemantapan Mutu Eksternal (PME).
\end{enumerate}

\hypertarget{pelayanan-dan-asuhan-pasien-pap}{%
\section*{4. Pelayanan dan Asuhan Pasien (PAP)}\label{pelayanan-dan-asuhan-pasien-pap}}
\addcontentsline{toc}{section}{4. Pelayanan dan Asuhan Pasien (PAP)}

\textbf{Gambaran Umum}

Tanggung jawab rumah sakit dan staf yang terpenting adalah memberikan asuhan dan pelayanan pasien yang efektif dan aman. Hal ini membutuhkan komunikasi yang efektif, kolaborasi, dan standardisasi proses untuk memastikan bahwa rencana, koordinasi, dan implementasi asuhan mendukung serta merespons setiap kebutuhan unik pasien dan target.

Asuhan tersebut dapat berupa upaya pencegahan, paliatif, kuratif, atau rehabilitatif termasuk anestesia, tindakan bedah, pengobatan, terapi suportif, atau kombinasinya, yang berdasar atas pengkajian awal dan pengkajian ulang pasien.

Area asuhan risiko tinggi (termasuk resusitasi dan transfusi) serta asuhan untuk pasien risiko tinggi atau kebutuhan populasi khusus yang membutuhkan perhatian tambahan.

Asuhan pasien dilakukan oleh profesional pemberi asuhan (PPA) dengan banyak disiplin dan staf klinis. Semua staf yang terlibat dalam asuhan pasien harus memiliki peran yang jelas, ditentukan oleh kompetensi dan kewenangan, kredensial, sertifikasi, hukum dan regulasi, keterampilan individu, pengetahuan, pengalaman, dan kebijakan rumah sakit, atau uraian tugas wewenang (UTW). Beberapa asuhan dapat dilakukan oleh pasien/keluarganya atau pemberi asuhan terlatih (caregiver). Pelaksanaan asuhan dan pelayanan harus dikoordinasikan dan diintegrasikan oleh semua profesional pemberi asuhan (PPA) dapat dibantu oleh staf klinis. Asuhan pasien terintegrasi dilaksanakan dengan beberapa elemen:

\begin{enumerate}
\def\labelenumi{\alph{enumi}.}
\tightlist
\item
  Dokter penanggung jawab pelayanan (DPJP) sebagai pimpinan klinis/ketua tim PPA (clinical leader).
\item
  PPA bekerja sebagai tim interdisiplin dengan kolaborasi interprofesional, menggunakan panduan praktik klinis (PPK), alur klinis/clinical pathway terintegrasi, algoritma, protokol, prosedur, standing order, dan catatan perkembangan pasien terintegrasi (CPPT).
\item
  Manajer Pelayanan Pasien (MPP)/Case Manager menjaga kesinambungan pelayanan.
\item
  Keterlibatan serta pemberdayaan pasien dan keluarga dalam asuhan bersama PPA harus memastikan:
\end{enumerate}

\begin{enumerate}
\def\labelenumi{\arabic{enumi}.}
\tightlist
\item
  Asuhan direncanakan untuk memenuhi kebutuhan pasien yang unik berdasar atas hasil pengkajian;
\item
  Rencana asuhan diberikan kepada tiap pasien;
\item
  Respons pasien terhadap asuhan dipantau; dan
\item
  Rencana asuhan dimodifikasi bila perlu berdasarkan respons pasien.
  Fokus Standar Pelayanan dan Asuhan Pasien (PAP) meliputi:
\end{enumerate}

\begin{enumerate}
\def\labelenumi{\alph{enumi}.}
\tightlist
\item
  Pemberian pelayanan untuk semua pasien
\item
  Pelayanan pasien risiko tinggi dan penyediaan pelayanan risiko tinggi;
\item
  Pemberian makanan dan terapi nutrisi;
\item
  Pengelolaan nyeri; dan
\item
  Pelayanan menjelang akhir hayat.
\end{enumerate}

\hypertarget{a.-pemberian-pelayanan-untuk-semua-pasien}{%
\subsection*{a. Pemberian pelayanan untuk semua pasien}\label{a.-pemberian-pelayanan-untuk-semua-pasien}}
\addcontentsline{toc}{subsection}{a. Pemberian pelayanan untuk semua pasien}

\hypertarget{standar-pap-1}{%
\subsubsection*{1. Standar PAP 1}\label{standar-pap-1}}
\addcontentsline{toc}{subsubsection}{1. Standar PAP 1}

Pelayanan dan asuhan yang seragam diberikan untuk semua pasien sesuai peraturan perundang-undangan.

\hypertarget{maksud-dan-tujuan-pap-1}{%
\subsubsection*{2. Maksud dan Tujuan PAP 1}\label{maksud-dan-tujuan-pap-1}}
\addcontentsline{toc}{subsubsection}{2. Maksud dan Tujuan PAP 1}

Pasien dengan masalah kesehatan dan kebutuhan pelayanan yang sama berhak mendapat mutu asuhan yang seragam di rumah sakit. Untuk melaksanakan prinsip mutu asuhan yang setingkat, pimpinan harus merencanakan dan mengkoordinasi pelayanan pasien. Secara khusus, pelayanan yang diberikan kepada populasi pasien yang sama pada berbagai unit kerja sesuai dengan regulasi yang ditetapkan rumah sakit. Sebagai tambahan, pimpinan harus menjamin bahwa rumah sakit menyediakan tingkat mutu asuhan yang sama setiap hari dalam seminggu dan pada setiap shift. Regulasi tersebut harus sesuai dengan peraturan perundangan yang berlaku sehingga proses pelayanan pasien dapat diberikan secara kolaboratif. Asuhan pasien yang seragam tercermin dalam hal-hal berikut:

\begin{enumerate}
\def\labelenumi{\alph{enumi}.}
\tightlist
\item
  Akses untuk mendapatkan asuhan dan pengobatan tidak bergantung pada kemampuan pasien untuk membayar atau sumber pembayaran.
\item
  Akses untuk mendapatkan asuhan dan pengobatan yang diberikan oleh PPA yang kompeten tidak bergantung pada hari atau jam yaitu 7 (tujuh) hari, 24 (dua puluh empat) jam
\item
  Kondisi pasien menentukan sumber daya yang akan dialokasikan untuk memenuhi kebutuhannya
\item
  Pemberian asuhan yang diberikan kepada pasien, sama di semua unit pelayanan di rumah sakit misalnya pelayanan anestesi.
\item
  Pasien yang membutuhkan asuhan keperawatan yang sama akan menerima tingkat asuhan keperawatan yang sama di semua unit pelayanan di rumah sakit.
\end{enumerate}

Keseragaman dalam memberikan asuhan pada semua pasien akan menghasilkan penggunaan sumber daya yang efektif dan memungkinkan dilakukan evaluasi terhadap hasil asuhan yang sama di semua unit pelyanan di rumah sakit.

\hypertarget{elemen-penilaian-pap-1}{%
\subsubsection*{3. Elemen Penilaian PAP 1}\label{elemen-penilaian-pap-1}}
\addcontentsline{toc}{subsubsection}{3. Elemen Penilaian PAP 1}

\begin{enumerate}
\def\labelenumi{\alph{enumi}.}
\tightlist
\item
  Rumah sakit menetapkan regulasi tentang Pelayanan dan Asuhan Pasien (PAP) yang meliputi poin a) -- e) dalam gambaran umum.
\item
  Asuhan yang seragam diberikan kepada setiap pasien meliputi poin a) -- e) dalam maksud dan tujuan
\end{enumerate}

\hypertarget{standar-pap-1.1}{%
\subsubsection*{4. Standar PAP 1.1}\label{standar-pap-1.1}}
\addcontentsline{toc}{subsubsection}{4. Standar PAP 1.1}

Proses pelayanan dan asuhan pasien yang terintegrasi serta terkoordinasi telah dilakukan sesuai instruksi.

\hypertarget{maksud-dan-tujuan-pap-1.1}{%
\subsubsection*{5. Maksud dan Tujuan PAP 1.1}\label{maksud-dan-tujuan-pap-1.1}}
\addcontentsline{toc}{subsubsection}{5. Maksud dan Tujuan PAP 1.1}

Proses pelayanan dan asuhan pasien bersifat dinamis dan melibatkan banyak PPA dan berbagai unit pelayanan. Agar proses pelayanan dan asuhan pasien menjadi efisien, penggunaan sumber daya manusia dan sumber lainnya menjadi efektif, dan hasil akhir kondisi pasien menjadi lebih baik maka diperlukan integrasi dan koordinasi. Kepala unit pelayanan menggunakan cara untuk melakukan integrasi dan koordinasi pelayanan serta asuhan lebih baik (misalnya, pemberian asuhan pasein secara tim oleh para PPA, ronde pasien multidisiplin, formulir catatan perkembangan pasien terintegrasi (CPPT), dan manajer pelayanan pasien/case manager).

Instruksi PPA dibutuhkan dalam pemberian asuhan pasien misalnya instruksi pemeriksaan di laboratorium (termasuk Patologi Anatomi), pemberian obat, asuhan keperawatan khusus, terapi nurtrisi, dan lain-lain. Instruksi ini harus tersedia dan mudah diakses sehingga dapat ditindaklanjuti tepat waktu misalnya dengan menuliskan instruksi pada formulir catatan perkembangan pasien terintegrasi (CPPT) dalam rekam medis atau didokumentasikan dalam elektronik rekam medik agar staf memahami kapan instruksi harus dilakukan, dan siapa yang akan melaksanakan instruksi tersebut.

Setiap rumah sakit harus mengatur dalam regulasinya:

\begin{enumerate}
\def\labelenumi{\alph{enumi}.}
\tightlist
\item
  Instruksi seperti apa yang harus tertulis/didokumentasikan (bukan instruksi melalui telepon atau instruksi lisan saat PPA yang memberi instruksi sedang berada di tempat/rumah sakit), antara lain:
\end{enumerate}

\begin{enumerate}
\def\labelenumi{\arabic{enumi}.}
\tightlist
\item
  Instruksi yang diijinkan melalui telepon terbatas pada situasi darurat dan ketika dokter tidak berada di tempat/di rumah sakit.
\item
  Instruksi verbal diijinkan terbatas pada situasi dimana dokter yang memberi instruksi sedang melakukan tindakan/prosedur steril.
\end{enumerate}

\begin{enumerate}
\def\labelenumi{\alph{enumi}.}
\setcounter{enumi}{1}
\tightlist
\item
  Permintaan pemeriksaan laboratorium (termasuk pemeriksaan Patologi Anatomi) dan diagnostik imajing tertentu harus disertai indikasi klinik
\item
  Pengecualian dalam kondisi khusus, misalnya di unit darurat dan unit intensif
\item
  Siapa yang diberi kewenangan memberi instruksi dan perintah catat di dalam berkas rekam medik/sistem elektronik rekam medik sesuai regulasi rumah sakit
\end{enumerate}

Prosedur diagnostik dan tindakan klinis, yang dilakukan sesuai instruksi serta hasilnya didokumentasikan di dalam rekam medis pasien. Contoh prosedur dan tindakan misalnya endoskopi, kateterisasi jantung, terapi radiasi, pemeriksaan Computerized Tomography (CT), dan tindakan serta prosedur diagnostik invasif dan non-invasif lainnya. Informasi mengenai siapa yang meminta dilakukannya prosedur atau tindakan, dan alasan dilakukannya prosedur atau tindakan tersebut didokumentasikan dalam rekam medik.

Di rawat jalan bila dilakukan tindakan diagnostik invasif/berisiko, termasuk pasien yang dirujuk dari luar, juga harus dilakukan pengkajian serta pencatatannya dalam rekam medis.

\hypertarget{elemen-penilaian-standar-pap-1.1}{%
\subsubsection*{6. Elemen Penilaian Standar PAP 1.1}\label{elemen-penilaian-standar-pap-1.1}}
\addcontentsline{toc}{subsubsection}{6. Elemen Penilaian Standar PAP 1.1}

\begin{enumerate}
\def\labelenumi{\alph{enumi}.}
\tightlist
\item
  Rumah sakit telah melakukan pelayanan dan asuhan yang terintegrasi serta terkoordinasi kepada setiap pasien.
\item
  Rumah sakit telah menetapkan kewenangan pemberian instruksi oleh PPA yang kompeten, tata cara pemberian instruksi dan pendokumentasiannya.
\item
  Permintaan pemeriksaan laboratorium dan diagnostik imajing harus disertai indikasi klinis apabila meminta hasilnya berupa interpretasi.
\item
  Prosedur dan tindakan telah dilakukan sesuai instruksi dan PPA yang memberikan instruksi, alasan dilakukan prosedur atau tindakan serta hasilnya telah didokumentasikan di dalam rekam medis pasien.
\item
  Pasien yang menjalani tindakan invasif/berisiko di rawat jalan telah dilakukan pengkajian dan didokumentasikan dalam rekam medis.
\end{enumerate}

\hypertarget{standar-pap-1.2}{%
\subsubsection*{7. Standar PAP 1.2}\label{standar-pap-1.2}}
\addcontentsline{toc}{subsubsection}{7. Standar PAP 1.2}

Rencana asuhan individual setiap pasien dibuat dan didokumentasikan

\hypertarget{maksud-dan-tujuan-standar-pap-1.2}{%
\subsubsection*{8. Maksud dan Tujuan Standar PAP 1.2}\label{maksud-dan-tujuan-standar-pap-1.2}}
\addcontentsline{toc}{subsubsection}{8. Maksud dan Tujuan Standar PAP 1.2}

Rencana asuhan merangkum asuhan dan pengobatan/tindakan yang akan diberikan kepada seorang pasien. Rencana asuhan memuat satu rangkaian tindakan yang dilakukan oleh PPA untuk menegakkan atau mendukung diagnosis yang disusun dari hasil pengkajian. Tujuan utama rencana asuhan adalah memperoleh hasil klinis yang optimal.

Proses perencanaan bersifat kolaboratif menggunakan data yang berasal dari pengkajian awal dan pengkajian ulang yang di buat oleh para PPA (dokter, perawat, ahli gizi, apoteker, dan lain-lainnya)

Rencana asuhan dibuat setelah melakukan pengkajian awal dalam waktu 24 jam terhitung sejak pasien diterima sebagai pasien rawat inap. Rencana asuhan yang baik menjelaskan asuhan pasien yang objektif dan memiliki sasaran yang dapat diukur untuk memudahkan pengkajian ulang serta mengkaji atau merevisi rencana asuhan. Pasien dan keluarga dapat dilibatkan dalam proses perencanaan asuhan. Rencana asuhan harus disertai target terukur, misalnya:

\begin{enumerate}
\def\labelenumi{\alph{enumi}.}
\tightlist
\item
  Detak jantung, irama jantung, dan tekanan darah menjadi normal atau sesuai dengan rencana yang ditetapkan;
\item
  Pasien mampu menyuntik sendiri insulin sebelum pulang dari rumah sakit;
\item
  Pasien mampu berjalan dengan ``walker'' (alat bantu untuk berjalan).
\end{enumerate}

Berdasarkan hasil pengkajian ulang, rencana asuhan diperbaharui untuk dapat menggambarkan kondisi pasien terkini. Rencana asuhan pasien harus terkait dengan kebutuhan pasien. Kebutuhan ini mungkin berubah sebagai hasil dari proses penyembuhan klinis atau terdapat informasi baru hasil pengkajian ulang (contoh, hilangnya kesadaran, hasil laboratorium yang abnormal). Rencana asuhan dan revisinya didokumentasikan dalam rekam medis pasien sebagai rencana asuhan baru.

DPJP sebagai ketua tim PPA melakukan evaluasi / reviu berkala dan verifikasi harian untuk memantau terlaksananya asuhan secara terintegrasi dan membuat notasi sesuai dengan kebutuhan.

Catatan: satu rencana asuhan terintegrasi dengan sasaran- sasaran yang diharapkan oleh PPA lebih baik daripada rencana terpisah oleh masing-masing PPA. Rencana asuhan yang baik menjelaskan asuhan individual, objektif, dan sasaran dapat diukur untuk memudahkan pengkajian ulang serta revisi rencana asuhan.

\hypertarget{elemen-penilaian-pap-1.2}{%
\subsubsection*{9. Elemen Penilaian PAP 1.2}\label{elemen-penilaian-pap-1.2}}
\addcontentsline{toc}{subsubsection}{9. Elemen Penilaian PAP 1.2}

\begin{enumerate}
\def\labelenumi{\alph{enumi}.}
\tightlist
\item
  PPA telah membuat rencana asuhan untuk setiap pasien setelah diterima sebagai pasien rawat inap dalam waktu 24 jam berdasarkan hasil pengkajian awal.
\item
  Rencana asuhan dievaluasi secara berkala, direvisi atau dimutakhirkan serta didokumentasikan dalam rekam medis oleh setiap PPA.
\item
  Instruksi berdasarkan rencana asuhan dibuat oleh PPA yang kompeten dan berwenang, dengan cara yang seragam, dan didokumentasikan di CPPT.
\item
  Rencana asuhan pasien dibuat dengan membuat sasaran yang terukur dan di dokumentasikan.
\item
  DPJP telah melakukan evaluasi/review berkala dan verifikasi harian untuk memantau terlaksananya asuhan secara terintegrasi dan membuat notasi sesuai dengan kebutuhan.
\end{enumerate}

\hypertarget{b.-pelayanan-pasien-risiko-tinggi-dan-penyediaan-pelayanan-risiko-tinggi}{%
\subsection*{b. Pelayanan pasien risiko tinggi dan penyediaan pelayanan risiko tinggi}\label{b.-pelayanan-pasien-risiko-tinggi-dan-penyediaan-pelayanan-risiko-tinggi}}
\addcontentsline{toc}{subsection}{b. Pelayanan pasien risiko tinggi dan penyediaan pelayanan risiko tinggi}

\hypertarget{standar-pap-2}{%
\subsubsection*{1. Standar PAP 2}\label{standar-pap-2}}
\addcontentsline{toc}{subsubsection}{1. Standar PAP 2}

Rumah sakit menetapkan pasien risiko tinggi dan pelayanan risiko tinggi sesuai dengan kemampuan, sumber daya dan sarana prasarana yang dimiliki.

\hypertarget{maksud-dan-tujuan-pap-2}{%
\subsubsection*{2. Maksud dan Tujuan PAP 2}\label{maksud-dan-tujuan-pap-2}}
\addcontentsline{toc}{subsubsection}{2. Maksud dan Tujuan PAP 2}

Rumah sakit memberikan pelayanan untuk pasien dengan berbagai keperluan. Pelayanan pada pasien berisiko tinggi membutuhkan prosedur, panduan praktik klinis (PPK) clinical pathway dan rencana perawatan yang akan mendukung PPA memberikan pelayanan kepada pasien secara menyeluruh, kompeten dan seragam.

Dalam memberikan asuhan pada pasien risiko tinggi dan pelayanan berisiko tinggi, Pimpinan rumah sakit bertanggung jawab untuk:

\begin{enumerate}
\def\labelenumi{\alph{enumi}.}
\tightlist
\item
  Mengidentifikasi pasien dan pelayanan yang dianggap berisiko tinggi di rumah sakit;
\item
  Menetapkan prosedur, panduan praktik klinis (PPK), clinical pathway dan rencana perawatan secara kolaboratif
\item
  Melatih staf untuk menerapkan prosedur, panduan praktik klinis (PPK), clinical pathway dan rencana perawatan rencana perawatan tersebut.
\end{enumerate}

Pelayanan pada pasien berisiko tinggi atau pelayanan berisiko tinggi dibuat berdasarkan populasi yaitu pasien anak, pasien dewasa dan pasien geriatri. Hal-hal yang perlu diterapkan dalam pelayanan tersebut meliputi Prosedur, dokumentasi, kualifikasi staf dan peralatan medis meliputi:

\begin{enumerate}
\def\labelenumi{\alph{enumi}.}
\item
  Rencana asuhan perawatan pasien;
\item
  Perawatan terintegrasi dan mekanisme komunikasi antar PPA secara efektif;
\item
  Pemberian informed consent, jika diperlukan;
\item
  \begin{enumerate}
  \def\labelenumii{\alph{enumii})}
  \setcounter{enumii}{3}
  \tightlist
  \item
    Pemantauan/observasi pasien selama memberikan pelayanan;
  \end{enumerate}
\item
  Kualifikasi atau kompetensi staf yang memberikan pelayanan; dan
\item
  Ketersediaan dan penggunaan peralatan medis khusus untuk pemberian pelayanan.
\end{enumerate}

Rumah sakit mengidentifikasi dan memberikan asuhan pada pasien risiko tinggi dan pelayanan risiko tinggi sesuai kemampuan, sumber daya dan sarana prasarana yang dimiliki meliputi:

\begin{enumerate}
\def\labelenumi{\alph{enumi}.}
\tightlist
\item
  Pasien emergensi;
\item
  Pasien koma;
\item
  Pasien dengan alat bantuan hidup;
\item
  Pasien risiko tinggi lainnya yaitu pasien dengan penyakit jantung, hipertensi, stroke dan diabetes;
\item
  Pasien dengan risiko bunuh diri;
\item
  Pelayanan pasien dengan penyakit menular dan penyakit yang berpotensi menyebabkan kejadian luar biasa;
\item
  Pelayanan pada pasien dengan ``immuno-suppressed'';
\item
  Pelayanan pada pasien yang mendapatkan pelayanan dialisis;
\item
  Pelayanan pada pasien yang direstrain;
\item
  Pelayanan pada pasien yang menerima kemoterapi;
\item
  Pelayanan pasien paliatif;
\item
  Pelayanan pada pasien yang menerima radioterapi;
\item
  Pelayanan pada pasien risiko tinggi lainnya (misalnya terapi hiperbarik dan pelayanan radiologi intervensi);
\item
  Pelayanan pada populasi pasien rentan, pasien lanjut usia (geriatri) misalnya anak-anak, dan pasien berisiko tindak kekerasan atau diterlantarkan misalnya pasien dengan gangguan jiwa.
\end{enumerate}

Rumah sakit juga menetapkan jika terdapat risiko tambahan setelah dilakukan tindakan atau rencana asuhan (contoh, kebutuhan mencegah trombosis vena dalam, luka dekubitus, infeksi terkait penggunaan ventilator pada pasien, cedera neurologis dan pembuluh darah pada pasien restrain, infeksi melalui pembuluh darah pada pasien dialisis, infeksi saluran/slang sentral, dan pasien jatuh. Jika terjadi risiko tambahan tersebut, dilakukan penanganan dan pencegahan dengan membuat regulasi, memberikan pelatihan dan edukasi kepada staf.

Rumah sakit menggunakan informasi tersebut untuk mengevaluasi pelayanan yang diberikan kepada pasien risiko tinggi dan pelayanan berisiko tinggi serta mengintegrasikan informasi tersebut dalam pemilihan prioritas perbaikan tingkat rumah sakit pada program peningkatan mutu dan keselamatan pasien.

\hypertarget{elemen-penilaian-pap-2}{%
\subsubsection*{3. Elemen Penilaian PAP 2}\label{elemen-penilaian-pap-2}}
\addcontentsline{toc}{subsubsection}{3. Elemen Penilaian PAP 2}

\begin{enumerate}
\def\labelenumi{\alph{enumi}.}
\tightlist
\item
  Pimpinan rumah sakit telah melaksanakan tanggung jawabnya untuk memberikan pelayanan pada pasien berisiko tinggi dan pelayanan berisiko tinggi meliputi a) - c) dalam maksud dan tujuan.
\item
  Rumah sakit telah memberikan pelayanan pada pasien risiko tinggi dan pelayanan risiko tinggi yang telah diidentifikasi berdasarkan populasi yaitu pasien anak, pasien dewasa dan pasien geriatri sesuai dalam maksud dan tujuan.
\item
  Pimpinan rumah sakit telah mengidentifikasi risiko tambahan yang dapat mempengaruhi pasien dan pelayanan risiko tinggi.
\end{enumerate}

\hypertarget{standar-pap-2.1}{%
\subsubsection*{4. Standar PAP 2.1}\label{standar-pap-2.1}}
\addcontentsline{toc}{subsubsection}{4. Standar PAP 2.1}

Rumah sakit memberikan pelayanan geriatri rawat jalan, rawat inap akut dan rawat inap kronis sesuai dengan tingkat jenis pelayanan.

\hypertarget{standar-pap-2.2}{%
\subsubsection*{5. Standar PAP 2.2}\label{standar-pap-2.2}}
\addcontentsline{toc}{subsubsection}{5. Standar PAP 2.2}

Rumah Sakit melakukan promosi dan edukasi sebagai bagian dari Pelayanan Kesehatan Warga Lanjut usia di Masyarakat Berbasis Rumah Sakit (Hospital Based Community Geriatric Service).

\hypertarget{maksud-dan-tujuan-pap-2.1-dan-pap-2.2}{%
\subsubsection*{6. Maksud dan Tujuan PAP 2.1 dan PAP 2.2}\label{maksud-dan-tujuan-pap-2.1-dan-pap-2.2}}
\addcontentsline{toc}{subsubsection}{6. Maksud dan Tujuan PAP 2.1 dan PAP 2.2}

Pasien geriatri adalah pasien lanjut usia dengan multi penyakit/gangguan akibat penurunan fungsi organ, psikologi, sosial, ekonomi dan lingkungan yang membutuhkan pelayanan kesehatan secara tepadu dengan pendekatan multi disiplin yang bekerja sama secara interdisiplin. Dengan meningkatnya sosial ekonomi dan pelayanan kesehatan maka usia harapan hidup semakin meningkat, sehingga secara demografi terjadi peningkatan populasi lanjut usia. Sehubungan dengan itu rumah sakit perlu menyelenggarakan pelayanan geriatri sesuai dengan tingkat jenis pelayanan geriatri:

\begin{enumerate}
\def\labelenumi{\alph{enumi}.}
\tightlist
\item
  Tingkat sederhana (rawat jalan dan home care)
\item
  Tingkat lengkap (rawat jalan, rawat inap akut dan home care)
\item
  Tingkat sempurna (rawat jalan, rawat inap akut dan home care klinik asuhan siang)
\item
  Tingkat paripurna (rawat jalan, klinik asuhan siang, rawat inap akut, rawat inap kronis, rawat inap psychogeriatri, penitipan pasien Respit care dan home care)
\end{enumerate}

\hypertarget{elemen-penilaian-pap-2.1}{%
\subsubsection*{7. Elemen Penilaian PAP 2.1}\label{elemen-penilaian-pap-2.1}}
\addcontentsline{toc}{subsubsection}{7. Elemen Penilaian PAP 2.1}

\begin{enumerate}
\def\labelenumi{\alph{enumi}.}
\tightlist
\item
  Rumah sakit telah menetapkan regulasi tentang penyelenggaraan pelayanan geriatri di rumah sakit sesuai dengan kemampuan, sumber daya dan sarana prasarana nya.
\item
  Rumah sakit telah menetapkan tim terpadu geriatri dan telah menyelenggarakan pelayanan sesuai tingkat jenis layanan
\item
  Rumah sakit telah melaksanakan proses pemantauan dan evaluasi kegiatan pelayanan geriatri
\item
  Ada pelaporan penyelenggaraan pelayanan geriatri di rumah sakit.
\end{enumerate}

\hypertarget{elemen-penilaian-pap-2.2}{%
\subsubsection*{8. Elemen Penilaian PAP 2.2}\label{elemen-penilaian-pap-2.2}}
\addcontentsline{toc}{subsubsection}{8. Elemen Penilaian PAP 2.2}

\begin{enumerate}
\def\labelenumi{\alph{enumi}.}
\tightlist
\item
  Ada program PKRS terkait Pelayanan Kesehatan Warga Lanjut usia di Masyarakat Berbasis Rumah Sakit (Hospital Based Community Geriatric Service).
\item
  Rumah sakit telah memberikan edukasi sebagai bagian dari Pelayanan Kesehatan Warga Lanjut usia di Masyarakat Berbasis Rumah Sakit (Hospital Based Community Geriatric Service).
\item
  Rumah sakit telah melaksanakan kegiatan sesuai program dan tersedia leaflet atau alat bantu kegiatan (brosur, leaflet, dan lain-lainnya).
\item
  Rumah sakit telah melakukan evaluasi dan membuat laporan kegiatan pelayanan secara berkala.
\end{enumerate}

\hypertarget{standar-pap-2.3}{%
\subsubsection*{9. Standar PAP 2.3}\label{standar-pap-2.3}}
\addcontentsline{toc}{subsubsection}{9. Standar PAP 2.3}

Rumah sakit menerapkan proses pengenalan perubahan kondisi pasien yang memburuk.

\hypertarget{maksud-dan-tujuan-pap-2.3}{%
\subsubsection*{10. Maksud dan Tujuan PAP 2.3}\label{maksud-dan-tujuan-pap-2.3}}
\addcontentsline{toc}{subsubsection}{10. Maksud dan Tujuan PAP 2.3}

Staf yang tidak bekerja di daerah pelayanan kritis/intensif mungkin tidak mempunyai pengetahuan dan pelatihan yang cukup untuk melakukan pengkajian, serta mengetahui pasien yang akan masuk dalam kondisi kritis. Padahal, banyak pasien di luar daerah pelayanan kritis mengalami keadaan kritis selama dirawat inap. Seringkali pasien memperlihatkan tanda bahaya dini (contoh, tanda- tanda vital yang memburuk dan perubahan kecil status neurologis) sebelum mengalami penurunan kondisi klinis yang meluas sehingga mengalami kejadian yang tidak diharapkan.

Ada kriteria fisiologis yang dapat membantu staf untuk mengenali sedini-dininya pasien yang kondisinya memburuk. Sebagian besar pasien yang mengalami gagal jantung atau gagal paru sebelumnya memperlihatkan tanda-tanda fisiologis di luar kisaran normal yang merupakan indikasi keadaan pasien memburuk. Hal ini dapat diketahui dengan early warning system (EWS).

Penerapan EWS membuat staf mampu mengidentifikasi keadaan pasien memburuk sedini-dininya dan bila perlu mencari bantuan staf yang kompeten. Dengan demikian, hasil asuhan akan lebih baik. Pelaksanaan EWS dapat dilakukan menggunakan sistem skor oleh PPA yang terlatih.

\hypertarget{elemen-penilaian-pap-2.3}{%
\subsubsection*{11. Elemen Penilaian PAP 2.3}\label{elemen-penilaian-pap-2.3}}
\addcontentsline{toc}{subsubsection}{11. Elemen Penilaian PAP 2.3}

\begin{enumerate}
\def\labelenumi{\alph{enumi}.}
\tightlist
\item
  Rumah sakit telah menerapkan proses pengenalan perubahan kondisi pasien yang memburuk (EWS) dan mendokumentasikannya di dalam rekam medik pasien.
\item
  Rumah sakit memiliki bukti PPA dilatih menggunakan EWS.
\end{enumerate}

\hypertarget{standar-pap-2.4}{%
\subsubsection*{12. Standar PAP 2.4}\label{standar-pap-2.4}}
\addcontentsline{toc}{subsubsection}{12. Standar PAP 2.4}

Pelayanan resusitasi tersedia di seluruh area rumah sakit.

\hypertarget{maksud-dan-tujuan-pap-2.4}{%
\subsubsection*{13. Maksud dan Tujuan PAP 2.4}\label{maksud-dan-tujuan-pap-2.4}}
\addcontentsline{toc}{subsubsection}{13. Maksud dan Tujuan PAP 2.4}

Pelayanan resusitasi diartikan sebagai intervensi klinis pada pasien yang mengalami kejadian mengancam hidupnya seperti henti jantung atau paru. Pada saat henti jantung atau paru maka pemberian kompresi pada dada atau bantuan pernapasan akan berdampak pada hidup atau matinya pasien, setidak-tidaknya menghindari kerusakan jaringan otak. Resusitasi yang berhasil pada pasien dengan henti jantung-paru bergantung pada intervensi yang kritikal/penting seperti kecepatan pemberian bantuan hidup dasar, bantuan hidup lanjut yang akurat (code blue) dan kecepatan melakukan defibrilasi. Pelayanan seperti ini harus tersedia untuk semua pasien selama 24 jam setiap hari.

Sangat penting untuk dapat memberikan pelayanan intervensi yang kritikal, yaitu tersedia dengan cepat peralatan medis terstandar, obat resusitasi, dan staf terlatih yang baik untuk resusitasi. Bantuan hidup dasar harus dilakukan secepatnya saat diketahui ada tanda henti jantung-paru dan proses pemberian bantuan hidup lanjut kurang dari 5 (lima) menit. Hal ini termasuk evaluasi terhadap pelaksanaan sebenarnya resusitasi atau terhadap simulasi pelatihan resusitasi di rumah sakit. Pelayanan resusitasi tersedia di seluruh area rumah sakit termasuk peralatan medis dan staf terlatih, berbasis bukti klinis, dan populasi pasien yang dilayani

\hypertarget{elemen-penilaian-pap-2.4}{%
\subsubsection*{14. Elemen Penilaian PAP 2.4}\label{elemen-penilaian-pap-2.4}}
\addcontentsline{toc}{subsubsection}{14. Elemen Penilaian PAP 2.4}

\begin{enumerate}
\def\labelenumi{\alph{enumi}.}
\tightlist
\item
  Pelayanan resusitasi tersedia dan diberikan selama 24 jam setiap hari di seluruh area rumah sakit.
\item
  Peralatan medis untuk resusitasi dan obat untuk bantuan hidup dasar dan lanjut terstandar sesuai dengan kebutuhan populasi pasien.
\item
  Di seluruh area rumah sakit, bantuan hidup dasar diberikan segera saat dikenali henti jantung-paru dan bantuan hidup lanjut diberikan kurang dari 5 menit.
\item
  Staf diberi pelatihan pelayanan bantuan hidup dasar/lanjut sesuai dengan ketentuan rumah sakit.
\end{enumerate}

\hypertarget{standar-pap-2.5}{%
\subsubsection*{15. Standar PAP 2.5}\label{standar-pap-2.5}}
\addcontentsline{toc}{subsubsection}{15. Standar PAP 2.5}

Pelayanan darah dan produk darah dilaksanakan sesuai dengan panduan klinis serta prosedur yang ditetapkan rumah sakit.

\hypertarget{maksud-dan-tujuan-pap-2.5}{%
\subsubsection*{16. Maksud dan tujuan PAP 2.5}\label{maksud-dan-tujuan-pap-2.5}}
\addcontentsline{toc}{subsubsection}{16. Maksud dan tujuan PAP 2.5}

Pelayanan darah dan produk darah harus diberikan sesuai peraturan perundangan meliputi antara lain:

\begin{enumerate}
\def\labelenumi{\alph{enumi}.}
\tightlist
\item
  Pemberian persetujuan (informed consent);
\item
  Permintaan darah;
\item
  Tes kecocokan;
\item
  Pengadaan darah;
\item
  Penyimpanan darah;
\item
  Identifikasi pasien;
\item
  Distribusi dan pemberian darah; dan
\item
  Pemantauan pasien dan respons terhadap reaksi transfusi.
\end{enumerate}

Staf kompeten dan berwenang melaksanakan pelayanan darah dan produk darah serta melakukan pemantauan dan evaluasi.

\hypertarget{elemen-penilaian-pap-2.5}{%
\subsubsection*{17. Elemen Penilaian PAP 2.5}\label{elemen-penilaian-pap-2.5}}
\addcontentsline{toc}{subsubsection}{17. Elemen Penilaian PAP 2.5}

\begin{enumerate}
\def\labelenumi{\alph{enumi}.}
\tightlist
\item
  Rumah sakit menerapkan penyelenggaraan pelayanan darah.
\item
  Panduan klinis dan prosedur disusun dan diterapkan untuk pelayanan darah serta produk darah.
\item
  Staf yang kompeten bertanggungjawab terhadap pelayanan darah di rumah sakit.
\end{enumerate}

\hypertarget{c.-pemberian-makanan-dan-terapi-nutrisi}{%
\subsection*{c.~Pemberian makanan dan terapi nutrisi}\label{c.-pemberian-makanan-dan-terapi-nutrisi}}
\addcontentsline{toc}{subsection}{c.~Pemberian makanan dan terapi nutrisi}

\hypertarget{standar-pap-3}{%
\subsubsection*{1. Standar PAP 3}\label{standar-pap-3}}
\addcontentsline{toc}{subsubsection}{1. Standar PAP 3}

Rumah sakit memberikan makanan untuk pasien rawat inap dan terapi nutrisi terintegrasi untuk pasien dengan risiko nutrisional.

\hypertarget{maksud-dan-tujuan-pap-3}{%
\subsubsection*{2. Maksud dan Tujuan PAP 3}\label{maksud-dan-tujuan-pap-3}}
\addcontentsline{toc}{subsubsection}{2. Maksud dan Tujuan PAP 3}

Makanan dan terapi nutrisi yang sesuai sangat penting bagi kesehatan pasien dan penyembuhannya. Pilihan makanan disesuaikan dengan usia, budaya, pilihan, rencana asuhan, diagnosis pasien termasuk juga antara lain diet khusus seperti rendah kolesterol dan diet diabetes melitus. Berdasarkan pengkajian kebutuhan dan rencana asuhan, maka DPJP atau PPA lain yang kompeten memesan makanan dan nutrisi lainnya untuk pasien. Pasien berhak menentukan makanan sesuai dengan nilai yang dianut. Bila memungkinkan pasien ditawarkan pilihan makanan yang konsisten dengan status gizi.

Jika keluarga pasien atau ada orang lain mau membawa makanan untuk pasien, maka mereka diberikan edukasi tentang makanan yang merupakan kontraindikasi terhadap rencana, kebersihan makanan, dan kebutuhan asuhan pasien, termasuk informasi terkait interaksi antara obat dan makanan. Makanan yang dibawa oleh keluarga atau orang lain disimpan dengan benar untuk mencegah kontaminasi. Skrining risiko gizi dilakukan pada pengkajian awal. Jika pada saat skrining ditemukan pasien dengan risiko gizi maka terapi gizi terintegrasi diberikan, dipantau, dan dievaluasi.

\hypertarget{elemen-penilaian-pap-3}{%
\subsubsection*{3. Elemen Penilaian PAP 3}\label{elemen-penilaian-pap-3}}
\addcontentsline{toc}{subsubsection}{3. Elemen Penilaian PAP 3}

\begin{enumerate}
\def\labelenumi{\alph{enumi}.}
\tightlist
\item
  Berbagai pilihan makanan atau terapi nutrisi yang sesuai untuk kondisi, perawatan, dan kebutuhan pasien tersedia dan disediakan tepat waktu.
\item
  Sebelum pasien rawat inap diberi makanan, terdapat instruksi pemberian makanan dalam rekam medis pasien yang didasarkan pada status gizi dan kebutuhan pasien.
\item
  Untuk makanan yang disediakan keluarga, edukasi diberikan mengenai batasan-batasan diet pasien dan penyimpanan yang baik untuk mencegah kontaminasi.
\item
  Memiliki bukti pemberian terapi gizi terintegrasi (rencana, pemberian dan evaluasi) pada pasien risiko gizi.
\item
  Pemantauan dan evaluasi terapi gizi dicatat di rekam medis pasien.
\end{enumerate}

\hypertarget{d.-pengelolaan-nyeri}{%
\subsection*{d.~Pengelolaan nyeri}\label{d.-pengelolaan-nyeri}}
\addcontentsline{toc}{subsection}{d.~Pengelolaan nyeri}

\hypertarget{standar-pap-4}{%
\subsubsection*{1. Standar PAP 4}\label{standar-pap-4}}
\addcontentsline{toc}{subsubsection}{1. Standar PAP 4}

Pasien mendapatkan pengelolaan nyeri yang efektif.

\hypertarget{maksud-dan-tujuan-pap-4}{%
\subsubsection*{2. Maksud dan Tujuan PAP 4}\label{maksud-dan-tujuan-pap-4}}
\addcontentsline{toc}{subsubsection}{2. Maksud dan Tujuan PAP 4}

Pasien berhak mendapatkan pengkajian dan pengelolaan nyeri yang tepat. Rumah sakit harus memiliki proses untuk melakukan skrining, pengkajian, dan tata laksana untuk mengatasi rasa nyeri, yang terdiri dari:

\begin{enumerate}
\def\labelenumi{\alph{enumi}.}
\tightlist
\item
  Identifikasi pasien dengan rasa nyeri pada pengkajian awal dan pengkajian ulang.
\item
  Memberi informasi kepada pasien bahwa rasa nyeri dapat merupakan akibat dari terapi, prosedur, atau pemeriksaan.
\item
  Memberikan tata laksana untuk mengatasi rasa nyeri, terlepas dari mana nyeri berasal, sesuai dengan regulasi rumah sakit.
\item
  Melakukan komunikasi dan edukasi kepada pasien dan keluarga mengenai pengelolaan nyeri sesuai dengan latar belakang agama, budaya, nilai-nilai yang dianut.
\item
  Memberikan edukasi kepada seluruh PPA mengenai pengkajian dan pengelolaan nyeri.
\end{enumerate}

\hypertarget{elemen-penilaian-pap-4}{%
\subsubsection*{3. Elemen Penilaian PAP 4}\label{elemen-penilaian-pap-4}}
\addcontentsline{toc}{subsubsection}{3. Elemen Penilaian PAP 4}

\begin{enumerate}
\def\labelenumi{\alph{enumi}.}
\tightlist
\item
  Rumah sakit memiliki proses untuk melakukan skrining, pengkajian, dan tata laksana nyeri meliputi poin a) - e) pada maksud dan tujuan.
\item
  Informasi mengenai kemungkinan adanya nyeri dan pilihan tata laksananya diberikan kepada pasien yang menerima terapi/prosedur/pemeriksaan terencana yang sudah dapat diprediksi menimbulkan rasa nyeri.
\item
  Pasien dan keluarga mendapatkan edukasi mengenai pengelolaan nyeri sesuai dengan latar belakang agama, budaya, nilai-nilai yang dianut.
\item
  Staf rumah sakit mendapatkan pelatihan mengenai cara melakukan edukasi bagi pengelolaan nyeri.
\end{enumerate}

\hypertarget{e.-pelayanan-menjelang-akhir-hayat}{%
\subsection*{e. Pelayanan menjelang akhir hayat}\label{e.-pelayanan-menjelang-akhir-hayat}}
\addcontentsline{toc}{subsection}{e. Pelayanan menjelang akhir hayat}

\hypertarget{standar-pap-5}{%
\subsubsection*{1. Standar PAP 5}\label{standar-pap-5}}
\addcontentsline{toc}{subsubsection}{1. Standar PAP 5}

Rumah sakit memberikan asuhan pasien menjelang akhir kehidupan dengan memperhatikan kebutuhan pasien dan keluarga, mengoptimalkan kenyamanan dan martabat pasien, serta mendokumentasikan dalam rekam medis.

\hypertarget{maksud-dan-tujuan-pap-5}{%
\subsubsection*{2. Maksud dan Tujuan PAP 5}\label{maksud-dan-tujuan-pap-5}}
\addcontentsline{toc}{subsubsection}{2. Maksud dan Tujuan PAP 5}

Skrining dilakukan untuk menetapkan bahwa kondisi pasien masuk dalam fase menjelang ajal. Selanjutnya, PPA melakukan pengkajian menjelang akhir kehidupan yang bersifat individual untuk mengidentifikasi kebutuhan pasien dan keluarganya.
Pengkajian pada pasien menjelang akhir kehidupan harus menilai kondisi pasien seperti:

\begin{enumerate}
\def\labelenumi{\arabic{enumi}.}
\tightlist
\item
  Manajemen gejala dan respons pasien, termasuk mual, kesulitan bernapas, dan nyeri.
\item
  Faktor yang memperparah gejala fisik.
\item
  Orientasi spiritual pasien dan keluarganya, termasuk keterlibatan dalam kelompok agama tertentu.
\item
  Keprihatinan spiritual pasien dan keluarganya, seperti putus asa, penderitaan, rasa bersalah.
\item
  Status psikososial pasien dan keluarganya, seperti kekerabatan, kelayakan perumahan, pemeliharaan lingkungan, cara mengatasi, reaksi pasien dan keluarganya menghadapi penyakit.
\item
  Kebutuhan bantuan atau penundaan layanan untuk pasien dan keluarganya.
\item
  Kebutuhan alternatif layanan atau tingkat layanan.
\item
  Faktor risiko bagi yang ditinggalkan dalam hal cara mengatasi dan potensi reaksi patologis.
\item
  Pasien dan keluarga dilibatkan dalam pengambilan keputusan asuhan.
\end{enumerate}

\hypertarget{elemen-penilaian-pap-5}{%
\subsubsection*{3. Elemen Penilaian PAP 5}\label{elemen-penilaian-pap-5}}
\addcontentsline{toc}{subsubsection}{3. Elemen Penilaian PAP 5}

\begin{enumerate}
\def\labelenumi{\alph{enumi}.}
\tightlist
\item
  Rumah sakit menerapkan pengkajian pasien menjelang akhir kehidupan dan dapat dilakukan pengkajian ulang sampai pasien yang memasuki fase akhir kehidupannya, dengan memperhatikan poin 1) -- 9) pada maksud dan tujuan.
\item
  Asuhan menjelang akhir kehidupan ditujukan terhadap kebutuhan psikososial, emosional, kultural dan spiritual pasien dan keluarganya.
\end{enumerate}

\hypertarget{pelayanan-anestesi-dan-bedah-pab}{%
\section*{5. Pelayanan Anestesi dan Bedah (PAB)}\label{pelayanan-anestesi-dan-bedah-pab}}
\addcontentsline{toc}{section}{5. Pelayanan Anestesi dan Bedah (PAB)}

\textbf{Gambaran Umum}

Tindakan anestesi, sedasi, dan intervensi bedah merupakan proses yang kompleks dan sering dilaksanakan di rumah sakit. Hal tersebut memerlukan:

\begin{enumerate}
\def\labelenumi{\alph{enumi}.}
\tightlist
\item
  Pengkajian pasien yang lengkap dan menyeluruh;
\item
  Perencanaan asuhan yang terintegrasi;
\item
  Pemantauan yang terus menerus;
\item
  Transfer ke ruang perawatan berdasar atas kriteria tertentu;
\item
  Rehabilitasi; dan
\item
  Transfer ke ruangan perawatan dan pemulangan.
\end{enumerate}

Anestesi dan sedasi umumnya merupakan suatu rangkaian proses yang dimulai dari sedasi minimal hingga anastesi penuh. Tindakan sedasi ditandai dengan hilangnya refleks pertahanan jalan nafas secara perlahan seperti batuk dan tersedak. Karena respon pasien terhadap tindakan sedasi dan anestesi berbeda-beda secara individu dan memberikan efek yang panjang, maka prosedur tersebut harus dilakukan pengelolaan yang baik dan terintegrasi. Bab ini tidak mencakup pelayanan sedasi di ICU untuk penggunaan ventilator dan alat invasive lainnya.

Karena tindakan bedah juga merupakan tindakan yang berisiko tinggi maka harus direncanakan dan dilaksanakan secara hati-hati. Rencana prosedur operasi dan asuhan pascaoperasi dibuat berdasar atas pengkajian pasien dan didokumentasikan. Bila rumah sakit memberikan pelayanan pembedahan dengan pemasangan implant, maka harus dibuat laporan jika terjadi ketidak berfungsinya alat tersebut dan proses tindak lanjutnya.

Standar pelayanan anestesi dan bedah berlaku di area manapun dalam rumah sakit yang menggunakan anestesi, sedasi ringan, sedang dan dalam, dan juga pada tempat dilaksanakannya prosedur pembedahan dan tindakan invasif lainnya yang membutuhkan persetujuan tertulis (informed consent). Area ini meliputi ruang operasi rumah sakit, rawat sehari (ODC), poliklinik gigi, poliklinik rawat jalan, endoskopi, radiologi, gawat darurat, perawatan intensif, dan tempat lainnya.

Fokus pada standard ini mencakup:

\begin{enumerate}
\def\labelenumi{\alph{enumi}.}
\tightlist
\item
  Pengorganisasian dan pengelolaan pelayanan anastesi dan sedasi.
\item
  Pelayanan sedasi.
\item
  Pelayanan anastesi.
\item
  Pelayanan pembedahan.
\end{enumerate}

\hypertarget{a.-pengorganisasian-dan-pengelolaan-pelayanan-anastesi-dan-sedasi}{%
\subsection*{a. Pengorganisasian dan pengelolaan pelayanan anastesi dan sedasi}\label{a.-pengorganisasian-dan-pengelolaan-pelayanan-anastesi-dan-sedasi}}
\addcontentsline{toc}{subsection}{a. Pengorganisasian dan pengelolaan pelayanan anastesi dan sedasi}

\hypertarget{standar-pab-1}{%
\subsubsection*{1. Standar PAB 1}\label{standar-pab-1}}
\addcontentsline{toc}{subsubsection}{1. Standar PAB 1}

Rumah sakit menerapkan pelayanan anestesi, sedasi moderat dan dalam untuk memenuhi kebutuhan pasien sesuai dengan kapasitas pelayanan, standar profesi dan perundang undangan yang berlaku.

\hypertarget{maksud-dan-tujuan-pab-1}{%
\subsubsection*{2. Maksud dan Tujuan PAB 1}\label{maksud-dan-tujuan-pab-1}}
\addcontentsline{toc}{subsubsection}{2. Maksud dan Tujuan PAB 1}

Anestesi dan sedasi diartikan sebagai satu alur layanan berkesinambungan mulai dari sedasi minimal sampai anestesi dalam. Anestesi dan sedasi menyebabkan refleks proteksi jalan nafas dapat menghilang sehingga pasien berisiko untuk terjadi sumbatan jalan nafas dan aspirasi cairan lambung. Anestesi dan sedasi adalah proses kompleks sehingga harus diintegrasikan ke dalam rencana asuhan. Anestesi dan sedasi membutuhkan pengkajian lengkap dan komprehensif serta pemantaun pasien secara terus menerus.

Rumah sakit mempunyai suatu sistem untuk pelayanan anestesi, sedasi ringan, moderat dan dalam untuk melayani kebutuhan pasien oleh PPA berdasarkan kewenangan klinis yang diberikan kepadanya, termasuk juga sistim penanganan bila terjadi kegawat daruratan selama tindakan sedasi. Pelayanan anestesi, sedasi ringan, moderat dan dalam (termasuk layanan yang diperlukan untuk kegawatdaruratan) tersedia 24 jam 7 (tujuh) hari.

\hypertarget{elemen-penilaian-pab-1}{%
\subsubsection*{3. Elemen Penilaian PAB 1}\label{elemen-penilaian-pab-1}}
\addcontentsline{toc}{subsubsection}{3. Elemen Penilaian PAB 1}

\begin{enumerate}
\def\labelenumi{\alph{enumi}.}
\tightlist
\item
  Rumah sakit telah menetapkan regulasi pelayanan anestesi dan sedasi dan pembedahan meliputi poin a) --
\item
  pada gambaran umum.
\item
  Pelayanan anestesi dan sedasi yang telah diberikan dapat memenuhi kebutuhan pasien.
\item
  Pelayanan anestesi dan sedasi tersedia selama 24 (dua puluh empat) jam 7 (tujuh) hari sesuai dengan kebutuhan pasien.
\end{enumerate}

\hypertarget{standar-pab-2}{%
\subsubsection*{4. Standar PAB 2}\label{standar-pab-2}}
\addcontentsline{toc}{subsubsection}{4. Standar PAB 2}

Rumah sakit menetapkan penanggung jawab pelayanan anestesi, sedasi moderat dan dalam adalah seorang dokter anastesi yang kompeten.

\hypertarget{maksud-dan-tujuan-pab-2}{%
\subsubsection*{5. Maksud dan Tujuan PAB 2}\label{maksud-dan-tujuan-pab-2}}
\addcontentsline{toc}{subsubsection}{5. Maksud dan Tujuan PAB 2}

Pelayanan anestesi, sedasi moderat dan dalam berada dibawah tanggung jawab seorang dokter anastesi yang kompeten sesuai dengan peraturan perundang undangan. Tanggung jawab pelayanan anestesi, sedasi moderat dan dalam tersebut meliputi:

\begin{enumerate}
\def\labelenumi{\alph{enumi}.}
\tightlist
\item
  Mengembangkan, menerapkan, dan menjaga regulasi;
\item
  Melakukan pengawasan administratif;
\item
  Melaksanakan program pengendalian mutu yang dibutuhkan; dan
\item
  Memantau dan mengevaluasi pelayanan sedasi dan anestesi.
\end{enumerate}

\hypertarget{elemen-penilaian-pab-2}{%
\subsubsection*{6. Elemen Penilaian PAB 2}\label{elemen-penilaian-pab-2}}
\addcontentsline{toc}{subsubsection}{6. Elemen Penilaian PAB 2}

\begin{enumerate}
\def\labelenumi{\alph{enumi}.}
\tightlist
\item
  Rumah sakit telah menerapkan pelayanan anestesi dan sedasi secara seragam di seluruh area seusai regulasi yang ditetapkan.
\item
  Rumah sakit telah menetapkan penanggung jawab pelayanan anestesi dan sedasi adalah seorang dokter anastesi yang kompeten yang melaksanakan tanggung jawabnya meliputi poin a) -- d) pada maksud dan tujuan.
\item
  Bila memerlukan profesional pemberi asuhan terdapat PPA dari luar rumah sakit untuk memberikan pelayanan anestesi dan sedasi, maka ada bukti rekomendasi dan evaluasi pelayanan dari penanggung jawab pelayanan anastesi dan sedasi terhadap PPA tersebut.
\end{enumerate}

\hypertarget{b.-pelayanan-sedasi}{%
\subsection*{b. Pelayanan sedasi}\label{b.-pelayanan-sedasi}}
\addcontentsline{toc}{subsection}{b. Pelayanan sedasi}

\hypertarget{standar-pab-3}{%
\subsubsection*{1. Standar PAB 3}\label{standar-pab-3}}
\addcontentsline{toc}{subsubsection}{1. Standar PAB 3}

Pemberian sedasi moderat dan dalam dilakukan sesuai dengan regulasi dan ditetapkan rumah sakit.

\hypertarget{maksud-dan-tujuan-pab-3}{%
\subsubsection*{2. Maksud dan Tujuan PAB 3}\label{maksud-dan-tujuan-pab-3}}
\addcontentsline{toc}{subsubsection}{2. Maksud dan Tujuan PAB 3}

Prosedur pemberian sedasi moderat dan dalam yang diberikan secara intravena tidak bergantung pada berapa dosisnya. oleh karena prosedur pemberian sedasi seperti

layaknya anestesi mengandung risiko potensial pada pasien. Pemberian sedasi pada pasien harus dilakukan seragam dan sama di semua tempat di rumah sakit termasuk unit di luar kamar operasi.

Keseragaman dalam pelayanan sedasi sesuai kebijakan dan prosedur yang ditetapkan dan dilaksanakan oleh tenaga medis yang kompeten dan telah diberikan kewenangan klinis untuk melakukan sedasi moderat dan dalam meliputi:

\begin{enumerate}
\def\labelenumi{\alph{enumi}.}
\tightlist
\item
  Area-area di dalam rumah sakit tempat sedasi moderat dan dalam dapat dilakukan;
\item
  Kualifikasi staf yang memberikan sedasi;
\item
  Persetujuan medis (informed consent) untuk prosedur maupun sedasinya;
\item
  Perbedaan populasi anak, dewasa, dan geriatri ataupun pertimbangan khusus lainnya;
\item
  Peralatan medis dan bahan yang digunakan sesuai dengan populasi yang diberikan sedasi moderat atau dalam; dan
\item
  Cara memantau.
\end{enumerate}

\hypertarget{elemen-penilaian-pab-3}{%
\subsubsection*{3. Elemen Penilaian PAB 3}\label{elemen-penilaian-pab-3}}
\addcontentsline{toc}{subsubsection}{3. Elemen Penilaian PAB 3}

\begin{enumerate}
\def\labelenumi{\alph{enumi}.}
\tightlist
\item
  Rumah sakit telah melaksanakan pemberian sedasi moderat dan dalam yang seragam di semua tempat di rumah sakit sesuai dengan poin a) - f) pada maksud dan tujuan.
\item
  Peralatan dan perbekalan gawat darurat tersedia di tempat dilakukan sedasi moderat dan dalam serta dipergunakan sesuai jenis sedasi, usia, dan kondisi pasien.
\item
  PPA yang terlatih dan berpengalaman dalam memberikan bantuan hidup lanjut (advance) harus selalu mendampingi dan siaga selama tindakan sedasi dikerjakan.
\end{enumerate}

\hypertarget{standar-pab-3.1}{%
\subsubsection*{4. Standar PAB 3.1}\label{standar-pab-3.1}}
\addcontentsline{toc}{subsubsection}{4. Standar PAB 3.1}

Tenaga medis yang kompeten dan berwenang memberikan pelayanan sedasi moderat dan dalam serta melaksanakan pemantauan.

\hypertarget{maksud-dan-tujuan-pab-3.1}{%
\subsubsection*{5. Maksud dan Tujuan PAB 3.1}\label{maksud-dan-tujuan-pab-3.1}}
\addcontentsline{toc}{subsubsection}{5. Maksud dan Tujuan PAB 3.1}

Kualifikasi tenaga medis yang diberikan kewenangan klinis untuk melakukan sedasi moderat dan dalam terhadap pasien sangat penting. Pemahaman metode pemberikan sedasi moderat dan dalam terkait kondisi pasien dan jenis tindakan yang diberikan dapat meningkatkan toleransi pasien terhadap rasa tidak nyaman, nyeri, dan atau risiko komplikasi.

Komplikasi terkait pemberian sedasi terutama gangguan jantung dan paru. Oleh sebab itu, diperlukan Sertifikasi bantuan hidup lanjut. Sebagai tambahan, pengetahuan farmakologi zat sedasi yang digunakan termasuk zat reversal mengurangi risiko terjadi kejadian yang tidak diharapkan. Oleh karena itu, tenaga medis yang diberikan kewenangan klinis memberikan sedasi moderat dan dalam harus kompeten dalam hal:

\begin{enumerate}
\def\labelenumi{\alph{enumi}.}
\tightlist
\item
  Teknik dan berbagai cara sedasi;
\item
  Farmakologi obat sedasi dan penggunaaan zat reversal (antidot);
\item
  Persyaratan pemantauan pasien; dan
\item
  Bertindak jika ada komplikasi.
\end{enumerate}

Tenaga medis yang melakukan prosedur sedasi harus mampu bertanggung jawab melakukan pemantauan terhadap pasien. PPA yang kompeten melakukan prosedur sedasi, seperti dokter spesialis anestesi atau perawat yang terlatih yang bertanggung jawab melakukan pemantauan berkesinambungan terhadap parameter fisiologis pasien dan membantu tindakan resusitasi. PPA yang bertanggung jawab melakukan pemantauan harus kompeten dalam:
a. Pemantauan yang diperlukan;
b. Bertindak jika ada komplikasi;
c.~Penggunaan zat reversal (antidot); dan
d.~Kriteria pemulihan.

\hypertarget{elemen-penilaian-pab-3.1}{%
\subsubsection*{6. Elemen Penilaian PAB 3.1}\label{elemen-penilaian-pab-3.1}}
\addcontentsline{toc}{subsubsection}{6. Elemen Penilaian PAB 3.1}

\begin{enumerate}
\def\labelenumi{\alph{enumi}.}
\tightlist
\item
  Tenaga medis yang diberikan kewenangan klinis memberikan sedasi moderat dan dalam harus kompeten dalam poin a) -- d) pada maksud dan tujuan.
\item
  Profesional pemberi asuhan (PPA) yang bertanggung jawab melakukan pemantauan selama pelayanan sedasi moderat dan dalam harus kompeten meliputi poin a) -- d) pada maksud dan tujuan.
\item
  Kompetensi semua PPA yang terlibat dalam sedasi moderat dan dalam tercatat di file kepegawaian.
\end{enumerate}

\hypertarget{standar-pab-3.2}{%
\subsubsection*{7. Standar PAB 3.2}\label{standar-pab-3.2}}
\addcontentsline{toc}{subsubsection}{7. Standar PAB 3.2}

Rumah sakit menetapkan panduan praktik klinis untuk pelayanan sedasi moderat dan dalam

\hypertarget{maksud-dan-tujuan-pab-3.2}{%
\subsubsection*{8. Maksud dan Tujuan PAB 3.2}\label{maksud-dan-tujuan-pab-3.2}}
\addcontentsline{toc}{subsubsection}{8. Maksud dan Tujuan PAB 3.2}

Tingkat kedalaman sedasi berlangsung dalam suatu kesinambungan mulai ringan sampai sedasi dalam dan pasien dapat berubah dari satu tingkat ke tingkat lainnya. Banyak faktor berpengaruh terhadap respons pasien dan hal ini memengaruhi tingkat sedasi pasien. Faktor-faktor tersebut termasuk obat-obatan yang diberikan, rute pemberian obat dan dosis, usia pasien (anak, dewasa, serta lanjut usia), dan riwayat kesehatan pasien. Misalnya, pasien memiliki riwayat gangguan organ utama maka kemungkinan obat yang digunakan pasien dapat berinteraksi dengan obat sedasi, alergi obat, efek samping obat sedasi atau anastesi sebelumnya. Jika status fisik pasien berisiko tinggi maka dipertimbangkan pemberian tambahan kebutuhan klinis lainnya dan diberikan tindakan sedasi yang sesuai.

Pengkajian prasedasi membantu mengidentifikasi faktor yang dapat yang berpengaruh pada respons pasien terhadap tindakan sedasi dan juga dapat diidentifikasi temuan-temuan penting dari hasil pemantaun selama dan sesudah sedasi.

Profesional pemberi asuhan (PPA) yang kompeten dan bertanggung jawab melakukan pengkajian prasedasi meliputi:

\begin{enumerate}
\def\labelenumi{\alph{enumi}.}
\tightlist
\item
  Mengidentifikasi masalah saluran pernapasan yang dapat memengaruhi jenis sedasi yang digunakan;
\item
  Mengevaluasi pasien terhadap risiko tindakan sedasi;
\item
  Merencanakan jenis sedasi dan tingkat kedalaman sedasi yang diperlukan pasien berdasarkan prosedur/tindakan yang akan dilakukan;
\item
  Pemberian sedasi secara aman; dan
\item
  Menyimpulkan temuan hasil pemantauan pasien selama prosedur sedasi dan pemulihan.
\end{enumerate}

Cakupan dan isi pengkajian dibuat berdasar atas Panduan Praktik Klinis dan kebijakan pelayanan anastesi dan sedasi yang ditetapkan oleh rumah sakit.

Pasien yang sedang menjalani tindakan sedasi dipantau tingkat kesadarannya, ventilasi dan status oksigenasi, variabel hemodinamik berdasar atas jenis obat sedasi yang diberikan, jangka waktu sedasi, jenis kelamin, dan kondisi pasien. Perhatian khusus ditujukan pada kemampuan pasien mempertahankan refleks protektif, jalan napas yang teratur dan lancar, serta respons terhadap stimulasi fisik dan perintah verbal. Seorang yang kompeten bertanggung jawab melakukan pemantauan status fisiologis pasien secara terus menerus dan membantu memberikan bantuan resusitasi sampai pasien pulih dengan selamat.

Setelah tindakan selesai dikerjakan, pasien masih tetap berisiko terhadap komplikasi karena keterlambatan absorsi obat sedasi, dapat terjadi depresi pernapasan, dan kekurangan stimulasi akibat tindakan.

Ditetapkan kriteria pemulihan untuk mengidentifikasi pasien yang sudah pulih kembali dan atau siap untuk ditransfer/dipulangkan.

\hypertarget{elemen-penilaian-pab-3.2}{%
\subsubsection*{9. Elemen Penilaian PAB 3.2}\label{elemen-penilaian-pab-3.2}}
\addcontentsline{toc}{subsubsection}{9. Elemen Penilaian PAB 3.2}

\begin{enumerate}
\def\labelenumi{\alph{enumi}.}
\item
  Rumah sakit telah menerapkan pengkajian prasedasi dan dicatat dalam rekam medis meliputi poin a) -- e) pada maksud dan tujuan.
\item
  Rumah sakit telah menerapakn pemantauan pasien selama dilakukan pelayanan sedasi moderat dan dalam oleh PPA yang kompeten dan di catat di rekam medik.
\item
  \begin{enumerate}
  \def\labelenumii{\alph{enumii})}
  \setcounter{enumii}{2}
  \tightlist
  \item
    Kriteria pemulihan telah digunakan dan didokumentasikan untuk mengidentifikasi pasien yang sudah pulih kembali dan atau siap untuk ditransfer/dipulangkan.
  \end{enumerate}
\end{enumerate}

\hypertarget{c.-pelayanan-anastesi}{%
\subsection*{c.~Pelayanan anastesi}\label{c.-pelayanan-anastesi}}
\addcontentsline{toc}{subsection}{c.~Pelayanan anastesi}

\hypertarget{standar-pab-4}{%
\subsubsection*{1. Standar PAB 4}\label{standar-pab-4}}
\addcontentsline{toc}{subsubsection}{1. Standar PAB 4}

Profesional pemberi asuhan (PPA) yang kompeten dan telah diberikan kewenangan klinis pelayanan anestesi melakukan asesmen pra-anestesi dan prainduksi.

\hypertarget{maksud-dan-tujuan-pab-4}{%
\subsubsection*{2. Maksud dan Tujuan PAB 4}\label{maksud-dan-tujuan-pab-4}}
\addcontentsline{toc}{subsubsection}{2. Maksud dan Tujuan PAB 4}

Oleh karena anestesi memiliki risiko tinggi maka pemberiannya harus direncanakan dengan hati-hati. Pengkajian pra-anestesi adalah dasar perencanaan ini untuk mengetahui temuan pemantauan selama anestesi dan pemulihan yang mungkin bermakna, dan juga untuk menentukan obat analgesi apa untuk pascaoperasi.

Pengkajian pra-anestesi juga memberikan informasi yang diperlukan untuk:

\begin{enumerate}
\def\labelenumi{\alph{enumi}.}
\tightlist
\item
  Mengetahui masalah saluran pernapasan;
\item
  Memilih anestesi dan rencana asuhan anestesi;
\item
  Memberikan anestesi yang aman berdasar atas pengkajian pasien, risiko yang ditemukan, dan jenis tindakan;
\item
  Menafsirkan temuan pada waktu pemantauan selama anestesi dan pemulihan; dan
\item
  Memberikan informasi obat analgesia yang akan digunakan pascaoperasi.
\end{enumerate}

Dokter spesialis anestesi akan melakukan pengkajian pra- anestesi yang dapat dilakukan sebelum masuk rawat inap atau sebelum dilakukan tindakan bedah atau sesaat menjelang operasi, misalnya pada pasien darurat.

Asesmen prainduksi terpisah dari asesmen pra-anestesi, karena difokuskan pada stabilitas fisiologis dan kesiapan pasien untuk tindakan anestesi, dan berlangsung sesaat sebelum induksi anestesi. Jika anestesi diberikan secara darurat maka pengkajian pra-anestesi dan prainduksi dapat dilakukan berurutan atau simultan, namun dicatat secara terpisah.

\hypertarget{elemen-penilaian-pab-4}{%
\subsubsection*{3. Elemen Penilaian PAB 4}\label{elemen-penilaian-pab-4}}
\addcontentsline{toc}{subsubsection}{3. Elemen Penilaian PAB 4}

\begin{enumerate}
\def\labelenumi{\alph{enumi}.}
\tightlist
\item
  Pengkajian pra-anestesi telah dilakukan untuk setiap pasien yang akan dilakukan anestesi.
\item
  Pengkajian prainduksi telah dilakukan secara terpisah untuk mengevaluasi ulang pasien segera sebelum induksi anestesi.
\item
  Kedua pengkajian tersebut telah dilakukan oleh PPA yang kompeten dan telah diberikan kewenangan klinis didokumentasikan dalam rekam medis pasien.
\end{enumerate}

\hypertarget{standar-pab-5}{%
\subsubsection*{4. Standar PAB 5}\label{standar-pab-5}}
\addcontentsline{toc}{subsubsection}{4. Standar PAB 5}

Risiko, manfaat, dan alternatif tindakan sedasi atau anestesi didiskusikan dengan pasien dan keluarga atau orang yang dapat membuat keputusan mewakili pasien sesuai dengan peraturan perundang-undangan.

\hypertarget{maksud-dan-tujuan-pab-5}{%
\subsubsection*{5. Maksud dan Tujuan PAB 5}\label{maksud-dan-tujuan-pab-5}}
\addcontentsline{toc}{subsubsection}{5. Maksud dan Tujuan PAB 5}

Rencana tindakan sedasi atau anastesi harus diinformasikan kepada pasien, keluarga pasien, atau mereka yang membuat keputusan mewakili pasien tentang jenis sedasi, risiko, manfaat, dan alternatif terkait tindakan tersebut. Informasi tersebut sebagai bagian dari proses mendapat persetujuan tindakan kedokteran untuk tindakan sedasi atau anestesi sesuai dengan peraturan perundang-undangan yang berlaku.

\hypertarget{elemen-penilaian-pab-5}{%
\subsubsection*{6. Elemen Penilaian PAB 5}\label{elemen-penilaian-pab-5}}
\addcontentsline{toc}{subsubsection}{6. Elemen Penilaian PAB 5}

\begin{enumerate}
\def\labelenumi{\alph{enumi}.}
\tightlist
\item
  Rumah sakit telah menerapkan pemberian informasi kepada pasien dan atau keluarga atau pihak yang akan memberikan keputusan tentang jenis, risiko, manfaat, alternatif dan analagsia pasca tindakan sedasi atau anastesi.
\item
  Pemberian informasi dilakukan oleh dokter spesialis anastesi dan didokumentasikan dalam formulir persetujuan tindakan anastesi/sedasi.
\end{enumerate}

\hypertarget{standar-pab-6}{%
\subsubsection*{7. Standar PAB 6}\label{standar-pab-6}}
\addcontentsline{toc}{subsubsection}{7. Standar PAB 6}

Status fisiologis setiap pasien selama tindakan sedasi atau anestesi dipantau sesuai dengan panduan praktik klinis (PPK) dan didokumentasikan dalam rekam medis pasien.

\hypertarget{maksud-dan-tujuan-pab-6}{%
\subsubsection*{8. Maksud dan Tujuan PAB 6}\label{maksud-dan-tujuan-pab-6}}
\addcontentsline{toc}{subsubsection}{8. Maksud dan Tujuan PAB 6}

Pemantauan fisiologis akan memberikan informasi mengenai status pasien selama tindakan anestesi (umum, spinal, regional dan lokal) dan masa pemulihan. Hasil pemantauan akan menjadi dasar untuk mengambil keputusan intraoperasi yang penting dan juga menjadi dasar pengambilan keputusan pascaoperasi seperti pembedahan ulang, pemindahan ke tingkat perawatan lain, atau pemulangan pasien.

Informasi hasil pemantauan akan memandu perawatan medis dan keperawatan serta mengidentifikasi kebutuhan diagnostik dan layanan lainnya. Temuan pemantauan dimasukkan ke dalam rekam medis pasien. Metode pemantauan bergantung pada status praanestesi pasien, pemilihan jenis tindakan anestesi, dan kerumitan pembedahan atau prosedur lainnya yang dilakukan selama tindakan anestesi. Meskipun demikian, pemantauan menyeluruh selama tindakan anestesi dan pembedahan dalam semua kasus harus sesuai dengan panduan praktik klinis (PPK) dan kebijakan rumah sakit. Hasilpemantauan didokumentasikan dalam rekam medis.

\hypertarget{elemen-penilaian-pab-6}{%
\subsubsection*{9. Elemen Penilaian PAB 6}\label{elemen-penilaian-pab-6}}
\addcontentsline{toc}{subsubsection}{9. Elemen Penilaian PAB 6}

\begin{enumerate}
\def\labelenumi{\alph{enumi}.}
\tightlist
\item
  Frekuensi dan jenis pemantauan selama tindakan anestesi dan pembedahan didasarkan pada status praanestesi pasien, anestesi yang digunakan, serta prosedur pembedahan yang dilakukan.
\item
  Pemantauan status fisiologis pasien sesuai dengan panduan praktik klinis (PPK) dan didokumentasikan dalam rekam medis pasien.
\end{enumerate}

\hypertarget{standar-pab.-6.1}{%
\subsubsection*{10. Standar PAB. 6.1}\label{standar-pab.-6.1}}
\addcontentsline{toc}{subsubsection}{10. Standar PAB. 6.1}

Status pasca anestesi pasien dipantau dan didokumentasikan, dan pasien dipindahkan/ditransfer/dipulangkan dari area pemulihan oleh PPA yang kompeten dengan menggunakan kriteria baku yang ditetapkan rumah sakit.

\hypertarget{maksud-dan-tujuan-pab-6.1}{%
\subsubsection*{11. Maksud dan Tujuan PAB 6.1}\label{maksud-dan-tujuan-pab-6.1}}
\addcontentsline{toc}{subsubsection}{11. Maksud dan Tujuan PAB 6.1}

Pemantauan selama anestesi menjadi dasar pemantauan saat pemulihan pascaanestesi. Pemantauan pasca anestesi dapat dilakukan di ruang rawat intensif atau di ruang pulih. Pemantauan pasca anestesi di ruang rawat intensif bisa direncanakan sejak awal sebelum tindakan operasi atau sebelumnya tidak direncanakan berubah dilakukan pemantauan di ruang intensif atas hasil keputusan PPA anestesi dan atau PPA bedah berdasarkan penilaian selama prosedur anestesi dan atau pembedahan.

Bila pemantauan pasca anestesi dilakukan di ruang intensif maka pasien langsung di transfer ke ruang rawat intensif dan tatalaksana pemantauan selanjutnya secara berkesinambungan dan sistematis berdasarkan instruksi DPJP di ruang rawat intensif serta didokumentasikan. Bila pemantauan dilakukan di ruang pulih maka pasien dipantau secara berkesinambungan dan sistematis serta didokumentasikan.

Pemindahan pasien dari area pemulihan pascaanestesi atau penghentian pemantauan pemulihan dilakukan dengan salah satu berdasarkan beberapa alternatif sebagai berikut:

\begin{enumerate}
\def\labelenumi{\alph{enumi}.}
\tightlist
\item
  pasien dipindahkan (atau pemantauan pemulihan dihentikan) oleh seorang ahli anestesi yang kompeten.
\item
  pasien dipindahkan (atau pemantauan pemulihan dihentikan) oleh seorang perawat atau penata anastesi yang kompeten berdasarkan kriteria pascaanestesi yang ditetapkan oleh rumah sakit, tercatat dalam rekam medis bahwa kriteria tersebut terpenuhi.
\item
  pasien dipindahkan ke unit yang mampu menyediakan perawatan pascaanestesi misalnya di unit perawatan intensif.
\item
  Waktu masuk dan keluar dari ruang pemulihan (atau waktu mulai dan dihentikannya pemantauan pemulihan) didokumentasikan dalam rekam medis pasien.
\end{enumerate}

\hypertarget{elemen-penilaian-pab-6.1}{%
\subsubsection*{12. Elemen Penilaian PAB 6.1}\label{elemen-penilaian-pab-6.1}}
\addcontentsline{toc}{subsubsection}{12. Elemen Penilaian PAB 6.1}

\begin{enumerate}
\def\labelenumi{\alph{enumi}.}
\tightlist
\item
  Rumah sakit telah menerapkan pemantauan pasien pascaanestesi baik di ruang intensif maupun di ruang pemulihan dan didokumentasikan dalam rekam medis pasien.
\item
  Pasien dipindahkan dari unit pascaanestesi (atau pemantauan pemulihan dihentikan) sesuai dengan kriteria baku yang ditetapkan dengan alternatif a) - c) pada maksud dan tujuan.
\item
  Waktu dimulai dan dihentikannya proses pemulihan dicatat di dalam rekam medis pasien.
\end{enumerate}

\hypertarget{d.-pelayanan-pembedahan}{%
\subsection*{d.~Pelayanan pembedahan}\label{d.-pelayanan-pembedahan}}
\addcontentsline{toc}{subsection}{d.~Pelayanan pembedahan}

\hypertarget{standar-pab-7}{%
\subsubsection*{1. Standar PAB 7}\label{standar-pab-7}}
\addcontentsline{toc}{subsubsection}{1. Standar PAB 7}

Asuhan setiap pasien bedah direncanakan berdasar atas hasil pengkajian dan dicatat dalam rekam medis pasien.

\hypertarget{maksud-dan-tujuan-pab-7}{%
\subsubsection*{2. Maksud dan Tujuan PAB 7}\label{maksud-dan-tujuan-pab-7}}
\addcontentsline{toc}{subsubsection}{2. Maksud dan Tujuan PAB 7}

Karena prosedur bedah mengandung risiko tinggi maka pelaksanaannya harus direncanakan dengan saksama. Pengkajian prabedah menjadi acuan untuk menentukan jenis tindakan bedah yang tepat dan mencatat temuan penting. Hasil pengkajian prabedah memberikan informasi tentang:

\begin{enumerate}
\def\labelenumi{\alph{enumi}.}
\tightlist
\item
  Tindakan bedah yang sesuai dan waktu pelaksanaannya;
\item
  Melakukan tindakan dengan aman; dan
\item
  Menyimpulkan temuan selama pemantauan.
\end{enumerate}

Pemilihan teknik operasi bergantung pada riwayat pasien, status fisik, data diagnostik, serta manfaat dan risiko tindakan yang dipilih. Untuk pasien yang saat masuk rumah sakit langsung dilayani oleh dokter bedah, pengkajian prabedah menggunakan formulir pengkajian awal rawat inap. Sedangkan pasien yang dikonsultasikan di tengah perawatan oleh dokter penanggung jawab pelayanan (DPJP) lain dan diputuskan operasi maka pengkajian prabedah dapat dicatat di rekam medis sesuai kebijakan rumah sakit. Hal ini termasuk diagnosis praoperasi dan pascaoperasi serta nama tindakan operasi.

\hypertarget{elemen-penilaian-pab-7}{%
\subsubsection*{3. Elemen Penilaian PAB 7}\label{elemen-penilaian-pab-7}}
\addcontentsline{toc}{subsubsection}{3. Elemen Penilaian PAB 7}

\begin{enumerate}
\def\labelenumi{\alph{enumi}.}
\tightlist
\item
  Rumah sakit telah menerapkan pengkajian prabedah pada pasien yang akan dioperasi oleh dokter penanggung jawab pelayanan (DPJP) sebelum operasi dimulai.
\item
  Diagnosis praoperasi dan rencana prosedur/tindakan operasi berdasarkan hasil pengkajian prabedah dan didokumentasikan di rekam medik.
\end{enumerate}

\hypertarget{standar-pab-7.1}{%
\subsubsection*{4. Standar PAB 7.1}\label{standar-pab-7.1}}
\addcontentsline{toc}{subsubsection}{4. Standar PAB 7.1}

Risiko, manfaat dan alternatif tindakan pembedahan didiskusikan dengan pasien dan atau keluarga atau pihak lain yang berwenang yang memberikan keputusan.

\hypertarget{maksud-dan-tujuan-pab-7.1}{%
\subsubsection*{5. Maksud dan Tujuan PAB 7.1}\label{maksud-dan-tujuan-pab-7.1}}
\addcontentsline{toc}{subsubsection}{5. Maksud dan Tujuan PAB 7.1}

Pasien, keluarga, dan mereka yang memutuskan mendapatkan penjelasan untuk berpartisipasi dalam keputusan asuhan pasien dengan memberikan persetujuan (consent).

Untuk memenuhi kebutuhan pasien maka penjelasan tersebut diberikan oleh dokter penanggung jawab pelayanan (DPJP) yang dalam keadaan darurat dapat dibantu oleh dokter di unit gawat darurat. Informasi yang disampaikan meliputi:

\begin{enumerate}
\def\labelenumi{\alph{enumi}.}
\tightlist
\item
  Risiko dari rencana tindakan operasi;
\item
  Manfaat dari rencana tindakan operasi;
\item
  Memungkinan komplikasi dan dampak;
\item
  Pilihan operasi atau nonoperasi (alternatif) yang tersedia untuk menangani pasien;
\item
  Sebagai tambahan jika dibutuhkan darah atau produk darah, sedangkan risiko dan alternatifnya didiskusikan.
\end{enumerate}

\hypertarget{elemen-penilaian-pab-7.1}{%
\subsubsection*{6. Elemen Penilaian PAB 7.1}\label{elemen-penilaian-pab-7.1}}
\addcontentsline{toc}{subsubsection}{6. Elemen Penilaian PAB 7.1}

\begin{enumerate}
\def\labelenumi{\alph{enumi}.}
\tightlist
\item
  Rumah sakit telah menerapkan pemberian informasi kepada pasien dan atau keluarga atau pihak yang akan memberikan keputusan tentang jenis, risiko, manfaat, komplikasi dan dampak serta alternatif prosedur/teknik terkait dengan rencana operasi (termasuk pemakaian produk darah bila diperlukan) kepada pasien dan atau keluarga atau mereka yang berwenang memberi keputusan.
\item
  Pemberian informasi dilakukan oleh dokter penanggung jawab pelayanan (DPJP) didokumentasikan dalam formulir persetujuan tindakan kedokteran.
\end{enumerate}

\hypertarget{standar-pab-7.2}{%
\subsubsection*{7. Standar PAB 7.2}\label{standar-pab-7.2}}
\addcontentsline{toc}{subsubsection}{7. Standar PAB 7.2}

Informasi yang terkait dengan operasi dicatat dalam laporan operasi dan digunakan untuk menyusun rencana asuhan lanjutan.

\hypertarget{maksud-dan-tujuan-pab-7.2}{%
\subsubsection*{8. Maksud dan Tujuan PAB 7.2}\label{maksud-dan-tujuan-pab-7.2}}
\addcontentsline{toc}{subsubsection}{8. Maksud dan Tujuan PAB 7.2}

Asuhan pasien pascaoperasi bergantung pada temuan dalam operasi. Hal yang terpenting adalah semua tindakan dan hasilnya dicatat di rekam medis pasien. Laporan ini dapat dibuat dalam bentuk format template atau dalam bentuk laporan operasi tertulis sesuai dengan regulasi rumah sakit. Laporan yang tercatat tentang operasi memuat paling sedikit:

\begin{enumerate}
\def\labelenumi{\alph{enumi}.}
\tightlist
\item
  Diagnosis pascaoperasi;
\item
  Nama dokter bedah dan asistennya;
\item
  Prosedur operasi yang dilakukan dan rincian temuan;
\item
  Ada dan tidak ada komplikasi;
\item
  Spesimen operasi yang dikirim untuk diperiksa;
\item
  Jumlah darah yang hilang dan jumlah yang masuk lewat transfusi;
\item
  Nomor pendaftaran alat yang dipasang (implan), (bila mempergunakan)
\item
  Tanggal, waktu, dan tanda tangan dokter yang bertanggung jawab.
\end{enumerate}

\hypertarget{elemen-penilaian-pab-7.2}{%
\subsubsection*{9. Elemen Penilaian PAB 7.2}\label{elemen-penilaian-pab-7.2}}
\addcontentsline{toc}{subsubsection}{9. Elemen Penilaian PAB 7.2}

\begin{enumerate}
\def\labelenumi{\alph{enumi}.}
\tightlist
\item
  Laporan operasi memuat poin a) -- h) pada maksud dan tujuan serta dicatat pada formular/template yang ditetapkan rumah sakit.
\item
  Laporan operasi telah tersedia segera setelah operasi selesai dan sebelum pasien dipindah ke ruang lain untuk perawatan selanjutnya.
\end{enumerate}

\hypertarget{standar-pab.-7.3}{%
\subsubsection*{10. Standar PAB. 7.3}\label{standar-pab.-7.3}}
\addcontentsline{toc}{subsubsection}{10. Standar PAB. 7.3}

Rencana asuhan pascaoperasi disusun, ditetapkan dan dicatat dalam rekam medis.

\hypertarget{maksud-dan-tujuan-pab-7.3}{%
\subsubsection*{11. Maksud dan Tujuan PAB 7.3}\label{maksud-dan-tujuan-pab-7.3}}
\addcontentsline{toc}{subsubsection}{11. Maksud dan Tujuan PAB 7.3}

Kebutuhan asuhan medis, keperawatan, dan professional pemberi asuhan (PPA) lainnya sesuai dengan kebutuhan setiap pasien pascaoperasi berbeda bergantung pada tindakan operasi dan riwayat kesehatan pasien. Beberapa pasien mungkin membutuhkan pelayanan dari profesional pemberi asuhan (PPA) lain atau unit lain seperti rehabilitasi medik atau terapi fisik. Penting membuat rencana asuhan tersebut termasuk tingkat asuhan, metode asuhan, tindak lanjut monitor atau tindak lanjut tindakan, kebutuhan obat, dan asuhan lain atau tindakan serta layanan lain. Rencana asuhan pascaoperasi dapat dimulai sebelum tindakan operasi berdasarkan asesmen kebutuhan dan kondisi pasien serta jenis operasi yang dilakukan.
Rencana asuhan pasca operasi juga memuat kebutuhan pasien yang segera. Rencana asuhan dicacat di rekam medik pasien dalam waktu 24 jam dan diverifikasi oleh dokter penanggung jawab pelayanan (DPJP) sebagai pimpinan tim klinis untuk memastikan kontuinitas asuhan selama waktu pemulihan dan masa rehabilitasi.

\hypertarget{elemen-penilaian-pab-7.3}{%
\subsubsection*{12. Elemen Penilaian PAB 7.3}\label{elemen-penilaian-pab-7.3}}
\addcontentsline{toc}{subsubsection}{12. Elemen Penilaian PAB 7.3}

\begin{enumerate}
\def\labelenumi{\alph{enumi}.}
\tightlist
\item
  Rencana asuhan pascaoperasi dicatat di rekam medis pasien dalam waktu 24 jam oleh dokter penanggung jawab pelayanan (DPJP).
\item
  Rencana asuhan pascaoperasi termasuk rencana asuhan medis, keperawatan, oleh PPA lainnya berdasar atas kebutuhan pasien.
\item
  Rencana asuhan pascaoperasi diubah berdasarkan pengkajian ulang pasien.
\end{enumerate}

\hypertarget{standar-pab-7.4}{%
\subsubsection*{13. Standar PAB 7.4}\label{standar-pab-7.4}}
\addcontentsline{toc}{subsubsection}{13. Standar PAB 7.4}

Perawatan bedah yang mencakup implantasi alat medis direncanakan dengan pertimbangan khusus tentang bagaimana memodifikasi proses dan prosedur standar.

\hypertarget{maksud-dan-tujuan-pab-7.4}{%
\subsubsection*{14. Maksud dan Tujuan PAB 7.4}\label{maksud-dan-tujuan-pab-7.4}}
\addcontentsline{toc}{subsubsection}{14. Maksud dan Tujuan PAB 7.4}

Banyak tindakan bedah menggunakan implan yang menetap/permanen maupun temporer antara lain panggul/lutut prostetik, pacu jantung, pompa insulin. Tindakan operasi seperti ini mengharuskan tindakan operasi rutin yang dimodifikasi dgn mempertimbangkan faktor khusus seperti:

\begin{enumerate}
\def\labelenumi{\alph{enumi}.}
\tightlist
\item
  Pemilihan implan berdasarkan peraturan perundangan.
\item
  Modifikasi surgical safety checklist utk memastikan ketersediaan implan di kamar operasi dan pertimbangan khusus utk penandaan lokasi operasi.
\item
  Kualifikasi dan pelatihan setiap staf dari luar yang dibutuhkan untuk pemasangan implan (staf dari pabrik/perusahaan implan untukmengkalibrasi).
\item
  Proses pelaporan jika ada kejadian yang tidak diharapkan terkait implant.
\item
  Proses pelaporan malfungsi implan sesuai dgn standar/aturan pabrik.
\item
  Pertimbangan pengendalian infeksi yang khusus.
\item
  Instruksi khusus kepada pasien setelah operasi.
\item
  kemampuan penelusuran (traceability) alat jika terjadi penarikan kembali (recall) alat medis misalnya dengan menempelkan barcode alat di rekam medis.
\end{enumerate}

\hypertarget{elemen-penilaian-pab-7.4}{%
\subsubsection*{15. Elemen Penilaian PAB 7.4}\label{elemen-penilaian-pab-7.4}}
\addcontentsline{toc}{subsubsection}{15. Elemen Penilaian PAB 7.4}

\begin{enumerate}
\def\labelenumi{\alph{enumi}.}
\tightlist
\item
  Rumah sakit telah mengidentifikasi jenis alat implan yang termasuk dalam cakupan layanannya.
\item
  Kebijakan dan praktik mencakup poin a) -- h) pada maksud dan tujuan.
\item
  Rumah sakit mempunyai proses untuk melacak implan medis yang telah digunakan pasien.
\item
  Rumah sakit menerapkan proses untuk menghubungi dan memantau pasien dalam jangka waktu yang ditentukan setelah menerima pemberitahuan adanya penarikan/recall suatu implan medis.
\end{enumerate}

\hypertarget{pelayanan-kefarmasian-dan-penggunaan-obat-pkpo}{%
\section*{6. Pelayanan Kefarmasian dan Penggunaan Obat (PKPO)}\label{pelayanan-kefarmasian-dan-penggunaan-obat-pkpo}}
\addcontentsline{toc}{section}{6. Pelayanan Kefarmasian dan Penggunaan Obat (PKPO)}

\textbf{Gambaran Umum}

Pelayanan kefarmasian dan penggunaan obat merupakan bagian penting dalam pelayanan pasien. Pelayanan kefarmasian yang diselenggarakan di rumah sakit harus mampu menjamin ketersediaan obat dan alat kesehatan yang bermutu, bermanfaat, aman, dan terjangkau untuk memenuhi kebutuhan pasien.

Standar Pelayanan Kefarmasian meliputi pengelolaan sediaan farmasi, alat kesehatan dan bahan medis habis pakai (BMHP), serta pelayanan farmasi klinik. Pengaturan Standar Pelayanan Kefarmasian di rumah sakit bertujuan untuk:

\begin{enumerate}
\def\labelenumi{\arabic{enumi}.}
\tightlist
\item
  Meningkatkan mutu pelayanan kefarmasian;
\item
  Menjamin kepastian hukum bagi tenaga kefarmasian; dan
\item
  Melindungi pasien dan masyarakat dari penggunaan obat yang tidak rasional dalam rangka keselamatan pasien (patient safety).
\end{enumerate}

Pada bab ini penilaian terhadap pelayananan kefarmasian difokuskan pada sediaan farmasi dan BMHP.

Obat merupakan komponen penting dalam pengobatan simptomatik, preventif, kuratif, paliatif dan rehabilitatif terhadap penyakit dan berbagai kondisi. Proses penggunaan obat yang mencakup peresepan, penyiapan (dispensing), pemberian dan pemantauan dilakukan secara multidisipliner dan terkoordinasi sehingga dapat menjamin penggunaan obat yang aman dan efektif.

Sistem pelayanan kefarmasian dan penggunaan obat di rumah sakit dirancang, diimplementasikan, dan dilakukan peningkatan mutu secara berkesinambungan terhadap proses-proses: pemilihan, perencanaan dan pengadaan, penyimpanan, pendistribusian, peresepan/permintaan obat/instruksi pengobatan, penyalinan (transcribing), penyiapan, pemberian dan pemantauan terapi obat.

Kejadian kesalahan obat (medication error) merupakan penyebab utama cedera pada pasien yang seharusnya dapat dicegah. Untuk meningkatkan keselamatan pasien, rumah sakit harus berupaya mengurangi terjadinya kesalahan obat dengan membuat sistem pelayanan kefarmasian dan penggunaan obat yang lebih aman (medication safety).

Masalah resistansi antimikroba merupakan masalah global yang disebabkan penggunaan antimikroba yang berlebihan dan tidak tepat. Untuk mengurangi laju resistansi antimikroba dan meningkatkan patient outcome, maka rumah sakit harus melaksanakan program pengendalian resistansi antimikroba sesuai peraturan perundang-undangan. Salah satu program kerja yang harus dilakukan adalah optimalisasi penggunaan antimikroba secara bijak melalui penerapan penatagunaan antimikroba (PGA).

\hypertarget{a.-pengorganisasian}{%
\subsection*{a. Pengorganisasian}\label{a.-pengorganisasian}}
\addcontentsline{toc}{subsection}{a. Pengorganisasian}

\hypertarget{standar-pkpo-1}{%
\subsubsection*{1. Standar PKPO 1}\label{standar-pkpo-1}}
\addcontentsline{toc}{subsubsection}{1. Standar PKPO 1}

Sistem pelayanan kefarmasian dan penggunaan obat dikelola untuk memenuhi kebutuhan pasien sesuai dengan peraturan perundang-undangan.

\hypertarget{maksud-dan-tujuan-pkpo-1}{%
\subsubsection*{2. Maksud dan Tujuan PKPO 1}\label{maksud-dan-tujuan-pkpo-1}}
\addcontentsline{toc}{subsubsection}{2. Maksud dan Tujuan PKPO 1}

Rumah sakit menetapkan dan menerapkan sistem pelayanan kefarmasian dan penggunaan obat yang meliputi:

\begin{enumerate}
\def\labelenumi{\alph{enumi}.}
\tightlist
\item
  Perencanaan sistem pelayanan kefarmasian dan penggunaan obat.
\item
  Pemilihan.
\item
  Perencanaan dan pengadaan sediaan farmasi dan BMHP.
\item
  Penyimpanan.
\item
  Pendistribusian.
\item
  Peresepan/permintaan obat/instruksi pengobatan.
\item
  Penyiapan (dispensing).
\item
  Pemberian.
\item
  Pemantauan terapi obat.
\end{enumerate}

Untuk memastikan efektivitas sistem pelayanan kefarmasian dan penggunaan obat, maka rumah sakit melakukan kajian sekurang-kurangnya sekali setahun. Kajian tahunan dilakukan dengan mengumpulkan semua informasi dan pengalaman yang berhubungan dengan pelayanan kefarmasian dan penggunaan obat, termasuk jumlah laporan insiden kesalahan obat serta upaya untuk menurunkannya.

Pelaksanaan kajian melibatkan Komite/Tim Farmasi dan Terapi, Komite/ Tim Penyelenggara Mutu, serta unit kerja terkait. Kajian bertujuan agar rumah sakit memahami kebutuhan dan prioritas perbaikan sistem berkelanjutan. Kajian meliputi proses-proses poin a) sampai dengan i), termasuk insiden kesalahan obat (medication error).

Pelayanan kefarmasian dipimpin oleh apoteker yang memiliki izin dan kompeten dalam melakukan supervisi semua aktivitas pelayanan kefarmasian dan penggunaan obat di rumah sakit. Pelayanan kefarmasian dan penggunaan obat bukan hanya tanggung jawab apoteker, tetapi juga staf lainnya yang terlibat, misalnya dokter, perawat, tenaga teknis kefarmasian, staf non klinis. Struktur organisasi dan tata hubungan kerja operasional pelayanan kefarmasian dan penggunaan obat di rumah sakit mengacu pada peraturan perundang-undangan.

Rumah sakit harus menyediakan sumber informasi yang dibutuhkan staf yang terlibat dalam pelayanan kefarmasian dan penggunaan obat, misalnya informasi tentang dosis, interaksi obat, efek samping obat, stabilitas dan kompatibilitas dalam bentuk cetak dan/atau elektronik.

\hypertarget{elemen-penilaian-pkpo-1}{%
\subsubsection*{3. Elemen Penilaian PKPO 1}\label{elemen-penilaian-pkpo-1}}
\addcontentsline{toc}{subsubsection}{3. Elemen Penilaian PKPO 1}

\begin{enumerate}
\def\labelenumi{\alph{enumi}.}
\tightlist
\item
  Rumah sakit telah menetapkan regulasi tentang sistem pelayanan kefarmasian dan penggunaan obat, termasuk pengorganisasiannya sesuai dengan peraturan perundang-undangan.
\item
  Rumah sakit memiliki bukti seluruh apoteker memiliki izin dan kompeten, serta telah melakukan supervisi pelayanan kefarmasian dan memastikan kepatuhan terhadap peraturan perundang- undangan.
\item
  Rumah sakit memiliki bukti kajian sistem pelayanan kefarmasian dan penggunaan obat yang dilakukan setiap tahun.
\item
  Rumah sakit memiliki sumber informasi obat untuk semua staf yang terlibat dalam penggunaan obat.
\end{enumerate}

\hypertarget{b.-pemilihan-perencanaan-dan-pengadaan}{%
\subsection*{b. Pemilihan, Perencanaan, dan Pengadaan}\label{b.-pemilihan-perencanaan-dan-pengadaan}}
\addcontentsline{toc}{subsection}{b. Pemilihan, Perencanaan, dan Pengadaan}

\hypertarget{standar-pkpo-2}{%
\subsubsection*{1. Standar PKPO 2}\label{standar-pkpo-2}}
\addcontentsline{toc}{subsubsection}{1. Standar PKPO 2}

Rumah sakit menetapkan dan menerapkan formularium yang digunakan untuk peresepan/permintaan obat/instruksi pengobatan. Obat dalam formularium senantiasa tersedia di rumah sakit.

\hypertarget{maksud-dan-tujuan-pkpo-2}{%
\subsubsection*{2. Maksud dan Tujuan PKPO 2}\label{maksud-dan-tujuan-pkpo-2}}
\addcontentsline{toc}{subsubsection}{2. Maksud dan Tujuan PKPO 2}

Rumah sakit menetapkan formularium obat mengacu pada peraturan perundang-undangan. Formularium ini didasarkan atas misi rumah sakit, kebutuhan pasien, dan jenis pelayanan yang diberikan. Penyusunan formularium merupakan suatu proses kolaboratif mempertimbangkan kebutuhan, keselamatan pasien dan aspek biaya.

Formularium harus dijadikan acuan dan dipatuhi dalam peresepan dan pengadaan obat. Komite/Tim Farmasi dan Terapi melakukan evaluasi terhadap formularium rumah sakit sekurang-kurangnya setahun sekali dengan mempertimbangkan efektivitas, keamanan dan biaya. Rumah sakit merencanakan kebutuhan obat, dan BMHP dengan baik agar tidak terjadi kekosongan yang dapat menghambat pelayanan.

Apabila terjadi kekosongan, maka tenaga kefarmasian harus menginformasikan kepada profesional pemberi asuhan (PPA) serta saran substitusinya. Rumah sakit menetapkan dan menerapkan regulasi pengadaan sediaan farmasi dan BMHP yang melibatkan apoteker untuk memastikan proses berjalan sesuai dengan peraturan perundang-undangan.

\hypertarget{elemen-penilaian-pkpo-2}{%
\subsubsection*{3. Elemen Penilaian PKPO 2}\label{elemen-penilaian-pkpo-2}}
\addcontentsline{toc}{subsubsection}{3. Elemen Penilaian PKPO 2}

\begin{enumerate}
\def\labelenumi{\alph{enumi}.}
\tightlist
\item
  Rumah sakit telah memiliki proses penyusunan formularium rumah sakit secara kolaboratif.
\item
  Rumah sakit melakukan pemantauan kepatuhan terhadap formularium baik dari persediaan maupun penggunaannya.
\item
  Rumah sakit melakukan evaluasi terhadap formularium sekurang-kurangnya setahun sekali berdasarkan informasi tentang efektivitas, keamanan dan biaya.
\item
  Rumah sakit melakukan pelaksanaan dan evaluasi terhadap perencanaan dan pengadaan sediaan farmasi, dan BMHP.
\item
  Rumah sakit melakukan pengadaan sediaan farmasi, dan BMHP melibatkan apoteker untuk memastikan proses berjalan sesuai peraturan perundang- undangan.
\end{enumerate}

\hypertarget{c.-penyimpanan}{%
\subsection*{c.~Penyimpanan}\label{c.-penyimpanan}}
\addcontentsline{toc}{subsection}{c.~Penyimpanan}

\hypertarget{standar-pkpo-3}{%
\subsubsection*{1. Standar PKPO 3}\label{standar-pkpo-3}}
\addcontentsline{toc}{subsubsection}{1. Standar PKPO 3}

Rumah sakit menetapkan dan menerapkan regulasi penyimpanan sediaan farmasi dan BMHP disimpan dengan benar dan aman sesuai peraturan perundang-undang dan standar profesi.

\hypertarget{standar-pkpo-3.1}{%
\subsubsection*{2. Standar PKPO 3.1}\label{standar-pkpo-3.1}}
\addcontentsline{toc}{subsubsection}{2. Standar PKPO 3.1}

Rumah sakit menetapkan dan menerapkan regulasi pengelolaan obat atau produk yang memerlukan penanganan khusus, misalnya obat dan bahan berbahaya, radioaktif, obat penelitian, produk nutrisi parenteral, obat/BMHP dari program/donasi sesuai peraturan perundang-undangan.

\hypertarget{standar-pkpo-3.2}{%
\subsubsection*{3. Standar PKPO 3.2}\label{standar-pkpo-3.2}}
\addcontentsline{toc}{subsubsection}{3. Standar PKPO 3.2}

Rumah sakit menetapkan dan menerapkan regulasi pengelolaan obat, dan BMHP untuk kondisi emergensi yang disimpan di luar Instalasi Farmasi untuk memastikan selalu tersedia, dimonitor dan aman.

\hypertarget{standar-pkpo-3.3}{%
\subsubsection*{4. Standar PKPO 3.3}\label{standar-pkpo-3.3}}
\addcontentsline{toc}{subsubsection}{4. Standar PKPO 3.3}

Rumah sakit menetapkan dan menerapkan regulasi penarikan kembali (recall) dan pemusnahan sediaan farmasi, BMHP dan implan sesuai peraturan perundang- undangan.

\hypertarget{maksud-dan-tujuan-pkpo-3-pkpo-3.1-pkpo-3.2-pkpo-3.3}{%
\subsubsection*{5. Maksud dan Tujuan PKPO 3, PKPO 3.1, PKPO 3.2, PKPO 3.3}\label{maksud-dan-tujuan-pkpo-3-pkpo-3.1-pkpo-3.2-pkpo-3.3}}
\addcontentsline{toc}{subsubsection}{5. Maksud dan Tujuan PKPO 3, PKPO 3.1, PKPO 3.2, PKPO 3.3}

Rumah sakit mempunyai ruang penyimpanan sediaan farmasi dan BMHP yang disesuaikan dengan kebutuhan, serta memperhatikan persyaratan penyimpanan dari produsen, kondisi sanitasi, suhu, cahaya, kelembaban, ventilasi, dan memiliki system keamanan penyimpanan yang bertujuan untuk menjamin mutu dan keamanan produk serta keselamatan staf.

Beberapa sediaan farmasi harus disimpan dengan cara khusus, yaitu:

\begin{enumerate}
\def\labelenumi{\alph{enumi}.}
\tightlist
\item
  Bahan berbahaya dan beracun (B3) disimpan sesuai sifat dan risiko bahan agar dapat mencegah staf dan lingkungan dari risiko terpapar bahan berbahaya dan beracun, atau mencegah terjadinya bahaya seperti kebakaran.
\item
  Narkotika dan psikotropika harus disimpan dengan cara yang dapat mencegah risiko kehilangan obat yang berpotensi disalahgunakan (drug abuse). Penyimpanan dan pelaporan penggunaan narkotika dan psikotropika dilakukan sesuai peraturan perundang-undangan.
\item
  Elektrolit konsentrat dan elektrolit dengan konsentrasi tertentu diatur penyimpanannya agar tidak salah dalam pengambilan.
\item
  Obat emergensi diatur penyimpanannya agar selalu siap pakai bila sewaktu-waktu diperlukan. Ketersediaan dan kemudahan akses terhadap obat, dan BMHP pada kondisi emergensi sangat menentukan penyelamatan jiwa pasien. Oleh karena itu rumah sakit harus menetapkan lokasi penempatan troli/tas/lemari/kotak berisi khusus obat, dan BMHP emergensi, termasuk di ambulans. Pengelolaan obat dan BMHP emergensi harus sama/seragam di seluruh rumah sakit dalam hal penyimpanan (termasuk tata letaknya), pemantauan dan pemeliharaannya. Rumah sakit menerapkan tata laksana obat emergensi untuk meningkatkan ketepatan dan kecepatan pemberian obat, misalnya:
\end{enumerate}

\begin{enumerate}
\def\labelenumi{\arabic{enumi}.}
\tightlist
\item
  Penyimpanan obat emergensi harus sudah dikeluarkan dari kotak kemasannya agar tidak menghambat kecepatan penyiapan dan pemberian obat, misalnya: obat dalam bentuk ampul atau vial.
\item
  Pemisahan penempatan BMHP untuk pasien dewasa dan pasien anak.
\item
  Tata letak obat yang seragam.
\item
  Tersedia panduan cepat untuk dosis dan penyiapan obat.
\end{enumerate}

Beberapa sediaan farmasi memiliki risiko khusus yang memerlukan ketentuan tersendiri dalam penyimpanan, pelabelan dan pengawasan penggunaannya, yaitu:

\begin{enumerate}
\def\labelenumi{\alph{enumi}.}
\tightlist
\item
  Produk nutrisi parenteral dikelola sesuai stabilitas produk;
\item
  Obat/bahan radioaktif dikelola sesuai sifat dan bahan radioaktif;
\item
  Obat yang dibawa pasien;
\item
  Obat/BMHP dari program atau bantuan pemerintah/pihak lain dikelola sesuai peraturan perundang-undangan dan pedoman; dan
\item
  Obat yang digunakan untuk penelitian dikelola sesuai protokol penelitian.
\end{enumerate}

Obat dan zat kimia yang digunakan untuk peracikan obat harus diberi label yang memuat informasi nama, kadar/kekuatan, tanggal kedaluwarsa dan peringatan khusus untuk menghindari kesalahan dalam penyimpanan dan penggunaannya.

Apoteker melakukan supervisi secara rutin ke lokasi penyimpanan sediaan farmasi dan BMHP, untuk memastikan penyimpanannya dilakukan dengan benar dan aman.

Rumah sakit harus memiliki sistem yang menjamin bahwa sediaan farmasi dan BMHP yang tidak layak pakai karena rusak, mutu substandar atau kedaluwarsa tidak digunakan serta dimusnahkan.

Obat yang sudah dibuka dari kemasan primer (wadah yang bersentuhan langsung dengan obat) atau sudah dilakukan perubahan, misalnya: dipindahkan dari wadah aslinya, sudah dilakukan peracikan, maka tanggal kedaluwarsanya (ED=Expired Date) tidak lagi mengikuti tanggal kedaluwarsa dari pabrik yang tertera di kemasan obat. Rumah sakit harus menetapkan tanggal kedaluwarsa sediaan obat tersebut (BUD=Beyond Use Date). BUD harus dicantumkan pada label obat.

Rumah sakit memiliki sistem pelaporan obat dan BMHP yang substandar (rusak) untuk perbaikan dan peningkatan mutu.

Obat yang ditarik dari peredaran (recall) dapat disebabkan mutu produk substandar atau obat berpotensi menimbulkan efek yang membahayakan pasien. Inisiatif recall dapat dilakukan oleh produsen secara sukarela atau oleh Badan POM. Rumah sakit harus memiliki sistem penarikan kembali (recall) yang meliputi identifikasi keberadaan obat yang di-recall di semua lokasi penyimpanan di rumah sakit, penarikan dari semua lokasi penyimpanan, dan pengembaliannya ke distributor. Rumah sakit memastikan bahwa proses recall dikomunikasikan dan dilaksanakan secepatnya untuk mencegah digunakannya produk yang di-recall.

\hypertarget{elemen-penilaian-pkpo-3}{%
\subsubsection*{6. Elemen Penilaian PKPO 3}\label{elemen-penilaian-pkpo-3}}
\addcontentsline{toc}{subsubsection}{6. Elemen Penilaian PKPO 3}

\begin{enumerate}
\def\labelenumi{\alph{enumi}.}
\tightlist
\item
  Sediaan farmasi dan BMHP disimpan dengan benar dan aman dalam kondisi yang sesuai untuk stabilitas produk, termasuk yang disimpan di luar Instalasi Farmasi.
\item
  Narkotika dan psikotropika disimpan dan dilaporkan penggunaannya sesuai peraturan perundang- undangan.
\item
  Rumah sakit melaksanakan supervisi secara rutin oleh apoteker untuk memastikan penyimpanan sediaan farmasi dan BMHP dilakukan dengan benar dan aman.
\item
  Obat dan zat kimia yang digunakan untuk peracikan obat diberi label secara akurat yang terdiri atas nama zat dan kadarnya, tanggal kedaluwarsa, dan peringatan khusus.
\end{enumerate}

\hypertarget{elemen-penilaian-pkpo-3.1}{%
\subsubsection*{7. Elemen Penilaian PKPO 3.1}\label{elemen-penilaian-pkpo-3.1}}
\addcontentsline{toc}{subsubsection}{7. Elemen Penilaian PKPO 3.1}

\begin{enumerate}
\def\labelenumi{\alph{enumi}.}
\tightlist
\item
  Obat yang memerlukan penanganan khusus dan bahan berbahaya dikelola sesuai sifat dan risiko bahan.
\item
  Radioaktif dikelola sesuai sifat dan risiko bahan radioaktif.
\item
  Obat penelitian dikelola sesuai protokol penelitian.
\item
  Produk nutrisi parenteral dikelola sesuai stabilitas produk.
\item
  Obat/BMHP dari program/donasi dikelola sesuai peraturan perundang-undangan dan pedoman terkait.
\end{enumerate}

\hypertarget{elemen-penilaian-pkpo-3.2}{%
\subsubsection*{8. Elemen Penilaian PKPO 3.2}\label{elemen-penilaian-pkpo-3.2}}
\addcontentsline{toc}{subsubsection}{8. Elemen Penilaian PKPO 3.2}

\begin{enumerate}
\def\labelenumi{\alph{enumi}.}
\tightlist
\item
  Obat dan BMHP untuk kondisi emergensi yang tersimpan di luar Instalasi Farmasi termasuk di ambulans dikelola secara seragam dalam hal penyimpanan, pemantauan, penggantian karena digunakan, rusak atau kedaluwarsa, dan dilindungi dari kehilangan dan pencurian.
\item
  Rumah sakit menerapkan tata laksana obat emergensi untuk meningkatkan ketepatan dan kecepatan pemberian obat.
\end{enumerate}

\hypertarget{elemen-penilaian-pkpo-3.3}{%
\subsubsection*{9. Elemen Penilaian PKPO 3.3}\label{elemen-penilaian-pkpo-3.3}}
\addcontentsline{toc}{subsubsection}{9. Elemen Penilaian PKPO 3.3}

\begin{enumerate}
\def\labelenumi{\alph{enumi}.}
\tightlist
\item
  Batas waktu obat dapat digunakan (beyond use date) tercantum pada label obat.
\item
  Rumah sakit memiliki sistem pelaporan sediaan farmasi dan BMHP substandar (rusak).
\item
  Rumah sakit menerapkan proses recall obat, BMHP dan implan yang meliputi identifikasi, penarikan, dan pengembalian produk yang di-recall.
\item
  Rumah sakit menerapkan proses pemusnahan sediaan farmasi dan BMHP.
\end{enumerate}

\hypertarget{d.-peresepan}{%
\subsection*{d.~Peresepan}\label{d.-peresepan}}
\addcontentsline{toc}{subsection}{d.~Peresepan}

\hypertarget{standar-pkpo-4}{%
\subsubsection*{1. Standar PKPO 4}\label{standar-pkpo-4}}
\addcontentsline{toc}{subsubsection}{1. Standar PKPO 4}

Rumah sakit menetapkan dan menerapkan regulasi rekonsiliasi obat.

\hypertarget{maksud-dan-tujuan-pkpo-4}{%
\subsubsection*{2. Maksud dan Tujuan PKPO 4}\label{maksud-dan-tujuan-pkpo-4}}
\addcontentsline{toc}{subsubsection}{2. Maksud dan Tujuan PKPO 4}

Pasien yang dirawat di rumah sakit mungkin sebelum masuk rumah sakit sedang menggunakan obat baik obat resep maupun non resep. Adanya diskrepansi (perbedaan) terapi obat yang diterima pasien sebelum dirawat dan saat dirawat dapat membahayakan kesehatan pasien. Kajian sistematik yang dilakukan oleh Cochrane pada tahun 2018 menunjukkan 55,9\% pasien berisiko mengalami diskrepansi terapi obat saat perpindahan perawatan (transition of care).

Untuk mencegah terjadinya kesalahan obat (medication error) akibat adanya diskrepansi tersebut, maka rumah sakit harus menetapkan dan menerapkan proses rekonsiliasi obat. Rekonsiliasi obat di rumah sakit adalah proses membandingkan daftar obat yang digunakan oleh pasien sebelum masuk rumah sakit dengan obat yang diresepkan pertama kali sejak pasien masuk, saat pindah antar unit pelayanan (transfer) di dalam rumah sakit dan sebelum pasien pulang.

Rekonsiliasi obat merupakan proses kolaboratif yang dilakukan oleh dokter, apoteker dan perawat, serta melibatkan pasien/keluarga. Rekonsiliasi obat dimulai dengan menelusuri riwayat penggunaan obat pasien sebelum masuk rumah sakit, kemudian membandingkan daftar obat tersebut dengan obat yang baru diresepkan saat perawatan. Jika ada diskrepansi, maka dokter yang merawat memutuskan apakah terapi obat yang digunakan oleh pasien sebelum masuk rumah sakit akan dilanjutkan atau tidak. Hasil rekonsiliasi obat didokumentasikan dan dikomunikasikan kepada profesional pemberi asuhan (PPA) terkait dan pasien/keluarga.

Kajian sistematik membuktikan bahwa rekonsiliasi obat dapat menurunkan diskrepansi dan kejadian yang tidak diharapkan terkait penggunaan obat (adverse drug event).

\hypertarget{elemen-penilaian-pkpo-4}{%
\subsubsection*{3. Elemen Penilaian PKPO 4}\label{elemen-penilaian-pkpo-4}}
\addcontentsline{toc}{subsubsection}{3. Elemen Penilaian PKPO 4}

\begin{enumerate}
\def\labelenumi{\alph{enumi}.}
\tightlist
\item
  Rumah sakit menerapkan rekonsiliasi obat saat pasien masuk rumah sakit, pindah antar unit pelayanan di dalam rumah sakit dan sebelum pasien pulang.
\item
  Hasil rekonsiliasi obat didokumentasikan di rekam medis.
\end{enumerate}

\hypertarget{standar-pkpo-4.1}{%
\subsubsection*{4. Standar PKPO 4.1}\label{standar-pkpo-4.1}}
\addcontentsline{toc}{subsubsection}{4. Standar PKPO 4.1}

Rumah sakit menetapkan dan menerapkan regulasi peresepan/permintaan obat dan BMHP/instruksi pengobatan sesuai peraturan perundang-undangan.

\hypertarget{maksud-dan-tujuan-pkpo-4.1}{%
\subsubsection*{5. Maksud dan Tujuan PKPO 4.1}\label{maksud-dan-tujuan-pkpo-4.1}}
\addcontentsline{toc}{subsubsection}{5. Maksud dan Tujuan PKPO 4.1}

Di banyak hasil penelitian, kesalahan obat (medication error) yang tersering terjadi di tahap peresepan. Jenis kesalahan peresepan antara lain: resep yang tidak lengkap, ketidaktepatan obat, dosis, rute dan frekuensi pemberian. Peresepan menggunakan tulisan tangan berpotensi tidak dapat dibaca. Penulisan resep yang tidak lengkap dan tidak terbaca dapat menyebabkan kesalahan dan tertundanya pasien mendapatkan obat.

Rumah sakit harus menetapkan dan menerapkan regulasi tentang peresepan/permintaan obat dan BMHP/instruksi pengobatan yang benar, lengkap dan terbaca. Rumah sakit menetapkan dan melatih tenaga medis yang kompeten dan berwenang untuk melakukan peresepan/permintaan obat dan BMHP/instruksi pengobatan.

Untuk menghindari keragaman dan mencegah kesalahan obat yang berdampak pada keselamatan pasien, maka rumah sakit menetapkan persyaratan bahwa semua resep/permintaan obat/instruksi pengobatan harus mencantumkan identitas pasien (lihat SKP 1), nama obat, dosis, frekuensi pemberian, rute pemberian, nama dan tanda tangan dokter. Persyaratan kelengkapan lain ditambahkan disesuaikan dengan jenis resep/permintaan obat/instruksi pengobatan, misalnya:

\begin{enumerate}
\def\labelenumi{\alph{enumi}.}
\tightlist
\item
  Penulisan nama dagang atau nama generik pada sediaan dengan zat aktif tunggal.
\item
  Penulisan indikasi dan dosis maksimal sehari pada obat PRN (pro renata atau ``jika perlu'').
\item
  Penulisan berat badan dan/atau tinggi badan untuk pasien anak-anak, lansia, pasien yang mendapatkan kemoterapi, dan populasi khusus lainnya.
\item
  Penulisan kecepatan pemberian infus di instruksi pengobatan.
\item
  Penulisan instruksi khusus seperti: titrasi, tapering, rentang dosis.
  Instruksi titrasi adalah instruksi pengobatan dimana dosis obat dinaikkan/diturunkan secara bertahap tergantung status klinis pasien. Instruksi harus terdiri dari: dosis awal, dosis titrasi, parameter penilaian, dan titik akhir penggunaan, misalnya: infus nitrogliserin, dosis awal 5 mcg/menit. Naikkan dosis 5 mcg/menit setiap 5 menit jika nyeri dada menetap, jaga tekanan darah 110-140 mmHg.
  Instruksi tapering down/tapering off adalah instruksi pengobatan dimana dosis obat diturunkan secara bertahap sampai akhirnya dihentikan. Cara ini dimaksudkan agar tidak terjadi efek yang tidak diharapkan akibat penghentian mendadak. Contoh obat yang harus dilakukan tapering down/off: pemakaian jangka panjang kortikosteroid, psikotropika. Instruksi harus rinci dituliskan tahapan penurunan dosis dan waktunya.
  Instruksi rentang dosis adalah instruksi pengobatan dimana dosis obat dinyatakan dalam rentang, misalnya morfin inj 2-4 mg IV tiap 3 jam jika nyeri. Dosis disesuaikan berdasarkan kebutuhan pasien.
\end{enumerate}

Rumah sakit menetapkan dan menerapkan proses untuk menangani resep/ permintaan obat dan BMHP/instruksi pengobatan:

\begin{enumerate}
\def\labelenumi{\alph{enumi}.}
\tightlist
\item
  Tidak lengkap, tidak benar dan tidak terbaca.
\item
  NORUM (Nama Obat Rupa Ucapan Mirip) atau LASA (Look Alike Sound Alike).
\item
  Jenis resep khusus seperti emergensi, cito, automatic stop order, tapering dan lainnya.
\item
  Secara lisan atau melalui telepon, wajib dilakukan komunikasi efektif meliputi: tulis lengkap, baca ulang (read back), dan meminta konfirmasi kepada dokter yang memberikan resep/instruksi melalui telepon dan mencatat di rekam medik bahwa sudah dilakukan konfirmasi. (Lihat standar SKP 2)
\end{enumerate}

Rumah sakit melakukan evaluasi terhadap penulisan resep/instruksi pengobatan yang tidak lengkap dan tidak terbaca dengan cara uji petik atau cara lain yang valid.

Daftar obat yang diresepkan tercatat dalam rekam medis pasien yang mencantumkan identitas pasien (lihat SKP 1), nama obat, dosis, rute pemberian, waktu pemberian, nama dan tanda tangan dokter. Daftar ini menyertai pasien ketika dipindahkan sehingga profesional pemberi asuhan (PPA) yang merawat pasien dengan mudah dapat mengakses informasi tentang penggunaan obat pasien. Daftar obat pulang diserahkan kepada pasien disertai edukasi penggunaannya agar pasien dapat menggunakan obat dengan benar dan mematuhi aturan pakai yang sudah ditetapkan.

\hypertarget{elemen-penilaian-pkpo-4.1}{%
\subsubsection*{6. Elemen Penilaian PKPO 4.1}\label{elemen-penilaian-pkpo-4.1}}
\addcontentsline{toc}{subsubsection}{6. Elemen Penilaian PKPO 4.1}

\begin{enumerate}
\def\labelenumi{\alph{enumi}.}
\tightlist
\item
  Resep dibuat lengkap sesuai regulasi.
\item
  Telah dilakukan evaluasi terhadap penulisan resep/instruksi pengobatan yang tidak lengkap dan tidak terbaca.
\item
  Telah dilaksanaan proses untuk mengelola resep khusus seperti emergensi, automatic stop order, tapering,
\item
  Daftar obat yang diresepkan tercatat dalam rekam medis pasien dan menyertai pasien ketika dipindahkan/transfer.
\item
  Daftar obat pulang diserahkan kepada pasien disertai edukasi penggunaannya.
\end{enumerate}

\hypertarget{e.-penyiapan-dispensing}{%
\subsection*{e. Penyiapan (Dispensing)}\label{e.-penyiapan-dispensing}}
\addcontentsline{toc}{subsection}{e. Penyiapan (Dispensing)}

\hypertarget{standar-pkpo-5}{%
\subsubsection*{1. Standar PKPO 5}\label{standar-pkpo-5}}
\addcontentsline{toc}{subsubsection}{1. Standar PKPO 5}

Rumah sakit menetapkan dan menerapkan regulasi dispensing sediaan farmasi dan bahan medis habis pakai sesuai standar profesi dan peraturan perundang-undangan.

\hypertarget{maksud-dan-tujuan-pkpo-5}{%
\subsubsection*{2. Maksud dan Tujuan PKPO 5}\label{maksud-dan-tujuan-pkpo-5}}
\addcontentsline{toc}{subsubsection}{2. Maksud dan Tujuan PKPO 5}

Penyiapan (dispensing) adalah rangkaian proses mulai dari diterimanya resep/permintaan obat/instruksi pengobatan sampai dengan penyerahan obat dan BMHP kepada dokter/perawat atau kepada pasien/keluarga. Penyiapan obat dilakukan oleh staf yang terlatih dalam lingkungan yang aman bagi pasien, staf dan lingkungan sesuai peraturan perundang-undangan dan standar praktik kefarmasian untuk menjamin keamanan, mutu, manfaat dan khasiatnya. Untuk menghindari kesalahan pemberian obat pada pasien rawat inap, maka obat yang diserahkan harus dalam bentuk yang siap digunakan, dan disertai dengan informasi lengkap tentang pasien dan obat.

\hypertarget{elemen-penilaian-pkpo-5}{%
\subsubsection*{3. Elemen Penilaian PKPO 5}\label{elemen-penilaian-pkpo-5}}
\addcontentsline{toc}{subsubsection}{3. Elemen Penilaian PKPO 5}

\begin{enumerate}
\def\labelenumi{\alph{enumi}.}
\tightlist
\item
  Telah memiliki sistem distribusi dan dispensing yang sama/seragam diterapkan di rumah sakit sesuai peraturan perundang-undangan.
\item
  Staf yang melakukan dispensing sediaan obat non steril kompeten.
\item
  Staf yang melakukan dispensing sediaan obat steril non sitostatika terlatih dan kompeten.
\item
  Staf yang melakukan pencampuran sitostatika terlatih dan kompeten.
\item
  Tersedia fasilitas dispensing sesuai standar praktik kefarmasian.
\item
  Telah melaksanakan penyerahan obat dalam bentuk yang siap diberikan untuk pasien rawat inap.
\item
  Obat yang sudah disiapkan diberi etiket yang meliputi identitas pasien, nama obat, dosis atau konsentrasi, cara pemakaian, waktu pemberian, tanggal dispensing dan tanggal kedaluwarsa/beyond use date (BUD).
\end{enumerate}

\hypertarget{standar-pkpo-5.1}{%
\subsubsection*{4. Standar PKPO 5.1}\label{standar-pkpo-5.1}}
\addcontentsline{toc}{subsubsection}{4. Standar PKPO 5.1}

Rumah sakit menetapkan dan menerapkan regulasi pengkajian resep dan telaah obat sesuai peraturan perundang-undangan dan standar praktik profesi.

\hypertarget{maksud-dan-tujuan-pkpo-5.1}{%
\subsubsection*{5. Maksud dan Tujuan PKPO 5.1}\label{maksud-dan-tujuan-pkpo-5.1}}
\addcontentsline{toc}{subsubsection}{5. Maksud dan Tujuan PKPO 5.1}

Pengkajian resep adalah kegiatan menelaah resep sebelum obat disiapkan, yang meliputi pengkajian aspek administratif, farmasetik dan klinis. Pengkajian resep dilakukan oleh tenaga kefarmasian yang kompeten dan diberi kewenangan dengan tujuan untuk mengidentifikasi dan menyelesaikan masalah terkait obat sebelum obat disiapkan.

Pengkajian resep aspek administratif meliputi: kesesuaian identitas pasien (lihat SKP 1), ruang rawat, status pembiayaan, tanggal resep, identitas dokter penulis resep.

Pengkajian resep aspek farmasetik meliputi: nama obat, bentuk dan kekuatan sediaan, jumlah obat, instruksi cara pembuatan (jika diperlukan peracikan), stabilitas dan inkompatibilitas sediaan.
Pengkajian resep aspek klinis meliputi:

\begin{enumerate}
\def\labelenumi{\alph{enumi}.}
\tightlist
\item
  Ketepatan identitas pasien, obat, dosis, frekuensi, aturan pakai dan waktu pemberian.
\item
  Duplikasi pengobatan.
\item
  Potensi alergi atau hipersensitivitas.
\item
  Interaksi antara obat dan obat lain atau dengan makanan.
\item
  Variasi kriteria penggunaan dari rumah sakit, misalnya membandingkan dengan panduan praktik klinis, formularium nasional.
\item
  Berat badan pasien dan atau informasi fisiologis lainnya.
\item
  Kontraindikasi.
\end{enumerate}

Dalam pengkajian resep tenaga teknis kefarmasian diberi kewenangan terbatas hanya aspek administratif dan farmasetik.

Pengkajian resep aspek klinis yang baik oleh apoteker memerlukan data klinis pasien, sehingga apoteker harus diberi kemudahan akses untuk mendapatkan informasi klinis pasien.

Apoteker/tenaga teknis kefarmasian harus melakukan telaah obat sebelum obat diserahkan kepada perawat/pasien.untuk memastikan bahwa obat yang sudah disiapkan tepat:

\begin{enumerate}
\def\labelenumi{\alph{enumi}.}
\tightlist
\item
  Pasien.
\item
  Nama obat.
\item
  Dosis dan jumlah obat.
\item
  Rute pemberian.
\item
  Waktu pemberian.
\end{enumerate}

\hypertarget{elemen-penilaian-pkpo-5.1}{%
\subsubsection*{6. Elemen Penilaian PKPO 5.1}\label{elemen-penilaian-pkpo-5.1}}
\addcontentsline{toc}{subsubsection}{6. Elemen Penilaian PKPO 5.1}

\begin{enumerate}
\def\labelenumi{\alph{enumi}.}
\tightlist
\item
  Telah melaksanakan pengkajian resep yang dilakukan oleh staf yang kompeten dan berwenang serta didukung tersedianya informasi klinis pasien yang memadai.
\item
  Telah memiliki proses telaah obat sebelum diserahkan.
\end{enumerate}

\hypertarget{f.-pemberian-obat}{%
\subsection*{f.~Pemberian Obat}\label{f.-pemberian-obat}}
\addcontentsline{toc}{subsection}{f.~Pemberian Obat}

\hypertarget{standar-pkpo-6}{%
\subsubsection*{1. Standar PKPO 6}\label{standar-pkpo-6}}
\addcontentsline{toc}{subsubsection}{1. Standar PKPO 6}

Rumah sakit menetapkan dan menerapkan regulasi pemberian obat sesuai peraturan perundang-undangan.

\hypertarget{maksud-dan-tujuan-pkpo-6}{%
\subsubsection*{2. Maksud dan Tujuan PKPO 6}\label{maksud-dan-tujuan-pkpo-6}}
\addcontentsline{toc}{subsubsection}{2. Maksud dan Tujuan PKPO 6}

Tahap pemberian obat merupakan tahap akhir dalam proses penggunaan obat sebelum obat masuk ke dalam tubuh pasien. Tahap ini merupakan tahap yang kritikal ketika terjadi kesalahan obat (medication error) karena pasien akan langsung terpapar dan dapat menimbulkan cedera. Rumah sakit harus menetapkan dan menerapkan regulasi pemberian obat. Rumah sakit menetapkan professional pemberi asuhan (PPA) yang kompeten dan berwenang memberikan obat sesuai peraturan perundang- undangan. Rumah sakit dapat membatasi kewenangan staf klinis dalam melakukan pemberian obat, misalnya pemberian obat anestesi, kemoterapi, radioaktif, obat penelitian.

Sebelum pemberian obat kepada pasien, dilakukan verifikasi kesesuaian obat dengan instruksi pengobatan yang meliputi:

\begin{enumerate}
\def\labelenumi{\alph{enumi}.}
\tightlist
\item
  Identitas pasien.
\item
  Nama obat.
\item
  Dosis.
\item
  Rute pemberian.
\item
  Waktu pemberian.
\end{enumerate}

Obat yang termasuk golongan obat high alert, harus dilakukan double-checking untuk menjamin ketepatan pemberian obat.

\hypertarget{elemen-penilaian-pkpo-6}{%
\subsubsection*{3. Elemen Penilaian PKPO 6}\label{elemen-penilaian-pkpo-6}}
\addcontentsline{toc}{subsubsection}{3. Elemen Penilaian PKPO 6}

\begin{enumerate}
\def\labelenumi{\alph{enumi}.}
\tightlist
\item
  Staf yang melakukan pemberian obat kompeten dan berwenang dengan pembatasan yang ditetapkan.
\item
  Telah dilaksanaan verifikasi sebelum obat diberikan kepada pasien minimal meliputi: identitas pasien, nama obat, dosis, rute, dan waktu pemberian.
\item
  Telah melaksanakan double checking untuk obat high alert.
\item
  Pasien diberi informasi tentang obat yang akan diberikan.
\end{enumerate}

\hypertarget{standar-pkpo-6.1}{%
\subsubsection*{4. Standar PKPO 6.1}\label{standar-pkpo-6.1}}
\addcontentsline{toc}{subsubsection}{4. Standar PKPO 6.1}

Rumah sakit menetapkan dan menerapkan regulasi penggunaan obat yang dibawa pasien dari luar rumah sakit dan penggunaan obat oleh pasien secara mandiri.

\hypertarget{maksud-dan-tujuan-pkpo-6.1}{%
\subsubsection*{5. Maksud dan Tujuan PKPO 6.1}\label{maksud-dan-tujuan-pkpo-6.1}}
\addcontentsline{toc}{subsubsection}{5. Maksud dan Tujuan PKPO 6.1}

Obat yang dibawa pasien/keluarga dari luar rumah sakit berisiko dalam hal identifikasi/keaslian dan mutu obat. Oleh sebab itu rumah sakit harus melakukan penilaian terhadap obat tersebut terkait kelayakan penggunaannya di rumah sakit. Penggunaan obat oleh pasien secara mandiri, baik yang dibawa dari luar rumah sakit atau yang diresepkan dari rumah sakit harus diketahui oleh dokter yang merawat dan dicatat di rekam medis pasien. Penggunaan obat secara mandiri harus ada proses edukasi dan pemantauan penggunaannya untuk menghindari penggunaan obat yang tidak tepat.

\hypertarget{elemen-penilaian-pkpo-6.1}{%
\subsubsection*{6. Elemen Penilaian PKPO 6.1}\label{elemen-penilaian-pkpo-6.1}}
\addcontentsline{toc}{subsubsection}{6. Elemen Penilaian PKPO 6.1}

\begin{enumerate}
\def\labelenumi{\alph{enumi}.}
\tightlist
\item
  Telah melakukan penilaian obat yang dibawa pasien dari luar rumah sakit untuk kelayakan penggunaannya di rumah sakit.
\item
  Telah melaksanakan edukasi kepada pasien/keluarga jika obat akan digunakan secara mandiri.
\item
  Telah memantau pelaksanaan penggunaan obat secara mandiri sesuai edukasi.
\end{enumerate}

\hypertarget{g.-pemantauan}{%
\subsection*{g. Pemantauan}\label{g.-pemantauan}}
\addcontentsline{toc}{subsection}{g. Pemantauan}

\hypertarget{standar-pkpo-7}{%
\subsubsection*{1. Standar PKPO 7}\label{standar-pkpo-7}}
\addcontentsline{toc}{subsubsection}{1. Standar PKPO 7}

Rumah sakit menerapkan pemantauan terapi obat secara kolaboratif.

\hypertarget{maksud-dan-tujuan-pkpo-7}{%
\subsubsection*{2. Maksud dan Tujuan PKPO 7}\label{maksud-dan-tujuan-pkpo-7}}
\addcontentsline{toc}{subsubsection}{2. Maksud dan Tujuan PKPO 7}

Untuk mengoptimalkan terapi obat pasien, maka dilakukan pemantauan terapi obat secara kolaboratif yang melibatkan profesional pemberi asuhan (PPA) dan pasien. Pemantauan meliputi efek yang diharapkan dan efek samping obat. Pemantauan terapi obat didokumentasikan di dalam catatan perkembangan pasien terintegrasi (CPPT) di rekam medis.

Rumah sakit menerapkan sistem pemantauan dan pelaporan efek samping obat untuk meningkatkan keamanan penggunaan obat sesuai peraturan perundang- undangan. Efek samping obat dilaporkan ke Komite/Tim Farmasi dan Terapi. Rumah sakit melaporkan efek samping obat ke Badan Pengawas Obat dan Makanan (BPOM).

\hypertarget{elemen-penilaian-pkpo-7}{%
\subsubsection*{3. Elemen Penilaian PKPO 7}\label{elemen-penilaian-pkpo-7}}
\addcontentsline{toc}{subsubsection}{3. Elemen Penilaian PKPO 7}

\begin{enumerate}
\def\labelenumi{\alph{enumi}.}
\tightlist
\item
  Telah melaksanakan pemantauan terapi obat secara kolaboratif.
\item
  Telah melaksanakan pemantauan dan pelaporan efek samping obat serta analisis laporan untuk meningkatkan keamanan penggunaan obat.
\end{enumerate}

\hypertarget{standar-pkpo-7.1}{%
\subsubsection*{4. Standar PKPO 7.1}\label{standar-pkpo-7.1}}
\addcontentsline{toc}{subsubsection}{4. Standar PKPO 7.1}

Rumah sakit menetapkan dan menerapkan proses pelaporan serta tindak lanjut terhadap kesalahan obat (medication error) dan berupaya menurunkan kejadiannya.

\hypertarget{maksud-dan-tujuan-pkpo-7.1}{%
\subsubsection*{5. Maksud dan Tujuan PKPO 7.1}\label{maksud-dan-tujuan-pkpo-7.1}}
\addcontentsline{toc}{subsubsection}{5. Maksud dan Tujuan PKPO 7.1}

Insiden kesalahan obat (medication error) merupakan penyebab utama cedera pada pasien yang seharusnya dapat dicegah. Untuk meningkatkan keselamatan pasien, rumah sakit harus berupaya mengurangi terjadinya kesalahan obat dengan membuat sistem pelayanan kefarmasian dan penggunaan obat yang lebih aman (medication safety).

Insiden kesalahan obat harus dijadikan sebagai pembelajaran bagi rumah sakit agar kesalahan tersebut tidak terulang lagi.

Rumah sakit menerapkan pelaporan insiden keselamatan pasien serta tindak lanjut terhadap kejadian kesalahan obat serta upaya perbaikannya. Proses pelaporan kesalahan obat yang mencakup kejadian sentinel, kejadian yang tidak diharapkan (KTD), kejadian tidak cedera (KTC) maupun kejadian nyaris cedera (KNC), menjadi bagian dari program peningkatan mutu dan keselamatan pasien. Rumah sakit memberikan pelatihan kepada staf rumah sakit tentang kesalahan obat dalam rangka upaya perbaikan dan untuk mencegah kesalahan obat, serta meningkatkan keselamatan pasien.

\hypertarget{elemen-penilaian-pkpo-7.1}{%
\subsubsection*{6. Elemen Penilaian PKPO 7.1}\label{elemen-penilaian-pkpo-7.1}}
\addcontentsline{toc}{subsubsection}{6. Elemen Penilaian PKPO 7.1}

\begin{enumerate}
\def\labelenumi{\alph{enumi}.}
\tightlist
\item
  Rumah sakit telah memiliki regulasi tentang medication safety yang bertujuan mengarahkan penggunaan obat yang aman dan meminimalkan risiko kesalahan penggunaan obat sesuai dengan peraturan perundang- undangan.
\item
  Rumah sakit menerapkan sistem pelaporan kesalahan obat yang menjamin laporan akurat dan tepat waktu yang merupakan bagian program peningkatan mutu dan keselamatan pasien.
\item
  Rumah sakit memiliki upaya untuk mendeteksi, mencegah dan menurunkan kesalahan obat dalam meningkatkan mutu proses penggunaan obat.
\item
  Seluruh staf rumah sakit dilatih terkait kesalahan obat (medication error).
\end{enumerate}

\hypertarget{h.-program-pengendalian-resistansi-antimikroba}{%
\subsection*{h. Program Pengendalian Resistansi Antimikroba}\label{h.-program-pengendalian-resistansi-antimikroba}}
\addcontentsline{toc}{subsection}{h. Program Pengendalian Resistansi Antimikroba}

\hypertarget{standar-pkpo-8}{%
\subsubsection*{1. Standar PKPO 8}\label{standar-pkpo-8}}
\addcontentsline{toc}{subsubsection}{1. Standar PKPO 8}

Rumah sakit menyelenggarakan program pengendalian resistansi antimikroba (PPRA) sesuai peraturan perundang- undangan.

\hypertarget{maksud-dan-tujuan-pkpo-8}{%
\subsubsection*{2. Maksud dan Tujuan PKPO 8}\label{maksud-dan-tujuan-pkpo-8}}
\addcontentsline{toc}{subsubsection}{2. Maksud dan Tujuan PKPO 8}

Resistansi antimikroba (antimicrobial resistance = AMR) telah menjadi masalah kesehatan nasional dan global. Pemberian obat antimikroba (antibiotik atau antibakteri, antijamur, antivirus, antiprotozoa) yang tidak rasional dan tidak bijak dapat memicu terjadinya resistansi yaitu ketidakmampuan membunuh atau menghambat pertumbuhan mikroba sehingga penggunaan pada penanganan penyakit infeksi tidak efektif.

Meningkatnya kejadian resistansi antimikroba akibat dari penggunaan antimikroba yang tidak bijak dan pencegahan pengendalian infeksi yang belum optimal. Resistansi antimikroba di rumah sakit menyebabkan menurunnya mutu pelayanan, meningkatkan morbiditas dan mortalitas, serta meningkatnya beban biaya perawatan dan pengobatan pasien.
Tersedia regulasi pengendalian resistensi antimikroba di rumah sakit yang meliputi:

\begin{enumerate}
\def\labelenumi{\alph{enumi}.}
\tightlist
\item
  kebijakan dan panduan penggunaan antibiotik
\item
  pembentukan komite/tim PRA yang terdiri dari tenaga kesehatan yang kompeten dari unsur:
\end{enumerate}

\begin{enumerate}
\def\labelenumi{\arabic{enumi}.}
\tightlist
\item
  Klinisi perwakilan SMF/bagian;
\item
  Keperawatan;
\item
  Instalasi farmasi;
\item
  Laboratorium mikrobiologi klinik;
\item
  Komite/Tim Pencegahan Pengendalian Infeksi (PPI);
\item
  Komite/tim Farmasi dan Terapi (KFT)
\end{enumerate}

Tim pelaksana Program Pengendalian Resistensi Antimikroba mempunyai tugas dan fungsi:

\begin{enumerate}
\def\labelenumi{\alph{enumi}.}
\tightlist
\item
  Membantu kepala/direktur rumah rakit dalam menetapkan kebijakan tentang pengendalian resistensi antimikroba;
\item
  Membantu kepala/direktur rumah sakit dalam menetapkan kebijakan umum dan panduan penggunaan antibiotik di rumah sakit;
\item
  Membantu kepala/direktur rumah sakit dalam pelaksanaan program pengendalian resistensi antimikroba;
\item
  Membantu kepala/direktur rumah sakit dalam mengawasi dan mengevaluasi pelaksanaan program pengendalian resistensi antimikoba;
\item
  Menyelenggarakan forum kajian kasus pengelolaan penyakit infeksi terintegrasi;
\item
  Melakukan surveilans pola penggunaan antibiotik;
\item
  Melakukan surveilans pola mikroba penyebab infeksi dan kepekaannya terhadap antibiotik;
\item
  Menyebarluaskan serta meningkatkan pemahaman dan kesadaran tentang prinsip pengendalian resistensi antimikroba, penggunaan antibiotik secara bijak, dan ketaatan terhadap pencegahan pengendalian infeksi melalui kegiatan pendidikan dan pelatihan;
\item
  Mengembangkan penelitian di bidang pengendalian resistensi antimikroba;
\item
  Melaporkan kegiatan program pengendalian resistensi antimikroba kepada kepala/direktur rumah sakit.
\end{enumerate}

Rumah sakit menjalankan program pengendalian resistansi antimikroba sesuai peraturan perundang-undangan. Implementasi PPRA di rumah sakit dapat berjalan dengan baik, apabila mendapat dukungan penuh dari pimpinan rumah sakit dengan penetapan kebijakan, pembentukan organisasi pengelola program dalam bentuk komite/tim yang bertanggungjawab langsung kepada pimpinan rumah sakit, penyediaan fasilitas, sarana, SDM dan dukungan finansial dalam mendukung pelaksanaan kegiatan PPRA. Rumah sakit menyusun program kerja PPRA meliputi:

\begin{enumerate}
\def\labelenumi{\alph{enumi}.}
\tightlist
\item
  Peningkatan pemahaman dan kesadaran penggunaan antimikroba bijak bagi seluruh tenaga kesehatan dan staf di rumah sakit, serta pasien dan keluarga, melalui pelatihan dan edukasi.
\item
  Optimalisasi penggunaan antimikroba secara bijak melalui penerapan penatagunaan antimikroba (PGA).
\item
  Surveilans penggunaan antimikroba secara kuantitatif dan kualitatif.
\item
  Surveilans resistansi antimikroba dengan indikator mikroba multi drugs resistance organism (MDRO).
\item
  Peningkatan mutu penanganan tata laksana infeksi, melalui pelaksanaan forum kajian kasus infeksi terintegrasi (FORKKIT).
\end{enumerate}

Program dan kegiatan pengendalian resistansi antimikroba di rumah sakit sesuai peraturan perundang-undangan dilaksanakan, dipantau, dievaluasi dan dilaporkan kepada Kementerian Kesehatan.

Rumah sakit membuat laporan pelaksanaan program/ kegiatan PRA meliputi:

\begin{enumerate}
\def\labelenumi{\alph{enumi}.}
\tightlist
\item
  Kegiatan sosialisasi dan pelatihan staf tenaga resistensi kesehatan tentang pengendalian antimikroba;
\item
  Surveilans pola penggunaan antibiotik di rumah sakit (termasuk laporan pelaksanaan pengendalian antibiotik);
\item
  Surveilans pola resistensi antimikroba;
\item
  Forum kajian penyakit inteksi terintegrasi.
\end{enumerate}

\hypertarget{elemen-penilaian-pkpo-8}{%
\subsubsection*{3. Elemen Penilaian PKPO 8}\label{elemen-penilaian-pkpo-8}}
\addcontentsline{toc}{subsubsection}{3. Elemen Penilaian PKPO 8}

\begin{enumerate}
\def\labelenumi{\alph{enumi}.}
\tightlist
\item
  Rumah sakit telah menetapkan regulasi tentang pengendalian resistansi antimikroba sesuai dengan ketentuan peraturan perundang-undangan.
\item
  Rumah sakit telah menetapkan komite/tim PPRA dengan melibatkan unsur terkait sesuai regulasi yang akan mengelola dan menyusun program pengendalian resistansi antimikroba dan bertanggungjawab langsung kepada Direktur rumah sakit.
\item
  Rumah sakit telah melaksanakan program kerja sesuai maksud dan tujuan.
\item
  Rumah sakit telah melaksanakan pemantauan dan evaluasi kegiatan PPRA sesuai maksud dan tujuan.
\item
  Memiliki telah membuat laporan kepada pimpinan rumah sakit secara berkala dan kepada Kementerian Kesehatan sesuai peraturan perundang-undangan.
\end{enumerate}

\hypertarget{standar-pkpo-8.1}{%
\subsubsection*{4. Standar PKPO 8.1}\label{standar-pkpo-8.1}}
\addcontentsline{toc}{subsubsection}{4. Standar PKPO 8.1}

Rumah sakit mengembangkan dan menerapkan penggunaan antimikroba secara bijak berdasarkan prinsip penatagunaan antimikroba (PGA).

\hypertarget{maksud-dan-tujuan-pkpo-8.1}{%
\subsubsection*{5. Maksud dan Tujuan PKPO 8.1}\label{maksud-dan-tujuan-pkpo-8.1}}
\addcontentsline{toc}{subsubsection}{5. Maksud dan Tujuan PKPO 8.1}

Penggunaan antimikroba secara bijak adalah penggunaan antimikroba secara rasional dengan mempertimbangkan dampak muncul dan menyebarnya mikroba resistan. Penerapan penggunaan antimikroba secara bijak berdasarkan prinsip penatagunaan antimikroba (PGA), atau antimicrobial stewardship (AMS) adalah kegiatan strategis dan sistematis, yang terpadu dan terorganisasi di rumah sakit, bertujuan mengoptimalkan penggunaan antimikroba secara bijak, baik kuantitas maupun kualitasnya, diharapkan dapat menurunkan tekanan selektif terhadap mikroba, sehingga dapat mengendalikan resistansi antimikroba.

Kegiatan ini dimulai dari tahap penegakan diagnosis penyakit infeksi, penggunaan antimikroba berdasarkan indikasi, pemilihan jenis antimikroba yang tepat, termasuk dosis, rute, saat, dan lama pemberiannya. Dilanjutkan dengan pencatatan dan pemantauan keberhasilan dan/atau kegagalan terapi, potensial dan aktual jika terjadi reaksi yang tidak dikehendaki, interaksi antimikroba dengan obat lain, dengan makanan, dengan pemeriksaan laboratorium, dan reaksi alergi.

Yang dimaksud obat antimikroba meliputi: antibiotik (antibakteri), antijamur, antivirus, dan antiprotozoa. Pada penatagunaan antibiotik, dalam melaksanakan pengendaliannya dilakukan dengan cara mengelompokkan antibiotik dalam kategori Access, Watch, Reserve (AWaRe).

Kebijakan kategorisasi ini mendukung rencana aksi nasional dan global WHO dalam menekan munculnya bakteri resistan dan mempertahankan kemanfaatan antibiotik dalam jangka panjang. Rumah sakit menyusun dan mengembangkan panduan penggunaan antimikroba untuk pengobatan infeksi (terapi) dan pencegahan infeksi pada tindakan pembedahan (profilaksis), serta panduan praktik klinis penyakit infeksi yang berbasis bukti ilmiah dan peraturan perundangan.

Rumah sakit menetapkan mekanisme untuk mengawasi pelaksanaan PGA dan memantau berdasarkan indikator keberhasilan program sesuai dengan ketentuan peraturan perundang-undangan.

\hypertarget{elemen-penilaian-pkpo-8.1}{%
\subsubsection*{6. Elemen Penilaian PKPO 8.1}\label{elemen-penilaian-pkpo-8.1}}
\addcontentsline{toc}{subsubsection}{6. Elemen Penilaian PKPO 8.1}

\begin{enumerate}
\def\labelenumi{\alph{enumi}.}
\tightlist
\item
  Rumah sakit telah melaksanakan dan mengembangkan penatagunaan antimikroba di unit pelayanan yang melibatkan dokter, apoteker, perawat, dan peserta didik.
\item
  Rumah sakit telah menyusun dan mengembangkan panduan praktik klinis (PPK), panduan penggunaan antimikroba untuk terapi dan profilaksis (PPAB), berdasarkan kajian ilmiah dan kebijakan rumah sakit serta mengacu regulasi yang berlaku secara nasional. Ada mekanisme untuk mengawasi pelaksanaan penatagunaan antimikroba.
\item
  Rumah sakit telah melaksanakan pemantauan dan evaluasi ditujukan untuk mengetahui efektivitas indikator keberhasilan program.
\end{enumerate}

\hypertarget{komunikasi-dan-edukasi-ke}{%
\section*{7. Komunikasi dan Edukasi (KE)}\label{komunikasi-dan-edukasi-ke}}
\addcontentsline{toc}{section}{7. Komunikasi dan Edukasi (KE)}

\textbf{Gambaran Umum}

Perawatan pasien di rumah sakit merupakan pelayanan yang kompleks dan melibatkan berbagai tenaga kesehatan serta pasien dan keluarga. Keadaan tersebut memerlukan komunikasi yang efektif, baik antar Profesional Pemberi Asuhan (PPA) maupun antara Profesional Pemberi Asuhan (PPA) dengan pasien dan keluarga. Setiap pasien memiliki keunikan dalam hal kebutuhan, nilai dan keyakinan.

Rumah sakit harus membangun kepercayaan dan komunikasi terbuka dengan pasien. Komunikasi dan edukasi yang efektif akan membantu pasien untuk memahami dan berpartisipasi dalam pengambilan keputusan yang berkaitan dengan pengobatan yang dijalaninya. Keberhasilan pengobatan dapat ditingkatkan jika pasien dan keluarga diberi informasi yang dibutuhkan dan dilibatkan dalam pengambilan keputusan serta proses yang sesuai dengan harapan mereka.

Rumah sakit menyediakan program edukasi yang didasarkan pada misi rumah sakit, layanan yang diberikan rumah sakit, serta populasi pasien. Profesional Pemberi Asuhan (PPA) berkolaborasi untuk memberikan edukasi tersebut.

Edukasi akan efektif apabila dilakukan sesuai dengan pilihan pembelajaran yang tepat, mempertimbangkan keyakinan, nilai budaya, kemampuan membaca, serta bahasa.

Edukasi yang efektif diawali dengan pengkajian kebutuhan edukasi pasien dan keluarganya. Pengkajian ini akan menentukan jenis dan proses edukasi yang dibutuhkan agar edukasi dapat menjadi efektif. Edukasi akan berdampak positif bila diberikan sepanjang proses asuhan. Edukasi yang diberikan meliputi pengetahuan dan informasi yang diperlukan selama proses asuhan maupun setelah pasien dipulangkan.

Dengan demikian, edukasi juga mencakup informasi sumber-sumber di komunitas untuk tindak lanjut pelayanan apabila diperlukan, serta bagaimana akses ke pelayanan gawat darurat bila dibutuhkan. Edukasi yang efektif menggunakan berbagai format yang sesuai sehingga dapat dipahami dengan baik oleh pasien dan keluarga, misalnya informasi diberikan secara tertulis atau audiovisual, serta memanfaatkan teknologi informasi dan komunikasi.
Standar ini akan membahas lebih lanjut mengenai:

\begin{enumerate}
\def\labelenumi{\alph{enumi}.}
\tightlist
\item
  Pengelolaan kegiatan Promosi Kesehatan Rumah Sakit (PKRS)
\item
  Komunikasi dengan pasien dan keluarga.
\end{enumerate}

\hypertarget{a.-pengelolaan-kegiatan-promosi-kesehatan-rumah-sakit-pkrs}{%
\subsection*{a. Pengelolaan kegiatan Promosi Kesehatan Rumah Sakit (PKRS)}\label{a.-pengelolaan-kegiatan-promosi-kesehatan-rumah-sakit-pkrs}}
\addcontentsline{toc}{subsection}{a. Pengelolaan kegiatan Promosi Kesehatan Rumah Sakit (PKRS)}

\hypertarget{standar-ke-1}{%
\subsubsection*{1. Standar KE 1}\label{standar-ke-1}}
\addcontentsline{toc}{subsubsection}{1. Standar KE 1}

Rumah sakit menetapkan tim atau unit Promosi Kesehatan Rumah Sakit (PKRS) dengan tugas dan tanggung jawab sesuai peraturan perundangan.

\hypertarget{maksud-dan-tujuan-ke-1}{%
\subsubsection*{2. Maksud dan Tujuan KE 1}\label{maksud-dan-tujuan-ke-1}}
\addcontentsline{toc}{subsubsection}{2. Maksud dan Tujuan KE 1}

Setiap rumah sakit mengintegrasikan edukasi pasien dan keluarga sebagai bagian dari proses perawatan, disesuaikan dengan misi, pelayanan yang disediakan, serta populasi pasiennya. Edukasi direncanakan sedemikian rupa sehingga setiap pasien mendapatkan edukasi yang dibutuhkan oleh pasien tersebut. Rumah sakit menetapkan pengaturan sumber daya edukasi secara efisien dan efektif. Oleh karena itu, rumah sakit dapat menetapkan tim atau unit Promosi Kesehatan Rumah Sakit (PKRS), menyelenggarakan pelayanan edukasi, dan mengatur penugasan seluruh staf yang memberikan edukasi secara terkoordinasi.

Staf klinis memahami kontribusinya masing-masing dalam pemberian edukasi pasien, sehingga mereka dapat berkolaborasi secara efektif. Kolaborasi menjamin bahwa informasi yang diterima pasien dan keluarga adalah komprehensif, konsisten, dan efektif. Kolaborasi ini didasarkan pada kebutuhan pasien, oleh karena itu mungkin tidak selalu diperlukan. Agar edukasi yang diberikan dapat berhasil guna, dibutuhkan pertimbangan- pertimbangan penting seperti pengetahuan tentang materi yang akan diedukasikan, waktu yang cukup untuk memberi edukasi, dan kemampuan untuk berkomunikasi secara efektif.

\hypertarget{elemen-penilaian-ke-1}{%
\subsubsection*{3. Elemen Penilaian KE 1}\label{elemen-penilaian-ke-1}}
\addcontentsline{toc}{subsubsection}{3. Elemen Penilaian KE 1}

\begin{enumerate}
\def\labelenumi{\alph{enumi}.}
\tightlist
\item
  Rumah sakit menetapkan regulasi tentang pelaksanaan PKRS di rumah sakit sesuai poin a) -- b) pada gambaran umum.
\item
  Terdapat penetapan tim atau unit Promosi Kesehatan Rumah Sakit (PKRS) yang mengkoordinasikan pemberian edukasi kepada pasien sesuai dengan peraturan perundang-undangan.
\item
  Tim atau unit PKRS menyusun program kegiatan promosi kesehatan rumah sakit setiap tahunnya, termasuk kegiatan edukasi rutin sesuai dengan misi rumah sakit, layanan, dan populasi pasiennya.
\item
  Rumah sakit telah menerapkan pemberian edukasi kepada pasien dan keluarga menggunakan media, format, dan metode yang yang telah ditetapkan.
\end{enumerate}

\hypertarget{b.-komunikasi-dengan-pasien-dan-keluarga}{%
\subsection*{b. Komunikasi dengan pasien dan keluarga}\label{b.-komunikasi-dengan-pasien-dan-keluarga}}
\addcontentsline{toc}{subsection}{b. Komunikasi dengan pasien dan keluarga}

\hypertarget{standar-ke-2}{%
\subsubsection*{1. Standar KE 2}\label{standar-ke-2}}
\addcontentsline{toc}{subsubsection}{1. Standar KE 2}

Rumah sakit memberikan informasi kepada pasien dan keluarga tentang jenis asuhan dan pelayanan, serta akses untuk mendapatkan pelayanan.

\hypertarget{maksud-dan-tujuan-ke-2}{%
\subsubsection*{2. Maksud dan Tujuan KE 2}\label{maksud-dan-tujuan-ke-2}}
\addcontentsline{toc}{subsubsection}{2. Maksud dan Tujuan KE 2}

Pasien dan keluarga membutuhkan informasi lengkap mengenai asuhan dan pelayanan yang disediakan oleh rumah sakit, serta bagaimana untuk mengakses pelayanan tersebut. Hal ini akan membantu menghubungkan harapan pasien dengan kemampuan rumah sakit. Rumah sakit memberikan informasi tentang sumber alternatif asuhan dan pelayanan di tempat lain, jika rumah sakit tidak dapat menyediakan asuhan serta pelayanan yang dibutuhkan pasien.

Akses mendapatkan informasi kesehatan diberikan secara tepat waktu, dan status sosial ekonomi perawatan pasien tidak menghalangi pasien dan keluarga untuk mendapatkan informasi yang dibutuhkan.

\hypertarget{elemen-penilaian-ke-2}{%
\subsubsection*{3. Elemen Penilaian KE 2}\label{elemen-penilaian-ke-2}}
\addcontentsline{toc}{subsubsection}{3. Elemen Penilaian KE 2}

\begin{enumerate}
\def\labelenumi{\alph{enumi}.}
\tightlist
\item
  Tersedia informasi untuk pasien dan keluarga mengenai asuhan dan pelayanan yang disediakan oleh rumah sakit serta akses untuk mendapatkan layanan tersebut. Informasi dapat disampaikan secara langsung dan/atau tidak langsung.
\item
  Rumah sakit menyampaikan informasi kepada pasien dan keluarga terkait alternatif asuhan dan pelayanan di tempat lain, apabila rumah sakit tidak dapat memberikan asuhan dan pelayanan yang dibutuhkan pasien.
\item
  Akses mendapatkan informasi kesehatan diberikan secara tepat waktu, dan status sosial ekonomi perawatan pasien tidak menghalangi pasien dan keluarga untuk mendapatkan informasi yang dibutuhkan.
\item
  Terdapat bukti pemberian informasi untuk pasien dan keluarga mengenai asuhan dan pelayanan di rumah sakit.
\end{enumerate}

\hypertarget{standar-ke-3}{%
\subsubsection*{4. Standar KE 3}\label{standar-ke-3}}
\addcontentsline{toc}{subsubsection}{4. Standar KE 3}

Rumah sakit melakukan pengkajian terhadap kebutuhan edukasi setiap pasien, beserta kesiapan dan kemampuan pasien untuk menerima edukasi.

\hypertarget{maksud-dan-tujuan-ke-3}{%
\subsubsection*{5. Maksud dan Tujuan KE 3}\label{maksud-dan-tujuan-ke-3}}
\addcontentsline{toc}{subsubsection}{5. Maksud dan Tujuan KE 3}

Edukasi berfokus pada pemahaman yang dibutuhkan pasien dan keluarga dalam pengambilan keputusan, berpartisipasi dalam asuhan dan asuhan berkelanjutan di rumah. Untuk memahami kebutuhan edukasi dari setiap pasien beserta keluarganya, perlu dilakukan pengkajian. Pengkajian ini memungkinkan staf rumah sakit untuk merencanakan dan memberikan edukasi sesuai kebutuhan pasien. Pengetahuan dan keterampilan pasien dan keluarga yang menjadi kekuatan dan kekurangan diidentifikasi untuk digunakan dalam membuat rencana edukasi.
Pengkajian kemampuan dan kemauan belajar pasien/keluarga meliputi:

\begin{enumerate}
\def\labelenumi{\alph{enumi}.}
\tightlist
\item
  Kemampuan membaca, tingkat Pendidikan;
\item
  Bahasa yang digunakan (apakah diperlukan penerjemah atau penggunaan bahasa isyarat);
\item
  Hambatan emosional dan motivasi;
\item
  Keterbatasan fisik dan kognitif;
\item
  Kesediaan pasien untuk menerima informasi; dan
\item
  Nilai-nilai dan pilihan pasien.
\end{enumerate}

Hasil pengkajian tersebut dijadikan dasar oleh staf klinis dalam merencanakan dan melaksanakan pemberian informasi dan edukasi kepada pasien dan keluarga. Hasil pengkajian didokumentasikan di rekam medis pasien agar PPA yang terlibat merawat pasien dapat berpartisipasi dalam proses edukasi.

\hypertarget{elemen-penilaian-ke-3}{%
\subsubsection*{6. Elemen Penilaian KE 3}\label{elemen-penilaian-ke-3}}
\addcontentsline{toc}{subsubsection}{6. Elemen Penilaian KE 3}

\begin{enumerate}
\def\labelenumi{\alph{enumi}.}
\tightlist
\item
  Kebutuhan edukasi pasien dan keluarga dinilai berdasarkan pengkajian terhadap kemampuan dan kemauan belajar pasien dan keluarga yang meliputi poin a) -- f) pada maksud dan tujuan, dan dicatat di rekam medis.
\item
  Hambatan dari pasien dan keluarga dalam menerima edukasi dinilai sebelum pemberian edukasi dan dicatat di rekam medis.
\item
  Terdapat bukti dilakukan pengkajian kemampuan dan kemauan belajar pasien/keluarga, serta hasil pengkajian digunakan PPA untuk membuat perencanaan kebutuhan edukasi.
\end{enumerate}

\hypertarget{standar-ke-4}{%
\subsubsection*{7. Standar KE 4}\label{standar-ke-4}}
\addcontentsline{toc}{subsubsection}{7. Standar KE 4}

Edukasi tentang proses asuhan disampaikan kepada pasien dan keluarga disesuaikan dengan tingkat pemahaman dan bahasa yang dimengerti oleh pasien dan keluarga.

\hypertarget{maksud-dan-tujuan-ke-4}{%
\subsubsection*{8. Maksud dan Tujuan KE 4}\label{maksud-dan-tujuan-ke-4}}
\addcontentsline{toc}{subsubsection}{8. Maksud dan Tujuan KE 4}

Informasi dan edukasi yang diberikan kepada pasien dan keluarga sesuai dengan bahasa yang dipahaminya sesuai hasil pengkajian.

Mereka ikut terlibat dalam pembuatan keputusan dan berpartisipasi dalam asuhannya, serta dapat melanjutkan asuhan di rumah. Pasien/keluarga diberitahu tentang hasil pengkajian, diagnosis, rencana asuhan dan hasil pengobatan, termasuk hasil pengobatan yang tidak diharapkan.

Pasien dan keluarga diedukasi terkait cara cuci tangan yang aman, penggunaan obat yang aman, penggunaan peralatan medis yang aman, potensi interaksi antara obat dan makanan, pedoman nutrisi, manajemen nyeri, dan teknik rehabilitasi serta edukasi asuhan lanjutan di rumah.

\hypertarget{elemen-penilaian-ke-4}{%
\subsubsection*{9. Elemen penilaian KE 4}\label{elemen-penilaian-ke-4}}
\addcontentsline{toc}{subsubsection}{9. Elemen penilaian KE 4}

\begin{enumerate}
\def\labelenumi{\alph{enumi}.}
\tightlist
\item
  Terdapat bukti bahwa edukasi yang diberikan kepada pasien dan keluarga telah diberikan dengan cara dan bahasa yang mudah dipahami.
\item
  Terdapat bukti bahwa pasien/keluarga telah dijelaskan mengenai hasil pengkajian, diagnosis, rencana asuhan, dan hasil pengobatan, termasuk hasil pengobatan yang tidak diharapkan.
\item
  Terdapat bukti edukasi kepada pasien dan keluarga terkait dengan cara cuci tangan yang aman, penggunaan obat yang aman, penggunaan peralatan medis yang aman, potensi interaksi obat-obat dan obat-makanan, pedoman nutrisi, manajemen nyeri, dan teknik rehabilitasi serta edukasi asuhan lanjutan di rumah.
\end{enumerate}

\hypertarget{standar-ke-5}{%
\subsubsection*{10. Standar KE 5}\label{standar-ke-5}}
\addcontentsline{toc}{subsubsection}{10. Standar KE 5}

Metode edukasi dipilih dengan mempertimbangkan nilai yang dianut serta preferensi pasien dan keluarganya, untuk memungkinkan terjadinya interaksi yang memadai antara pasien, keluarga pasien dan staf.

\hypertarget{maksud-dan-tujuan-ke-5}{%
\subsubsection*{11. Maksud dan Tujuan KE 5}\label{maksud-dan-tujuan-ke-5}}
\addcontentsline{toc}{subsubsection}{11. Maksud dan Tujuan KE 5}

Proses edukasi akan berlangsung dengan baik bila mengunakan metode yang tepat. Pemahaman tentang kebutuhan edukasi pasien serta keluarganya akan membantu rumah sakit untuk memilih edukator dan metode edukasi yang sesuai dengan nilai dan preferensi dari pasien dan keluarganya, serta mengidentifikasi peran pasien/keluarga.

Dalam proses edukasi pasien dan keluarga didorong untuk bertanya/berdiskusi agar dapat berpartisipasi dalam proses asuhan. Materi edukasi yang diberikan harus selalu diperbaharui dan dapat dipahami oleh pasien dan keluarga. Pasien dan keluarga diberi kesempatan untuk berinteraksi aktif sehingga mereka dapat memberikan umpan balik untuk memastikan bahwa informasi dimengerti dan bermanfaat untuk diterapkan. Edukasi lisan dapat diperkuat dengan materi tertulis agar pemahaman pasien meningkat dan sebagai referensi untuk bahan edukasi selanjutnya.

Rumah sakit harus menyediakan penerjemah sesuai dengan kebutuhan pasien dan keluarga. Bila di rumah sakit tidak ada petugas penerjemah maka dapat dilakukan kerja sama dengan pihak ketiga diluar rumah sakit.

\hypertarget{elemen-penilaian-ke-5}{%
\subsubsection*{12. Elemen Penilaian KE 5}\label{elemen-penilaian-ke-5}}
\addcontentsline{toc}{subsubsection}{12. Elemen Penilaian KE 5}

\begin{enumerate}
\def\labelenumi{\alph{enumi}.}
\tightlist
\item
  Rumah sakit memiliki proses untuk memastikan bahwa pasien dan keluarganya memahami edukasi yang diberikan.
\item
  Proses pemberian edukasi di dokumentasikan dalam rekam medik sesuai dengan metode edukasi yang dapat diterima pasien dan keluarganya.
\item
  Materi edukasi untuk pasien dan keluarga selalu tersedia dan diperbaharui secara berkala.
\item
  Informasi dan edukasi disampaikan kepada pasien dan keluarga dengan menggunakan format yang praktis dan dengan bahasa yang dipahami pasien dan keluarga.
\item
  Rumah sakit menyediakan penerjemah (bahasa dan bahasa isyarat) sesuai dengan kebutuhan pasien dan keluarga.
\end{enumerate}

\hypertarget{standar-ke-6}{%
\subsubsection*{13. Standar KE 6}\label{standar-ke-6}}
\addcontentsline{toc}{subsubsection}{13. Standar KE 6}

Dalam menunjang keberhasilan asuhan yang berkesinambungan, upaya promosi kesehatan harus dilakukan berkelanjutan.

\hypertarget{maksud-dan-tujuan-ke-6}{%
\subsubsection*{14. Maksud dan Tujuan KE 6}\label{maksud-dan-tujuan-ke-6}}
\addcontentsline{toc}{subsubsection}{14. Maksud dan Tujuan KE 6}

Setelah mendapatkan pelayanan di rumah sakit, pasien terkadang membutuhkan pelayanan kesehatan berkelanjutan. Untuk itu rumah sakit perlu mengidentifikasi sumber-sumber yang dapat memberikan edukasi dan pelatihan yang tersedia di komunitas, khususnya organisasi dan fasilitas pelayanan kesehatan yang memberikan dukungan promosi kesehatan serta pencegahan penyakit.

Fasilitas pelayanan Kesehatan tersebut mencakup Fasilitas Kesehatan Tingkat Pertama (FKTP). Hal ini dilakukan agar tercapai hasil asuhan yang optimal setelah meninggalkan rumah sakit.

\hypertarget{elemen-penilaian-ke-6}{%
\subsubsection*{15. Elemen penilaian KE 6}\label{elemen-penilaian-ke-6}}
\addcontentsline{toc}{subsubsection}{15. Elemen penilaian KE 6}

\begin{enumerate}
\def\labelenumi{\alph{enumi}.}
\tightlist
\item
  Rumah sakit mengidentifikasi sumber-sumber yang ada di komunitas untuk mendukung promosi kesehatan berkelanjutan dan edukasi untuk menunjang asuhan pasien yang berkelanjutan.
\item
  Rumah sakit telah memiliki jejaring di komunitas untuk mendukung asuhan pasien berkelanjutan.
\item
  Memiliki bukti telah disampaikan kepada pasien dan keluarga tentang edukasi lanjutan dikomunitas. Rujukan edukasi tersebut dilaksanakan oleh jejaring utama yaitu Fasilitas Kesehatan Tingkat Pertama (FKTP).
\item
  Terdapat bukti edukasi berkelanjutan tersebut diberikan kepada pasien sesuai dengan kebutuhan.
\end{enumerate}

\hypertarget{standar-ke-7}{%
\subsubsection*{16. Standar KE 7}\label{standar-ke-7}}
\addcontentsline{toc}{subsubsection}{16. Standar KE 7}

Profesional Pemberi Asuhan (PPA) mampu memberikan edukasi secara efektif.

\hypertarget{maksud-dan-tujuan-ke-7}{%
\subsubsection*{17. Maksud dan Tujuan KE 7}\label{maksud-dan-tujuan-ke-7}}
\addcontentsline{toc}{subsubsection}{17. Maksud dan Tujuan KE 7}

Profesional Pemberi Asuhan (PPA) yang memberi asuhan memahami kontribusinya masing-masing dalam pemberian edukasi pasien. Informasi yang diterima pasien dan keluarga harus komprehensif, konsisten, dan efektif. Profesional Pemberi Asuhan (PPA) diberikan pelatihan sehingga terampil melaksanakan komunikasi efektif.

\hypertarget{elemen-penilaian-ke-7}{%
\subsubsection*{18. Elemen penilaian KE 7}\label{elemen-penilaian-ke-7}}
\addcontentsline{toc}{subsubsection}{18. Elemen penilaian KE 7}

\begin{enumerate}
\def\labelenumi{\alph{enumi}.}
\tightlist
\item
  Profesional Pemberi Asuhan (PPA) telah diberikan pelatihan dan terampil melaksanakan komunikasi efektif.
\item
  PPA telah memberikan edukasi yang efektif kepada pasien dan keluarga secara kolaboratif.
\end{enumerate}

\hypertarget{c.-kelompok-sasaran-keselamatan-pasien}{%
\chapter*{C. Kelompok Sasaran Keselamatan Pasien}\label{c.-kelompok-sasaran-keselamatan-pasien}}
\addcontentsline{toc}{chapter}{C. Kelompok Sasaran Keselamatan Pasien}

\textbf{Gambaran Umum}

Sasaran Keselamatan Pasien wajib diterapkan di rumah sakit untuk mencegah terjadinya insiden keselamatan pasien serta meningkatkan mutu pelayanan kesehatan sesuai dengan standar WHO Patient Safety (2007) yang digunakan juga oleh pemerintah. Tujuan SKP adalah untuk mendorong rumah sakit melakukan perbaikan-perbaikan yang menunjang tercapainya keselamatan pasien. Sasaran sasaran dalam SKP menyoroti bidang-bidang yang bermasalah dalam pelayanan kesehatan, memberikan bukti dan solusi hasil konsensus yang berdasarkan nasihat para pakar serta penelitian berbasis bukti.

Di Indonesia secara nasional untuk seluruh Fasilitas pelayanan Kesehatan, diberlakukan Sasaran Keselamatan Pasien Nasional yang terdiri dari:

\begin{enumerate}
\def\labelenumi{\arabic{enumi}.}
\tightlist
\item
  Sasaran 1 mengidentifikasi pasien dengan benar;
\item
  Sasaran 2 meningkatkan komunikasi yang efektif;
\item
  Sasaran 3 meningkatkan keamanan obat-obatan yang harus diwaspadai;
\item
  Sasaran 4 memastikan sisi yang benar, prosedur yang benar, pasien yang benar pada pembedahan/tindakan invasif;
\item
  Sasaran 5 mengurangi risiko infeksi akibat perawatan kesehatan; dan
\item
  Sasaran 6 mengurangi risiko cedera pasien akibat jatuh.
\end{enumerate}

\hypertarget{mengidentifikasi-pasien-dengan-benar}{%
\section*{1. Mengidentifikasi pasien dengan benar}\label{mengidentifikasi-pasien-dengan-benar}}
\addcontentsline{toc}{section}{1. Mengidentifikasi pasien dengan benar}

\hypertarget{a.-standar-skp-1}{%
\subsubsection*{a. Standar SKP 1}\label{a.-standar-skp-1}}
\addcontentsline{toc}{subsubsection}{a. Standar SKP 1}

Rumah sakit menerapkan proses untuk menjamin ketepatan identifikasi pasien

\hypertarget{b.-maksud-dan-tujuan-skp-1}{%
\subsubsection*{b. Maksud dan Tujuan SKP 1}\label{b.-maksud-dan-tujuan-skp-1}}
\addcontentsline{toc}{subsubsection}{b. Maksud dan Tujuan SKP 1}

Kesalahan mengidentifikasi pasien dapat terjadi di semua aspek pelayanan baik diagnosis, proses pengobatan serta tindakan. Misalnya saat keadaan pasien masih dibius, mengalami disorientasi atau belum sepenuhnya sadar; adanya kemungkinan pindah tempat tidur, pindah kamar, atau pindah lokasi di dalam rumah sakit; atau apabila pasien memiliki cacat indra atau rentan terhadap situasi berbeda.

Adapun tujuan dari identifikasi pasien secara benar ini adalah:

\begin{enumerate}
\def\labelenumi{\arabic{enumi}.}
\tightlist
\item
  mengidentifikasi pasien sebagai individu yang akan diberi layanan, tindakan atau pengobatan tertentu secara tepat.
\item
  mencocokkan layanan atau perawatan yang akan diberikan dengan pasien yang akan menerima layanan.
\end{enumerate}

Identifikasi pasien dilakukan setidaknya menggunakan minimal 2 (dua) identitas yaitu nama lengkap dan tanggal lahir/bar code, dan tidak termasuk nomor kamar atau lokasi pasien agar tepat pasien dan tepat pelayanan sesuai dengan regulasi rumah sakit. Pasien diidentifikasi menggunakan minimal dua jenis identitas pada saat:

\begin{enumerate}
\def\labelenumi{\arabic{enumi}.}
\tightlist
\item
  melakukan tindakan intervensi/terapi (misalnya pemberian obat, pemberian darah atau produk darah, melakukan terapi radiasi);
\item
  melakukan tindakan (misalnya memasang jalur intravena atau hemodialisis);
\item
  sebelum tindakan diagnostik apa pun (misalnya mengambil darah dan spesimen lain untuk pemeriksaan laboratorium penunjang, atau sebelum melakukan kateterisasi jantung ataupun tindakan radiologi diagnostik); dan
\item
  menyajikan makanan pasien.
\end{enumerate}

Rumah sakit memastikan pasien teridentifikasi dengan tepat pada situasi khusus, seperti pada pasien koma atau pada bayi baru lahir yang tidak segera diberi nama serta identifikasi pasien pada saat terjadi darurat bencana.

Penggunaan dua identitas juga digunakan dalam pelabelan. misalnya, sampel darah dan sampel patologi, nampan makanan pasien, label ASI yang disimpan untuk bayi yang dirawat di rumah sakit.

\hypertarget{c.-elemen-penilaian-skp-1}{%
\subsubsection*{c.~Elemen Penilaian SKP 1}\label{c.-elemen-penilaian-skp-1}}
\addcontentsline{toc}{subsubsection}{c.~Elemen Penilaian SKP 1}

\begin{enumerate}
\def\labelenumi{\arabic{enumi}.}
\tightlist
\item
  Rumah sakit telah menetapkan regulasi terkait Sasaran keselamatan pasien meliputi poin 1 -- 6 pada gambaran umum.
\item
  Rumah sakit telah menerapkan proses identifikasi pasien menggunakan minimal 2 (dua) identitas, dapat memenuhi tujuan identifikasi pasien dan sesuai dengan ketentuan rumah sakit.
\item
  Pasien telah diidentifikasi menggunakan minimal dua jenis identitas meliputi poin 1) - 4) dalam maksud dan tujuan.
\item
  Rumah sakit memastikan pasien teridentifikasi dengan tepat pada situasi khusus, dan penggunaan label seperti tercantum dalam maksud dan tujuan.
\end{enumerate}

\hypertarget{meningkatkan-komunikasi-yang-efektif}{%
\section*{2. Meningkatkan komunikasi yang efektif;}\label{meningkatkan-komunikasi-yang-efektif}}
\addcontentsline{toc}{section}{2. Meningkatkan komunikasi yang efektif;}

\hypertarget{a.-standar-skp-2}{%
\subsubsection*{a. Standar SKP 2}\label{a.-standar-skp-2}}
\addcontentsline{toc}{subsubsection}{a. Standar SKP 2}

Rumah sakit menerapkan proses untuk meningkatkan efektivitas komunikasi lisan dan/atau telepon di antara para profesional pemberi asuhan (PPA), proses pelaporan hasil kritis pada pemeriksaan diagnostic termasuk POCT dan proses komunikasi saat serah terima (hand over) .

\hypertarget{b.-maksud-dan-tujuan-skp-2}{%
\subsubsection*{b. Maksud dan Tujuan SKP 2}\label{b.-maksud-dan-tujuan-skp-2}}
\addcontentsline{toc}{subsubsection}{b. Maksud dan Tujuan SKP 2}

Komunikasi efektif adalah komunikasi yang tepat waktu, akurat, lengkap, jelas, dan dipahami oleh resipien/penerima pesan akan mengurangi potensi terjadinya kesalahan serta meningkatkan keselamatan pasien. Komunikasi dapat dilakukan secara lisan, tertulis dan elektronik.

Komunikasi yang paling banyak memiliki potensi terjadinya kesalahan adalah pemberian instruksi secara lisan atau melalui telpon, pelaporan hasil kritis dan saat serah terima.. Latar belakang suara, gangguan, nama obat yang mirip dan istilah yang tidak umum sering kali menjadi masalah.

Metode, formulir dan alat bantu ditetapkan sesuai dengan jenis komunikasi agar dapat dilakukan secara konsisten dan lengkap.

\begin{enumerate}
\def\labelenumi{\arabic{enumi}.}
\item
  Metode komunikasi saat menerima instruksi melalui telpon adalah: ``menulis/menginput ke komputer - membacakan - konfirmasi kembali'' (writedown, read back, confirmation) kepada pemberi instruksi misalnya kepada DPJP. Konfirmasi harus dilakukan saat itu juga melalui telpon untuk menanyakan apakah ``yang dibacakan'' sudah sesuai dengan instruksi yang diberikan. Sedangkan metode komunikasi saat melaporkan kondisi pasien kepada DPJP dapat menggunakan metode misalnya Situation --background - assessment - recommendation (SBAR).
\item
  Metode komunikasi saat melaporkan nilai kritis pemeriksaan diagnostik melalui telpon juga dapat dengan: ``menulis/menginput ke komputer - membacakan - konfirmasi kembali'' (writedown, read back). Hasil kritis didefinisikan sebagai varian dari rentang normal yang menunjukkan adanya kondisi patofisiologis yang berisiko tinggi atau mengancam nyawa, yang dianggap gawat atau darurat, dan mungkin memerlukan tindakan medis segera untuk menyelamatkan nyawa atau mencegah kejadian yang tidak diinginkan. Hasil kritis dapat dijumpai pada pemeriksaan pasien rawat jalan maupun rawat inap. Rumah sakit menentukan mekanisme pelaporan hasil kritis di rawat jalan dan rawat inap. Pemeriksaan diagnostik mencakup semua pemeriksaan seperti laboratorium, pencitraan/radiologi, diagnostik jantung juga pada hasil pemeriksaan yang dilakukan di tempat tidur pasien (point- of-care testing (POCT). Pada pasien rawat inap pelaporan hasil kritis dapat dilaporkan melalui perawat yang akan meneruskan laporan kepada DPJP yang meminta pemeriksaan. Rentang waktu pelaporan hasil kritis ditentukan kurang dari 30 menit sejak hasil di verifikasi oleh PPA yang berwenang di unit pemeriksaan penunjang diagnostik.
\item
  Metode komunikasi saat serah terima distandardisasi pada jenis serah terima yang sama misalnya serah terima antar ruangan di rawat inap. Untuk jenis serah terima yang berbeda maka dapat menggunakan metode, formulir dan alat yang berbeda. Misalnya serah terima dari IGD ke ruang rawat inap dapat berbeda dengan serah terima dari kamar operasi ke unit intensif;
  Jenis serah terima (handover) di dalam rumah sakit dapat mencakup:
\item
  antara PPA (misalnya, antar dokter, dari dokter ke perawat, antar perawat, dan seterusnya);
\item
  antara unit perawatan yang berbeda di dalam rumah sakit (misalnya saat pasien dipindahkan dari ruang perawatan intensif ke ruang perawatan atau dari instalasi gawat darurat ke ruang operasi); dan
\item
  dari ruang perawatan pasien ke unit layanan diagnostik seperti radiologi atau fisioterapi.
\end{enumerate}

Formulir serah terima antara PPA, tidak perlu dimasukkan ke dalam rekam medis. Namun demikian, rumah sakit harus memastikan bahwa proses serah terima telah dilakukan. misalnya PPA mencatat serah terima telah dilakukan dan kepada siapa tanggung jawab pelayanan diserahterimakan, kemudian dapat dibubuhkan tanda tangan, tanggal dan waktu pencatatan).

\hypertarget{c.-elemen-penilaian-skp-2}{%
\subsubsection*{c.~Elemen Penilaian SKP 2}\label{c.-elemen-penilaian-skp-2}}
\addcontentsline{toc}{subsubsection}{c.~Elemen Penilaian SKP 2}

\begin{enumerate}
\def\labelenumi{\arabic{enumi}.}
\tightlist
\item
  Rumah sakit telah menerapkan komunikasi saat menerima instruksi melalui telepon: menulis/menginput ke komputer - membacakan - konfirmasi kembali'' (writedown, read back, confirmation dan SBAR saat melaporkan kondisi pasien kepada DPJP serta di dokumentasikan dalam rekam medik.
\item
  Rumah sakit telah menerapkan komunikasi saat pelaporan hasil kritis pemeriksaan penunjang diagnostic melalui telepon: menulis/menginput ke komputer -- membacakan -- konfirmasi kembali'' (writedown, read back, confirmation dan di dokumentasikan dalam rekam medik.
\item
  Rumah sakit telah menerapkan komunikasi saat serah terima sesuai dengan jenis serah terima meliputi poin 1) - 3) dalam maksud dan tujuan.
\end{enumerate}

\hypertarget{meningkatkan-keamanan-obat-obatan-yang-harus-diwaspadai}{%
\section*{3. Meningkatkan keamanan obat-obatan yang harus diwaspadai}\label{meningkatkan-keamanan-obat-obatan-yang-harus-diwaspadai}}
\addcontentsline{toc}{section}{3. Meningkatkan keamanan obat-obatan yang harus diwaspadai}

\hypertarget{a.-standar-skp-3}{%
\subsubsection*{a. Standar SKP 3}\label{a.-standar-skp-3}}
\addcontentsline{toc}{subsubsection}{a. Standar SKP 3}

Rumah sakit menerapkan proses untuk meningkatkan keamanan penggunaan obat yang memerlukan kewaspadaan tinggi (high alert medication) termasuk obat Look - Alike Sound Alike (LASA).

\hypertarget{b.-standar-skp-3.1}{%
\subsubsection*{b. Standar SKP 3.1}\label{b.-standar-skp-3.1}}
\addcontentsline{toc}{subsubsection}{b. Standar SKP 3.1}

Rumah sakit menerapkan proses untuk meningkatkan keamanan penggunaan elektrolit konsentrat

\hypertarget{c.-maksud-dan-tujuan-skp-3-dan-skp-3.1}{%
\subsubsection*{c.~Maksud dan Tujuan SKP 3 dan SKP 3.1}\label{c.-maksud-dan-tujuan-skp-3-dan-skp-3.1}}
\addcontentsline{toc}{subsubsection}{c.~Maksud dan Tujuan SKP 3 dan SKP 3.1}

Obat-obatan yang perlu diwaspadai (high-alert medications) adalah obat-obatan yang memiliki risiko menyebabkan cedera serius pada pasien jika digunakan dengan tidak tepat.

Obat high alert mencakup:
1. Obat risiko tinggi, yaitu obat dengan zat aktif yang dapat menimbulkan kematian atau kecacatan bila terjadi kesalahan (error) dalam penggunaannya (contoh: insulin, heparin atau sitostatika).
2. Obat yang terlihat mirip dan kedengarannya mirip (Nama Obat Rupa dan Ucapan Mirip/NORUM, atau Look Alike Sound Alike/LASA)
3. Elektrolit konsentrat contoh: kalium klorida dengan konsentrasi sama atau lebih dari 1 mEq/ml, natrium klorida dengan konsentrasi lebih dari 0,9\% dan magnesium sulfat injeksi dengan konsentrasi sama atau lebih dari 50\%

Rumah sakit harus menetapkan dan menerapkan strategi untuk mengurangi risiko dan cedera akibat kesalahan penggunaan obat high alert, antara lain: penataan penyimpanan, pelabelan yang jelas, penerapan double checking, pembatasan akses, penerapan panduan penggunaan obat high alert.

Rumah sakit perlu membuat daftar obat-obatan berisiko tinggi berdasarkan pola penggunaan obat-obatan yang berisiko dari data internalnya sendiri tentang laporan inisiden keselamatan pasien. Daftar ini sebaiknya diperbarui setiap tahun. Daftar ini dapat diperbarui secara sementara jika ada penambahan atau perubahan pada layanan rumah sakit.

Obat dengan nama dan rupa yang mirip (look-alike/sound-alike, LASA) adalah obat yang memiliki tampilan dan nama yang serupa dengan obat lain, baik saat ditulis maupun diucapkan secara lisan. Obat dengan kemasan serupa (look-alike packaging) adalah obat dengan wadah atau kemasan yang mirip dengan obat lainnya.

Obat-obatan yang berisiko terjadinya kesalahan terkait LASA, atau obat dengan kemasan produk yang serupa, dapat menyebabkan terjadinya kesalahan pengobatan yang berpotensi cedera. Terdapat banyak nama obat yang terdengar serupa dengan nama obat lainnya, sebagai contoh, dopamin dan dobutamin.

Hal lain yang sering dimasukkan dalam isu keamanan obat adalah kesalahan dalam pemberian elektrolit konsentrat yang tidak disengaja (misalnya, kalium/potasium klorida {[}sama dengan 1 mEq/ml atau yang lebih pekat), kalium/potasium fosfat {[}(sama dengan atau lebih besar dari 3 mmol/ml){]}, natrium/sodium klorida {[}lebih pekat dari 0.9\%{]}, dan magnesium sulfat {[}sama dengan 50\% atau lebih pekat{]}. Kesalahan ini dapat terjadi apabila staf tidak mendapatkan orientasi dengan baik di unit asuhan pasien, bila perawat kontrak tidak diorientasikan sebagaimana mestinya terhadap unit asuhan pasien, atau pada keadaan gawat darurat/emergensi. Cara yang paling efektif untuk mengurangi atau mengeliminasi kejadian tersebut adalah dengan menerapkan proses pengelolaan obat-obat yang perlu diwaspadai termasuk penyimpanan elektrolit konsentrat di unit farmasi di rumah sakit.

Penyimpanan elektrolit konsentrat di luar Instalasi Farmasi diperbolehkan hanya dalam situasi klinis yang berisiko dan harus memenuhi persyaratan yaitu staf yang dapat mengakes dan memberikan elektrolit konsentrat adalah staf yang kompeten dan terlatih, disimpan terpisah dari obat lain, diberikan pelabelan secara jelas, lengkap dengan peringatan kewaspadaan.

\hypertarget{d.-elemen-penilaian-skp-3}{%
\subsubsection*{d.~Elemen Penilaian SKP 3}\label{d.-elemen-penilaian-skp-3}}
\addcontentsline{toc}{subsubsection}{d.~Elemen Penilaian SKP 3}

\begin{enumerate}
\def\labelenumi{\arabic{enumi}.}
\tightlist
\item
  Rumah sakit menetapkan daftar obat kewaspadaan tinggi
\item
  (High Alert) termasuk obat Look -Alike Sound Alike (LASA).
\item
  Rumah sakit menerapkan pengelolaan obat kewaspadaan tinggi (High Alert) termasuk obat Look -Alike Sound Alike (LASA) secara seragam di seluruh area rumah sakit untuk mengurangi risiko dan cedera
\item
  Rumah sakit mengevaluasi dan memperbaharui daftar obat High-Alert dan obat Look -Alike Sound Alike (LASA) yang sekurang-kurangnya 1 (satu) tahun sekali berdasarkan laporan insiden lokal, nasional dan internasional.
\end{enumerate}

\hypertarget{e.-elemen-penilaian-skp-3.1}{%
\subsubsection*{e. Elemen Penilaian SKP 3.1}\label{e.-elemen-penilaian-skp-3.1}}
\addcontentsline{toc}{subsubsection}{e. Elemen Penilaian SKP 3.1}

\begin{enumerate}
\def\labelenumi{\arabic{enumi}.}
\tightlist
\item
  Rumah sakit menerapkan proses penyimpanan elektrolit konsentrat tertentu hanya di Instalasi Farmasi, kecuali di unit pelayanan dengan pertimbangan klinis untuk mengurangi risiko dan cedera pada penggunaan elektrolit konsentrat.
\item
  Penyimpanan elektrolit konsentrat di luar Instalasi Farmasi diperbolehkan hanya dalam untuk situasi yang ditentukan sesuai dalam maksud dan tujuan.
\item
  Rumah sakit menetapkan dan menerapkan protokol koreksi hipokalemia, hiponatremia, hipofosfatemia.
\end{enumerate}

\hypertarget{memastikan-sisi-yang-benar-prosedur-yang-benar-pasien-yang-benar-pada-pembedahantindakan-invasif}{%
\section*{4. Memastikan sisi yang benar, prosedur yang benar, pasien yang benar pada pembedahan/tindakan invasif}\label{memastikan-sisi-yang-benar-prosedur-yang-benar-pasien-yang-benar-pada-pembedahantindakan-invasif}}
\addcontentsline{toc}{section}{4. Memastikan sisi yang benar, prosedur yang benar, pasien yang benar pada pembedahan/tindakan invasif}

\hypertarget{a.-standar-skp-4}{%
\subsubsection*{a. Standar SKP 4}\label{a.-standar-skp-4}}
\addcontentsline{toc}{subsubsection}{a. Standar SKP 4}

Rumah sakit menetapkan proses untuk melaksanakan verifikasi pra opearsi, penandaan lokasi operasi dan proses time-out yang dilaksanakan sesaat sebelum tindakan pembedahan/invasif dimulai serta proses sign-out yang dilakukan setelah tindakan selesai.

\hypertarget{b.-maksud-dan-tujuan-skp-4}{%
\subsubsection*{b. Maksud dan Tujuan SKP 4}\label{b.-maksud-dan-tujuan-skp-4}}
\addcontentsline{toc}{subsubsection}{b. Maksud dan Tujuan SKP 4}

Salah-sisi, salah-prosedur, salah-pasien operasi, adalah kejadian yang mengkhawatirkan dan dapat terjadi di rumah sakit. Kesalahan ini terjadi akibat adanya komunikasi yang tidak efektif atau tidak adekuat antara anggota tim bedah, kurangnya keterlibatan pasien di dalam penandaan lokasi (site marking), serta tidak adanya prosedur untuk memverifikasi sisi operasi. Rumah sakit memerlukan upaya kolaboratif untuk mengembangkan proses dalam mengeliminasi masalah ini.

Tindakan operasi dan invasif meliputi semua tindakan yang melibatkan insisi atau pungsi, termasuk, tetapi tidak terbatas pada, operasi terbuka, aspirasi perkutan, injeksi obat tertentu, biopsi, tindakan intervensi atau diagnostik vaskuler dan kardiak perkutan, laparoskopi, dan endoskopi. Rumah sakit perlu mengidentifikasi semua area di rumah sakit mana operasi dan tindakan invasif dilakukan Protokol umum (universal protocol) untuk pencegahan salah sisi, salah prosedur dan salah pasien pembedahan meliputi:

\begin{enumerate}
\def\labelenumi{\arabic{enumi}.}
\tightlist
\item
  Proses verifikasi sebelum operasi.
\item
  Penandaan sisi operasi.
\item
  Time-out dilakukan sesaat sebelum memulai tindakan.
\end{enumerate}

\hypertarget{c.-proses-verifikasi-praoperasi}{%
\subsubsection*{c.~Proses Verifikasi Praoperasi}\label{c.-proses-verifikasi-praoperasi}}
\addcontentsline{toc}{subsubsection}{c.~Proses Verifikasi Praoperasi}

Verifikasi praoperasi merupakan proses pengumpulan informasi dan konfirmasi secara terus-menerus. Tujuan dari proses verifikasi praoperasi adalah:

\begin{enumerate}
\def\labelenumi{\arabic{enumi}.}
\tightlist
\item
  melakukan verifikasi terhadap sisi yang benar, prosedur yang benar dan pasien yang benar;
\item
  memastikan bahwa semua dokumen, foto hasil radiologi atau pencitraan, dan pemeriksaan yang terkait operasi telah tersedia, sudah diberi label dan di siapkan;
\item
  melakukan verifikasi bahwa produk darah, peralatan medis khusus dan/atau implan yang diperlukan sudah tersedia.
\end{enumerate}

Di dalam proses verifikasi praoperasi terdapat beberapa elemen yang dapat dilengkapi sebelum pasien tiba di area praoperasi. seperti memastikan bahwa dokumen, foto hasil radiologi, dan hasil pemeriksaan sudah tersedia, di beri label dan sesuai dengan penanda identitas pasien.

Menunggu sampai pada saat proses time-out untuk melengkapi proses verifikasi praoperasi dapat menyebabkan penundaan yang tidak perlu. Beberapa proses verifikasi praoperasi dapat dilakukan lebih dari sekali dan tidak hanya di satu tempat saja. Misalnya persetujuan tindakan bedah dapat diambil di ruang periksa dokter spesialis bedah dan verifikasi kelengkapannya dapat dilakukan di area tunggu praoperasi.

\hypertarget{d.-penandaan-lokasi}{%
\subsubsection*{d.~Penandaan Lokasi}\label{d.-penandaan-lokasi}}
\addcontentsline{toc}{subsubsection}{d.~Penandaan Lokasi}

Penandaan sisi operasi dilakukan dengan melibatkan pasien serta dengan tanda yang tidak memiliki arti ganda serta segera dapat dikenali. Tanda tersebut harus digunakan secara konsisten di dalam rumah sakit; dan harus dibuat oleh PPA yang akan melakukan tindakan; harus dibuat saat pasien terjaga dan sadar jika memungkinkan, dan harus terlihat sampai pasien disiapkan. Penandaan sisi operasi hanya ditandai pada semua kasus yang memiliki dua sisi kiri dan kanan (lateralisasi), struktur multipel (jari tangan, jari kaki, lesi), atau multiple level (tulang belakang).

Penandaan lokasi operasi harus melibatkan pasien dan dilakukan dengan tanda yang langsung dapat dikenali dan tidak bermakna ganda. Tanda ``X'' tidak digunakan sebagai penanda karena dapat diartikan sebagai ``bukan di sini'' atau ``salah sisi'' serta dapat berpotensi menyebabkan kesalahan dalam penandaan lokasi operasi. Tanda yang dibuat harus seragam dan konsisten digunakan di rumah sakit. Dalam semua kasus yang melibatkan lateralitas, struktur ganda (jari tangan, jari kaki, lesi), atau tingkatan berlapis (tulang belakang), lokasi operasi harus ditandai.

Penandaan lokasi tindakan operasi/invasif dilakukan oleh PPA yang akan melakukan tindakan tersebut. PPA tersebut akan melakukan seluruh prosedur operasi/invasif dan tetap berada dengan pasien selama tindakan berlangsung. Pada tindakan operasi, DPJP bedah pada umumnya yang akan melakukan operasi dan kemudian melakukan penandaan lokasi.. Untuk tindakan invasif non-operasi, penandaan dapat dilakukan oleh dokter yang akan melakukan tindakan, dan dapat dilakukan di area di luar area kamar operasi. Terdapat situasi di mana peserta didik (trainee) dapat melakukan penandaan lokasi, misalnya ketika peserta didik akan melakukan keseluruhan tindakan, tidak memerlukan supervisi atau memerlukan supervisi minimal dari operator/dokter penanggung jawab.

Pada situasi tersebut, peserta didik dapat menandai lokasi operasi. Ketika seorang peserta didik menjadi asisten dari operator/dokter penanggung jawab, hanya operator/dokter penanggung jawab yang dapat melakukan penandaan lokasi. Penandaan lokasi dapat terjadi kapan saja sebelum tindakan operasi/invasif selama pasien terlibat secara aktif dalam proses penandaan lokasi jika memungkinkan dan tanda tersebut harus tetap dapat terlihat walaupun setelah pasien dipersiapkan dan telah ditutup kain. Contoh keadaan di mana partisipasi pasien tidak memungkinkan meliputi : kasus di mana pasien tidak kompeten untuk membuat keputusan perawatan, pasien anak, dan pasien yang memerlukan operasi darurat.

\hypertarget{e.-time-out}{%
\subsubsection*{e. Time-Out}\label{e.-time-out}}
\addcontentsline{toc}{subsubsection}{e. Time-Out}

Time-out dilakukan sesaat sebelum tindakan dimulai dan dihadiri semua anggota tim yang akan melaksanakan tindakan operasi. Selama time-out, tim menyetujui komponen sebagai berikut:

\begin{enumerate}
\def\labelenumi{\arabic{enumi}.}
\tightlist
\item
  Benar identitas pasien.
\item
  Benar prosedur yang akan dilakukan.
\item
  Benar sisi operasi/tindakan invasif.
\end{enumerate}

Time-out dilakukan di tempat di mana tindakan akan dilakukan dan melibatkan secara aktif seluruh tim bedah. Pasien tidak berpartisipasi dalam time-out. Keseluruhan proses time-out didokumentasikan dan meliputi tanggal serta jam time-out selesai. Rumah sakit menentukan bagaimana proses time-out didokumentasikan.

\hypertarget{f.-sign-out}{%
\subsubsection*{f.~Sign-Out}\label{f.-sign-out}}
\addcontentsline{toc}{subsubsection}{f.~Sign-Out}

Sign out yang dilakukan di area tempat tindakan berlangsung sebelum pasien meninggalkan ruangan. Pada umumnya, perawat sebagai anggota tim melakukan konfirmasi secara lisan untuk komponen sign-out sebagai berikut:

\begin{enumerate}
\def\labelenumi{\arabic{enumi}.}
\tightlist
\item
  Nama tindakan operasi/invasif yang dicatat/ditulis.
\item
  Kelengkapan perhitungan instrumen, kasa dan jarum (bila ada).
\item
  Pelabelan spesimen (ketika terdapat spesimen selama proses sign-out, label dibacakan dengan jelas, meliputi nama pasien, tanggal lahir).
\item
  Masalah peralatan yang perlu ditangani (bila ada).
\end{enumerate}

Rumah sakit dapat menggunakan Daftar tilik keselamatan operasi (Surgical Safety Checklist dari WHO terkini)

\hypertarget{g.-elemen-penilaian-skp-4}{%
\subsubsection*{g. Elemen Penilaian SKP 4}\label{g.-elemen-penilaian-skp-4}}
\addcontentsline{toc}{subsubsection}{g. Elemen Penilaian SKP 4}

\begin{enumerate}
\def\labelenumi{\arabic{enumi}.}
\tightlist
\item
  Rumah sakit telah melaksanakan proses verifikasi pra operasi dengan daftar tilik untuk memastikan benar pasien, benar tindakan dan benar sisi.
\item
  Rumah sakit telah menetapkan dan menerapkan tanda yang seragam, mudah dikenali dan tidak bermakna ganda untuk mengidentifikasi sisi operasi atau tindakan invasif.
\item
  Rumah sakit telah menerapkan penandaan sisi operasi atau tindakan invasif (site marking) dilakukan oleh dokter operator/dokter asisten yang melakukan operasi atau tindakan invasif dengan melibatkan pasien bila memungkinkan.
\item
  Rumah sakit telah menerapkan proses Time-Out menggunakan ``surgical check list'' (Surgical Safety Checklist dari WHO terkini pada tindakan operasi termasuk tindakan medis invasif.
\end{enumerate}

\hypertarget{mengurangi-risiko-infeksi-akibat-perawatan-kesehatan}{%
\section*{5. Mengurangi risiko infeksi akibat perawatan kesehatan;}\label{mengurangi-risiko-infeksi-akibat-perawatan-kesehatan}}
\addcontentsline{toc}{section}{5. Mengurangi risiko infeksi akibat perawatan kesehatan;}

\hypertarget{a.-standar-skp-5}{%
\subsubsection*{a. Standar SKP 5}\label{a.-standar-skp-5}}
\addcontentsline{toc}{subsubsection}{a. Standar SKP 5}

Rumah sakit menerapkan kebersihan tangan (hand hygiene) untuk menurunkan risiko infeksi terkait layanan kesehatan.

\hypertarget{b.-maksud-dan-tujuan-skp-5}{%
\subsubsection*{b. Maksud dan Tujuan SKP 5}\label{b.-maksud-dan-tujuan-skp-5}}
\addcontentsline{toc}{subsubsection}{b. Maksud dan Tujuan SKP 5}

Pencegahan dan pengendalian infeksi merupakan tantangan praktisi dalam tatanan pelayanan kesehatan, dan peningkatan biaya untuk mengatasi infeksi yang berhubungan dengan pelayanan kesehatan merupakan hal yang sangat membebani pasien serta profesional pemberi asuhan (PPA) pada pelayanan kesehatan.

Infeksi umumnya dijumpai dalam semua bentuk pelayanan kesehatan termasuk infeksi saluran kemih-terkait kateter, infeksi aliran darah (blood stream infections) dan pneumonia (sering kali dihubungkan dengan ventilasi mekanis). Kegiatan utama dari upaya eliminasi infeksi ini maupun infeksi lainnya adalah dengan melakukan tindakan cuci tangan (hand hygiene) yang tepat.

Pedoman hand hygiene yang berlaku secara internasional dapat diperoleh di situs web WHO. Rumah sakit harus memiliki proses kolaboratif untuk mengembangkan kebijakan dan/atau prosedur yang menyesuaikan atau mengadopsi pedoman hand hygiene yang diterima secara luas untuk implementasinya di rumah sakit.

\hypertarget{c.-elemen-penilaian-skp-5}{%
\subsubsection*{c.~Elemen Penilaian SKP 5}\label{c.-elemen-penilaian-skp-5}}
\addcontentsline{toc}{subsubsection}{c.~Elemen Penilaian SKP 5}

\begin{enumerate}
\def\labelenumi{\arabic{enumi}.}
\tightlist
\item
  Rumah sakit telah menerapkan kebersihan tangan (hand hygiene) yang mengacu pada standar WHO terkini.
\item
  Terdapat proses evaluasi terhadap pelaksanaan program kebersihan tangan di rumah sakit serta upaya perbaikan yang dilakukan untuk meningkatkan pelaksanaan program.
\end{enumerate}

\hypertarget{mengurangi-risiko-cedera-pasien-akibat-jatuh}{%
\section*{6. Mengurangi risiko cedera pasien akibat jatuh}\label{mengurangi-risiko-cedera-pasien-akibat-jatuh}}
\addcontentsline{toc}{section}{6. Mengurangi risiko cedera pasien akibat jatuh}

\hypertarget{a.-standar-skp-6}{%
\subsubsection*{a. Standar SKP 6}\label{a.-standar-skp-6}}
\addcontentsline{toc}{subsubsection}{a. Standar SKP 6}

Rumah sakit menerapkan proses untuk mengurangi risiko cedera pasien akibat jatuh di rawat jalan.

\hypertarget{b.-standar-skp-6.1}{%
\subsubsection*{b. Standar SKP 6.1}\label{b.-standar-skp-6.1}}
\addcontentsline{toc}{subsubsection}{b. Standar SKP 6.1}

Rumah sakit menerapkan proses untuk mengurangi risiko cedera pasien akibat jatuh di rawat inap.

\hypertarget{c.-maksud-dan-tujuan-skp-6-dan-6.1}{%
\subsubsection*{c.~Maksud dan Tujuan SKP 6 dan 6.1}\label{c.-maksud-dan-tujuan-skp-6-dan-6.1}}
\addcontentsline{toc}{subsubsection}{c.~Maksud dan Tujuan SKP 6 dan 6.1}

Risiko jatuh pada pasien rawat jalan berhubungan dengan kondisi pasien, situasi, dan/atau lokasi di rumah sakit. Di unit rawat jalan, dilakukan skrining risiko jatuh pada pasien dengan kondisi, diagnosis, situasi, dan/atau lokasi yang menyebabkan risiko jatuh. Jika hasil skrining pasien berisiko jatuh, maka harus dilakukan intervensi untuk mengurangi risiko jatuh pasien tersebut. Skrining risiko jatuh di rawat jalan meliputi:

\begin{enumerate}
\def\labelenumi{\arabic{enumi}.}
\tightlist
\item
  kondisi pasien misalnya pasien geriatri, dizziness, vertigo, gangguan keseimbangan, gangguan penglihatan, penggunaan obat, sedasi, status kesadaran dan atau kejiwaan, konsumsi alkohol.
\item
  diagnosis, misalnya pasien dengan diagnosis penyakit Parkinson.
\item
  situasi misalnya pasien yang mendapatkan sedasi atau pasien dengan riwayat tirah baring/perawatan yang lama yang akan dipindahkan untuk pemeriksaan penunjang dari ambulans, perubahan posisi akan meningkatkan risiko jatuh.
\item
  lokasi misalnya area-area yang berisiko pasien jatuh, yaitu tangga, area yang penerangannya kurang atau mempunyai unit pelayanan dengan peralatan parallel bars, freestanding staircases seperti unit rehabilitasi medis. Ketika suatu lokasi tertentu diidentifikasi sebagai area risiko tinggi yang lebih rumah sakit dapat menentukan bahwa semua pasien yang mengunjungi lokasi tersebut akan dianggap berisiko jatuh dan menerapkan langkah-langkah untuk mengurangi risiko jatuh yang berlaku untuk semua pasien.
\end{enumerate}

Skrining umumnya berupa evaluasi sederhana meliputi pertanyaan dengan jawaban sederhana: ya/tidak, atau metode lain meliputi pemberian nilai/skor untuk setiap respons pasien. Rumah sakit dapat menentukan bagaimana proses skrining dilakukan. Misalnya skrining dapat dilakukan oleh petugas registrasi, atau pasien dapat melakukan skrining secara mandiri, seperti di anjungan mandiri untuk skrining di unit rawat jalan.
Contoh pertanyaan skrining sederhana dapat meliputi:

\begin{enumerate}
\def\labelenumi{\arabic{enumi}.}
\tightlist
\item
  Apakah Anda merasa tidak stabil ketika berdiri atau berjalan?;
\item
  Apakah Anda khawatir akan jatuh?;
\item
  Apakah Anda pernah jatuh dalam setahun terakhir?
\end{enumerate}

Rumah sakit dapat menentukan pasien rawat jalan mana yang akan dilakukan skrining risiko jatuh. Misalnya, semua pasien di unit rehabilitasi medis, semua pasien dalam perawatan lama/tirah baring lama datang dengan ambulans untuk pemeriksaan rawat jalan, pasien yang dijadwalkan untuk operasi rawat jalan dengan tindakan anestesi atau sedasi, pasien dengan gangguan keseimbangan, pasien dengan gangguan penglihatan, pasien anak di bawah usia 2 (dua) tahun, dan seterusnya.

Untuk semua pasien rawat inap baik dewasa maupun anak harus dilakukan pengkajian risiko jatuh menggunakan metode pengkajian yang baku sesuai ketentuan rumah sakit. Kriteria risiko jatuh dan intervensi yang dilakukan harus didokumentasikan dalam rekam medis pasien.

Pasien yang sebelumnya risiko rendah jatuh dapat meningkat risikonya secara mendadak menjadi risiko tinggi jatuh. Perubahan risiko ini dapat diakibatkan, namun tidak terbatas pada tindakan pembedahan dan/atau anestesi, perubahan mendadak pada kondisi pasien, dan penyesuaian obat-obatan yang diberikan sehingga pasien memerlukan pengkajian ulang jatuh selama dirawat inap dan paska pembedahan.

\hypertarget{d.-elemen-penilaian-skp-6}{%
\subsubsection*{d.~Elemen Penilaian SKP 6}\label{d.-elemen-penilaian-skp-6}}
\addcontentsline{toc}{subsubsection}{d.~Elemen Penilaian SKP 6}

\begin{enumerate}
\def\labelenumi{\arabic{enumi}.}
\tightlist
\item
  Rumah sakit telah melaksanakan skrining pasien rawat jalan pada kondisi, diagnosis, situasi atau lokasi yang dapat menyebabkan pasien berisiko jatuh, dengan menggunakan alat bantu/metode skrining yang ditetapkan rumah sakit
\item
  Tindakan dan/atau intervensi dilakukan untuk mengurangi risiko jatuh pada pasien jika hasil skrining menunjukkan adanya risiko jatuh dan hasil skrining serta intervensi didokumentasikan.
\end{enumerate}

\hypertarget{e.-elemen-penilaian-skp-6.1}{%
\subsubsection*{e. Elemen Penilaian SKP 6.1}\label{e.-elemen-penilaian-skp-6.1}}
\addcontentsline{toc}{subsubsection}{e. Elemen Penilaian SKP 6.1}

\begin{enumerate}
\def\labelenumi{\arabic{enumi}.}
\tightlist
\item
  Rumah sakit telah melakukan pengkajian risiko jatuh untuk semua pasien rawat inap baik dewasa maupun anak menggunakan metode pengkajian yang baku sesuai dengan ketentuan rumah sakit.
\item
  Rumah sakit telah melaksanakan pengkajian ulang risiko jatuh pada pasien rawat inap karena adanya perubahan kondisi, atau memang sudah mempunyai risiko jatuh dari hasil pengkajian.
\item
  Tindakan dan/atau intervensi untuk mengurangi risiko jatuh pada pasien rawat inap telah dilakukan dan didokumentasikan.
\end{enumerate}

\hypertarget{d.-program-nasional}{%
\chapter*{D. Program Nasional}\label{d.-program-nasional}}
\addcontentsline{toc}{chapter}{D. Program Nasional}

\textbf{Gambaran Umum}

Pada Rencana Pembangunan Jangka Menengah Nasional (RPJMN) bidang kesehatan telah ditentukan prioritas pelayanan kesehatan dengan target yang harus dicapai. Salah satu fungsi rumah sakit adalah melaksanakan program pemerintah dan mendukung tercapainya target target pembangunan nasional. Pada standar akreditasi ini Program Nasional (Prognas) meliputi:

\begin{enumerate}
\def\labelenumi{\arabic{enumi}.}
\tightlist
\item
  Peningkatan kesehatan ibu dan bayi.
\item
  Penurunan angka kesakitan Tuberkulosis/TBC.
\item
  Penurunan angka kesakitan HIV/AIDS.
\item
  Penurunan prevalensi stunting dan wasting.
\item
  Pelayanan Keluarga Berencana Rumah Sakit.
\end{enumerate}

Pelaksanaan program nasional oleh rumah sakit diharapkan mampu meningkatkan akselerasi pencapaian target RPJMN bidang kesehatan sehingga upaya mingkatkan derajat kesehatan masyarakat meningkat segera terwujud.

\hypertarget{peningkatan-kesehatan-ibu-dan-bayi}{%
\section*{1. Peningkatan kesehatan ibu dan bayi}\label{peningkatan-kesehatan-ibu-dan-bayi}}
\addcontentsline{toc}{section}{1. Peningkatan kesehatan ibu dan bayi}

\hypertarget{a.-standar-prognas-1}{%
\subsubsection*{a. Standar Prognas 1}\label{a.-standar-prognas-1}}
\addcontentsline{toc}{subsubsection}{a. Standar Prognas 1}

Rumah sakit melaksanakan program PONEK 24 jam dan 7 (tujuh) hari seminggu.

\hypertarget{b.-maksud-dan-tujuan-prognas-1}{%
\subsubsection*{b. Maksud dan Tujuan Prognas 1}\label{b.-maksud-dan-tujuan-prognas-1}}
\addcontentsline{toc}{subsubsection}{b. Maksud dan Tujuan Prognas 1}

Rumah sakit melaksanakan program PONEK sesuai dengan pedoman PONEK yang berlaku dengan langkah langkah sebagai berikut:

\begin{enumerate}
\def\labelenumi{\arabic{enumi}.}
\tightlist
\item
  Melaksanakan dan menerapkan standar pelayanan perlindungan ibu dan bayi secara terpadu.
\item
  Mengembangkan kebijakan dan standar pelayanan ibu dan bayi.
\item
  Meningkatkan kualitas pelayanan kesehatan ibu dan bayi.
\item
  Meningkatkan kesiapan rumah sakit dalam melaksanakan fungsi pelayanan obstetric dan neonates termasuk pelayanan kegawatdaruratan (PONEK 24 jam).
\item
  Meningkatkan fungsi rumah sakit sebagai model dan Pembina teknis dalam pelaksanaan IMD dan ASI Eksklusif serta Perawatan Metode Kanguru (PMK) pada BBLR
\item
  Meningkatkan fungsi rumah sakit sebagai pusat rujukan pelayanan kesehatan ibu dan bayi bagi sarana pelayanan kesehatan lainnya.
\item
  Melaksanakan pemantauan dan evaluasi pelaksanaan program RSSIB 10 langkah menyusui dan peningkatan kesehatan ibu
\item
  Melakukan pemantauan dan analisis yang meliputi:
\end{enumerate}

\begin{enumerate}
\def\labelenumi{\alph{enumi}.}
\tightlist
\item
  Angka keterlambatan operasi section caesaria
\item
  Angka kematian ibu dan anak
\item
  Kejadian tidak dilakukannya inisiasi menyusui dini (IMD) pada bayi baru lahir
\end{enumerate}

\hypertarget{c.-elemen-penilaian-prognas-1}{%
\subsubsection*{c.~Elemen Penilaian Prognas 1}\label{c.-elemen-penilaian-prognas-1}}
\addcontentsline{toc}{subsubsection}{c.~Elemen Penilaian Prognas 1}

\begin{enumerate}
\def\labelenumi{\arabic{enumi}.}
\tightlist
\item
  Rumah sakit menetapkan regulasi tentang pelaksanaan PONEK 24 jam.
\item
  Terdapat Tim PONEK yang ditetapkan oleh rumah sakit dengan rincian tugas dan tanggungjawabnya.
\item
  Terdapat program kerja yang menjadi acuan dalam pelaksanaan program PONEK Rumah Sakit sesuai maksud dan tujuan.
\item
  Terdapat bukti pelaksanaan program PONEK Rumah Sakit.
\item
  Program PONEK Rumah Sakit dipantau dan dievaluasi secara rutin.
\end{enumerate}

\hypertarget{d.-standar-prognas-1.1}{%
\subsubsection*{d.~Standar Prognas 1.1}\label{d.-standar-prognas-1.1}}
\addcontentsline{toc}{subsubsection}{d.~Standar Prognas 1.1}

Untuk meningkatkan efektifitas sistem rujukan maka Rumah sakit melakukan pembinaan kepada jejaring fasilitas Kesehatan rujukan yang ada.

\hypertarget{e.-maksud-dan-tujuan-prognas-1.1}{%
\subsubsection*{e. Maksud dan Tujuan Prognas 1.1}\label{e.-maksud-dan-tujuan-prognas-1.1}}
\addcontentsline{toc}{subsubsection}{e. Maksud dan Tujuan Prognas 1.1}

Salah satu tugas dari rumah sakit dengan kemampuan PONEK adalah melakukan pembinaan kepada jejaring rujukan seperti Puskesmas, Klinik bersalin, praktek perseorangan dan fasilitas pelayanan kesehatan lainnya. Pembinaan jejaring rujukan dapat dilakukan dengan mengadakan pelatihan kepada fasilitas kesehatan jejaring, berbagi pengalaman dalam pelayanan ibu dan anak serta peningkatanan kompetensi jejaring rujukan secara berkala. Rumah sakit memetakan jejaring rujukan yang ada dan membuat program pembinaan setiap tahun.

\hypertarget{f.-elemen-penilaian-standar-prognas-1.1}{%
\subsubsection*{f.~Elemen Penilaian Standar Prognas 1.1}\label{f.-elemen-penilaian-standar-prognas-1.1}}
\addcontentsline{toc}{subsubsection}{f.~Elemen Penilaian Standar Prognas 1.1}

\begin{enumerate}
\def\labelenumi{\arabic{enumi}.}
\tightlist
\item
  Rumah sakit menetapkan program pembinaan jejaring rujukan rumah sakit.
\item
  Rumah sakit melakukan pembinaan terhadap jejaring secara berkala.
\item
  Telah dilakukan evaluasi program pembinaan jejaring rujukan.
\end{enumerate}

\hypertarget{penurunan-angka-kesakitan-tuberkulosistbc}{%
\section*{2. Penurunan angka kesakitan Tuberkulosis/TBC}\label{penurunan-angka-kesakitan-tuberkulosistbc}}
\addcontentsline{toc}{section}{2. Penurunan angka kesakitan Tuberkulosis/TBC}

\hypertarget{a.-standar-prognas-2}{%
\subsubsection*{a. Standar Prognas 2}\label{a.-standar-prognas-2}}
\addcontentsline{toc}{subsubsection}{a. Standar Prognas 2}

Rumah sakit melaksanakan program penanggulangan tuberkulosis.

\hypertarget{b.-maksud-dan-tujuan-prognas-2}{%
\subsubsection*{b. Maksud dan Tujuan Prognas 2}\label{b.-maksud-dan-tujuan-prognas-2}}
\addcontentsline{toc}{subsubsection}{b. Maksud dan Tujuan Prognas 2}

Pemerintah mengeluarkan kebijakan penanggulangan tuberkulosis berupa upaya kesehatan yang mengutamakan aspek promotif, preventif, tanpa mengabaikan aspek kuratif dan rehabilitatif yang ditujukan untuk melindungi kesehatan masyarakat, menurunkan angka kesakitan, kecatatan atau kematian, memutuskan penularan mencegah resistensi obat dan mengurangi dampak negatif yang ditimbulkan akibat tuberkulosis.
Rumah sakit dalam melaksanakan penanggulangan tubekulosis melakukan kegiatan yang meliputi:

\begin{enumerate}
\def\labelenumi{\arabic{enumi}.}
\tightlist
\item
  Promosi kesehatan yang diarahkan untuk meningkatkan pengetahuan yang benar dan komprehensif mengenai pencegahan penularan, penobatan, pola hidup bersih dan sehat (PHBS) sehingga terjadi perubahan sikap dan perilaku sasaran yaitu pasien dan keluarga, pengunjung serta staf rumah sakit.
\item
  Surveilans tuberkulosis, merupakan kegiatan memperoleh data epidemiologi yang diperlukan dalam sistem informasi program penanggulangan tuberkulosis, seperti pencatatan dan pelaporan tuberkulosis sensitif obat, pencatatan dan pelaporan tuberkulosis resistensi obat.
\item
  Pengendalian faktor risiko tuberkulosis, ditujukan untuk mencegah, mengurangi penularan dan kejadian penyakit tuberkulosis, yang pelaksanaannya sesuai dengan pedoman pengendalian pencegahan infeksi tuberkulosis di rumah sakit pengendalian faktor risiko tuberkulosis, ditujukan untuk mencegah, mengurangi penularan dan kejadian penyakit tuberkulosis, yang pelaksanaannya sesuai dengan pedoman pengendalian pencegahan infeksi tuberkulosis di rumah sakit.
\item
  Penemuan dan penanganan kasus tuberkulosis.
  Penemuan kasus tuberkulosis dilakukan melalui pasien yang datang kerumah sakit, setelah pemeriksaan, penegakan diagnosis, penetapan klarifikasi dan tipe pasien tuberkulosis. Sedangkan untuk penanganan kasus dilaksanakan sesuai tata laksana pada pedoman nasional pelayanan kedokteran tuberkulosis dan standar lainnya sesuai dengan peraturan perundang- undangan.
\item
  Pemberian kekebalan.
  Pemberian kekebalan dilakukan melalui pemberian imunisasi BCG terhadap bayi dalam upaya penurunan risiko tingkat pemahaman tuberkulosis sesuai dengan peraturan perundang-undangan.
\item
  Pemberian obat pencegahan.
  Pemberian obat pencegahan selama 6 (enam) bulan yang ditujukan pada anak usia dibawah 5 (lima) tahun yang kontak erat dengan pasien tuberkulosis aktif; orang dengan HIV dan AIDS (ODHA) yang tidak terdiagnosis tuberkulosis; populasi tertentu lainnya sesuai peraturan perundang-undangan.
\end{enumerate}

Untuk menjalankan kegiatan tersebut maka rumah sakit dapat membentuk tim/panitia pelaksana program TB Paru Rumah Sakit.

\hypertarget{c.-elemen-penilaian-prognas-2}{%
\subsubsection*{c.~Elemen Penilaian Prognas 2}\label{c.-elemen-penilaian-prognas-2}}
\addcontentsline{toc}{subsubsection}{c.~Elemen Penilaian Prognas 2}

\begin{enumerate}
\def\labelenumi{\arabic{enumi}.}
\tightlist
\item
  Rumah sakit menerapkan regulasi tentang pelaksanaan penanggulangan tuberkulosis di rumah sakit.
\item
  Direktur menetapkan tim TB Paru Rumah sakit beserta program kerjanya.
\item
  Ada bukti pelaksanaan promosi kesehatan, surveilans dan upaya pencegahan tuberkulosis
\item
  Tersedianya laporan pelaksanaan promosi Kesehatan.
\end{enumerate}

\hypertarget{d.-standar-prognas-2.1}{%
\subsubsection*{d.~Standar Prognas 2.1}\label{d.-standar-prognas-2.1}}
\addcontentsline{toc}{subsubsection}{d.~Standar Prognas 2.1}

Rumah sakit menyediakan sarana dan prasarana pelayanan tuberkulosis sesuai peraturan perundang-undangan.

\hypertarget{e.-maksud-dan-tujuan-prognas-2.1}{%
\subsubsection*{e. Maksud dan Tujuan Prognas 2.1}\label{e.-maksud-dan-tujuan-prognas-2.1}}
\addcontentsline{toc}{subsubsection}{e. Maksud dan Tujuan Prognas 2.1}

Dalam melaksanakan pelayanan kepada penderita TB Paru dan program TB Paru di rumah sakit, maka harus tersedia sarana dan prasarana yang memenuhi syarat pelayanan TB Paru sesuai dengan Pedoman Pelayanan TB Paru.

\hypertarget{f.-elemen-penilaian-prognas-2.1}{%
\subsubsection*{f.~Elemen Penilaian Prognas 2.1}\label{f.-elemen-penilaian-prognas-2.1}}
\addcontentsline{toc}{subsubsection}{f.~Elemen Penilaian Prognas 2.1}

\begin{enumerate}
\def\labelenumi{\arabic{enumi}.}
\tightlist
\item
  Tersedia ruang pelayanan rawat jalan yang memenuhi pedoman pencegahan dan pengendalian infeksi tuberkulosis.
\item
  Bila rumah sakit memberikan pelayanan rawat inap bagi pasien tuberkulosis paru dewasa maka rumah sakit harus memiliki ruang rawat inap yang memenuhi pedoman pencegahan danpengendalian infeksi tuberkulosis.
\item
  Tersedia ruang pengambilan spesimen sputum yang memenuhi pedoman pencegahan dan pengendalian infeksi tuberkulosis.
\end{enumerate}

\hypertarget{g.-standar-prognas-2.2}{%
\subsubsection*{g. Standar Prognas 2.2}\label{g.-standar-prognas-2.2}}
\addcontentsline{toc}{subsubsection}{g. Standar Prognas 2.2}

Rumah sakit telah melaksanakan pelayanan tuberkulosis dan upaya pengendalian faktor risiko tuberkulosis sesuai peraturan perundang-undangan.

\hypertarget{h.-elemen-penilaian-prognas-2.2}{%
\subsubsection*{h. Elemen Penilaian Prognas 2.2}\label{h.-elemen-penilaian-prognas-2.2}}
\addcontentsline{toc}{subsubsection}{h. Elemen Penilaian Prognas 2.2}

\begin{enumerate}
\def\labelenumi{\arabic{enumi}.}
\tightlist
\item
  Rumah sakit telah menerapkan kepatuhan staf medis terhadap panduan praktik klinis tuberkulosis.
\item
  Rumah sakit merencanakan dan mengadakan penyediaan Obat Anti Tuberkulosis.
\item
  Rumah sakit melaksanakan pelayanan TB MDR (bagi rumah sakit rujukan TB MDR).
\item
  Rumah sakit melaksanakan pencatatan dan pelaporan kasus TB Paru sesuai ketentuan.
\end{enumerate}

\hypertarget{penurunan-angka-kesakitan-hivaids}{%
\section*{3. Penurunan angka kesakitan HIV/AIDS}\label{penurunan-angka-kesakitan-hivaids}}
\addcontentsline{toc}{section}{3. Penurunan angka kesakitan HIV/AIDS}

\hypertarget{a.-standar-prognas-3}{%
\subsubsection*{a. Standar Prognas 3}\label{a.-standar-prognas-3}}
\addcontentsline{toc}{subsubsection}{a. Standar Prognas 3}

Rumah sakit melaksanakan penanggulangan HIV/AIDS sesuai dengan peraturan perundang-undangan.

\hypertarget{b.-maksud-dan-tujuan-prognas-3}{%
\subsubsection*{b. Maksud dan Tujuan Prognas 3}\label{b.-maksud-dan-tujuan-prognas-3}}
\addcontentsline{toc}{subsubsection}{b. Maksud dan Tujuan Prognas 3}

Rumah sakit dalam melaksanakan penanggulangan HIV/AIDS sesuai standar pelayanan bagi rujukan orang dengan HIV/AIDS (ODHA) dan satelitnya dengan langkah-langkah sebagai berikut:

\begin{enumerate}
\def\labelenumi{\arabic{enumi}.}
\tightlist
\item
  Meningkatkan fungsi pelayanan Voluntary Counseling and Testing (VCT).
\item
  Meningkatkan fungsi pelayanan Antiretroviral Therapy (ART) atau bekerja sama dengan rumah sakit yang ditunjuk.
\item
  Meningkatkan fungsi pelayanan Infeksi Oportunistik (IO).
\item
  Meningkatkan fungsi pelayanan pada ODHA dengan factor risiko Injection Drug Use (IDU).
\item
  Meningkatkan fungsi pelayanan penunjang yang meliputi pelayanan gizi, laboratorium dan radiologi, pencatatan dan pelaporan.
\end{enumerate}

\hypertarget{c.-elemen-penilaian-prognas-3}{%
\subsubsection*{c.~Elemen Penilaian Prognas 3}\label{c.-elemen-penilaian-prognas-3}}
\addcontentsline{toc}{subsubsection}{c.~Elemen Penilaian Prognas 3}

\begin{enumerate}
\def\labelenumi{\arabic{enumi}.}
\tightlist
\item
  Rumah sakit telah melaksanakan kebijakan program penanggulangan HIV/AIDS sesuai ketentuan perundangan.
\item
  Rumah sakit telah menerapkan fungsi rujukan HIV/AIDS pada rumah sakit sesuai dengan kebijakan yang berlaku.
\item
  Rumah sakit melaksanakan pelayanan PITC dan PMTC.
\item
  Rumah sakit memberikan pelayanan ODHA dengan faktor risiko IO.
\item
  Rumah sakit merencanakan dan mengadakan penyediaan ART.
\item
  Rumah sakit melakukan pemantauan dan evaluasi program penanggulangan HIV/AIDS.
\end{enumerate}

\hypertarget{penurunan-prevalensi-stunting-dan-wasting}{%
\section*{4. Penurunan prevalensi stunting dan wasting}\label{penurunan-prevalensi-stunting-dan-wasting}}
\addcontentsline{toc}{section}{4. Penurunan prevalensi stunting dan wasting}

\hypertarget{a.-standar-prognas-4}{%
\subsubsection*{a. Standar Prognas 4}\label{a.-standar-prognas-4}}
\addcontentsline{toc}{subsubsection}{a. Standar Prognas 4}

Rumah Sakit melaksanakan program penurunan prevalensi
stunting dan wasting.

\hypertarget{b.-standar-prognas-4.1}{%
\subsubsection*{b. Standar Prognas 4.1}\label{b.-standar-prognas-4.1}}
\addcontentsline{toc}{subsubsection}{b. Standar Prognas 4.1}

Rumah Sakit melakukan edukasi, pendampingan intervensi dan pengelolaan gizi serta penguatan jejaring rujukan kepada rumah sakit kelas di bawahnya dan FKTP di wilayahnya serta rujukan masalah gizi.

\hypertarget{c.-maksud-dan-tujuan-prognas-4-dan-prognas-4.1}{%
\subsubsection*{c.~Maksud dan Tujuan Prognas 4 dan Prognas 4.1}\label{c.-maksud-dan-tujuan-prognas-4-dan-prognas-4.1}}
\addcontentsline{toc}{subsubsection}{c.~Maksud dan Tujuan Prognas 4 dan Prognas 4.1}

Tersedia regulasi penyelenggaraan program penurunan prevalensi stunting dan prevalensi wasting di rumah sakit yang meliputi:

\begin{enumerate}
\def\labelenumi{\arabic{enumi}.}
\tightlist
\item
  Program penurunan prevalensi stunting dan prevalensi
\item
  wasting.
\item
  Panduan tata laksana.
\item
  Organisasi pelaksana program terdiri dari tenaga kesehatan yang kompeten dari unsur:
\end{enumerate}

\begin{enumerate}
\def\labelenumi{\alph{enumi}.}
\tightlist
\item
  Staf Medis.
\item
  Staf Keperawatan.
\item
  Staf Instalasi Farmasi.
\item
  Staf Instalasi Gizi.
\item
  Tim Tumbuh Kembang.
\item
  Tim Humas Rumah Sakit.
\end{enumerate}

Organisasi program penurunan prevalensi stunting dan wasting dipimpin oleh staf medis atau dokter spesialis anak. Rumah sakit menyusun program penurunan prevalensi stunting dan wasting di rumah sakit terdiri dari:

\begin{enumerate}
\def\labelenumi{\arabic{enumi}.}
\tightlist
\item
  Peningkatan pemahaman dan kesadaran seluruh staf, pasien dan keluarga tentang masalah stunting dan wasting;
\item
  Intervensi spesifik di rumah sakit;
\item
  Penerapan Rumah Sakit Sayang Ibu Bayi;
\item
  Rumah sakit sebagai pusat rujukan kasus stunting dan
\item
  wasting;
\item
  Rumah sakit sebagai pendamping klinis dan manajemen serta merupakan jejaring rujukan
\item
  Program pemantauan dan evaluasi.
\end{enumerate}

Penurunan prevalensi stunting dan prevalensi wasting
meliputi:

\begin{enumerate}
\def\labelenumi{\arabic{enumi}.}
\tightlist
\item
  Kegiatan sosialisasi dan pelatihan staf tenaga kesehatan rumah sakit tentang Program Penurunan Stunting dan Wasting.
\item
  Peningkatan efektifitas intervensi spesifik.
\end{enumerate}

\begin{enumerate}
\def\labelenumi{\alph{enumi}.}
\tightlist
\item
  Program 1000 HPK.
\item
  Suplementasi Tablet Besi Folat pada ibu hamil.
\item
  Pemberian Makanan Tambahan (PMT) pada ibu hamil.
\item
  Promosi dan konseling IMD dan ASI Eksklusif.
\item
  Pemberian Makanan Bayi dan Anak (PMBA).
\item
  Pemantauan Pertumbuhan (Pelayanan Tumbuh Kembang bayi dan balita).
\item
  Pemberian Imunisasi.
\item
  Pemberian Makanan Tambahan Balita Gizi Kurang.
\item
  Pemberian Vitamin A.
\item
  Pemberian taburia pada Baduta (0-23 bulan).
\item
  Pemberian obat cacing pada ibu hamil.
\end{enumerate}

\begin{enumerate}
\def\labelenumi{\arabic{enumi}.}
\setcounter{enumi}{2}
\tightlist
\item
  Penguatan sistem surveilans gizi
\end{enumerate}

\begin{enumerate}
\def\labelenumi{\alph{enumi}.}
\tightlist
\item
  Tata laksana tim asuhan gizi meliputi Tata laksana Gizi Stunting, Tata Laksana Gizi Kurang, Tata Laksana Gizi Buruk (Pedoman Pencegahan dan Tata Laksana Gizi Buruk pada Balita).
\item
  Pencatatan dan Pelaporan kasus masalah gizi melalui aplikasi ePPGBM (Aplikasi Pencatatan dan Pelaporan Gizi Berbasis Masyarakat).
\item
  Melakukan evaluasi pelayanan, audit kesakitan dan kematian, pencatatan dan pelaporan gizi buruk dan stunting dalam Sistem Informasi Rumah Sakit (SIRS).
\end{enumerate}

Rumah sakit melaksanakan pelayanan sebagai pusat rujukan kasus stunting dan kasus wasting dengan menyiapkan sebagai:

\begin{enumerate}
\def\labelenumi{\arabic{enumi}.}
\tightlist
\item
  Rumah sakit sebagai pusat rujukan kasus stunting untuk memastikan kasus, penyebab dan tata laksana lanjut oleh dokter spesialis anak.
\item
  Rumah sakit sebagai pusat rujukan balita gizi buruk dengan komplikasi medis.
\item
  Rumah sakit dapat melaksanakan pendampingan klinis dan manajemen serta penguatan jejaring rujukan kepada rumah sakit dengan kelas di bawahnya dan Fasilitas Kesehatan Tingkat Pertama (FKTP) di wilayahnya dalam tata laksana stunting dan gizi buruk.
\end{enumerate}

\hypertarget{d.-elemen-penilaian-prognas-4}{%
\subsubsection*{d.~Elemen Penilaian Prognas 4}\label{d.-elemen-penilaian-prognas-4}}
\addcontentsline{toc}{subsubsection}{d.~Elemen Penilaian Prognas 4}

\begin{enumerate}
\def\labelenumi{\arabic{enumi}.}
\tightlist
\item
  Rumah sakit telah menetapkan kebijakan tentang pelaksanaan program gizi.
\item
  Terdapat tim untuk program penurunan prevalensi stunting dan wasting di rumah sakit.
\item
  Rumah sakit telah menetapkan sistem rujukan untuk kasus gangguan gizi yang perlu penanganan lanjut.
\end{enumerate}

\hypertarget{e.-elemen-penilaian-prognas-4.1}{%
\subsubsection*{e. Elemen Penilaian Prognas 4.1}\label{e.-elemen-penilaian-prognas-4.1}}
\addcontentsline{toc}{subsubsection}{e. Elemen Penilaian Prognas 4.1}

\begin{enumerate}
\def\labelenumi{\arabic{enumi}.}
\tightlist
\item
  Rumah sakit membuktikan telah melakukan pendampingan intervensi dan pengelolaan gizi serta penguatan jejaring rujukan kepada rumah sakit kelas di bawahnya dan FKTP di wilayahnya serta rujukan masalah gizi.
\item
  Rumah sakit telah menerapkan sistem pemantauan dan evaluasi, bukti pelaporan, dan analisis.
\end{enumerate}

\hypertarget{pelayanan-keluarga-berencana-rumah-sakit}{%
\section*{5. Pelayanan Keluarga Berencana Rumah Sakit}\label{pelayanan-keluarga-berencana-rumah-sakit}}
\addcontentsline{toc}{section}{5. Pelayanan Keluarga Berencana Rumah Sakit}

\hypertarget{a.-standar-prognas-5}{%
\subsubsection*{a. Standar Prognas 5}\label{a.-standar-prognas-5}}
\addcontentsline{toc}{subsubsection}{a. Standar Prognas 5}

Rumah sakit melaksanakan program pelayanan keluarga berencana dan kesehatan reproduksi di rumah sakit beserta pemantauan dan evaluasinya.

\hypertarget{b.-standar-prognas-5.1}{%
\subsubsection*{b. Standar Prognas 5.1}\label{b.-standar-prognas-5.1}}
\addcontentsline{toc}{subsubsection}{b. Standar Prognas 5.1}

Rumah sakit menyiapkan sumber daya untuk penyelenggaraan pelayanan keluarga dan kesehatan reproduksi.

\hypertarget{c.-maksud-dan-tujuan-prognas-5-dan-prognas-5.1}{%
\subsubsection*{c.~Maksud dan Tujuan Prognas 5 dan Prognas 5.1}\label{c.-maksud-dan-tujuan-prognas-5-dan-prognas-5.1}}
\addcontentsline{toc}{subsubsection}{c.~Maksud dan Tujuan Prognas 5 dan Prognas 5.1}

Pelayanan Keluarga Berencana di Rumah Sakit (PKBRS) merupakan bagian dari program keluarga berencana (KB), yang sangat berperan dalam menurunkan angka kematian ibu dan percepatan penurunan stunting. Kunci keberhasilan PKBRS adalah ketersediaan alat dan obat kontrasepsi, sarana penunjang pelayanan kontrasepsi dan tenaga kesehatan yang sesuai kompetensi serta manjemen yang handal. Rumah sakit dalam melaksanakan PKBRS sesuai dengan pedoman pelayanan KB yang berlaku, dengan langkah-langkah pelaksanaan sebagai berikut:

\begin{enumerate}
\def\labelenumi{\arabic{enumi}.}
\tightlist
\item
  Melaksanakan dan menerapkan standar pelayanaan KB secara terpadu dan paripurna.
\item
  Mengembangkan kebijakan dan Standar Prosedur Operasional (SPO) pelayanan KB dan meningkatkan kualitas pelayanan KB.
\item
  Meningkatkan kesiapan rumah sakit dalam melaksanakan PKBRS termasuk pelayanan KB Pasca Persalinan dan Pasca Keguguran.
\item
  Meningkatkan fungsi rumah sakit sebagai model dan pembinaan teknis dalam melaksanakan PKBRS.
\item
  Meningkatkan fungsi rumah sakit sebagai pusat rujukan pelayanan KB bagi sarana pelayanan kesehatan lainnya.
\item
  Melaksanakan sistem pemantauan dan evaluasi pelaksanaan PKBRS.
\item
  Adanya regulasi rumah sakit yang menjamin pelaksanaan PKBRS, meliputi SPO pelayanan KB per metode kontrasepsi termasuk pelayanan KB Pasca Persalinan dan Pasca Keguguran.
\item
  Upaya peningkatan PKBRS masuk dalam rencana strategis (Renstra) dan rencana kerja anggaran (RKA) rumah sakit.
\item
  Tersedia ruang pelayanan yang memenuhi persyaratan untuk PKBRS antara lain ruang konseling dan ruang pelayanan KB.
\item
  Pembentukan tim PKBR serta program kerja dan bukti pelaksanaanya.
\item
  Terselenggara kegiatan peningkatan kapasitas untuk meningkatkan kemampuan pelayanan PKBRS, termasuk KB Pasca Persalinan dan Pasca Keguguran.
\item
  Pelaksanaan rujukan sesuai dengan ketentuan peraturan perundangan-undangan.
\item
  Pelaporan dan analisis meliputi:
\end{enumerate}

\begin{enumerate}
\def\labelenumi{\alph{enumi}.}
\tightlist
\item
  Ketersediaan semua jenis alat dan obat kontrasepsi sesuai dengan kapasitas rumah sakit dan kebutuhan pelayanan KB.
\item
  Ketersediaan sarana penunjang pelayanan KB.
\item
  Ketersediaan tenaga kesehatan yang memberikan pelayanan KB.
\item
  Angka capaian pelayanan KB per metode kontrasepsi, baik Metode Kontrasepsi Jangka Panjang (MKJP) dan Non MKJP.
\item
  Angka capaian pelayanan KB Pasca Persalinan dan Pasca Keguguran.
\item
  Kejadian tidak dilakukannya KB Pasca Persalinan pada ibu baru bersalin dan KB Pasca Keguguran pada Ibu pasca keguguran.
\end{enumerate}

\hypertarget{d.-elemen-penilaian-prognas-5}{%
\subsubsection*{d.~Elemen Penilaian Prognas 5}\label{d.-elemen-penilaian-prognas-5}}
\addcontentsline{toc}{subsubsection}{d.~Elemen Penilaian Prognas 5}

\begin{enumerate}
\def\labelenumi{\arabic{enumi}.}
\tightlist
\item
  Rumah sakit telah menetapkan kebijakan tentang pelaksanaan PKBRS.
\item
  Terdapat tim PKBRS yang ditetapkan oleh direktur disertai program kerjanya.
\item
  Rumah sakit telah melaksanakan program KB Pasca Persalinan dan Pasca Keguguran.
\item
  Rumah sakit telah melakukan pemantauan dan evaluasi pelaksanaan PKBRS.
\end{enumerate}

\hypertarget{e.-elemen-penilaian-prognas-5.1}{%
\subsubsection*{e. Elemen Penilaian Prognas 5.1}\label{e.-elemen-penilaian-prognas-5.1}}
\addcontentsline{toc}{subsubsection}{e. Elemen Penilaian Prognas 5.1}

\begin{enumerate}
\def\labelenumi{\arabic{enumi}.}
\tightlist
\item
  Rumah sakit telah menyediakan alat dan obat kontrasepsi dan sarana penunjang pelayanan KB.
\item
  Rumah sakit menyediakan layanan konseling bagi peserta dan calon peserta program KB.
\item
  Rumah sakit telah merancang dan menyediakan ruang pelayanan KB yang memadai.
\end{enumerate}

  \bibliography{book.bib,packages.bib}

\end{document}
